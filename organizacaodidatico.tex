\newpage
\section{Organiza\c{c}\~ao Did\'atico-Pedag\'ogica}

\subsection{Concep\c{c}\~ao do Curso}

A Paraíba está inserida há um bom tempo no circuito nacional e internacional de tecnologia de informação e comunicação, tendo como destaque a cidade de Campina Grande que apresenta na área tecnológica uma das molas do seu desenvolvimento. A cidade destaca-se na vocação da região no desenvolvimento de novas tecnologias no campo da Engenharia Elétrica e de Informática, devido principalmente à influência da UFCG, com seus Cursos de Engenharia Elétrica e Ciências da Computação, ambos classificados entre os melhores do país.

Como resultado dessa vocação, observa-se o aumento do número de empresas de base tecnológica e incubadas no Parque Tecnológico da Paraíba. Cabe ressaltar, entretanto, que, apesar desta posição de destaque, há uma carência na formação de profissionais qualificados, para serem absorvidos pelo pólo de tecnologia da região.

Pelo panorama apresentado, necessita-se formar profissionais de nível superior, preparados para enfrentar os novos desafios que surgem no mercado, capacitados para atuar nas diversas áreas tecnológicas. Além disso, deve-se buscar a formação humana, necessária à condução de projetos, agregando ao indivíduo o espírito criativo, essencial à inovação tão exigida no mundo competitivo de hoje. Ciente desta realidade e consciente do seu papel no contexto da educação brasileira, o Campus Guarabira do IFPB apresenta o Curso Superior de Tecnologia em Sistemas para Internet, entendendo que este é um espaço promissor no que tange à geração de emprego, atendendo às demandas da sociedade e ao desenvolvimento econômico da região.

O curso de Sistemas para Internet no Campus Guarabira do IFPB foi concebido com base nas recomendações do MEC, estando fundamentado nas habilidades, competências e conhecimentos necessários à formação, inovador, ciente de seu papel e responsabilidade na sociedade. Assim, o curso tem por objetivo formar um profissional que possua uma boa e sólida formação formação tecnológica, para garantir a sua inserção e competitividade no mercado de trabalho.

Para atender a esses pressupostos, na definição do Curso de Sistemas para Internet, considerou-se obter a formação de um profissional com características que atendessem à atual demanda do mercado de trabalho. Assim, esse curso propôs-se habilitar profissionais com conhecimentos nas áreas de Computação, Desenvolvimento de Software, Redes de Computadores e Sistemas Distribu\'idos para o desenvolvimento de soluções inovadoras em projetos de sistemas computacionais, principalmente voltados para tecnologias emergentes, como \'e o caso de tecnologias para a WEB e dispositivos m\'oveis. O Curso de Sistemas para Internet possui uma formação sólida nos fundamentos de Computação. O profissional em Sistemas para Internet estará habilitado para atuar em todas nas áreas em que os conhecimentos de computação são essenciais e complementares.  

O egresso do Curso de Sistemas para Internet receberá o conhecimento necessário para prosseguir em estudos de pós-graduação em razão do fundamentado conhecimento obtido nas disciplinas da área básica do curso e nas atividades realizadas em projetos de pesquisa e extensão que incentivam a busca por novos desafios.

\subsubsection{Justificativas do curso}

A expansão da computação é verificada pela quantidade e diversidade de sistemas computacionais utilizados em diferentes segmentos, seja no trabalho, na educação e entretenimento. No trabalho, estes sistemas têm sido empregados nas mais diversas áreas e finalidades: desde a automatização de fábricas à terapia ocupacional, sendo essenciais nas comunicações, fortemente presente na Internet e nos aplicativos web. No âmbito da educação, os referidos têm auxiliado, seja como suporte gerencial ou como ferramenta de apoio ao processo de ensino-aprendizagem. Na área de entretenimento, estão os jogos que utilizam as mais sofisticadas técnicas de projeto gráfico e conceitos como os de Inteligência Artificial. Fora esse contexto, existem diversos dispositivos, como os eletrodomésticos e aparelhos eletrônicos, com funcionalidades implementadas por meio de hardware e software.


O interesse pelo Curso de Sistemas para Internet vêm da necessidade de mão de obra qualificada na área de Tecnologia da Informação e Comunicação (TIC), como vem sendo noticiado constantemente pela imprensa. É importante também notar que o tecnólogo em Sistemas para Internet é capaz de prestar concurso para qualquer cargo de nível superior na área de TI, uma vez que esses concursos não restringem a formação apenas para bacharelado. Além disso, a Paraíba se destaca como grande formadora de mão de obra qualificada na área de TI, possuindo dois programas de pós-graduação, contando com dois cursos de mestrado e um de doutorado.%, o que facilita a captação de mão de obra qualificada para atuar no curso.

A expansão das Instituições de Ensino na área de informática ou computação, tem respaldo na exigência do mercado pelo aumento do número de profissionais e pelo desenvolvimento dessa área no País e no mundo. Guarabira est\'a localizada de forma privilegiada, a cerca de 100 km da capital Paraibana e a cerca de 90 km de Campina Grande, cidade que possui atua\c{c}\~ao destacada nas \'area de TIC, contando com um parque tecnol\'ogico e v\'arias empresas dessa \'area. O curso de tecnologia em Sistemas para Internet do IFPB - Campus Guarabira ser\'a o primeiro curso superior da \'area de inform\'atica a ser ofertado por uma institui\c{c}\~ao p\'ublica na região do brejo e o Campus Guarabira ser\'a o segundo campus a oferecer o curso em todo o IFPB (atualmente apenas o campus Jo\~ao Pessoa oferece esse curso).

O curso de Sistemas para Internet trabalha com tecnologias emergentes e que oferecem muitas oportunidades de emprego, como é o caso de desenvolvimento de software para web e desenvolvimento de aplicativos para dispositivos móveis. Além disso, a formação do profissional de TSI prevê um bom nível de conhecimento em redes de computadores, incluindo segurança de redes, áreas que também estão sendo demandadas pelo mercado de trabalho.

Outra motiva\c{c}\~ao para a criação do curso é o fortalecimento da área de informática no campus Guarabira, que já conta com dois cursos técnicos (um integrado e um subsequente). A existência do curso técnico na área de informática contribui de duas formas para a criação do curso superior. Primeiramente, oferece estrutura e recursos humanos para permitir o início do curso. Em segundo lugar, os alunos egressos dos cursos técnicos em informática terão oportunidade de continuar os estudos em nível superior sem precisar sair de Guarabira. Esse ponto, além de aumentar a demanda pelo curso, provavelmente aumentará o nível dos alunos ingressantes, já que espera-se que muitos alunos virão com conhecimentos adquiridos nos cursos técnicos. Por último, é interessante observar que o Campus de João Pessoa é a única instituição pública a ofertar tal curso na Paraíba e a demanda por profissionais dessa área vem crescendo significativamente. Guarabira, por ser a cidade mais importante comercialmente para a Microrregião do Brejo, só tem a ganhar com a criação do curso no Campus, já que é notório o crescimento do \textit{e-commerce} no Brasil. Para inserir a cidade nessa nova realidade são necessários profissionais do curso de Sistemas para Internet. %Além disso, há a possibilidade de interdisciplinaridade entre com o curso Tecnólogo em Gestão Comercial, pois mesmo sendo de áreas distintas, há uma convergência muito forte entre os dois cursos devido à vocação comercial da região.

	Outro aspecto a ser levado em consideração são as atividades já sendo desenvolvidas no campus Guarabira na área de informática. Um grande conjunto de atividades já foi desenvolvido e outro já está em andamento, envolvendo principalmente atividades de pesquisa. A seguir são listadas as atividades e projetos desenvolvidos pelos professores de informática nos últimos tr\^es anos:

\begin{itemize}
\item Criação do Grupo de Pesquisa em Redes de Computadores e Sistemas Distribuídos (GPRSD), liderado pelos professores Ruan Delgado Gomes e Erick Augusto Gomes de Melo. O grupo  conta atualmente com 7 (sete) pesquisadores do IFPB e da UFPB, e 7 (sete) alunos dos cursos técnicos em informática;

\item Aprovação no GPRSD no Edital 07/2012 - Programa de Apoio ao Fortalecimento dos Grupos de Pesquisa do IFPB;

\item Participação do GPRSD no projeto ``Forma-Engenharia - Desenvolvimento de Sistemas de Monitoramento e Controle de Motores de Indução Trifásicos''. Financiado pelo CNPq, por meio do edital da chamada CNPQ/VALE S.A. Nº 05/2012. Nesse projeto dois alunos do curso técnico subsequente em informática foram bolsistas de iniciação tecnológica e industrial – nível B do CNPq;

\item Participação do GPRSD no projeto ``Projeto Ver Brasil: Sistema Brasileiro de Exibição Digital'', financiado pelo Ministério da Cultura e Secretaria de Audiovisual em parceria com o Núcleo de Pesquisa LAViD, da UFPB. Nesse projeto um aluno do curso integrado em informática e um aluno do curso subsequente em informática foram bolsistas, além de três professores do campus da área de informática;

\item Criação do Grupo de Pesquisa em Sistemas Digitais (GPSDi), liderado pelos professores Otacílio de Araújo Ramos Neto e Ruan Delgado Gomes. O grupo conta atualmente com 2 (dois) pesquisadores e 7 (sete) alunos dos cursos Técnicos em Informática;

\item Aprovação do GPSDi na chamada MEC/SETEC/CNPq N º 94/2013 – Apoio a Projetos Cooperativos de Pesquisa Aplicada e de Extensão Tecnológica, com o projeto “Grupo de Estudos em Robótica Educacional”.  Nesse projeto, alunos do curso técnico integrado em informática estão se preparando para a olimpíada brasileira de robótica;

\item Participação no projeto ``Projeto Olímpico de Informática no IFPB'', financiado por meio da chamada MEC/SETEC/CNPq N º 94/2013 – Apoio a Projetos Cooperativos de Pesquisa Aplicada e de Extensão Tecnológica. Por meio desse projeto foram obtidas quatro medalhas na olimp\'iada paraibana de inform\'atica e uma men\c{c}\~ao honrosa na olim\'ipada brasileira de inform\'atica;

\item Aprovação do professor Ruan Delgado Gomes no edital 25-2013 Bolsa Pesquisador do IFPB, com o projeto intitulado “Desafios e Aplicações de Redes de Sensores sem Fio Industriais”;

\item Um total de 8 (quatro) bolsistas PIBIC-EM e 2 (dois) voluntários envolvidos em projetos de pesquisa aprovados nos editais de PIBIC-EM;

\item Aprovação no GPRSD no Edital 17/2014 - Programa de Apoio ao Fortalecimento dos Grupos de Pesquisa do IFPB;

\item Classifica\c{c}\~ao para a final do concurso de trabalhos t\'ecnicos em inform\'atica do evento Computer on the Beach 2015;

\item Foram publicados oito artigos em peri\'odico, oito artigos em anais de eventos e dois cap\'itulos de livro.

\end{itemize}

%Os resultados descritos demonstram o potencial do corpo docente e discente da \'area de inform\'atica do campus Guarabira.

%Quanto à vocação regional da cidade de Campina Grande, a computação está alicerçada nas empresas da área de informática existentes, incluindo as de consultoria em tecnologia de informação e comunicação, de desenvolvimento de soluções de hardware e software para diversos segmentos econômicos. No ano de 2001, edição de abril, a revista norte-americana, Newsweek, escolheu Campina Grande dentre as nove cidades de destaque no mundo que representam um novo modelo de centro tecnológico. Tal escolha não foi por acaso, tendo em vista que, atualmente existem doze indústrias voltadas a atividades de fabricação e serviços relacionados a informática com sede em Campina Grande, além de empresas de confecção de material eletrônico e equipamentos de comunicação.
%Essa vocação é sustentada por entidades e empresas que se agregam para fomentar e prover o desenvolvimento da área de tecnologia, destacando-se a fundação do Parque Tecnológico (PaqTcPB) e do Centro de Inovação Tecnológica Telmo Araújo (CITTA), que possui o objetivo de auxiliar na consolidação de novos negócios, incubando e apoiando as empresas que possuem ênfase em tecnologia e inovação durante os diversos estágios do negócio.
%O PaqTcPB, em 2013, possuía um total de 21 empresas incubadas, dessas, 15 são de Campina Grande. O CITTA está com uma previsão de investimento inicial de aproximadamente R$ 4 milhões e deverá sediar uma média de 50 empresas voltadas à produção de tecnologia. Esses números ilustram uma Campina Grande com perfil empreendedor, com projetos nas áreas de produção de software, geoprocessamento, setor eletroeletrônico e biotecnologia.

%colocar subtopicos

\subsubsection{Objetivos do curso}

\textbf{Geral}

\vspace{5mm}
Formar profissional capaz de atender as demandas da sociedade e do mercado de trabalho, contribuindo para a evolução do conhecimento do ponto de vista científico e tecnológico, aplicando-o na avaliação, especificação e desenvolvimento de ferramentas, métodos e sistemas computacionais. O curso prima pela formação humanística, respeitando princípios éticos, permitindo ao profissional a compreensão do mundo, com visão crítica e consistente do impacto da profissão de Sistemas para Internet na sociedade.

\vspace{5mm}
\textbf{Específicos}

\vspace{5mm}

De acordo com o cat\'alogo do MEC, o profissional formado no curso superior de Tecnologia em Sistemas para Internet, poder\'a ter as seguintes atribui\c{c}\~oes:

\begin{itemize}

\item atuar no desenvolvimento de programas, interfaces e aplicativos;
\item atuar nos ramos de com\'ercio e marketing eletr\^onicos;
\item atuar no desenvolvimento de p\'aginas e portais para internet e intranet;
\item atuar no gerenciamento de projetos de sistemas, inclusive com acesso a banco de dados;
\item atuar no desenvolvimento de projetos e aplica\c{c}\~oes para a rede mundial de computadores e na intergra\c{c}\~ao de m\'idias nas p\'aginas da internet;
\item atuar com tecnologias emergentes como: computa\c{c}\~ao m\'ovel, redes sem fio e sistemas distribu\'idos;
\item cuidar da implanta\c{c}\~ao, atualiza\c{c}\~ao, manuten\c{c}\~ao e seguran\c{c}a dos sistemas para internet.
\end{itemize}

Al\'em disso, o curso de Sistemas para Internet tem como objetivo formar profissionais que possam:

\begin{itemize}
\item Comunicar-se eficientemente nas formas escrita, oral e gráfica;
\item Atuar em equipes multidisciplinares;
\item Compreender e aplicar ética e responsabilidade profissional;
\item Estar preparado para necessidade de atualização profissional constante;
\item Assumir a postura de permanente busca de atualização profissional;
\end{itemize}

\subsection{Pol\'iticas Institucionais e sua Correla\c{c}\~ao com o Curso}

%colocar subtopicos

\subsection{Organiza\c{c}\~ao Curricular}

%colocar subtopicos (aqui entra fluxograma, ementario etc)

\subsubsection{Ement\'ario e Bibliografia}

    \paragraph{Programação Orientada a Objetos}

%PREENCHER DADOS DA DISCIPLINA A SEGUIR
%\vspace{-12mm}
\begin{center}\textbf{Dados do Componente Curricular}\end{center}
\vspace{-5mm}
\noindent\rule{16.5cm}{0.4pt}
\\
\textbf{Nome:} Programação Orientada a Objetos
\\
\textbf{Curso:} Tecnologia em Sistemas para Internet
\\ 
\textbf{Período:} $2^{\circ}$ 
\\
\textbf{Carga Horária:} 83~h 
\\ 
\textbf{Docente Responsável:} José de Sousa Barros 
\\ 
\noindent\rule{16.5cm}{0.4pt}\\
\\
%PREENCHER A EMENTA A SEGUIR
\vspace{-12mm}
\begin{center}\textbf{Ementa}\end{center}
\vspace{-5mm}
\noindent\rule{16.5cm}{0.4pt}
\\
O paradigma de programação orientada a objetos: abstração, conceito de classes e objetos, troca de mensagens entre objetos, composição de objetos, encapsulamento, empacotamento de classes, visibilidade, coleções de objetos, herança, sobrescrita, sobrecarga, interface e polimorfismo, tratamento de exceções, persistência de dados em arquivos. \\
\noindent\rule{16.5cm}{0.4pt}\\
\\
%PREENCHER OS OBJETIVOS A SEGUIR
\vspace{-12mm}
\begin{center}\textbf{Objetivos}\end{center}
\vspace{-5mm}
\noindent\rule{16.5cm}{0.4pt}
\\
\begin{itemize}
\item Identificar os conceitos do paradigma de programação orientado a objetos;
\item Utilizar os conceitos do paradigma de programação orientado a objetos;
\item Desenvolver aplicações em uma linguagem de programação Orientada a Objetos.
\end{itemize}
\noindent\rule{16.5cm}{0.4pt}\\
\\
%PREENCHER OS CONTEUDOS PROGRAMATICOS A SEGUIR (CUIDADO PARA NAO DEIXAR A TABELA MUITO GRANDE)
\vspace{-12mm}
\begin{center}\textbf{Conteúdo Programático}\end{center}
\vspace{-5mm}
\noindent\rule{16.5cm}{0.4pt}
\\
\begin{itemize}
 \item \textbf{Introdução à Programação Orientada a Objetos:} Abstração; Modelagem orientada a objetos; Apresentação de uma linguagem de programação orientada a objetos;	Classes; Objetos; Construtores; Métodos; Encapsulamento e visibilidade, Pacotes.


 \item \textbf{Herança e Polimorfismo:} Membros de classe: atributos e métodos (de classe e de instância); Herança;	Classes abstratas; Métodos abstratos; Sobrescrita de métodos; Sobrecarga de métodos; Interfaces;	Polimorfismo; Coleções estáticas.

 \item \textbf{Coleções dinâmicas e Tratamento de exceções:} Generics; Coleções dinâmicas: Collection, List, Queue, Deque, Set e SortedSet; Tratamento de exceções; Interface gráfica; Manipulação de eventos; Persistência de dados em arquivos.
\end{itemize}
\noindent\rule{16.5cm}{0.4pt}\\
\\
%COLOCAR A METODOLOGIA DE ENSINO A SEGUIR
\vspace{-12mm}
\begin{center}\textbf{Metodologia de Ensino}\end{center} 
\vspace{-5mm}
\noindent\rule{16.5cm}{0.4pt}
\\
   Aulas expositivas utilizando recursos audiovisuais e quadro, além de aulas práticas utilizando computadores. Adicionalmente, serão realizadas atividades práticas individuais ou em grupo, para consolidação do conteúdo ministrado.\\
\noindent\rule{16.5cm}{0.4pt}\\
\\
%COLOCAR AVALIACAO DO PROCESSO DE ENSINO E APRENDIZAGEM A SEGUIR
\vspace{-12mm}
\begin{center}\textbf{Avaliação do Processo de Ensino e Apendizagem}\end{center}
\vspace{-5mm}
\noindent\rule{16.5cm}{0.4pt}
\\
   Avaliações escritas ao final de cada unidade. Prática baseada em Estudo de Caso ou problema real.\\
\noindent\rule{16.5cm}{0.4pt}\\
\\
%PREENCHER RECURSOS NECESSARIOS A SEGUIR
\vspace{-12mm}
\begin{center}\textbf{Recursos Necessários}\end{center}
\vspace{-5mm}
\noindent\rule{16.5cm}{0.4pt}
\\
\begin{itemize} 
  \item Listas de Exercícios;
  \item Livros e apostilas;
  \item Utilização de recursos da web;
  \item Quadro branco;
  \item Marcadores para quadro branco;
  \item Sala de aula com acesso à internet, microcomputador e TV ou projetor para apresentação de slides ou material multimídia;
  \item Laboratório de microcomputadores contendo componentes de hardware e software específicos;
\end{itemize}
\noindent\rule{16.5cm}{0.4pt}\\
\\
%PREENCHER BIBLIOGRAFIA A SEGUIR
\vspace{-12mm}
\begin{center}\textbf{Bibliografia}\end{center}
\vspace{-5mm}
\noindent\rule{16.5cm}{0.4pt}
\\
\begin{itemize} 
  \item Básica;
	\begin{enumerate}
  	\item DEITEL, H. M.; DEITEL, P. J. \textbf{Java: Como Programar.} Pearson, 8ª Edição, 2010;
	\item FURGERI, S. \textbf{Java 7 Ensino Didático.} Érica, 1ª Edição, 2010;
	\item SIERRA K.; BATES, B. \textbf{Use a Cabeça! - Java.} Alta Books, 2ª Edição, 2007.
	\end{enumerate}
  \item Complementar;
	\begin{enumerate}
  	\item HORSTMANN, C. S. \& CORNELL, G. \textbf{Core Java, Volume 1.} Pearson, 8ª edição, 2010;
	\item CADENHEAD, R.; LEMAY, L. \textbf{Aprenda Java em 21 Dias.} Campus, 4ª edição, 2005.
	\end{enumerate}
\end{itemize}
\noindent\rule{16.5cm}{0.4pt}\\
\\


%\section{Dados da Institui\c{c}\~ao}
\subsection{Sistemas Distribuídos}

\begin{table}[h!]
%\caption{Dados da Instituição}
\centering

% definindo o tamanho da fonte para small
% outros possíveis tamanhos: footnotesize, scriptsize
\begin{small} 
  
% redefinindo o espaçamento das colunas
\setlength{\tabcolsep}{3pt} 
\begin{tabular}{|p{15cm}|}\hline
 %& \multicolumn{4}{c|}{Trimestres}\\ \cline{2-5}
%\raisebox{1.5ex}{Etapa} & 01-03/2014 & 04-06/2014 & 07-09/2014 & 10-12/2014 \\ \hline

\begin{center}\textbf{Dados do Componente Curricular}\end{center}\\ \hline
\textbf{Nome:} Sistemas Distribuídos \\ \hline
\textbf{Curso:} Tecnologia em Sistemas para Internet \\ \hline
\textbf{Período:} $6^{\circ}$ \\ \hline
\textbf{Carga Horária:} 67~h \\ \hline
\textbf{Docente Responsável:} Ruan Delgado Gomes \\ \hline
\end{tabular} 
\end{small}
\label{dadosinstituicao}
\end{table} 

\begin{table}[h!]
%\caption{Dados da Instituição}
\centering

% definindo o tamanho da fonte para small
% outros possíveis tamanhos: footnotesize, scriptsize
\begin{small} 
  
% redefinindo o espaçamento das colunas
\setlength{\tabcolsep}{1pt} 
\begin{tabular}{|p{15cm}|}\hline
 %& \multicolumn{4}{c|}{Trimestres}\\ \cline{2-5}
%\raisebox{1.5ex}{Etapa} & 01-03/2014 & 04-06/2014 & 07-09/2014 & 10-12/2014 \\ \hline

\begin{center}\textbf{Ementa}\end{center}\\ \hline
Fundamentos de Sistemas Distribuídos. Estilos Arquiteturais para Sistemas Distribuídos. P2P. Processos e Threads. Arquitetura de Comunicação Cliente-Servidor. Comunicação: Sockets, RPC, RMI, MOM. Serviços: Conceitos, Arquitetura Orientada a Serviços, Tipos de Serviços, Design de Serviços, Registro e descoberta, Web Services. Tolerância a Faltas. Sincronização. \\ \hline
\end{tabular} 
\end{small}
\label{dadosinstituicao}
\end{table} 

\hspace{1cm}
\begin{table}[h!]
%\caption{Dados da Instituição}
\centering

% definindo o tamanho da fonte para small
% outros possíveis tamanhos: footnotesize, scriptsize
\begin{small} 
  
% redefinindo o espaçamento das colunas
\setlength{\tabcolsep}{3pt} 
\begin{tabular}{|p{15cm}|}\hline
 %& \multicolumn{4}{c|}{Trimestres}\\ \cline{2-5}
%\raisebox{1.5ex}{Etapa} & 01-03/2014 & 04-06/2014 & 07-09/2014 & 10-12/2014 \\ \hline

\begin{center}\textbf{Objetivos}\end{center}\\ \hline
\begin{itemize}
\item Proporcionar o entendimento sobre as possíveis formas de estruturação dos sistemas distribuídos;
\item Conhecer e utilizar técnicas para garantir a qualidade de sistemas distribuídos;
\item Saber como resolver problemas de faltas em sistemas distribuídos.
\end{itemize}
 \\ \hline
\end{tabular} 
\end{small}
\label{dadosinstituicao}
\end{table}

\hspace{1cm}
\begin{table}[h!]
%\caption{Dados da Instituição}
\centering

% definindo o tamanho da fonte para small
% outros possíveis tamanhos: footnotesize, scriptsize
\begin{small} 
  
% redefinindo o espaçamento das colunas
\setlength{\tabcolsep}{3pt} 
\begin{tabular}{|p{15cm}|}\hline
 %& \multicolumn{4}{c|}{Trimestres}\\ \cline{2-5}
%\raisebox{1.5ex}{Etapa} & 01-03/2014 & 04-06/2014 & 07-09/2014 & 10-12/2014 \\ \hline

\begin{center}\textbf{Conteúdo Programático}\end{center}\\ \hline
\begin{itemize}
 \item \textbf{Fundamentos de Sistemas Distribuídos:} Definição de Sistemas Distribuídos; Infraestrutura básica; Tipos de Sistemas Distribuídos.

 \item \textbf{Estilos Arquiteturais para SD:} Camadas; Baseada em Objetos; Baseada em Dados; Baseada em Eventos.

 \item \textbf{Visão Cliente-Servidor:} Requisição-Resposta; Comunicação síncrona; Comunicação assíncrona.

 \item \textbf{P2P:} Arquitetura Centralizada; Arquitetura Descentralizada.
 \item \textbf{Processos e Threads}
 \item \textbf{Comunicação:} Sockets; RPC; RMI; JMS.

 \item \textbf{Serviços:} Conceitos; Arquitetura Orientada a Serviço; Tipos de Serviços; Design de Serviços; Registro e descoberta; Web Services.

 \item \textbf{Tolerância a Faltas:} Definição; Dependabilidade; Tipos; Recuperação; Mascaramento.
 \item \textbf{Sincronização:} Cálculo de Latência; Ajuste de relógios.
\end{itemize}
 \\ \hline
\end{tabular} 
\end{small}
\label{dadosinstituicao}
\end{table}

\begin{table}[h!]
%\caption{Dados da Instituição}
\centering

% definindo o tamanho da fonte para small
% outros possíveis tamanhos: footnotesize, scriptsize
\begin{small} 
  
% redefinindo o espaçamento das colunas
\setlength{\tabcolsep}{3pt} 
\begin{tabular}{|p{15cm}|}\hline
 %& \multicolumn{4}{c|}{Trimestres}\\ \cline{2-5}
%\raisebox{1.5ex}{Etapa} & 01-03/2014 & 04-06/2014 & 07-09/2014 & 10-12/2014 \\ \hline

\begin{center}\textbf{Metodologia de Ensino}\end{center}\\ \hline
   Aulas expositivas utilizando recursos audiovisuais e quadro, além de aulas práticas utilizando computadores. Adicionalmente, serão realizadas atividades práticas individuais ou em grupo, para consolidação do conteúdo ministrado.
 \\ \hline
\end{tabular} 
\end{small}
\label{dadosinstituicao}
\end{table}


\begin{table}[h!]
%\caption{Dados da Instituição}
\centering

% definindo o tamanho da fonte para small
% outros possíveis tamanhos: footnotesize, scriptsize
\begin{small} 
  
% redefinindo o espaçamento das colunas
\setlength{\tabcolsep}{3pt} 
\begin{tabular}{|p{15cm}|}\hline
 %& \multicolumn{4}{c|}{Trimestres}\\ \cline{2-5}
%\raisebox{1.5ex}{Etapa} & 01-03/2014 & 04-06/2014 & 07-09/2014 & 10-12/2014 \\ \hline

\begin{center}\textbf{Avaliação do Processo de Ensino e Apendizagem}\end{center}\\ \hline
   Avaliações escritas ao final de cada unidade. Prática baseada em Estudo de Caso ou problema real.
 \\ \hline
\end{tabular} 
\end{small}
\label{dadosinstituicao}
\end{table}

\begin{table}[h!]
%\caption{Dados da Instituição}
\centering

% definindo o tamanho da fonte para small
% outros possíveis tamanhos: footnotesize, scriptsize
\begin{small} 
  
% redefinindo o espaçamento das colunas
\setlength{\tabcolsep}{3pt} 
\begin{tabular}{|p{15cm}|}\hline
 %& \multicolumn{4}{c|}{Trimestres}\\ \cline{2-5}
%\raisebox{1.5ex}{Etapa} & 01-03/2014 & 04-06/2014 & 07-09/2014 & 10-12/2014 \\ \hline

\begin{center}\textbf{Recursos Necessários}\end{center}\\ \hline
\begin{itemize} 
  \item Listas de Exercícios;
  \item Livros e apostilas;
  \item Utilização de recursos da web;
  \item Quadro branco;
  \item Marcadores para quadro branco;
  \item Sala de aula com acesso à internet, microcomputador e TV ou projetor para apresentação de slides ou material multimídia;
  \item Laboratório de microcomputadores contendo componentes de hardware e software específicos;
\end{itemize}
 \\ \hline
\end{tabular} 
\end{small}
\label{dadosinstituicao}
\end{table}


\begin{table}[h!]
%\caption{Dados da Instituição}
\centering

% definindo o tamanho da fonte para small
% outros possíveis tamanhos: footnotesize, scriptsize
\begin{small} 
  
% redefinindo o espaçamento das colunas
\setlength{\tabcolsep}{3pt} 
\begin{tabular}{|p{15cm}|}\hline
 %& \multicolumn{4}{c|}{Trimestres}\\ \cline{2-5}
%\raisebox{1.5ex}{Etapa} & 01-03/2014 & 04-06/2014 & 07-09/2014 & 10-12/2014 \\ \hline

\begin{center}\textbf{Bibliografia}\end{center}\\ \hline
\begin{itemize} 
  \item Básica;
  \item Complementar;
\end{itemize}
 \\ \hline
\end{tabular} 
\end{small}
\label{dadosinstituicao}
\end{table}

\newpage
 %arquivo que organiza as ementas

\newpage

\subsection{Proposta Pedag\'ogica}

%colocar subtopicos 

\subsection{Sistema de Avalia\c{c}\~ao do Curso}

%colocar subtopicos 


