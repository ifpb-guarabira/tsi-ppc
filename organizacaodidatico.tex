\newpage
\section{Organiza\c{c}\~ao Did\'atico-Pedag\'ogica}

\subsection{Concep\c{c}\~ao do Curso}

A Paraíba está inserida há um bom tempo no circuito nacional e internacional de tecnologia de informação e comunicação, tendo como destaque a cidade de Campina Grande que apresenta na área tecnológica uma das molas do seu desenvolvimento. A cidade destaca-se na vocação da região no desenvolvimento de novas tecnologias no campo da Engenharia Elétrica e de Informática, devido principalmente à influência da UFCG, com seus Cursos de Engenharia Elétrica e Ciências da Computação, ambos classificados entre os melhores do país.

Como resultado dessa vocação, observa-se o aumento do número de empresas de base tecnológica e incubadas no Parque Tecnológico da Paraíba. Cabe ressaltar, entretanto, que, apesar desta posição de destaque, há uma carência na formação de profissionais qualificados, para serem absorvidos pelo pólo de tecnologia da região.

Pelo panorama apresentado, necessita-se formar profissionais de nível superior, preparados para enfrentar os novos desafios que surgem no mercado, capacitados para atuar nas diversas áreas tecnológicas. Além disso, deve-se buscar a formação humana, necessária à condução de projetos, agregando ao indivíduo o espírito criativo, essencial à inovação tão exigida no mundo competitivo de hoje. Ciente desta realidade e consciente do seu papel no contexto da educação brasileira, o Campus Guarabira do IFPB apresenta o Curso Superior de Tecnologia em Sistemas para Internet, entendendo que este é um espaço promissor no que tange à geração de emprego, atendendo às demandas da sociedade e ao desenvolvimento econômico da região.

%O curso de Sistemas para Internet no Campus Guarabira do IFPB foi concebido com base nas recomendações do MEC, estando fundamentado nas habilidades, competências e conhecimentos necessários à formação, inovador, ciente de seu papel e responsabilidade na sociedade. Assim, o curso tem por objetivo formar um profissional que possua uma boa e sólida formação formação tecnológica, para garantir a sua inserção e competitividade no mercado de trabalho.

Para atender a esses pressupostos, na definição do Curso de Sistemas para Internet, considerou-se obter a formação de um profissional com características que atendessem à atual demanda do mercado de trabalho. Assim, esse curso propôs-se habilitar profissionais com conhecimentos nas áreas de Computação, Desenvolvimento de Software, Redes de Computadores e Sistemas Distribu\'idos para o desenvolvimento de soluções inovadoras em projetos de sistemas computacionais, principalmente voltados para tecnologias emergentes, como \'e o caso de tecnologias para a WEB e dispositivos m\'oveis. O Curso de Sistemas para Internet possui uma formação sólida nos fundamentos de Computação. O profissional em Sistemas para Internet estará habilitado para atuar em todas as áreas em que os conhecimentos de computação são essenciais e complementares.  

O egresso do Curso de Sistemas para Internet receberá o conhecimento necessário para prosseguir em estudos de pós-graduação em razão do fundamentado conhecimento obtido nas disciplinas da área básica do curso e nas atividades realizadas em projetos de pesquisa e extensão que incentivam a busca por novos desafios.

\subsubsection{Justificativas do curso}

A expansão da computação é verificada pela quantidade e diversidade de sistemas computacionais utilizados em diferentes segmentos, seja no trabalho, na educação e entretenimento. No trabalho, estes sistemas têm sido empregados nas mais diversas áreas e finalidades: desde a automatização de fábricas à terapia ocupacional, sendo essenciais nas comunicações, fortemente presente na Internet e nos aplicativos web. No âmbito da educação, os referidos têm auxiliado, seja como suporte gerencial ou como ferramenta de apoio ao processo de ensino-aprendizagem. Na área de entretenimento, estão os jogos que utilizam as mais sofisticadas técnicas de projeto gráfico e conceitos como os de Inteligência Artificial. Fora esse contexto, existem diversos dispositivos, como os eletrodomésticos e aparelhos eletrônicos, com funcionalidades implementadas por meio de \textit{hardware} e \textit{software}.


O interesse pelo Curso Superior de Tecnologia em Sistemas para Internet vêm da necessidade de mão de obra qualificada na área de Tecnologia da Informação e Comunicação (TI), como vem sendo noticiado constantemente pela imprensa. é importante também notar que o tecnólogo em Sistemas para Internet é capaz de prestar concurso para qualquer cargo de nível superior na área de TI, uma vez que esses concursos não restringem a formação apenas para bacharelado. Além disso, a Paraíba se destaca como grande formadora de mão de obra qualificada na área de TI, possuindo três programas de pós-graduação, contando com três cursos de mestrado e um de doutorado na área de computação.%, o que facilita a captação de mão de obra qualificada para atuar no curso.

A expansão das Instituições de Ensino na área de informática tem respaldo na exigência do mercado pelo aumento do número de profissionais e pelo desenvolvimento dessa área no País e no mundo. Guarabira est\'a localizada de forma privilegiada, a cerca de 100 km da capital Paraibana e a cerca de 90 km de Campina Grande, cidade que possui atua\c{c}\~ao destacada na \'area de TI, contando com um parque tecnol\'ogico e v\'arias empresas dessa \'area. O curso de tecnologia em Sistemas para Internet do IFPB - Campus Guarabira ser\'a o primeiro curso superior da \'area de inform\'atica a ser ofertado por uma institui\c{c}\~ao p\'ublica na região do brejo e o Campus Guarabira ser\'a o segundo campus a oferecer o curso em todo o IFPB (atualmente apenas o campus Jo\~ao Pessoa oferece esse curso).

O curso de Sistemas para Internet trabalha com tecnologias emergentes e que oferecem muitas oportunidades de emprego, como é o caso de desenvolvimento de \emph{software} para web e desenvolvimento de aplicativos para dispositivos móveis. Além disso, a formação do profissional de TSI prevê um bom nível de conhecimento em redes de computadores, incluindo segurança de redes, áreas que também estão sendo demandadas pelo mercado de trabalho.

Outra motiva\c{c}\~ao para a criação do curso é o fortalecimento da área de informática no campus Guarabira, que já conta com um curso técnico. A existência do curso técnico na área de informática contribui de duas formas para a criação do curso superior. Primeiramente, oferece estrutura e recursos humanos para permitir o início do curso. Em segundo lugar, os alunos egressos dos cursos técnicos em informática terão oportunidade de continuar os estudos em nível superior sem precisar sair de Guarabira. Esse ponto, além de aumentar a demanda pelo curso, provavelmente aumentará o nível dos alunos ingressantes, já que espera-se que muitos alunos virão com conhecimentos adquiridos no curso técnico. Por último, é interessante observar que o Campus de João Pessoa é a única instituição pública a ofertar tal curso na Paraíba e a demanda por profissionais dessa área vem crescendo significativamente. Guarabira, por ser a cidade mais importante comercialmente para a Microrregião do Brejo, só tem a ganhar com a criação do curso no Campus, já que é notório o crescimento do \textit{e-commerce} no Brasil. Para inserir a cidade nessa nova realidade são necessários profissionais do curso de Sistemas para Internet. %Além disso, há a possibilidade de interdisciplinaridade entre com o curso Tecnólogo em Gestão Comercial, pois mesmo sendo de áreas distintas, há uma convergência muito forte entre os dois cursos devido à vocação comercial da região.

	Outro aspecto a ser levado em consideração são as atividades já sendo desenvolvidas no campus Guarabira na área de informática. Um grande conjunto de atividades já foi desenvolvido e outro já está em andamento, envolvendo principalmente atividades de pesquisa. A seguir são listadas as atividades e projetos desenvolvidos pelos professores de informática nos últimos tr\^es anos:

\begin{itemize}
\item Criação do Grupo de Pesquisa em Redes de Computadores e Sistemas Distribuídos (GPRSD). O grupo conta atualmente com 7 (sete) pesquisadores do IFPB e da UFPB, e 7 (sete) alunos dos cursos técnicos em informática;

\item Aprovação no GPRSD no Edital 07/2012 - Programa de Apoio ao Fortalecimento dos Grupos de Pesquisa do IFPB;

\item Participação do GPRSD no projeto ``Forma-Engenharia - Desenvolvimento de Sistemas de Monitoramento e Controle de Motores de Indução Trifásicos''. Financiado pelo CNPq, por meio do edital da chamada CNPQ/VALE S.A. Nº 05/2012. Nesse projeto dois alunos do curso técnico subsequente em informática foram bolsistas de iniciação tecnológica e industrial – nível B do CNPq;

\item Participação do GPRSD no projeto ``Projeto Ver Brasil: Sistema Brasileiro de Exibição Digital'', financiado pelo Ministério da Cultura e Secretaria de Audiovisual em parceria com o Núcleo de Pesquisa LAViD, da UFPB. Nesse projeto um aluno do curso integrado em informática e um aluno do curso subsequente em informática foram bolsistas, além de três professores do campus da área de informática;

\item Criação do Grupo de Pesquisa em Sistemas Digitais (GPSDi). O grupo conta atualmente com 2 (dois) pesquisadores e 7 (sete) alunos dos cursos Técnicos em Informática;

\item Aprovação do GPSDi na chamada MEC/SETEC/CNPq N º 94/2013 – Apoio a Projetos Cooperativos de Pesquisa Aplicada e de Extensão Tecnológica, com o projeto “Grupo de Estudos em Robótica Educacional”.  Nesse projeto, alunos do curso técnico integrado em informática estão se preparando para a olimpíada brasileira de robótica;

\item Participação no projeto ``Projeto Olímpico de Informática no IFPB'', financiado por meio da chamada MEC/SETEC/CNPq N º 94/2013 – Apoio a Projetos Cooperativos de Pesquisa Aplicada e de Extensão Tecnológica. Por meio desse projeto foram obtidas quatro medalhas na olimp\'iada paraibana de inform\'atica e uma men\c{c}\~ao honrosa na olim\'ipada brasileira de inform\'atica;

%\item Aprovação do professor Ruan Delgado Gomes no edital 25-2013 Bolsa Pesquisador do IFPB, com o projeto intitulado “Desafios e Aplicações de Redes de Sensores sem Fio Industriais”;

\item Um total de 8 (oito) bolsistas e 2 (dois) voluntários envolvidos em projetos de pesquisa aprovados nos editais de PIBIC-EM;

\item Aprovação no GPRSD no Edital 17/2014 - Programa de Apoio ao Fortalecimento dos Grupos de Pesquisa do IFPB;

\item Classifica\c{c}\~ao para a final do concurso de trabalhos t\'ecnicos em inform\'atica do evento Computer on the Beach 2015;

\item Publicação de oito artigos em peri\'odicos, oito artigos em anais de eventos e dois cap\'itulos de livro pela equipe de informática nos últimos três anos.

\end{itemize}

%Os resultados descritos demonstram o potencial do corpo docente e discente da \'area de inform\'atica do campus Guarabira.

%Quanto à vocação regional da cidade de Campina Grande, a computação está alicerçada nas empresas da área de informática existentes, incluindo as de consultoria em tecnologia de informação e comunicação, de desenvolvimento de soluções de hardware e software para diversos segmentos econômicos. No ano de 2001, edição de abril, a revista norte-americana, Newsweek, escolheu Campina Grande dentre as nove cidades de destaque no mundo que representam um novo modelo de centro tecnológico. Tal escolha não foi por acaso, tendo em vista que, atualmente existem doze indústrias voltadas a atividades de fabricação e serviços relacionados a informática com sede em Campina Grande, além de empresas de confecção de material eletrônico e equipamentos de comunicação.
%Essa vocação é sustentada por entidades e empresas que se agregam para fomentar e prover o desenvolvimento da área de tecnologia, destacando-se a fundação do Parque Tecnológico (PaqTcPB) e do Centro de Inovação Tecnológica Telmo Araújo (CITTA), que possui o objetivo de auxiliar na consolidação de novos negócios, incubando e apoiando as empresas que possuem ênfase em tecnologia e inovação durante os diversos estágios do negócio.
%O PaqTcPB, em 2013, possuía um total de 21 empresas incubadas, dessas, 15 são de Campina Grande. O CITTA está com uma previsão de investimento inicial de aproximadamente R$ 4 milhões e deverá sediar uma média de 50 empresas voltadas à produção de tecnologia. Esses números ilustram uma Campina Grande com perfil empreendedor, com projetos nas áreas de produção de software, geoprocessamento, setor eletroeletrônico e biotecnologia.

%colocar subtopicos

\subsubsection{Objetivos do curso}

\textbf{Geral}

\vspace{5mm}

O Curso Superior de Tecnologia em Sistemas para Internet tem como objetivo formar profissionais de nível superior para atuar no desenvolvimento de programas, de interfaces e aplicativos, do comércio e do marketing eletrônicos, além de páginas e portais para internet e intranet, gerenciar projetos de sistemas, inclusive com acesso a banco de dados, desenvolver projetos de aplicações para a rede mundial de computadores, integrar mídias nas páginas da internet, cuidar da implantação, atualização, manutenção e segurança dos sistemas para internet, atuar com tecnologias emergentes como: computação móvel, redes sem fio, sistemas WEB e sistemas distribuídos, além de estar apto a pensar e desenvolver novas tecnologias, atuar de forma criativa e criticamente na identificação das demandas sociais e no desenvolvimento sustentável da região e do país.

\vspace{5mm}
\textbf{Específicos}

\vspace{5mm}

De acordo com o cat\'alogo do MEC, o profissional formado no curso superior de Tecnologia em Sistemas para Internet, poder\'a ter as seguintes atribui\c{c}\~oes:

\begin{itemize}

\item Formar cidadãos na área de conhecimento de tecnologia da informação e comunicação, aptos para inserção no mundo do trabalho e conscientes da sua responsabilidade profissional e social; 

\item Estimular o desenvolvimento de habilidade e competências filosóficas, científicas e tecnológicas a partir de uma base de pensamento reflexivo;

\item Possibilitar que os egressos do curso possam atuar em projetos e atividades que envolvam o desenvolvimento e uso da tecnologia da informação e comunicação;

\item Possibilitar que os alunos possam ter acesso a conhecimentos teóricos e práticos consistentes necessários a um processo de aprendizagem qualitativo;

\item Oportunizar que os alunos possam colocar na prática os conhecimentos adquiridos em laboratórios, projetos, monitorias ou estágios; 

\item Proporcionar formação humanística e ética, fundamental à integração do profissional à sociedade e ao trabalho multidisciplinar; 

\item Incentivar o trabalho de pesquisa e a investigação científica, visando o desenvolvimento da ciência e da tecnologia; 

\item Possibilitar que a sociedade possa ter acesso ao conjunto das pesquisas científicas e tecnológicas geradas na instituição, com o objetivo de dar visibilidade aos resultados e os benefícios resultantes das mesmas;

\item Proporcionar a formação de um tecnólogo crítico, criativo e empreendedor, capaz de entender os desafios e as necessidades impostas pelo mundo do trabalho na atualidade;

\item Desenvolver e apoiar projetos científicos e tecnológicos fundamentados na plataforma da interdisciplinaridade e que apresentem relevância mundial, nacional, regional e local;

\item Capacitar os egressos para assumirem a postura de permanente busca de atualização profissional;

\end{itemize}

\subsubsection{Perfil do Egresso do Curso}

De acordo com o Parecer CNE/CP no. 29/2002, os cursos de graduação tecnológica devem primar por uma formação em processo contínuo. Essa formação deve pautar-se pela descoberta do conhecimento e pelo desenvolvimento de competências profissionais necessárias ao longo da vida. Deve, ainda, privilegiar a construção do pensamento crítico e autônomo na elaboração de propostas educativas que possam garantir identidade aos cursos de graduação tecnológica e favorecer respostas às necessidades e demandas de formação tecnológica do contexto social, local e nacional.

A formação tecnológica proposta no modelo curricular deve propiciar ao aluno condições de assimilar, integrar e produzir conhecimentos científicos e tecnológicos na área específica de sua formação; analisar criticamente a dinâmica da sociedade brasileira e as diferentes formas de participação do cidadão-tecnólogo nesse contexto; e desenvolver as capacidades necessárias ao desempenho das atividades profissionais.

Nesse sentido, o profissional egresso do Curso Superior de Tecnologia em Sistemas para Internet deve ser capaz de processar informações, ter senso crítico e ser capaz de impulsionar o desenvolvimento econômico da região, integrando formação técnica à cidadania.

A base de conhecimentos científicos e tecnológicos deverá capacitar o profissional para:

\begin{itemize}
	\item Articular e inter-relacionar teoria e prática;
	\item Utilizar adequadamente a linguagem oral e escrita como instrumento de comunicação e interação social necessária ao desempenho de sua profissão;
	\item Realizar a investigação científica e a pesquisa aplicada como forma de contribuição para o
processo de produção do conhecimento;
	\item Resolver situações-problema que exijam raciocínio abstrato, percepção espacial, memória
auditiva, memória visual, atenção concentrada, operações numéricas e criatividade;
	\item Dominar conhecimentos científicos e tecnológicos na área específica de sua formação;
	\item Avaliar e especificar a necessidade de treinamento e de suporte técnico aos usuários;
	\item Executar ações de treinamento e suporte técnico em redes ou em sistemas de informação;
	\item Atuar na análise de sistemas, propondo soluções para incrementar a produção e diminuir
desperdício de tempo e de recursos de trabalho em uma empresa ou instituição;
	\item Criar manuais, relatórios técnicos ou especificações que descrevam um sistema (ou parte
dele) ou um projeto de redes de computadores ou sistema distribuído;
	\item Analisar problemas e desenvolver algoritmos que levem à solução deles de forma correta e otimizada;
	\item Compreender o funcionamento de estruturas de dados e ser capaz de construir
novas estruturas de dados;
	\item Desenvolver software em linguagens de diferentes paradigmas, como o paradigma imperativo e orientado a objetos;
	\item Entender o funcionamento dos sistemas computacionais em todos os seus níveis, desde a organização de computadores e sistemas operacionais, às linguagens de alto nível;
	\item Projetar e implementar sistemas computacionais que fazem uso de computação distribuída e paralela;
	\item Analisar e projetar soluções de software utilizando uma linguagem de modelagem;
	\item Interpretar diagramas em linguagem de modelagem e implementar o código
correspondente em uma linguagem de programação;
	\item Utilizar ferramentas de apoio ao desenvolvimento de sistemas, tais como ambientes de
desenvolvimento integrado (IDEs) e ferramentas que auxiliem ao desenvolvimento rápido
(RAD);
	\item Conhecer processos e metodologias para o desenvolvimento de software;
	\item Desenvolver sistemas corporativos;
	\item Compreender o comércio eletrônico, seus desafios e os meios para implantar soluções bem-
sucedidas para essa forma de comércio;
	\item Projetar e criar soluções para a aparência, funcionalidade e navegabilidade de páginas na
Web;
	\item Projetar e implementar bancos de dados convencionais e distribuídos;
	\item Criar páginas dinâmicas para a Internet, com consulta e atualização de informações em
bases de dados remotas;
	\item Utilizar frameworks que auxiliam na criação de páginas Web.
	\item Instalar e configurar sistemas operacionais de redes de computadores;
	\item Instalar e configurar protocolos e softwares de redes;
	\item Desenvolver serviços de administração de redes de computadores;
	\item Compreender e implementar projetos de redes de computadores e sistemas distribuídos;
	\item Desenvolver aplicações para dispositivos móveis;
	\item Lidar com requisitos de segurança em sistemas computacionais;
	\item Aplicar normas técnicas nas atividades específicas da sua área de formação profissional.
	\item Familiarizar-se com as práticas e procedimentos comuns em ambientes organizacionais;
	\item Empreender negócios em sua área de formação;
	\item Conhecer e aplicar normas de sustentabilidade ambiental, respeitando o meio ambiente e entendendo a sociedade como uma construção humana dotada de tempo, espaço e história;
	\item Ter atitude ética no trabalho e no convívio social, compreender os processos de socialização humana em âmbito coletivo e perceber-se como agente social que intervém na realidade;
	\item Ter iniciativa, criatividade, autonomia, responsabilidade, saber trabalhar em equipe, exercer liderança e ter capacidade empreendedora; e
	\item Posicionar-se crítica e eticamente frente às inovações tecnológicas, avaliando seu impacto no desenvolvimento e na construção da sociedade.
\end{itemize}

\subsubsection{Diferenciais competitivos do curso}

 O curso superior de Tecnologia em Sistemas para Internet do IFPB-Campus Guarabira surge levando em consideração à carência de profissionais de tecnologia da informação e comunicação em todo o país, além da possibilidade de impulsionar o desenvolvimento tecnológico na região do brejo paraibano, se transformando atualmente em fator importantíssimo para o desenvolvimento que a cidade vivencia. A economia de Guarabira é movimentada principalmente pelos setores de comércio, indústrias de alimentos, indústrias de cerâmica, indústrias de aguardante, entre outros. Em todos esses setores o uso da tecnologia da informação e comunicação é essencial, de modo que é importante a formação de mão de obra altamente qualificada na região. Além disso, o curso terá foco na formação de empreendedores, o que pode resultar no aparecimento na região de empresas para desenvolvimento de sistemas computacionais e prestação de serviços, o que pode movimentar ainda mais a economia da região.
 
 Além disso, o Curso de Tecnologia em Sistemas para Internet apresenta outros fatores competitivos, como por exemplo: 

\begin{itemize}
	
\item Será o curso de tecnologia em sistemas para internet pioneiro no brejo da Paraíba, especificamente no que tange a microrregião polarizada pela cidade de Guarabira, além de ser o primeiro curso na área de informática oferecido por instituição pública na região;

\item O curso será uma opção de acesso ao ensino superior para jovens e adultos, principalmente da microrregião de Guarabira, que não terão a necessidade de se deslocarem aos grandes centros para cursar um curso superior na área de informática em uma instituição pública;

\item O curso formará tecnólogos aptos a atuarem no mercado de trabalho, em empresas públicas ou privadas, e para a criação de novas empresas de tecnologia da informação e comunicação; 

\item Os alunos egressos do curso estarão contribuindo para a melhoria da qualidade da mão de obra especializada da cidade, permitindo o crescimento de todos os segmentos da economia da região;

\item Os alunos ao longo do curso estarão desenvolvendo atividades curriculares e de pesquisa, que poderão ser aplicados no desenvolvimento tecnológico da região, por meio de parceria com empresas dos diversos setores da economia.

\end{itemize}


\subsection{Pol\'iticas Institucionais e sua Correla\c{c}\~ao com o Curso}

        Atualmente o Campus Guarabira, em observância às suas obrigações previstas em lei, oferece Cursos Técnicos Integrados ao Ensino Médio, Cursos Técnicos Subsequentes e um Curso Superior de Tecnologia, todos em consonância com os princípios doutrinários consagrados na Lei de Diretrizes e Bases da Educação Nacional – LDB. 

        No Campus Guarabira existem dois cursos técnicos na \'area de inform\'atica, sendo um integrado e outro subsequente. Desta forma, com o objetivo de expandir a verticalização do ensino no Campus e em consonância com as políticas institucionais constantes no Plano de Desenvolvimento Institucional do IFPB, foi proposto o Curso de Tecnologia em Sistemas para Internet, com o objetivo de formar profissionais qualificados para atuarem no mercado de trabalho, bem como, capazes de prosseguirem seus estudos na pós-graduação. 

\newpage
\subsection{Organiza\c{c}\~ao Curricular}

\subsubsection{Estrutura curricular}
%\vspace{-1mm}
\begin{table}[h!]
\tiny
\centering
\begin{tabular}{llll}
\hline
\rowcolor[HTML]{34CDF9} 
\multicolumn{4}{|c|}{\cellcolor[HTML]{34CDF9}\textbf{Primeiro Período}}                                                                                                                                                                                          \\ \hline
\rowcolor[HTML]{34CDF9} 
\multicolumn{1}{|l|}{\cellcolor[HTML]{34CDF9}\textbf{Disciplinas}} & \multicolumn{1}{l|}{\cellcolor[HTML]{34CDF9}\textbf{Teórica}} & \multicolumn{1}{l|}{\cellcolor[HTML]{34CDF9}\textbf{Prática}} & \multicolumn{1}{l|}{\cellcolor[HTML]{34CDF9}\textbf{Total}} \\ \hline
\multicolumn{1}{|l|}{Inglês Instrumental}                          & \multicolumn{1}{l|}{67}                                       & \multicolumn{1}{l|}{}                                         & \multicolumn{1}{l|}{67}                                     \\ \hline
\multicolumn{1}{|l|}{Fundamentos de Redes de Computadores}         & \multicolumn{1}{l|}{47}                                       & \multicolumn{1}{l|}{20}                                       & \multicolumn{1}{l|}{67}                                     \\ \hline
\multicolumn{1}{|l|}{Cálculo Diferencial e Integral}               & \multicolumn{1}{l|}{100}                                      & \multicolumn{1}{l|}{}                                         & \multicolumn{1}{l|}{100}                                    \\ \hline
\multicolumn{1}{|l|}{Algoritmos e Lógica de Programação}           & \multicolumn{1}{l|}{50}                                       & \multicolumn{1}{l|}{50}                                       & \multicolumn{1}{l|}{100}                                    \\ \hline
\multicolumn{1}{|l|}{Fundamentos da Computação}                    & \multicolumn{1}{l|}{20}                                       & \multicolumn{1}{l|}{13}                                       & \multicolumn{1}{l|}{33}                                     \\ \hline
\multicolumn{1}{|l|}{Linguagens de Marcação}                       & \multicolumn{1}{l|}{30}                                       & \multicolumn{1}{l|}{20}                                       & \multicolumn{1}{l|}{50}                                     \\ \hline
\rowcolor[HTML]{34CDF9} 
\multicolumn{1}{|r|}{\cellcolor[HTML]{34CDF9}\textbf{Subtotal}}    & \multicolumn{1}{l|}{\cellcolor[HTML]{34CDF9}\textbf{314}}     & \multicolumn{1}{l|}{\cellcolor[HTML]
{34CDF9}\textbf{103}}     & \multicolumn{1}{l|}{\cellcolor[HTML]{34CDF9}\textbf{417}}   \\ \hline
\multicolumn{4}{l}{}                                                                                                                                                                                                                                             \\ \hline
\rowcolor[HTML]{34CDF9} 
\multicolumn{4}{|c|}{\cellcolor[HTML]{34CDF9}\textbf{Segundo Período}}                                                                                                                                                                                          \\ \hline
\rowcolor[HTML]{34CDF9} 
\multicolumn{1}{|l|}{\cellcolor[HTML]{34CDF9}\textbf{Disciplinas}} & \multicolumn{1}{l|}{\cellcolor[HTML]{34CDF9}\textbf{Teórica}} & \multicolumn{1}{l|}{\cellcolor[HTML]{34CDF9}\textbf{Prática}} & \multicolumn{1}{l|}{\cellcolor[HTML]{34CDF9}\textbf{Total}} \\ \hline
\multicolumn{1}{|l|}{Português Instrumental}                          & \multicolumn{1}{l|}{67}                                       & \multicolumn{1}{l|}{}                                         & \multicolumn{1}{l|}{67}                                     \\ \hline
\multicolumn{1}{|l|}{Protocolos de Interconexão de Redes}         & \multicolumn{1}{l|}{47}                                       & \multicolumn{1}{l|}{20}                                       & \multicolumn{1}{l|}{67}                                     \\ \hline
\multicolumn{1}{|l|}{Estruturas de Dados I}               & \multicolumn{1}{l|}{37}                                      & \multicolumn{1}{l|}{30}                                         & \multicolumn{1}{l|}{67}                                    \\ \hline
\multicolumn{1}{|l|}{Probabilidade e Estatística}           & \multicolumn{1}{l|}{70}                                       & \multicolumn{1}{l|}{13}                                       & \multicolumn{1}{l|}{83}                                    \\ \hline
\multicolumn{1}{|l|}{Arquitetura de Computadores}                    & \multicolumn{1}{l|}{40}                                       & \multicolumn{1}{l|}{27}                                       & \multicolumn{1}{l|}{67}                                     \\ \hline
\multicolumn{1}{|l|}{Linguagens de Script}                       & \multicolumn{1}{l|}{30}                                       & \multicolumn{1}{l|}{20}                                       & \multicolumn{1}{l|}{50}                                     \\ \hline
\rowcolor[HTML]{34CDF9} 
\multicolumn{1}{|r|}{\cellcolor[HTML]{34CDF9}\textbf{Subtotal}}    & \multicolumn{1}{l|}{\cellcolor[HTML]{34CDF9}\textbf{291}}     & \multicolumn{1}{l|}{\cellcolor[HTML]
{34CDF9}\textbf{110}}     & \multicolumn{1}{l|}{\cellcolor[HTML]{34CDF9}\textbf{401}}   \\ \hline
\multicolumn{4}{l}{}                                                                                                                                                                                                                                             \\ \hline
\rowcolor[HTML]{34CDF9} 
\multicolumn{4}{|c|}{\cellcolor[HTML]{34CDF9}\textbf{Terceiro Período}}                                                                                                                                                                                          \\ \hline
\rowcolor[HTML]{34CDF9} 
\multicolumn{1}{|l|}{\cellcolor[HTML]{34CDF9}\textbf{Disciplinas}} & \multicolumn{1}{l|}{\cellcolor[HTML]{34CDF9}\textbf{Teórica}} & \multicolumn{1}{l|}{\cellcolor[HTML]{34CDF9}\textbf{Prática}} & \multicolumn{1}{l|}{\cellcolor[HTML]{34CDF9}\textbf{Total}} \\ \hline
\multicolumn{1}{|l|}{Interação Humano-Computador}                          & \multicolumn{1}{l|}{50}                                       & \multicolumn{1}{l|}{17}                                         & \multicolumn{1}{l|}{67}                                     \\ \hline
\multicolumn{1}{|l|}{Bancos de Dados I}         & \multicolumn{1}{l|}{40}                                       & \multicolumn{1}{l|}{27}                                       & \multicolumn{1}{l|}{67}                                     \\ \hline
\multicolumn{1}{|l|}{Estruturas de Dados II}               & \multicolumn{1}{l|}{37}                                      & \multicolumn{1}{l|}{30}                                         & \multicolumn{1}{l|}{67}                                    \\ \hline
\multicolumn{1}{|l|}{Sistemas Operacionais}           & \multicolumn{1}{l|}{47}                                       & \multicolumn{1}{l|}{20}                                       & \multicolumn{1}{l|}{67}                                    \\ \hline
\multicolumn{1}{|l|}{Metodologia da Pesquisa Científica}                    & \multicolumn{1}{l|}{20}                                       & \multicolumn{1}{l|}{13}                                       & \multicolumn{1}{l|}{33}                                     \\ \hline
\multicolumn{1}{|l|}{Programação Orientada a Objetos}                       & \multicolumn{1}{l|}{50}                                       & \multicolumn{1}{l|}{33}                                       & \multicolumn{1}{l|}{83}                                     \\ \hline
\rowcolor[HTML]{34CDF9} 
\multicolumn{1}{|r|}{\cellcolor[HTML]{34CDF9}\textbf{Subtotal}}    & \multicolumn{1}{l|}{\cellcolor[HTML]{34CDF9}\textbf{244}}     & \multicolumn{1}{l|}{\cellcolor[HTML]
{34CDF9}\textbf{140}}     & \multicolumn{1}{l|}{\cellcolor[HTML]{34CDF9}\textbf{384}}   \\ \hline
\multicolumn{4}{l}{}                                                                                                                                                                                                                                             \\ \hline
\rowcolor[HTML]{34CDF9} 
\multicolumn{4}{|c|}{\cellcolor[HTML]{34CDF9}\textbf{Quarto Período}}                                                                                                                                                                                          \\ \hline
\rowcolor[HTML]{34CDF9} 
\multicolumn{1}{|l|}{\cellcolor[HTML]{34CDF9}\textbf{Disciplinas}} & \multicolumn{1}{l|}{\cellcolor[HTML]{34CDF9}\textbf{Teórica}} & \multicolumn{1}{l|}{\cellcolor[HTML]{34CDF9}\textbf{Prática}} & \multicolumn{1}{l|}{\cellcolor[HTML]{34CDF9}\textbf{Total}} \\ \hline
\multicolumn{1}{|l|}{Programação para a Web I}                          & \multicolumn{1}{l|}{43}                                       & \multicolumn{1}{l|}{40}                                         & \multicolumn{1}{l|}{83}                                     \\ \hline
\multicolumn{1}{|l|}{Legislação Social}         & \multicolumn{1}{l|}{67}                                       & \multicolumn{1}{l|}{}                                       & \multicolumn{1}{l|}{67}                                     \\ \hline
\multicolumn{1}{|l|}{Segurança da Informação}               & \multicolumn{1}{l|}{47}                                      & \multicolumn{1}{l|}{20}                                         & \multicolumn{1}{l|}{67}                                    \\ \hline
\multicolumn{1}{|l|}{Programação Paralela}           & \multicolumn{1}{l|}{37}                                       & \multicolumn{1}{l|}{30}                                       & \multicolumn{1}{l|}{67}                                    \\ \hline
\multicolumn{1}{|l|}{Bancos de Dados II}                    & \multicolumn{1}{l|}{40}                                       & \multicolumn{1}{l|}{27}                                       & \multicolumn{1}{l|}{67}                                     \\ \hline
\multicolumn{1}{|l|}{Análise e Projeto de Sistemas}                       & \multicolumn{1}{l|}{50}                                       & \multicolumn{1}{l|}{17}                                       & \multicolumn{1}{l|}{67}                                     \\ \hline
\rowcolor[HTML]{34CDF9} 
\multicolumn{1}{|r|}{\cellcolor[HTML]{34CDF9}\textbf{Subtotal}}    & \multicolumn{1}{l|}{\cellcolor[HTML]{34CDF9}\textbf{284}}     & \multicolumn{1}{l|}{\cellcolor[HTML]
{34CDF9}\textbf{134}}     & \multicolumn{1}{l|}{\cellcolor[HTML]{34CDF9}\textbf{418}}   \\ \hline
\multicolumn{4}{l}{}                                                                                                                                                                                                                                             \\ \hline
\rowcolor[HTML]{34CDF9} 
\multicolumn{4}{|c|}{\cellcolor[HTML]{34CDF9}\textbf{Quinto Período}}                                                                                                                                                                                          \\ \hline
\rowcolor[HTML]{34CDF9} 
\multicolumn{1}{|l|}{\cellcolor[HTML]{34CDF9}\textbf{Disciplinas}} & \multicolumn{1}{l|}{\cellcolor[HTML]{34CDF9}\textbf{Teórica}} & \multicolumn{1}{l|}{\cellcolor[HTML]{34CDF9}\textbf{Prática}} & \multicolumn{1}{l|}{\cellcolor[HTML]{34CDF9}\textbf{Total}} \\ \hline
\multicolumn{1}{|l|}{Programação para a Web II}                          & \multicolumn{1}{l|}{43}                                       & \multicolumn{1}{l|}{40}                                         & \multicolumn{1}{l|}{83}                                     \\ \hline
\multicolumn{1}{|l|}{Padrões de Projeto de Software}         & \multicolumn{1}{l|}{37}                                       & \multicolumn{1}{l|}{30}                                       & \multicolumn{1}{l|}{67}                                     \\ \hline
\multicolumn{1}{|l|}{Gerência e Configuração de Serviços para a Internet}               & \multicolumn{1}{l|}{30}                                      & \multicolumn{1}{l|}{37}                                         & \multicolumn{1}{l|}{67}                                    \\ \hline
\multicolumn{1}{|l|}{Engenharia de Software}           & \multicolumn{1}{l|}{50}                                       & \multicolumn{1}{l|}{17}                                       & \multicolumn{1}{l|}{67}                                    \\ \hline
\multicolumn{1}{|l|}{Programação para Dispositivos Móveis}                    & \multicolumn{1}{l|}{37}                                       & \multicolumn{1}{l|}{30}                                       & \multicolumn{1}{l|}{67}                                     \\ \hline
\multicolumn{1}{|l|}{Empreendedorismo em Software}                       & \multicolumn{1}{l|}{50}                                       & \multicolumn{1}{l|}{17}                                       & \multicolumn{1}{l|}{67}                                     \\ \hline
\rowcolor[HTML]{34CDF9} 
\multicolumn{1}{|r|}{\cellcolor[HTML]{34CDF9}\textbf{Subtotal}}    & \multicolumn{1}{l|}{\cellcolor[HTML]{34CDF9}\textbf{247}}     & \multicolumn{1}{l|}{\cellcolor[HTML]
{34CDF9}\textbf{171}}     & \multicolumn{1}{l|}{\cellcolor[HTML]{34CDF9}\textbf{418}}   \\ \hline
\multicolumn{4}{l}{}                                                                                                                                                                                                                                             \\ \hline
\rowcolor[HTML]{34CDF9} 
\multicolumn{4}{|c|}{\cellcolor[HTML]{34CDF9}\textbf{Sexto Período}}                                                                                                                                                                                          \\ \hline
\rowcolor[HTML]{34CDF9} 
\multicolumn{1}{|l|}{\cellcolor[HTML]{34CDF9}\textbf{Disciplinas}} & \multicolumn{1}{l|}{\cellcolor[HTML]{34CDF9}\textbf{Teórica}} & \multicolumn{1}{l|}{\cellcolor[HTML]{34CDF9}\textbf{Prática}} & \multicolumn{1}{l|}{\cellcolor[HTML]{34CDF9}\textbf{Total}} \\ \hline
\multicolumn{1}{|l|}{Sistemas Distribuídos}                          & \multicolumn{1}{l|}{47}                                       & \multicolumn{1}{l|}{20}                                         & \multicolumn{1}{l|}{67}                                     \\ \hline
\multicolumn{1}{|l|}{Comércio Eletrônico}         & \multicolumn{1}{l|}{33}                                       & \multicolumn{1}{l|}{}                                       & \multicolumn{1}{l|}{33}                                     \\ \hline
\multicolumn{1}{|l|}{Desenvolvimento de Aplicações Corporativas}               & \multicolumn{1}{l|}{37}                                      & \multicolumn{1}{l|}{30}                                         & \multicolumn{1}{l|}{67}                                    \\ \hline
\multicolumn{1}{|l|}{Projeto em TSI}                    & \multicolumn{1}{l|}{10}                                       & \multicolumn{1}{l|}{57}                                         & \multicolumn{1}{l|}{67}                                    \\ \hline
\multicolumn{1}{|l|}{Tópicos Especiais}                       & \multicolumn{1}{l|}{67}                                       & \multicolumn{1}{l|}{}                                        & \multicolumn{1}{l|}{67}                                     \\ \hline                               
\rowcolor[HTML]{34CDF9} 
\multicolumn{1}{|r|}{\cellcolor[HTML]{34CDF9}\textbf{Subtotal}}    & \multicolumn{1}{l|}{\cellcolor[HTML]{34CDF9}\textbf{194}}     & \multicolumn{1}{l|}{\cellcolor[HTML]
{34CDF9}\textbf{107}}     & \multicolumn{1}{l|}{\cellcolor[HTML]{34CDF9}\textbf{301}}   \\ \hline
\multicolumn{4}{l}{}                    

                                                                                         \\ \hline
\rowcolor[HTML]{34CDF9} 
\multicolumn{4}{|c|}{\cellcolor[HTML]{34CDF9}\textbf{Quadro Resumo}}                                                                                                                                                                                          \\ \hline
\rowcolor[HTML]{34CDF9} 
\multicolumn{1}{|l|}{\cellcolor[HTML]{34CDF9}\textbf{Demonstrativo}} & \multicolumn{2}{l|}{\cellcolor[HTML]{34CDF9}\textbf{Carga Horária Total}} & \multicolumn{1}{l|}{\cellcolor[HTML]{34CDF9}\textbf{\%}} \\ \hline

\multicolumn{1}{|l|}{Disciplinas}         & \multicolumn{2}{l|}{2339}             & \multicolumn{1}{l|}{85,2}                                     \\ \hline
\multicolumn{1}{|l|}{Trabalho de Conclusão de Curso}         & \multicolumn{2}{l|}{67}             & \multicolumn{1}{l|}{2,4}                                     \\ \hline
\multicolumn{1}{|l|}{Estágio Supervisionado}         & \multicolumn{2}{l|}{300}             & \multicolumn{1}{l|}{10,9}                                     \\ \hline
\multicolumn{1}{|l|}{Atividades Complementares}         & \multicolumn{2}{l|}{40}             & \multicolumn{1}{l|}{1,5}                                     \\ \hline
\rowcolor[HTML]{34CDF9} 
\multicolumn{1}{|r|}{\cellcolor[HTML]{34CDF9}\textbf{Carga Horária Total do Curso}}      & \multicolumn{2}{l|}{\cellcolor[HTML]
{34CDF9}\textbf{2.746}}     & \multicolumn{1}{l|}{\cellcolor[HTML]{34CDF9}\textbf{100}}   \\ \hline
\multicolumn{4}{l}{}                                                                                                                                                                                                                           
\end{tabular}
\end{table}

\newpage

\begin{landscape}

\subsubsection{Fluxograma}


\begin{table}[h!]
\centering
\tiny
\label{my-label}
\begin{tabular}{lcllcllcllcllcllcl}
\multicolumn{3}{c}{{\bf 1º Período}}                                                                                                                                                & \multicolumn{3}{c}{\cellcolor[HTML]{C0C0C0}{\bf 2º Período}}                                                                                                                                                                                                  & \multicolumn{3}{c}{{\bf 3º Período}}                                                                                                                                               & \multicolumn{3}{c}{\cellcolor[HTML]{C0C0C0}{\bf 4º Período}}                                                                                                                                                                                        & \multicolumn{3}{c}{{\bf 5º Período}}                                                                                                                                                          & \multicolumn{3}{c}{\cellcolor[HTML]{C0C0C0}{\bf 6º Período}}                                                                                                                                                                                                    \\
                          & \multicolumn{1}{l}{}                                                                                                     &                              &                                                 & \multicolumn{1}{l}{}                                                                                                                                &                                                       &                         & \multicolumn{1}{l}{}                                                                                                     &                               &                                                 & \multicolumn{1}{l}{}                                                                                                                      &                                                       &                         & \multicolumn{1}{l}{}                                                                                                                &                               &                                                 & \multicolumn{1}{l}{}                                                                                                                                  &                                                       \\ \hline
\multicolumn{1}{|l|}{A1}  & \multicolumn{1}{c|}{}                                                                                                    & \multicolumn{1}{l|}{}        & \multicolumn{1}{l|}{\cellcolor[HTML]{C0C0C0}B1} & \multicolumn{1}{c|}{\cellcolor[HTML]{C0C0C0}}                                                                                                       & \multicolumn{1}{l|}{\cellcolor[HTML]{C0C0C0}}         & \multicolumn{1}{l|}{C1} & \multicolumn{1}{c|}{}                                                                                                    & \multicolumn{1}{l|}{{\bf B6}} & \multicolumn{1}{l|}{\cellcolor[HTML]{C0C0C0}D1} & \multicolumn{1}{c|}{\cellcolor[HTML]{C0C0C0}}                                                                                             & \multicolumn{1}{l|}{\cellcolor[HTML]{C0C0C0}{\bf C1}} & \multicolumn{1}{l|}{E1} & \multicolumn{1}{c|}{}                                                                                                               & \multicolumn{1}{l|}{{\bf D1}} & \multicolumn{1}{l|}{\cellcolor[HTML]{C0C0C0}F1} & \multicolumn{1}{c|}{\cellcolor[HTML]{C0C0C0}}                                                                                                         & \multicolumn{1}{l|}{\cellcolor[HTML]{C0C0C0}{\bf D4}} \\ \cline{1-1} \cline{3-4} \cline{6-7} \cline{9-10} \cline{12-13} \cline{15-16} \cline{18-18} 
\multicolumn{1}{l|}{}     & \multicolumn{1}{c|}{}                                                                                                    &                              & \multicolumn{1}{l|}{\cellcolor[HTML]{C0C0C0}}   & \multicolumn{1}{c|}{\cellcolor[HTML]{C0C0C0}}                                                                                                       & \cellcolor[HTML]{C0C0C0}                              & \multicolumn{1}{l|}{}   & \multicolumn{1}{c|}{}                                                                                                    &                               & \multicolumn{1}{l|}{\cellcolor[HTML]{C0C0C0}}   & \multicolumn{1}{c|}{\cellcolor[HTML]{C0C0C0}}                                                                                             & \cellcolor[HTML]{C0C0C0}{\bf C2}                      & \multicolumn{1}{l|}{}   & \multicolumn{1}{c|}{}                                                                                                               & {\bf D5}                      & \multicolumn{1}{l|}{\cellcolor[HTML]{C0C0C0}}   & \multicolumn{1}{c|}{\cellcolor[HTML]{C0C0C0}}                                                                                                         & \cellcolor[HTML]{C0C0C0}                              \\
\multicolumn{1}{l|}{}     & \multicolumn{1}{c|}{}                                                                                                    &                              & \multicolumn{1}{l|}{\cellcolor[HTML]{C0C0C0}}   & \multicolumn{1}{c|}{\cellcolor[HTML]{C0C0C0}}                                                                                                       & \cellcolor[HTML]{C0C0C0}                              & \multicolumn{1}{l|}{}   & \multicolumn{1}{c|}{}                                                                                                    & {\bf }                        & \multicolumn{1}{l|}{\cellcolor[HTML]{C0C0C0}}   & \multicolumn{1}{c|}{\cellcolor[HTML]{C0C0C0}}                                                                                             & \cellcolor[HTML]{C0C0C0}{\bf C6}                      & \multicolumn{1}{l|}{}   & \multicolumn{1}{c|}{}                                                                                                               &                               & \multicolumn{1}{l|}{\cellcolor[HTML]{C0C0C0}}   & \multicolumn{1}{c|}{\cellcolor[HTML]{C0C0C0}}                                                                                                         & \cellcolor[HTML]{C0C0C0}                              \\ \cline{1-1} \cline{3-4} \cline{6-7} \cline{9-10} \cline{12-13} \cline{15-16} \cline{18-18} 
\multicolumn{1}{|l|}{67}  & \multicolumn{1}{c|}{\multirow{-4}{*}{\begin{tabular}[c]{@{}c@{}}Inglês \\ Instrumental\end{tabular}}}                    & \multicolumn{1}{l|}{}        & \multicolumn{1}{l|}{\cellcolor[HTML]{C0C0C0}67} & \multicolumn{1}{c|}{\multirow{-4}{*}{\cellcolor[HTML]{C0C0C0}\begin{tabular}[c]{@{}c@{}}Português\\      Instrumental\end{tabular}}}                & \multicolumn{1}{l|}{\cellcolor[HTML]{C0C0C0}}         & \multicolumn{1}{l|}{67} & \multicolumn{1}{c|}{\multirow{-4}{*}{\begin{tabular}[c]{@{}c@{}}Interação\\ Humano-\\ Computador\end{tabular}}}          & \multicolumn{1}{l|}{}         & \multicolumn{1}{l|}{\cellcolor[HTML]{C0C0C0}83} & \multicolumn{1}{c|}{\multirow{-4}{*}{\cellcolor[HTML]{C0C0C0}\begin{tabular}[c]{@{}c@{}}Programação \\ para\\ a Web I\end{tabular}}}      & \multicolumn{1}{l|}{\cellcolor[HTML]{C0C0C0}}         & \multicolumn{1}{l|}{83} & \multicolumn{1}{c|}{\multirow{-4}{*}{\begin{tabular}[c]{@{}c@{}}Programação \\ para\\ a Web II\end{tabular}}}                       & \multicolumn{1}{l|}{}         & \multicolumn{1}{l|}{\cellcolor[HTML]{C0C0C0}67} & \multicolumn{1}{c|}{\multirow{-4}{*}{\cellcolor[HTML]{C0C0C0}\begin{tabular}[c]{@{}c@{}}Sistemas\\ Distribuídos\end{tabular}}}                        & \multicolumn{1}{l|}{\cellcolor[HTML]{C0C0C0}}         \\ \hline
                          & \multicolumn{1}{l}{}                                                                                                     &                              &                                                 & \multicolumn{1}{l}{}                                                                                                                                &                                                       &                         & \multicolumn{1}{l}{}                                                                                                     &                               &                                                 & \multicolumn{1}{l}{}                                                                                                                      &                                                       &                         & \multicolumn{1}{l}{}                                                                                                                &                               &                                                 & \multicolumn{1}{l}{}                                                                                                                                  &                                                       \\ \hline
\multicolumn{1}{|l|}{A2}  & \multicolumn{1}{c|}{}                                                                                                    & \multicolumn{1}{l|}{}        & \multicolumn{1}{l|}{\cellcolor[HTML]{C0C0C0}B2} & \multicolumn{1}{c|}{\cellcolor[HTML]{C0C0C0}}                                                                                                       & \multicolumn{1}{l|}{\cellcolor[HTML]{C0C0C0}{\bf A2}} & \multicolumn{1}{l|}{C2} & \multicolumn{1}{c|}{}                                                                                                    & \multicolumn{1}{l|}{{\bf B3}} & \multicolumn{1}{l|}{\cellcolor[HTML]{C0C0C0}D2} & \multicolumn{1}{c|}{\cellcolor[HTML]{C0C0C0}}                                                                                             & \multicolumn{1}{l|}{\cellcolor[HTML]{C0C0C0}}         & \multicolumn{1}{l|}{E2} & \multicolumn{1}{c|}{}                                                                                                               & \multicolumn{1}{l|}{{\bf D6}} & \multicolumn{1}{l|}{\cellcolor[HTML]{C0C0C0}F2} & \multicolumn{1}{c|}{\cellcolor[HTML]{C0C0C0}}                                                                                                         & \multicolumn{1}{l|}{\cellcolor[HTML]{C0C0C0}}         \\ \cline{1-1} \cline{3-4} \cline{6-7} \cline{9-10} \cline{12-13} \cline{15-16} \cline{18-18} 
\multicolumn{1}{l|}{}     & \multicolumn{1}{c|}{}                                                                                                    &                              & \multicolumn{1}{l|}{\cellcolor[HTML]{C0C0C0}}   & \multicolumn{1}{c|}{\cellcolor[HTML]{C0C0C0}}                                                                                                       & \cellcolor[HTML]{C0C0C0}                              & \multicolumn{1}{l|}{}   & \multicolumn{1}{c|}{}                                                                                                    &                               & \multicolumn{1}{l|}{\cellcolor[HTML]{C0C0C0}}   & \multicolumn{1}{c|}{\cellcolor[HTML]{C0C0C0}}                                                                                             & \cellcolor[HTML]{C0C0C0}                              & \multicolumn{1}{l|}{}   & \multicolumn{1}{c|}{}                                                                                                               &                               & \multicolumn{1}{l|}{\cellcolor[HTML]{C0C0C0}}   & \multicolumn{1}{c|}{\cellcolor[HTML]{C0C0C0}}                                                                                                         & \cellcolor[HTML]{C0C0C0}                              \\
\multicolumn{1}{l|}{}     & \multicolumn{1}{c|}{}                                                                                                    &                              & \multicolumn{1}{l|}{\cellcolor[HTML]{C0C0C0}}   & \multicolumn{1}{c|}{\cellcolor[HTML]{C0C0C0}}                                                                                                       & \cellcolor[HTML]{C0C0C0}                              & \multicolumn{1}{l|}{}   & \multicolumn{1}{c|}{}                                                                                                    &                               & \multicolumn{1}{l|}{\cellcolor[HTML]{C0C0C0}}   & \multicolumn{1}{c|}{\cellcolor[HTML]{C0C0C0}}                                                                                             & \cellcolor[HTML]{C0C0C0}                              & \multicolumn{1}{l|}{}   & \multicolumn{1}{c|}{}                                                                                                               &                               & \multicolumn{1}{l|}{\cellcolor[HTML]{C0C0C0}}   & \multicolumn{1}{c|}{\cellcolor[HTML]{C0C0C0}}                                                                                                         & \cellcolor[HTML]{C0C0C0}                              \\ \cline{1-1} \cline{3-4} \cline{6-7} \cline{9-10} \cline{12-13} \cline{15-16} \cline{18-18} 
\multicolumn{1}{|l|}{67}  & \multicolumn{1}{c|}{\multirow{-4}{*}{\begin{tabular}[c]{@{}c@{}}Fundamentos \\ de Redes\\ de Computadores\end{tabular}}} & \multicolumn{1}{l|}{}        & \multicolumn{1}{l|}{\cellcolor[HTML]{C0C0C0}67} & \multicolumn{1}{c|}{\multirow{-4}{*}{\cellcolor[HTML]{C0C0C0}\begin{tabular}[c]{@{}c@{}}Protocolos de \\ Interconexão\\  de \\ Redes\end{tabular}}} & \multicolumn{1}{l|}{\cellcolor[HTML]{C0C0C0}}         & \multicolumn{1}{l|}{67} & \multicolumn{1}{c|}{\multirow{-4}{*}{\begin{tabular}[c]{@{}c@{}}Bancos de Dados\\ I\end{tabular}}}                       & \multicolumn{1}{l|}{}         & \multicolumn{1}{l|}{\cellcolor[HTML]{C0C0C0}67} & \multicolumn{1}{c|}{\multirow{-4}{*}{\cellcolor[HTML]{C0C0C0}\begin{tabular}[c]{@{}c@{}}Legislação\\ Social\end{tabular}}}                & \multicolumn{1}{l|}{\cellcolor[HTML]{C0C0C0}}         & \multicolumn{1}{l|}{67} & \multicolumn{1}{c|}{\multirow{-4}{*}{\begin{tabular}[c]{@{}c@{}}Padrões de \\ Projeto\\ de Software\end{tabular}}}                  & \multicolumn{1}{l|}{}         & \multicolumn{1}{l|}{\cellcolor[HTML]{C0C0C0}33} & \multicolumn{1}{c|}{\multirow{-4}{*}{\cellcolor[HTML]{C0C0C0}\begin{tabular}[c]{@{}c@{}}Comércio \\ Eletrônico\end{tabular}}}                         & \multicolumn{1}{l|}{\cellcolor[HTML]{C0C0C0}}         \\ \hline
                          & \multicolumn{1}{l}{}                                                                                                     &                              &                                                 & \multicolumn{1}{l}{}                                                                                                                                &                                                       &                         & \multicolumn{1}{l}{}                                                                                                     &                               &                                                 & \multicolumn{1}{l}{}                                                                                                                      &                                                       &                         & \multicolumn{1}{l}{}                                                                                                                &                               &                                                 & \multicolumn{1}{l}{}                                                                                                                                  &                                                       \\ \hline
\multicolumn{1}{|l|}{A3}  & \multicolumn{1}{c|}{}                                                                                                    & \multicolumn{1}{l|}{}        & \multicolumn{1}{l|}{\cellcolor[HTML]{C0C0C0}B3} & \multicolumn{1}{c|}{\cellcolor[HTML]{C0C0C0}}                                                                                                       & \multicolumn{1}{l|}{\cellcolor[HTML]{C0C0C0}{\bf A4}} & \multicolumn{1}{l|}{}   & \multicolumn{1}{c|}{}                                                                                                    & \multicolumn{1}{l|}{{\bf B3}} & \multicolumn{1}{l|}{\cellcolor[HTML]{C0C0C0}D3} & \multicolumn{1}{c|}{\cellcolor[HTML]{C0C0C0}}                                                                                             & \multicolumn{1}{l|}{\cellcolor[HTML]{C0C0C0}{\bf B2}} & \multicolumn{1}{l|}{E3} & \multicolumn{1}{c|}{}                                                                                                               & \multicolumn{1}{l|}{{\bf B2}} & \multicolumn{1}{l|}{\cellcolor[HTML]{C0C0C0}F3} & \multicolumn{1}{c|}{\cellcolor[HTML]{C0C0C0}}                                                                                                         & \multicolumn{1}{l|}{\cellcolor[HTML]{C0C0C0}{\bf E1}} \\ \cline{1-1} \cline{3-4} \cline{6-7} \cline{9-10} \cline{12-13} \cline{15-16} \cline{18-18} 
\multicolumn{1}{l|}{}     & \multicolumn{1}{c|}{}                                                                                                    &                              & \multicolumn{1}{l|}{\cellcolor[HTML]{C0C0C0}}   & \multicolumn{1}{c|}{\cellcolor[HTML]{C0C0C0}}                                                                                                       & \cellcolor[HTML]{C0C0C0}                              & \multicolumn{1}{l|}{}   & \multicolumn{1}{c|}{}                                                                                                    &                               & \multicolumn{1}{l|}{\cellcolor[HTML]{C0C0C0}}   & \multicolumn{1}{c|}{\cellcolor[HTML]{C0C0C0}}                                                                                             & \cellcolor[HTML]{C0C0C0}                              & \multicolumn{1}{l|}{}   & \multicolumn{1}{c|}{}                                                                                                               & {\bf D1}                      & \multicolumn{1}{l|}{\cellcolor[HTML]{C0C0C0}}   & \multicolumn{1}{c|}{\cellcolor[HTML]{C0C0C0}}                                                                                                         & \cellcolor[HTML]{C0C0C0}                              \\
\multicolumn{1}{l|}{}     & \multicolumn{1}{c|}{}                                                                                                    &                              & \multicolumn{1}{l|}{\cellcolor[HTML]{C0C0C0}}   & \multicolumn{1}{c|}{\cellcolor[HTML]{C0C0C0}}                                                                                                       & \cellcolor[HTML]{C0C0C0}                              & \multicolumn{1}{l|}{}   & \multicolumn{1}{c|}{}                                                                                                    &                               & \multicolumn{1}{l|}{\cellcolor[HTML]{C0C0C0}}   & \multicolumn{1}{c|}{\cellcolor[HTML]{C0C0C0}}                                                                                             & \cellcolor[HTML]{C0C0C0}                              & \multicolumn{1}{l|}{}   & \multicolumn{1}{c|}{}                                                                                                               & {\bf D3}                      & \multicolumn{1}{l|}{\cellcolor[HTML]{C0C0C0}}   & \multicolumn{1}{c|}{\cellcolor[HTML]{C0C0C0}}                                                                                                         & \cellcolor[HTML]{C0C0C0}                              \\ \cline{1-1} \cline{3-4} \cline{6-7} \cline{9-10} \cline{12-13} \cline{15-16} \cline{18-18} 
\multicolumn{1}{|l|}{100} & \multicolumn{1}{c|}{\multirow{-4}{*}{\begin{tabular}[c]{@{}c@{}}Cálculo \\ Diferencial \\ e Integral\end{tabular}}}      & \multicolumn{1}{l|}{}        & \multicolumn{1}{l|}{\cellcolor[HTML]{C0C0C0}67} & \multicolumn{1}{c|}{\multirow{-4}{*}{\cellcolor[HTML]{C0C0C0}\begin{tabular}[c]{@{}c@{}}Estruturas de \\ Dados I\end{tabular}}}                     & \multicolumn{1}{l|}{\cellcolor[HTML]{C0C0C0}}         & \multicolumn{1}{l|}{67} & \multicolumn{1}{c|}{\multirow{-4}{*}{\begin{tabular}[c]{@{}c@{}}Estruturas de\\ Dados II\end{tabular}}}                  & \multicolumn{1}{l|}{}         & \multicolumn{1}{l|}{\cellcolor[HTML]{C0C0C0}67} & \multicolumn{1}{c|}{\multirow{-4}{*}{\cellcolor[HTML]{C0C0C0}\begin{tabular}[c]{@{}c@{}}Segurança da\\ Informação\end{tabular}}}          & \multicolumn{1}{l|}{\cellcolor[HTML]{C0C0C0}}         & \multicolumn{1}{l|}{67} & \multicolumn{1}{c|}{\multirow{-4}{*}{\begin{tabular}[c]{@{}c@{}}Gerência e Config. \\ de Serviços \\ para a Internet\end{tabular}}} & \multicolumn{1}{l|}{}         & \multicolumn{1}{l|}{\cellcolor[HTML]{C0C0C0}67} & \multicolumn{1}{c|}{\multirow{-4}{*}{\cellcolor[HTML]{C0C0C0}\begin{tabular}[c]{@{}c@{}}Desenvolvimento\\ de Aplicações\\ Corporativas\end{tabular}}} & \multicolumn{1}{l|}{\cellcolor[HTML]{C0C0C0}}         \\ \hline
                          & \multicolumn{1}{l}{}                                                                                                     &                              &                                                 & \multicolumn{1}{l}{}                                                                                                                                &                                                       &                         & \multicolumn{1}{l}{}                                                                                                     &                               &                                                 & \multicolumn{1}{l}{}                                                                                                                      &                                                       &                         & \multicolumn{1}{l}{}                                                                                                                &                               &                                                 & \multicolumn{1}{l}{}                                                                                                                                  &                                                       \\ \hline
\multicolumn{1}{|l|}{A4}  & \multicolumn{1}{c|}{}                                                                                                    & \multicolumn{1}{l|}{}        & \multicolumn{1}{l|}{\cellcolor[HTML]{C0C0C0}B4} & \multicolumn{1}{c|}{\cellcolor[HTML]{C0C0C0}}                                                                                                       & \multicolumn{1}{l|}{\cellcolor[HTML]{C0C0C0}{\bf A3}} & \multicolumn{1}{l|}{C4} & \multicolumn{1}{c|}{}                                                                                                    & \multicolumn{1}{l|}{{\bf B5}} & \multicolumn{1}{l|}{\cellcolor[HTML]{C0C0C0}D4} & \multicolumn{1}{c|}{\cellcolor[HTML]{C0C0C0}}                                                                                             & \multicolumn{1}{l|}{\cellcolor[HTML]{C0C0C0}{\bf B2}} & \multicolumn{1}{l|}{E4} & \multicolumn{1}{c|}{}                                                                                                               & \multicolumn{1}{l|}{{\bf D6}} & \multicolumn{1}{l|}{\cellcolor[HTML]{C0C0C0}F4} & \multicolumn{1}{c|}{\cellcolor[HTML]{C0C0C0}}                                                                                                         & \multicolumn{1}{l|}{\cellcolor[HTML]{C0C0C0}{\bf E1}} \\ \cline{1-1} \cline{3-4} \cline{6-7} \cline{9-10} \cline{12-13} \cline{15-16} \cline{18-18} 
\multicolumn{1}{l|}{}     & \multicolumn{1}{c|}{}                                                                                                    &                              & \multicolumn{1}{l|}{\cellcolor[HTML]{C0C0C0}}   & \multicolumn{1}{c|}{\cellcolor[HTML]{C0C0C0}}                                                                                                       & \cellcolor[HTML]{C0C0C0}                              & \multicolumn{1}{l|}{}   & \multicolumn{1}{c|}{}                                                                                                    &                               & \multicolumn{1}{l|}{\cellcolor[HTML]{C0C0C0}}   & \multicolumn{1}{c|}{\cellcolor[HTML]{C0C0C0}}                                                                                             & \cellcolor[HTML]{C0C0C0}{\bf C4}                      & \multicolumn{1}{l|}{}   & \multicolumn{1}{c|}{}                                                                                                               &                               & \multicolumn{1}{l|}{\cellcolor[HTML]{C0C0C0}}   & \multicolumn{1}{c|}{\cellcolor[HTML]{C0C0C0}}                                                                                                         & \cellcolor[HTML]{C0C0C0}{\bf E5}                      \\
\multicolumn{1}{l|}{}     & \multicolumn{1}{c|}{}                                                                                                    &                              & \multicolumn{1}{l|}{\cellcolor[HTML]{C0C0C0}}   & \multicolumn{1}{c|}{\cellcolor[HTML]{C0C0C0}}                                                                                                       & \cellcolor[HTML]{C0C0C0}                              & \multicolumn{1}{l|}{}   & \multicolumn{1}{c|}{}                                                                                                    &                               & \multicolumn{1}{l|}{\cellcolor[HTML]{C0C0C0}}   & \multicolumn{1}{c|}{\cellcolor[HTML]{C0C0C0}}                                                                                             & \cellcolor[HTML]{C0C0C0}                              & \multicolumn{1}{l|}{}   & \multicolumn{1}{c|}{}                                                                                                               &                               & \multicolumn{1}{l|}{\cellcolor[HTML]{C0C0C0}}   & \multicolumn{1}{c|}{\cellcolor[HTML]{C0C0C0}}                                                                                                         & \cellcolor[HTML]{C0C0C0}{\bf E6}                      \\ \cline{1-1} \cline{3-4} \cline{6-7} \cline{9-10} \cline{12-13} \cline{15-16} \cline{18-18} 
\multicolumn{1}{|l|}{100} & \multicolumn{1}{c|}{\multirow{-4}{*}{\begin{tabular}[c]{@{}c@{}}Algoritmos e \\ Lógica de Programação\end{tabular}}}     & \multicolumn{1}{l|}{}        & \multicolumn{1}{l|}{\cellcolor[HTML]{C0C0C0}83} & \multicolumn{1}{c|}{\multirow{-4}{*}{\cellcolor[HTML]{C0C0C0}\begin{tabular}[c]{@{}c@{}}Probabilidade e \\ Estatística\end{tabular}}}               & \multicolumn{1}{l|}{\cellcolor[HTML]{C0C0C0}}         & \multicolumn{1}{l|}{67} & \multicolumn{1}{c|}{\multirow{-4}{*}{\begin{tabular}[c]{@{}c@{}}Sistemas\\ Operacionais\end{tabular}}}                   & \multicolumn{1}{l|}{}         & \multicolumn{1}{l|}{\cellcolor[HTML]{C0C0C0}67} & \multicolumn{1}{c|}{\multirow{-4}{*}{\cellcolor[HTML]{C0C0C0}\begin{tabular}[c]{@{}c@{}}Programação\\ Paralela\end{tabular}}}             & \multicolumn{1}{l|}{\cellcolor[HTML]{C0C0C0}}         & \multicolumn{1}{l|}{67} & \multicolumn{1}{c|}{\multirow{-4}{*}{\begin{tabular}[c]{@{}c@{}}Engenharia de\\ Software\end{tabular}}}                             & \multicolumn{1}{l|}{}         & \multicolumn{1}{l|}{\cellcolor[HTML]{C0C0C0}67} & \multicolumn{1}{c|}{\multirow{-4}{*}{\cellcolor[HTML]{C0C0C0}Projeto em TSI}}                                                                         & \multicolumn{1}{l|}{\cellcolor[HTML]{C0C0C0}}         \\ \hline
                          & \multicolumn{1}{l}{}                                                                                                     &                              &                                                 & \multicolumn{1}{l}{}                                                                                                                                &                                                       &                         & \multicolumn{1}{l}{}                                                                                                     &                               &                                                 & \multicolumn{1}{l}{}                                                                                                                      &                                                       &                         & \multicolumn{1}{l}{}                                                                                                                &                               &                                                 & \multicolumn{1}{l}{}                                                                                                                                  &                                                       \\ \hline
\multicolumn{1}{|l|}{A5}  & \multicolumn{1}{c|}{}                                                                                                    & \multicolumn{1}{l|}{}        & \multicolumn{1}{l|}{\cellcolor[HTML]{C0C0C0}B5} & \multicolumn{1}{c|}{\cellcolor[HTML]{C0C0C0}}                                                                                                       & \multicolumn{1}{l|}{\cellcolor[HTML]{C0C0C0}{\bf A5}} & \multicolumn{1}{l|}{C5} & \multicolumn{1}{c|}{}                                                                                                    & \multicolumn{1}{l|}{{\bf B1}} & \multicolumn{1}{l|}{\cellcolor[HTML]{C0C0C0}D5} & \multicolumn{1}{c|}{\cellcolor[HTML]{C0C0C0}}                                                                                             & \multicolumn{1}{l|}{\cellcolor[HTML]{C0C0C0}{\bf C2}} & \multicolumn{1}{l|}{E5} & \multicolumn{1}{c|}{}                                                                                                               & \multicolumn{1}{l|}{{\bf D1}} & \multicolumn{1}{l|}{\cellcolor[HTML]{C0C0C0}F5} & \multicolumn{1}{c|}{\cellcolor[HTML]{C0C0C0}}                                                                                                         & \multicolumn{1}{l|}{\cellcolor[HTML]{C0C0C0}}         \\ \cline{1-1} \cline{3-4} \cline{6-7} \cline{9-10} \cline{12-13} \cline{15-16} \cline{18-18} 
\multicolumn{1}{l|}{}     & \multicolumn{1}{c|}{}                                                                                                    &                              & \multicolumn{1}{l|}{\cellcolor[HTML]{C0C0C0}}   & \multicolumn{1}{c|}{\cellcolor[HTML]{C0C0C0}}                                                                                                       & \cellcolor[HTML]{C0C0C0}                              & \multicolumn{1}{l|}{}   & \multicolumn{1}{c|}{}                                                                                                    &                               & \multicolumn{1}{l|}{\cellcolor[HTML]{C0C0C0}}   & \multicolumn{1}{c|}{\cellcolor[HTML]{C0C0C0}}                                                                                             & \cellcolor[HTML]{C0C0C0}                              & \multicolumn{1}{l|}{}   & \multicolumn{1}{c|}{}                                                                                                               &                               & \multicolumn{1}{l|}{\cellcolor[HTML]{C0C0C0}}   & \multicolumn{1}{c|}{\cellcolor[HTML]{C0C0C0}}                                                                                                         & \cellcolor[HTML]{C0C0C0}                              \\
\multicolumn{1}{l|}{}     & \multicolumn{1}{c|}{}                                                                                                    &                              & \multicolumn{1}{l|}{\cellcolor[HTML]{C0C0C0}}   & \multicolumn{1}{c|}{\cellcolor[HTML]{C0C0C0}}                                                                                                       & \cellcolor[HTML]{C0C0C0}                              & \multicolumn{1}{l|}{}   & \multicolumn{1}{c|}{}                                                                                                    &                               & \multicolumn{1}{l|}{\cellcolor[HTML]{C0C0C0}}   & \multicolumn{1}{c|}{\cellcolor[HTML]{C0C0C0}}                                                                                             & \cellcolor[HTML]{C0C0C0}                              & \multicolumn{1}{l|}{}   & \multicolumn{1}{c|}{}                                                                                                               &                               & \multicolumn{1}{l|}{\cellcolor[HTML]{C0C0C0}}   & \multicolumn{1}{c|}{\cellcolor[HTML]{C0C0C0}}                                                                                                         & \cellcolor[HTML]{C0C0C0}                              \\ \cline{1-1} \cline{3-4} \cline{6-7} \cline{9-10} \cline{12-13} \cline{15-16} \cline{18-18} 
\multicolumn{1}{|l|}{33}  & \multicolumn{1}{c|}{\multirow{-4}{*}{\begin{tabular}[c]{@{}c@{}}Fundamentos \\ da\\ Computação\end{tabular}}}            & \multicolumn{1}{l|}{}        & \multicolumn{1}{l|}{\cellcolor[HTML]{C0C0C0}67} & \multicolumn{1}{c|}{\multirow{-4}{*}{\cellcolor[HTML]{C0C0C0}\begin{tabular}[c]{@{}c@{}}Arquitetura de\\ Computadores\end{tabular}}}                & \multicolumn{1}{l|}{\cellcolor[HTML]{C0C0C0}}         & \multicolumn{1}{l|}{33} & \multicolumn{1}{c|}{\multirow{-4}{*}{\begin{tabular}[c]{@{}c@{}}Metodologia \\ da\\ Pesquisa\\ Científica\end{tabular}}} & \multicolumn{1}{l|}{}         & \multicolumn{1}{l|}{\cellcolor[HTML]{C0C0C0}67} & \multicolumn{1}{c|}{\multirow{-4}{*}{\cellcolor[HTML]{C0C0C0}\begin{tabular}[c]{@{}c@{}}Bancos de Dados\\ II\end{tabular}}}               & \multicolumn{1}{l|}{\cellcolor[HTML]{C0C0C0}}         & \multicolumn{1}{l|}{67} & \multicolumn{1}{c|}{\multirow{-4}{*}{\begin{tabular}[c]{@{}c@{}}Programação\\  para \\ Dispositivos Móveis\end{tabular}}}           & \multicolumn{1}{l|}{}         & \multicolumn{1}{l|}{\cellcolor[HTML]{C0C0C0}67} & \multicolumn{1}{c|}{\multirow{-4}{*}{\cellcolor[HTML]{C0C0C0}\begin{tabular}[c]{@{}c@{}}Tópicos \\ Especiais\end{tabular}}}                           & \multicolumn{1}{l|}{\cellcolor[HTML]{C0C0C0}}         \\ \hline
                          & \multicolumn{1}{l}{}                                                                                                     &                              &                                                 & \multicolumn{1}{l}{}                                                                                                                                &                                                       &                         & \multicolumn{1}{l}{}                                                                                                     &                               &                                                 & \multicolumn{1}{l}{}                                                                                                                      &                                                       &                         & \multicolumn{1}{l}{}                                                                                                                &                               &                                                 & \multicolumn{1}{l}{}                                                                                                                                  &                                                       \\ \hline
\multicolumn{1}{|l|}{A6}  & \multicolumn{1}{c|}{}                                                                                                    & \multicolumn{1}{l|}{}        & \multicolumn{1}{l|}{\cellcolor[HTML]{C0C0C0}B6} & \multicolumn{1}{c|}{\cellcolor[HTML]{C0C0C0}}                                                                                                       & \multicolumn{1}{l|}{\cellcolor[HTML]{C0C0C0}{\bf A4}} & \multicolumn{1}{l|}{C6} & \multicolumn{1}{c|}{}                                                                                                    & \multicolumn{1}{l|}{{\bf B3}} & \multicolumn{1}{l|}{\cellcolor[HTML]{C0C0C0}D6} & \multicolumn{1}{c|}{\cellcolor[HTML]{C0C0C0}}                                                                                             & \multicolumn{1}{l|}{\cellcolor[HTML]{C0C0C0}{\bf C6}} & \multicolumn{1}{l|}{E6} & \multicolumn{1}{c|}{}                                                                                                               & \multicolumn{1}{l|}{}         & \multicolumn{1}{l|}{\cellcolor[HTML]{C0C0C0}F6} & \multicolumn{1}{c|}{\cellcolor[HTML]{C0C0C0}}                                                                                                         & \multicolumn{1}{l|}{\cellcolor[HTML]{C0C0C0}}         \\ \cline{1-1} \cline{3-4} \cline{6-7} \cline{9-10} \cline{12-13} \cline{15-16} \cline{18-18} 
\multicolumn{1}{l|}{}     & \multicolumn{1}{c|}{}                                                                                                    &                              & \multicolumn{1}{l|}{\cellcolor[HTML]{C0C0C0}}   & \multicolumn{1}{c|}{\cellcolor[HTML]{C0C0C0}}                                                                                                       & \cellcolor[HTML]{C0C0C0}{\bf A6}                      & \multicolumn{1}{l|}{}   & \multicolumn{1}{c|}{}                                                                                                    &                               & \multicolumn{1}{l|}{\cellcolor[HTML]{C0C0C0}}   & \multicolumn{1}{c|}{\cellcolor[HTML]{C0C0C0}}                                                                                             & \cellcolor[HTML]{C0C0C0}                              & \multicolumn{1}{l|}{}   & \multicolumn{1}{c|}{}                                                                                                               &                               & \multicolumn{1}{l|}{\cellcolor[HTML]{C0C0C0}}   & \multicolumn{1}{c|}{\cellcolor[HTML]{C0C0C0}}                                                                                                         & \cellcolor[HTML]{C0C0C0}                              \\
\multicolumn{1}{l|}{}     & \multicolumn{1}{c|}{}                                                                                                    &                              & \multicolumn{1}{l|}{\cellcolor[HTML]{C0C0C0}}   & \multicolumn{1}{c|}{\cellcolor[HTML]{C0C0C0}}                                                                                                       & \cellcolor[HTML]{C0C0C0}                              & \multicolumn{1}{l|}{}   & \multicolumn{1}{c|}{}                                                                                                    &                               & \multicolumn{1}{l|}{\cellcolor[HTML]{C0C0C0}}   & \multicolumn{1}{c|}{\cellcolor[HTML]{C0C0C0}}                                                                                             & \cellcolor[HTML]{C0C0C0}                              & \multicolumn{1}{l|}{}   & \multicolumn{1}{c|}{}                                                                                                               &                               & \multicolumn{1}{l|}{\cellcolor[HTML]{C0C0C0}}   & \multicolumn{1}{c|}{\cellcolor[HTML]{C0C0C0}}                                                                                                         & \cellcolor[HTML]{C0C0C0}                              \\ \cline{1-1} \cline{3-4} \cline{6-7} \cline{9-10} \cline{12-13} \cline{15-16} \cline{18-18} 
\multicolumn{1}{|l|}{50}  & \multicolumn{1}{c|}{\multirow{-4}{*}{\begin{tabular}[c]{@{}c@{}}Linguagens de \\ Marcação\end{tabular}}}                 & \multicolumn{1}{l|}{}        & \multicolumn{1}{l|}{\cellcolor[HTML]{C0C0C0}50} & \multicolumn{1}{c|}{\multirow{-4}{*}{\cellcolor[HTML]{C0C0C0}\begin{tabular}[c]{@{}c@{}}Linguagens de \\ Script\end{tabular}}}                      & \multicolumn{1}{l|}{\cellcolor[HTML]{C0C0C0}}         & \multicolumn{1}{l|}{83} & \multicolumn{1}{c|}{\multirow{-4}{*}{\begin{tabular}[c]{@{}c@{}}Programação\\ Orientada a\\ Objetos\end{tabular}}}       & \multicolumn{1}{l|}{}         & \multicolumn{1}{l|}{\cellcolor[HTML]{C0C0C0}67} & \multicolumn{1}{c|}{\multirow{-4}{*}{\cellcolor[HTML]{C0C0C0}\begin{tabular}[c]{@{}c@{}}Análise e \\ Projeto\\ de Sistemas\end{tabular}}} & \multicolumn{1}{l|}{\cellcolor[HTML]{C0C0C0}}         & \multicolumn{1}{l|}{67} & \multicolumn{1}{c|}{\multirow{-4}{*}{\begin{tabular}[c]{@{}c@{}}Empreende-\\ dorismo\\ em Software\end{tabular}}}                   & \multicolumn{1}{l|}{}         & \multicolumn{1}{l|}{\cellcolor[HTML]{C0C0C0}67} & \multicolumn{1}{c|}{\multirow{-4}{*}{\cellcolor[HTML]{C0C0C0}TCC}}                                                                                    & \multicolumn{1}{l|}{\cellcolor[HTML]{C0C0C0}}         \\ \hline
                          & CH Semestral                                                                                                             & 417                          &                                                 & \multicolumn{1}{l}{CH Semestral}                                                                                                                    & 401                                                   &                         & \multicolumn{1}{l}{CH Semestral}                                                                                         & 384                           &                                                 & \multicolumn{1}{l}{CH Semestral}                                                                                                          & 418                                                   &                         & \multicolumn{1}{l}{CH Semestral}                                                                                                    & 418                           &                                                 & \multicolumn{1}{l}{CH Semestral}                                                                                                                      & 368                                                   \\
                          & \multicolumn{1}{l}{}                                                                                                     &                              &                                                 & \multicolumn{1}{l}{}                                                                                                                                &                                                       &                         & \multicolumn{1}{l}{}                                                                                                     &                               &                                                 & \multicolumn{1}{l}{}                                                                                                                      &                                                       &                         & \multicolumn{1}{l}{}                                                                                                                &                               &                                                 & \multicolumn{1}{l}{}                                                                                                                                  &                                                       \\ \cline{1-3} \cline{16-18} 
\multicolumn{1}{|l|}{N}   & \multicolumn{1}{c|}{}                                                                                                    & \multicolumn{1}{l|}{{\bf P}} &                                                 & \multicolumn{2}{l}{N: Código da disciplina}                                                                                                                                                                 &                         & \multicolumn{1}{l}{}                                                                                                     & \multicolumn{3}{l}{CH na instituição: 2406}                                                                                                                                                                                 &                                                       &                         & \multicolumn{1}{l}{}                                                                                                                & \multicolumn{1}{l|}{}         & \multicolumn{1}{l|}{F7}                         & \multicolumn{1}{c|}{}                                                                                                                                 & \multicolumn{1}{l|}{}                                 \\ \cline{1-1} \cline{3-3} \cline{16-16} \cline{18-18} 
\multicolumn{1}{l|}{}     & \multicolumn{1}{c|}{}                                                                                                    &                              &                                                 & \multicolumn{2}{l}{P: Pré-requisitos}                                                                                                                                                                       &                         & \multicolumn{1}{l}{}                                                                                                     &                               &                                                 & \multicolumn{1}{l}{}                                                                                                                      &                                                       &                         & \multicolumn{1}{l}{}                                                                                                                &                               & \multicolumn{1}{l|}{}                           & \multicolumn{1}{c|}{}                                                                                                                                 &                                                       \\
\multicolumn{1}{l|}{}     & \multicolumn{1}{c|}{}                                                                                                    &                              &                                                 & \multicolumn{2}{l}{C: Carga Horária}                                                                                                                                                                        &                         & \multicolumn{1}{l}{}                                                                                                     &                               &                                                 & \multicolumn{1}{l}{}                                                                                                                      &                                                       &                         & \multicolumn{1}{l}{}                                                                                                                &                               & \multicolumn{1}{l|}{}                           & \multicolumn{1}{c|}{}                                                                                                                                 &                                                       \\ \cline{1-1} \cline{3-3} \cline{16-16} \cline{18-18} 
\multicolumn{1}{|l|}{C}   & \multicolumn{1}{c|}{\multirow{-4}{*}{\begin{tabular}[c]{@{}c@{}}Nome da \\ Disciplina\end{tabular}}}                     & \multicolumn{1}{l|}{}        &                                                 & \multicolumn{1}{l}{}                                                                                                                                &                                                       &                         & \multicolumn{1}{l}{}                                                                                                     &                               &                                                 & \multicolumn{1}{l}{}                                                                                                                      &                                                       &                         & \multicolumn{1}{l}{}                                                                                                                & \multicolumn{1}{l|}{}         & \multicolumn{1}{l|}{67}                         & \multicolumn{1}{c|}{\multirow{-4}{*}{Libras}}                                                                                                         & \multicolumn{1}{l|}{}                                 \\ \cline{1-3} \cline{16-18} 
\end{tabular}
\end{table}

\end{landscape}


\subsubsection{Coerência do PPC com as Diretrizes Curriculares}\

	A organização curricular do curso observa as determinações legais presentes na Lei de Diretrizes e Bases da Educação Nacional (LDB no. 9.394/96), no Decreto no 5.154/2004, na Resolução CNE/CP no 03/2002 e no Catálogo Nacional de Cursos Superiores de Tecnologia. Esses referenciais norteiam as instituições formadoras, definem o perfil, a atuação e os requisitos básicos necessários à formação profissional do Tecnólogo em Sistemas para Internet, quando estabelece competências e habilidades, conteúdos curriculares, prática profissional, bem como os procedimentos de organização e funcionamento dos cursos.

	Os cursos superiores de tecnologia possuem uma estrutura curricular fundamentada na concepção de eixos tecnológicos constantes do Catálogo Nacional de Cursos Superiores de Tecnologia (CNCST), instituído pela Portaria MEC no. 10/2006. Trata-se de uma concepção curricular que favorece o desenvolvimento de práticas pedagógicas integradoras e articula o conceito de trabalho, ciência, tecnologia e cultura, à medida que os eixos tecnológicos se constituem de agrupamentos dos fundamentos científicos comuns, de intervenções na natureza, de processos produtivos e culturais, além de aplicações científicas às atividades humanas.

	A proposta pedagógica do curso está organizada por núcleos politécnicos os quais favorecem a prática da interdisciplinaridade, apontando para o reconhecimento da necessidade de uma educação profissional e tecnológica integradora de conhecimentos científicos e experiências e saberes advindos do mundo do trabalho, e possibilitando, assim, a construção do pensamento tecnológico crítico e a capacidade de intervir em situações concretas.

	Essa proposta possibilita a realização de práticas interdisciplinares, concernente a conhecimentos científicos e tecnológicos, propostas metodológicas, tempos e espaços de formação.

	Desse modo, a matriz curricular dos cursos de graduação tecnológica organiza-se em dois núcleos, o núcleo fundamental e o núcleo científico e tecnológico. O núcleo fundamental compreende conhecimentos científicos imprescindíveis ao desempenho acadêmico dos ingressantes. Contempla, ainda, revisão de conhecimentos da formação geral, objetivando construir base científica para a formação tecnológica. Nesse núcleo, há dois propósitos pedagógicos indispensáveis: o domínio da língua portuguesa e, de acordo com as necessidades do curso, a apropriação dos conceitos científicos básicos.

	O núcleo científico e tecnológico compreende disciplinas destinadas à caracterização da identidade do profissional tecnólogo. Compõe-se por uma unidade básica (relativa a conhecimentos de formação científica para o ensino superior e de formação tecnológica básica) e por uma unidade tecnológica (relativa à formação tecnológica específica, de acordo com a área do curso). Essa última unidade contempla conhecimentos intrínsecos à área do curso, conhecimentos necessários à integração curricular e conhecimentos imprescindíveis à formação específica.

	A matriz curricular do curso está organizada por disciplinas em regime de crédito, com período semestral, com 2.339 horas destinadas às disciplinas que compõem os núcleos politécnicos, 67 horas destinadas ao trabalho de conclusão de curso, 40 horas destinadas a atividades complementares e 300 horas destinadas à prática profissional, totalizando a carga horária de 2.746 horas.
	
	Além disso, o curso apresenta:

	\begin{itemize}
		\item Regime de matrícula: semestral por disciplina; 
		\item Carga horária máxima por semestre: 25 aulas semanais (aulas de 50 minutos);
		\item Vagas totais anuais: 60 (sessenta vagas), divididas em duas turmas de 30 alunos, uma entrada de 30 alunos a cada semestre; 
		\item Turno funcionamento: integral (com aulas preferencialmente no período vespertino);
		\item Carga horária mínima para integralização do curso: 2.746 horas-relógio;
		\item Estágio curricular obrigatório: 300 horas-relógio, com defesa; 
		\item Trabalho de conclusão de curso: obrigatório, 67 horas-relógio, com defesa; 
		\item A disciplina Linguagem Brasileira de Sinais, LIBRAS, é optativa, sendo oferecida para os alunos do IFPB;
		\item Tempo mínimo para integralização do curso: 5 semestres letivos;
		\item Tempo máximo para integralização do curso: 12 semestres letivos;
		\item De acordo com a organização curricular o percurso de formação do egresso do curso superior de tecnologia em sistemas para internet se dará da seguinte forma:
		\begin{itemize}
			\item Só poderão ser cursadas as disciplinas ofertadas, respeitando-se a carga horária máxima semestral;
			\item Os alunos blocados terão prioridade na matrícula;
			\item Será permitido no máximo matrícula de 50 alunos por disciplina;
			\item O aluno só poderá se matricular em uma disciplina se tiver concluído seu respectivo pré-requisito;
		\end{itemize}
	\end{itemize}

\subsubsection{Coerência dos Conteúdos Curriculares com o Perfil do Egresso}

	Para a formação de um profissional inserido no perfil almejado, os conteúdos curriculares, bem como as ações educativas dos profissionais do IFPB, são fundamentados em objetivos que sustentem e impulsionem a prática do educando, de forma a repercutir na atuação docente cotidiana e edificar o profissional desejado.

	As disciplinas oferecidas durante todo o curso contemplam conhecimentos e saberes necessários à formação das competências elencadas no perfil do egresso a partir da congruência entre teoria e prática, pois aglutinam atividades que impulsionam o discente ao constante diálogo com o contexto profissional. Aplicando os conhecimentos com uma base sólida dos princípios e fundamentos, entendendo o contexto social em que opera, bem como as suas relações interinstitucionais, com a análise do impacto das tecnologias sobre os indivíduos, organização e sociedade, abrangendo os aspectos éticos, ambientais e de segurança.

	Para atingir esse perfil de profissional, o currículo do curso apresenta plena coerência com o perfil traçado para o egresso, sobretudo, porque a concepção dos componentes curriculares básicos e tecnológicos é abordada de maneira a desenvolver nos alunos os conceitos essenciais da computação de maneira sólida e propiciar-lhes facilidades para o acompanhamento futuro da evolução da tecnologia da informação e comunicação.

	Conscientes do contexto em constante mudança, o alinhamento dos conteúdos programáticos das atividades acadêmicas no IFPB será trabalhado constantemente pelas coordenações em conjunto com o Núcleo Docente Estruturante e demais professores, dedicando especial atenção para que os conteúdos curriculares ministrados, bem como o ementário e demais atividades do curso, sejam adequados ao perfil desejado do egresso em um processo de melhoria contínua.

	O educando deverá, de forma interdisciplinar, integrar-se no espaço de atuação profissional não só como agente cultural, mas também ator de transformação técnica e capacidade de abordagem do conhecimento, além de adequar-se às constantes mudanças no campo científico, cultural e tecnológico.

%\paragraph{Coerência dos conteúdos curriculares com os objetivos do curso}\

%\paragraph{Coerência dos conteúdos curriculares com o perfil do egresso}\

%\newpage

\subsubsection{Ementário e Bibliografia}

	As ementas do curso de Tecnologia em Sistemas para Internet estão adequadas e atualizadas de acordo com os objetivos e o perfil de conclusão do egresso que estão definidos no projeto pedagógico do curso. 
			
		Para que o curso esteja sempre atualizado existe a garantia de que os planos de ensino serão revisados semestralmente conforme a seguinte determinação que consta no Regulamento Didático Cursos Superiores, no seu artigo 4º: 
		
		“o planejamento acadêmico dos cursos de graduação, os planos de ensino e respectivos programas curriculares e demais atividades relacionadas ao desenvolvimento do processo educativo serão avaliados semestralmente pelo Colegiado de Curso.” 
		
		Além do Colegiado do Curso, o Núcleo Docente Estruturante (NDE) também estará participando da avaliação do processo ensino aprendizagem e propondo mudanças necessárias nos planos de ensino, que venham a ser apontadas por professores e alunos. Assim, o ementário, os planos de ensino e a bibliografia estarão em permanente processo de atualização. Para dar suporte a essas mudanças será utilizada a elaboração e aprovação semestral dos planos de ensino possibilitando aos professores a oportunidade de, ao detectar possíveis falhas, encaminhar imediatamente a solicitação de modificação do plano de ensino ao Colegiado do curso para que se tome as devidas providências no sentido de se garantir a contínua adequação dos planos de ensino aos objetivos e perfil de conclusão do egresso, previsto no projeto pedagógico do curso (PPC), fato que permite a utilização do ato educativo de formar o ser para o seu tempo.

\paragraph{Estruturas de Dados I}

%PREENCHER DADOS DA DISCIPLINA A SEGUIR
%\vspace{-12mm}
\begin{center}\textbf{Dados do Componente Curricular}\end{center}
\vspace{-5mm}
\noindent\rule{16.5cm}{0.4pt}
\\
\textbf{Nome:} Estruturas de Dados I
\\ 
\textbf{Curso:} Tecnologia em Sistemas para Internet
\\ 
\textbf{Período:} \unit{2}{\degree}
\\ 
\textbf{Carga Horária:} \unit{67}{\hour}
\\ 
\textbf{Docente Responsável:} Otacílio de Araújo Ramos Neto
\\ 
\noindent\rule{16.5cm}{0.4pt}\\
\\
%PREENCHER A EMENTA A SEGUIR
\vspace{-12mm}
\begin{center}\textbf{Ementa}\end{center}
\vspace{-5mm}
\noindent\rule{16.5cm}{0.4pt}
\\
Conceitos básicos, crescimento de funções e recorrências; Algoritmos de ordenação e busca; Estruturas de dados elementares; Árvores de busca binária.\\ 
\noindent\rule{16.5cm}{0.4pt}\\
\\
%PREENCHER OS OBJETIVOS A SEGUIR
\vspace{-12mm}
\begin{center}\textbf{Objetivos}\end{center}
\vspace{-5mm}
\noindent\rule{16.5cm}{0.4pt}
\\
\begin{itemize}
\item Apresentar os conceitos básicos para criação e análise de algoritmos;
\item Apresentar os algoritmos básicos de ordenação e busca;
\item Capacitar os alunos a utilizarem as estruturas de dados elementares em problemas reais;
\item Apresentar aos alunos as árvores de busca binária e capacitá-los no seu uso.
\end{itemize} 
\noindent\rule{16.5cm}{0.4pt}\\
\\
%PREENCHER OS CONTEUDOS PROGRAMATICOS A SEGUIR (CUIDADO PARA NAO DEIXAR A TABELA MUITO GRANDE)
\vspace{-12mm}
\begin{center}\textbf{Conteúdo Programático}\end{center}
\vspace{-5mm}
\noindent\rule{16.5cm}{0.4pt}
\\
\begin{itemize}
 \item \textbf{Conceitos básicos:} Análise e projeto de algoritmos; Notação assintótica; O Método da Substituição, Método da Árvore de Recursão e Método Mestre.
 \item \textbf{Algoritmos de ordenação e busca:} Ordenação por inserção, Heapsort, Quicksort e ordenação em tempo linear; Busca sequencial e busca binária.
 \item \textbf{Estruturas de dados elementares:} Implementações de ponteiros e objetos; Pilhas, filas e listas ligadas.
 \item \textbf{Árvores de pesquisa binária:} Conceitos fundamentais de árvores de pesquisa binária; Algoritmos de inserção, remoção e busca; Impressão \textit{In-Order}, \textit{Post-Order} e \textit{Pre-Order} .
\end{itemize}
\noindent\rule{16.5cm}{0.4pt}\\
\\
%COLOCAR A METODOLOGIA DE ENSINO A SEGUIR
\vspace{-12mm}
\begin{center}\textbf{Metodologia de Ensino}\end{center} 
\vspace{-5mm}
\noindent\rule{16.5cm}{0.4pt}
\\
   Aulas expositivas utilizando recursos audiovisuais e quadro, além de aulas práticas utilizando computadores. Adicionalmente, serão realizadas atividades práticas individuais ou em grupo, para consolidação do conteúdo ministrado.\\
\noindent\rule{16.5cm}{0.4pt}\\
\\
%COLOCAR AVALIACAO DO PROCESSO DE ENSINO E APRENDIZAGEM A SEGUIR
\vspace{-12mm}
\begin{center}\textbf{Avaliação do Processo de Ensino e Apendizagem}\end{center}
\vspace{-5mm}
\noindent\rule{16.5cm}{0.4pt}
\\
   Avaliações escritas. Avalia\c{c}\~oes pr\'aticas envolvendo a resolu\c{c}\~ao de problemas computacionais.\\
\noindent\rule{16.5cm}{0.4pt}\\
\\
%PREENCHER RECURSOS NECESSARIOS A SEGUIR
\vspace{-12mm}
\begin{center}\textbf{Recursos Necessários}\end{center}
\vspace{-5mm}
\noindent\rule{16.5cm}{0.4pt}
\\
\begin{itemize} 
  \item Listas de Exercícios;
  \item Livros e apostilas;
  \item Utilização de recursos da web;
  \item Quadro branco;
  \item Marcadores para quadro branco;
  \item Sala de aula com acesso à internet, microcomputador e TV ou projetor para apresentação de slides ou material multimídia;
  \item Laboratório de microcomputadores contendo componentes de hardware e software específicos;
\end{itemize}
\noindent\rule{16.5cm}{0.4pt}\\ - 
\\
%PREENCHER BIBLIOGRAFIA A SEGUIR
\vspace{-12mm}
\begin{center}\textbf{Bibliografia}\end{center}
\vspace{-5mm}
\noindent\rule{16.5cm}{0.4pt}
\\
\begin{itemize} 
  \item Básica:
	\begin{enumerate}
	\item CELES, Waldemar. Introdução a Estrutura de Dados -  ISBN 978-85-3521-228-0, Editora Campus Elsevier, 2004.
	\item T.H. Cormen, C.E. Leiserson, R.L. Rivest, C. Stein, "Algoritmos - Teoria e Prática", 3a. ed., ISBN: 8535236996, Editora Campus, 2012.
	\item TENENBAUM, Aaron M. LANGSAM, Yedidyah. AUGENSTEIN, Moshe J. Estrutura de Dados Usando C - ISBN 8534603480, Makron Books. 
	
	\end{enumerate}
  \item Complementar:
	\begin{enumerate} 
	\item Steven S Skiena, The Algorithm Design Manual, Springer; 2nd edition, ISBN: 978-1849967204, 2008.\\
	\end{enumerate}
\end{itemize}
\noindent\rule{16.5cm}{0.4pt}\\
\\

\clearpage
\paragraph{Fundamentos da Computa\c{c}\~ao}

%PREENCHER DADOS DA DISCIPLINA A SEGUIR
%\vspace{-12mm}
\begin{center}\textbf{Dados do Componente Curricular}\end{center}
\vspace{-5mm}
\noindent\rule{16.5cm}{0.4pt}
\\
\textbf{Nome:} Fundamentos da Computa\c{c}\~ao
\\ 
\textbf{Curso:} Tecnologia em Sistemas para Internet
\\ 
\textbf{Período:} \unit{1}{\degree}
\\ 
\textbf{Carga Horária:} \unit{33}{\hour}
\\ 
\textbf{Docente Responsável:} Ruan Delgado Gomes
\\ 
\noindent\rule{16.5cm}{0.4pt}\\
\\
%PREENCHER A EMENTA A SEGUIR
\vspace{-12mm}
\begin{center}\textbf{Ementa}\end{center}
\vspace{-5mm}
\noindent\rule{16.5cm}{0.4pt}
\\
Conceitos introdut\'orios de inform\'atica; Representa\c{c}\~ao de dados e convers\~ao de base; Opera\c{c}\~oes aritm\'eticas com n\'umeros bin\'arios; L\'ogica digital; T\'opicos especiais em computa\c{c}\~ao.\\
\noindent\rule{16.5cm}{0.4pt}\\
\\
%PREENCHER OS OBJETIVOS A SEGUIR
\vspace{-12mm}
\begin{center}\textbf{Objetivos}\end{center}
\vspace{-5mm}
\noindent\rule{16.5cm}{0.4pt}
\\
\begin{itemize}
\item Apresentar os conceitos de \textit{hardware} e \textit{software};
\item Apresentar a representação digital de dados e informação;
\item Introduzir conceitos de l\'ogica;
\item Apresentar o funcionamento das portas lógicas;
\item Apresentar as tecnologias e aplicações de computadores.
\end{itemize} 
\noindent\rule{16.5cm}{0.4pt}\\
\\
%PREENCHER OS CONTEUDOS PROGRAMATICOS A SEGUIR (CUIDADO PARA NAO DEIXAR A TABELA MUITO GRANDE)
\vspace{-12mm}
\begin{center}\textbf{Conteúdo Programático}\end{center}
\vspace{-5mm}
\noindent\rule{16.5cm}{0.4pt}
\\
\begin{itemize}

 \item \textbf{Conceitos introdut\'orios de inform\'atica:} Histórico e evolução dos computadores; Defini\c{c}\~oes de \textit{software} e \textit{hardware}; Modelo conceitual da arquitetura de organiza\c{c}\~ao de um computador; Classifica\c{c}\~ao dos computadores; Perif\'ericos de entrada e sa\'ida.

 \item \textbf{Representa\c{c}\~ao de dados e convers\~ao de base:} Representação de dados; Representação de números inteiros na base binária; Representação de números inteiros na base octal; Representação de números inteiros nas base hexadecimal; Convers\~ao de bases.

 \item \textbf{Opera\c{c}\~oes aritm\'eticas com n\'umeros bin\'arios:} Opera\c{c}\~oes aritm\'eticas b\'asicas com n\'umeros bin\'arios.

 \item \textbf{L\'ogica digital:} Introdu\c{c}\~ao \`a l\'ogica; L\'ogica digital; Portas l\'ogicas; Constru\c{c}\~ao de circuitos combinacionais simples.

 \item \textbf{T\'opicos especiais em computa\c{c}\~ao:} Conte\'udo vari\'avel, envolvendo temas relevantes e atuais da computa\c{c}\~ao.

\end{itemize}
\noindent\rule{16.5cm}{0.4pt}\\
\\
%COLOCAR A METODOLOGIA DE ENSINO A SEGUIR
\vspace{-12mm}
\begin{center}\textbf{Metodologia de Ensino}\end{center} 
\vspace{-5mm}
\noindent\rule{16.5cm}{0.4pt}
\\
   Aulas expositivas utilizando recursos audiovisuais e quadro, além de aulas práticas utilizando computadores. Adicionalmente, serão realizadas atividades práticas individuais ou em grupo, para consolidação do conteúdo ministrado.\\
\noindent\rule{16.5cm}{0.4pt}\\
\\
%COLOCAR AVALIACAO DO PROCESSO DE ENSINO E APRENDIZAGEM A SEGUIR
\vspace{-12mm}
\begin{center}\textbf{Avaliação do Processo de Ensino e Apendizagem}\end{center}
\vspace{-5mm}
\noindent\rule{16.5cm}{0.4pt}
\\
   Avaliações escritas e pr\'aticas\\
\noindent\rule{16.5cm}{0.4pt}\\
\\
%PREENCHER RECURSOS NECESSARIOS A SEGUIR
\vspace{-12mm}
\begin{center}\textbf{Recursos Necessários}\end{center}
\vspace{-5mm}
\noindent\rule{16.5cm}{0.4pt}
\\
\begin{itemize} 
  \item Listas de Exercícios;
  \item Livros e apostilas;
  \item Utilização de recursos da web;
  \item Quadro branco;
  \item Marcadores para quadro branco;
  \item Sala de aula com acesso à internet, microcomputador e TV ou projetor para apresentação de slides ou material multimídia;
  \item Laboratório de microcomputadores contendo componentes de hardware e software específicos;
\end{itemize}
\noindent\rule{16.5cm}{0.4pt}\\
\\
%PREENCHER BIBLIOGRAFIA A SEGUIR
\vspace{-12mm}
\begin{center}\textbf{Bibliografia}\end{center}
\vspace{-5mm}
\noindent\rule{16.5cm}{0.4pt}
\\
\begin{itemize} 
  \item Básica:
	\begin{enumerate}
	\item MONTEIRO, M. A. Introdução à Organização de Computadores. ISBN: 9788521615439. Editora LTC. 5 Ed., 2007; 
	\item IDOETA, I. V.; CAPUANO, F. G. Elementos de Eletrônica Digital. ISBN: 8571940193. Editora Érica, 40 Ed., 2007;
	\item VELLOSO, F. C. Informática: Conceitos Básicos. ISBN: 9788535243970. Editora Campus, 8 Ed., 2011. 
	\end{enumerate}
  \item Complementar:
	\begin{enumerate} 
	\item TANENBAUM, Andrew S. Organização Estruturada de Computadores. ISBN: 9788581435398. Editora Pearson. 6 Ed., 2013.
	\end{enumerate}
\end{itemize}
\noindent\rule{16.5cm}{0.4pt}\\
\\

\clearpage
\paragraph{Algoritmos e L\'ogica de Programa\c{c}\~ao}

%PREENCHER DADOS DA DISCIPLINA A SEGUIR
%\vspace{-12mm}
\begin{center}\textbf{Dados do Componente Curricular}\end{center}
\vspace{-5mm}
\noindent\rule{16.5cm}{0.4pt}
\\
\textbf{Nome:} Algoritmos e L\'ogica de Programa\c{c}\~ao
\\ 
\textbf{Curso:} Tecnologia em Sistemas para Internet
\\ 
\textbf{Período:} \unit{1}{\degree}
\\ 
\textbf{Carga Horária:} \unit{100}{\hour}
\\ 
\textbf{Docente Responsável:} Ruan Delgado Gomes
\\ 
\noindent\rule{16.5cm}{0.4pt}\\
\\
%PREENCHER A EMENTA A SEGUIR
\vspace{-12mm}
\begin{center}\textbf{Ementa}\end{center}
\vspace{-5mm}
\noindent\rule{16.5cm}{0.4pt}
\\
Conceito de Algoritmos e Linguagens de Programação; Estruturas de Decisão; Estruturas de Repetição; Vetores e Matrizes; Manipulação de Strings; Modularização; Recursividade; Registros (Estruturas).\\
\noindent\rule{16.5cm}{0.4pt}\\
\\
%PREENCHER OS OBJETIVOS A SEGUIR
\vspace{-12mm}
\begin{center}\textbf{Objetivos}\end{center}
\vspace{-5mm}
\noindent\rule{16.5cm}{0.4pt}
\\
\begin{itemize}
\item Saber construir programas de computador obedecendo aos princípios da programação estruturada;
\item Conhecer conceitos básicos relacionados à construção de algoritmos;
\item Compreender e elaborar estruturas de controle;
\item Saber manipular dados por meio de Strings, Vetores e Matrizes;
\item Aprender os conceitos para criação de sub-rotinas, passagem de parâmetros e escopos de variáveis;
\item Aprender o conceito de registro (estrutura).
\end{itemize} 
\noindent\rule{16.5cm}{0.4pt}\\
\\
%PREENCHER OS CONTEUDOS PROGRAMATICOS A SEGUIR (CUIDADO PARA NAO DEIXAR A TABELA MUITO GRANDE)
\vspace{-12mm}
\begin{center}\textbf{Conteúdo Programático}\end{center}
\vspace{-5mm}
\noindent\rule{16.5cm}{0.4pt}
\\
\begin{itemize}

 \item \textbf{Conceito de Algoritmos e Linguagens de Programação:} Defini\c{c}\~ao; Caracter\'isticas; Formas de representa\c{c}\~ao de algoritmos; Diferen\c{c}a entre linguagens de baixo n\'ivel e alto n\'ivel; Conceito de vari\'avel e mem\'oria; Comandos de entrada e saída; Express\~oes aritm\'eticas; Express\~oes l\'ogicas e relacionais; Preced\^encia de operadores.

 \item \textbf{Estruturas de Decisão:} Estrutura \textit{if else}; Estruturas de decis\~ao aninhadas; Estrutura \textit{switch case}.

 \item \textbf{Estruturas de Repetição:} Estruturas de repeti\c{c}\~ao \textit{while}, \textit{for} e \textit{do while}, ou estruturas equivalentes na linguagem de programa\c{c}\~ao adotada.

 \item \textbf{Vetores e Matrizes:} Conceito de vetores; Declara\c{c}\~ao e manipula\c{c}\~ao de vetores; Conceito de matrizes; Declara\c{c}\~ao e manipula\c{c}\~ao de matrizes; Vetores multidimensionais.

 \item \textbf{Manipulação de Strings:} Declara\c{c}\~ao e manipula\c{c}\~ao de strings; Codifica\c{c}\~ao de caracteres (ex: tabela ASCII, tabela Unicode). Fun\c{c}\~oes \'uteis para manipula\c{c}\~ao de strings.

 \item \textbf{Modulariza\c{c}\~ao:} Criação de sub-rotinas, passagem de parâmetros por valor e por refer\^encia, escopo de variáveis (variáveis locais e variáveis globais), ponteiros.

 \item \textbf{Recursividade:} Defini\c{c}\~ao recursiva de algoritmos; Pilha de execu\c{c}\~ao; Resolu\c{c}\~ao de problemas utilizando recursividade.

 \item \textbf{Registros (Estruturas):} Defini\c{c}\~ao de novos tipos de dados; Agrupamento de vari\'aveis para cria\c{c}\~ao de tipos mais complexos utilizando estruturas.

\end{itemize}
\noindent\rule{16.5cm}{0.4pt}\\
\\
%COLOCAR A METODOLOGIA DE ENSINO A SEGUIR
\vspace{-12mm}
\begin{center}\textbf{Metodologia de Ensino}\end{center} 
\vspace{-5mm}
\noindent\rule{16.5cm}{0.4pt}
\\
   Aulas expositivas utilizando recursos audiovisuais e quadro, além de aulas práticas utilizando computadores. Adicionalmente, serão realizadas atividades práticas individuais ou em grupo, para consolidação do conteúdo ministrado.\\
\noindent\rule{16.5cm}{0.4pt}\\
\\
%COLOCAR AVALIACAO DO PROCESSO DE ENSINO E APRENDIZAGEM A SEGUIR
\vspace{-12mm}
\begin{center}\textbf{Avaliação do Processo de Ensino e Apendizagem}\end{center}
\vspace{-5mm}
\noindent\rule{16.5cm}{0.4pt}
\\
   Avaliações escritas. Avalia\c{c}\~oes pr\'aticas envolvendo a resolu\c{c}\~ao de problemas computacionais.\\
\noindent\rule{16.5cm}{0.4pt}\\
\\
%PREENCHER RECURSOS NECESSARIOS A SEGUIR
\vspace{-12mm}
\begin{center}\textbf{Recursos Necessários}\end{center}
\vspace{-5mm}
\noindent\rule{16.5cm}{0.4pt}
\\
\begin{itemize} 
  \item Listas de Exercícios;
  \item Livros e apostilas;
  \item Utilização de recursos da web;
  \item Quadro branco;
  \item Marcadores para quadro branco;
  \item Sala de aula com acesso à internet, microcomputador e TV ou projetor para apresentação de slides ou material multimídia;
  \item Laboratório de microcomputadores contendo componentes de hardware e software específicos;
\end{itemize}
\noindent\rule{16.5cm}{0.4pt}\\
\\
%PREENCHER BIBLIOGRAFIA A SEGUIR
\vspace{-12mm}
\begin{center}\textbf{Bibliografia}\end{center}
\vspace{-5mm}
\noindent\rule{16.5cm}{0.4pt}
\\
\begin{itemize} 
  \item Básica:
	\begin{enumerate}
	\item Piva Junior, D., Engelbrecht,  A. M., Nakamiti,  G. S. e Bianchi, F.. Algoritmos e Programação de Computadores. ISBN: 9788535250312. Editora Campus. 1 ed, 2012;
	\item CELES, Waldemar. Introdução a Estrutura de Dados -  ISBN 978-85-3521-228-0, Editora Campus Elsevier, 2004.
	\end{enumerate}
  \item Complementar:
	\begin{enumerate} 
	\item Oliveira, U. Programando em C – Volume 1: Fundamentos. ISBN: 8573936592. Editora Ciência Moderna. 2007;
	\end{enumerate}
\end{itemize}
\noindent\rule{16.5cm}{0.4pt}\\
\\

\clearpage
%\section{Dados da Institui\c{c}\~ao}
%Sistemas Distribuídos

\begin{table}[h!]

\centering
% definindo o tamanho da fonte para small
% outros possíveis tamanhos: footnotesize, scriptsize
\begin{small} 
  % redefinindo o espaçamento das colunas
\setlength{\tabcolsep}{3pt} 
\begin{tabular}{|p{15cm}|}\hline


\begin{center}\textbf{Dados do Componente Curricular}\end{center}\\ \hline

%PREENCHER DADOS DA DISCIPLINA A SEGUIR
\textbf{Nome:} Probabilidade e Estatística \\ \hline
\textbf{Curso:} Tecnologia em Sistemas para Internet \\ \hline
\textbf{Período:} \unit{4}{\degree} \\ \hline
\textbf{Carga Horária:} \unit{67}{\hour} \\ \hline
\textbf{Docente Responsável:} Otacílio de Araújo Ramos Neto \\ \hline


\end{tabular} 
\end{small}
\label{ementa:ProbabilidadeeEstatistica}
\end{table} 

\begin{table}[h!]
\centering
\begin{small} 
\setlength{\tabcolsep}{1pt} 
\begin{tabular}{|p{15cm}|}\hline

%PREENCHER A EMENTA A SEGUIR
\begin{center}\textbf{Ementa}\end{center}\\ \hline

Análise Estatística de Dados. Espaço Amostral. Probabilidade e seus teoremas. Probabilidade Condicional e Independência de Eventos. Teorema de Bayes. Distribuições de Variáveis Aleatórias Discretas e Contínuas Unidimensionais. Valor Esperado, Variância e Desvio Padrão. Modelos Probabilísticos Discretos: Uniforme, Bernoulli, Binomial e Poisson. Modelos Probabilísticos Contínuos: Uniforme e Normal. Estimação. Testes de Hipóteses. Tomada de decisão utilizando Redes Bayesianas. \\ \hline

\end{tabular} 
\end{small}
%\label{dadosinstituicao}
\end{table} 

\hspace{1cm}
\begin{table}[h!]
\centering
\begin{small} 
\setlength{\tabcolsep}{3pt} 
\begin{tabular}{|p{15cm}|}\hline

%PREENCHER OS OBJETIVOS A SEGUIR
\begin{center}\textbf{Objetivos}\end{center}\\ \hline
\begin{itemize}

\item Utilizar métodos e técnicas estatísticas que possibilitem sumariar, calcular e analisar dados para uso na tomada de decisão auxiliada por computador;
\item Estudar resultados de experimentos aleatórios de maneira a modelar a previsão desses resultados e a probabilidade com que se pode confiar nas probabilidades obtidas;
\item Conhecer a representação gráfica, as medidas de posição e de dispersão;
\item Apresentar os principais modelos probabilísticos discretos e contínuos;
\item Conhecer a Estatística Inferencial e avaliar o tamanho do erro ao fazer generalizações;
\item Modelar a tomada de decisão por computador utilizando as redes de Bayes.
\end{itemize}
\\ \hline

\end{tabular} 
\end{small}
%\label{dadosinstituicao}
\end{table}

\hspace{1cm}
\begin{table}[h!]
\centering
\begin{small} 
\setlength{\tabcolsep}{3pt} 
\begin{tabular}{|p{15cm}|}\hline

%PREENCHER OS CONTEUDOS PROGRAMATICOS A SEGUIR (CUIDADO PARA NAO DEIXAR A TABELA MUITO GRANDE)
\begin{center}\textbf{Conteúdo Programático}\end{center}\\ \hline
\begin{itemize}
 \item \textbf{Estatística Descritiva:} Introdução à Estatística; Importância da Estatística; Grandes áreas da Estatística; Fases do método estatístico.
 \item \textbf{Distribuição de Frequência:} Elementos de uma distribuição de frequência; Amplitude total; Limites de classe; Amplitude do intervalo de classe; Ponto médio da classe; Frequência absoluta, relativa e acumulada; Regras gerais para a elaboração de uma distribuição de frequência; Gráficos representativos de uma distribuição de frequência: Histograma e gráfico de coluna.
 \item \textbf{Medidas de Posição:} Introdução; Média aritmética simples e ponderada e suas propriedades; Moda: Dados agrupados e não  agrupados em classes; Mediana: Dados agrupados e não agrupados em classes.
 \item \textbf{Medidas de Dispersão:} Variância; Desvio padrão; Coeficiente de variação.
 \item \textbf{Probabilidade:} Experimentos aleatórios, espaço amostral e eventos; Definição clássica da Probabilidade; Frequência relativa; Tipos de eventos; Axiomas de Probabilidade; Probabilidade condicional e independência de eventos; Teorema de Bayes, do Produto e da Probabilidade Total.
 \item \textbf{Variáveis Aleatórias:} Conceito de variável aleatória; Distribuição de probabilidade, função densidade de probabilidade, esperança matemática, variância, desvio padrão e suas propriedades para variáveis aleatórias discretas e contínuas.
 \item \textbf{Distribuições Discretas:} Bernoulli, Binomial e Poisson.
 \item \textbf{Distribuição Contínuas:} Uniforme; Normal Padrão (propriedades e distribuição); Aproximação Binomial da Distribuição Normal.
 \item \textbf{Inferência Estatística:} População e amostra; Estatísticas e parâmetros; Distribuições amostrais.
 \item \textbf{Estimação:} pontual e por intervalo.
 \item \textbf{Testes de Hipóteses:} Principais conceitos; Testes de hipóteses para média de populações normais com variância 
 \item \textbf{Redes Bayesianas:} Cálculo de probabilidades; Aplicando a regra de Bayes; Inferência em Redes Bayesianas; Aplicações em inteligência artificial.
\end{itemize}
 \\ \hline

\end{tabular} 
\end{small}
\label{dadosinstituicao}
\end{table}


\begin{table}[h!]
\centering
\begin{small} 
\setlength{\tabcolsep}{3pt} 
\begin{tabular}{|p{15cm}|}\hline

%COLOCAR A METODOLOGIA DE ENSINO A SEGUIR
\begin{center}\textbf{Metodologia de Ensino}\end{center}\\ \hline
   Aulas expositivas utilizando recursos audiovisuais e quadro.
 \\ \hline
\end{tabular} 
\end{small}
\label{dadosinstituicao}
\end{table}


\begin{table}[h!]
\centering
\begin{small} 
\setlength{\tabcolsep}{3pt} 
\begin{tabular}{|p{15cm}|}\hline

%COLOCAR AVALIACAO DO PROCESSO DE ENSINO E APRENDIZAGEM A SEGUIR
\begin{center}\textbf{Avaliação do Processo de Ensino e Apendizagem}\end{center}\\ \hline
   Avaliações escritas ao final de cada unidade.
 \\ \hline

\end{tabular} 
\end{small}
\label{dadosinstituicao}
\end{table}

\begin{table}[h!]
\centering
\begin{small} 
 
\setlength{\tabcolsep}{3pt} 
\begin{tabular}{|p{15cm}|}\hline

%PREENCHER RECURSOS NECESSARIOS A SEGUIR
\begin{center}\textbf{Recursos Necessários}\end{center}\\ \hline
\begin{itemize} 
  \item Listas de Exercícios;
  \item Livros e apostilas;
  \item Utilização de recursos da web;
  \item Quadro branco;
  \item Marcadores para quadro branco;
  \item Sala de aula com acesso à internet, microcomputador e TV ou projetor para apresentação de slides ou material multimídia;
\end{itemize}
 \\ \hline

\end{tabular} 
\end{small}
\label{dadosinstituicao}
\end{table}


\begin{table}[h!]
\centering
\begin{small} 
\setlength{\tabcolsep}{3pt} 
\begin{tabular}{|p{15cm}|}\hline

%PREENCHER BIBLIOGRAFIA A SEGUIR
\begin{center}\textbf{Bibliografia}\end{center}\\ \hline
\begin{itemize} 
  \item Básica;
  
  BARBETTA, P.A.; REIS, M.M. e BORNIA, A.C. Estatística para cursos de engenharia e informática. Editora Atlas, São Paulo, 2004. 410 p.
  
  BUSSAB, W. O. MORETTIN, P. A.Estatística Básica.  5 ed.  São Paulo: Saraiva, 2002.
  
  
  \item Complementar;
  
  MEYER, P.L. Probabilidade: Aplicações à Estatística. 2 ed.  Rio de Janeiro: LTC – Livros Técnicos e Científicos, 2000
  
  FONSECA, J.S. e Martins, G.A. Curso de Estatística. São Paulo: Atlas, 1993.
\end{itemize}
 \\ \hline

\end{tabular} 
\end{small}
\label{dadosinstituicao}
\end{table}

\clearpage
\paragraph{Portugu\^es Instrumental}

%PREENCHER DADOS DA DISCIPLINA A SEGUIR
%\vspace{-12mm}
\begin{center}\textbf{Dados do Componente Curricular}\end{center}
\vspace{-5mm}
\noindent\rule{16.5cm}{0.4pt}
\\
\textbf{Nome:} Portugu\^es Instrumental
\\ 
\textbf{Curso:} Tecnologia em Sistemas para Internet
\\ 
\textbf{Período:} \unit{2}{\degree}
\\ 
\textbf{Carga Horária:} \unit{67}{\hour}
\\ 
\textbf{Docente Responsável:} Golbery de Oliveira Chagas Aguiar Rodrigues
\\ 
\noindent\rule{16.5cm}{0.4pt}\\
\\
%PREENCHER A EMENTA A SEGUIR
\vspace{-12mm}
\begin{center}\textbf{Ementa}\end{center}
\vspace{-5mm}
\noindent\rule{16.5cm}{0.4pt}
\\
Níveis e Estratégias de leitura; Conceitos linguísticos: Norma culta, Variedades linbguísticas, Níveis de linguagem oral e escrita; Gêneros e tipos/sequências textuais. Noções metodológicas de leitura e interpretação de textos. Habilidades básicas de produção textual. Noções linguístico-gramaticais aplicada a textos de natureza diversa, inclusive, textos técnicos e científicos. Argumentação oral e escrita, a partir de diversas situações sociocomunicativas. Elementos/Fatores da Textualidade; Aspectos semânticos, pragmáticos e sintáticos aplicados ao texto. Redação oficial; Gêneros da correspondência oficial: Aviso, Ofício e Memorando. Gêneros de natureza diversa: Artigo científico, Relatório, Requerimento, Laudo técnico, Artigo de opinião, Resumo, Resenha crítica.\\
\noindent\rule{16.5cm}{0.4pt}\\
\\
%PREENCHER OS OBJETIVOS A SEGUIR
\vspace{-12mm}
\begin{center}\textbf{Objetivos}\end{center}
\vspace{-5mm}
\noindent\rule{16.5cm}{0.4pt}
\\
\begin{itemize}	
\item Proporcionar ao aluno a aquisição de conhecimentos sobre o funcionamento da linguagem e comunicação para a estruturação e elaboração de textos diversos, considerando o perfil do egresso.;
\item Conceituar e estabelecer as diferenças que marcam a língua escrita e a falada;
\item Reconhecer os diversos registros linguísticos (formal, coloquial, informal, familiar, entre outros), com ênfase na performance formal e sua contribuição para o perfil do egresso;
\item Reconhecer os fatores que definem um texto;
\item Contribuir para o desenvolvimento de uma consciência objetiva e crítica para a compreensão e a produção de textos;
\item Desenvolver habilidades para leitura – interpretação de textos – e escrita;
\item Tornar o aluno apto a reconhecer os gêneros e tipos/sequências textuais;
\item Tornar o aluno apto a produzir textos de diversos gêneros;
\item Reconhecer a argumentatividade de gêneros diversos;
\item Produzir construções argumentativas em diversas situações sociocomunicativas;
\item Entender o contexto de produção da redação oficial;
\item Produzir gêneros da correspondência oficial;
\item Produzir com proficiência gêneros acadêmico-científicos: Artigo científico, Relatório, Resumo, Resenha crítica.
\item Produzir com proficiência o Artigo de opinião, o Laudo técnico, o requerimento.
\end{itemize} 
\noindent\rule{16.5cm}{0.4pt}\\
\\
%PREENCHER OS CONTEUDOS PROGRAMATICOS A SEGUIR (CUIDADO PARA NAO DEIXAR A TABELA MUITO GRANDE)
\vspace{-12mm}
\begin{center}\textbf{Conteúdo Programático}\end{center}
\vspace{-5mm}
\noindent\rule{16.5cm}{0.4pt}
\\
\begin{itemize}
 \item \textbf{Elementos da teoria da comunicação:} Linguagem e comunicação; Níveis da linguagem; Funções da linguagem.

 \item \textbf{Gêneros e tipos textuais:} Tipologia textual: o texto e seus formatos físicos e eletrônicos; Gêneros textuais diversos; Estrutura e Produção de gêneros diversos: Artigo de opinião, Laudo técnico, Requerimento.

 \item \textbf{Noções metodológicas de leitura e interpretação de textos:} Mecanismo de coerência e coesão textuais; Habilidades básicas de produção textual; Noções linguístico-gramaticais aplicadas a textos de natureza diversa; Elementos/fatores da textualidade; Aspectos semânticos, sintáticos aplicados ao texto.

 \item \textbf{Gêneros acadêmico-científicos:} Estrutura e produção do Artigo científico, Relatório, Resumo, Resenha crítica.

 \item \textbf{Redação oficial:} Estrutura e produção dos gêneros oficiais: Aviso, Ofício e Memorando.

\end{itemize}
\noindent\rule{16.5cm}{0.4pt}\\
\\
%COLOCAR A METODOLOGIA DE ENSINO A SEGUIR
\vspace{-12mm}
\begin{center}\textbf{Metodologia de Ensino}\end{center} 
\vspace{-5mm}
\noindent\rule{16.5cm}{0.4pt}
\\
   As aulas serão desenvolvidas por meio de metodologia participativa, com a utilização de técnicas didáticas, como: aulas expositivas, debates, seminários, trabalhos de pesquisa - individualmente e em grupos.\\
\noindent\rule{16.5cm}{0.4pt}\\
\\
%COLOCAR AVALIACAO DO PROCESSO DE ENSINO E APRENDIZAGEM A SEGUIR
\vspace{-12mm}
\begin{center}\textbf{Avaliação do Processo de Ensino e Apendizagem}\end{center}
\vspace{-5mm}
\noindent\rule{16.5cm}{0.4pt}
\\
\begin{itemize}
\item Observação geral do aluno como parte integrante e atuante do processo ensino-aprendizagem.
\item Apresentação de seminários e outras atividades discursivas;
\item Atividades escritas coletivas com o objetivo de aprofundamento do conteúdo;
\item Avaliação oral e escrita;
\item Avaliação contínua.
\end{itemize}
\noindent\rule{16.5cm}{0.4pt}\\
\\
%PREENCHER RECURSOS NECESSARIOS A SEGUIR
\vspace{-12mm}
\begin{center}\textbf{Recursos Necessários}\end{center}
\vspace{-5mm}
\noindent\rule{16.5cm}{0.4pt}
\\
\begin{itemize} 	
  \item Quadro branco;
  \item Marcadores para quadro branco;
  \item Projetor de dados multimídia;
  \item Espaços adequados para aulas extras;
  \item Mini auditório;
  \item Outros espaços circunstanciais.
\end{itemize}
\noindent\rule{16.5cm}{0.4pt}\\
\\
%PREENCHER BIBLIOGRAFIA A SEGUIR
\vspace{-12mm}
\begin{center}\textbf{Bibliografia}\end{center}
\vspace{-5mm}
\noindent\rule{16.5cm}{0.4pt}
\\
\begin{itemize} 	 
  \item Básica:
	\begin{enumerate}
	\item SAVIOLI, F. P.; FIORIN, J. L.  Para entender o texto: leitura e redação. Ática, 1990;  
	\item SAVIOLI, F. P.; FIORIN, J. L. Lições de texto: leitura e redação. São Paulo: Ática, 1996. 
	\item MARCUSCHI, L. A.; XAVIER, A. C. Hipertexto e gêneros digitais: novas formas de construção de sentido. Lucerna, 2004;
	\end{enumerate}
  \item Complementar:
	\begin{enumerate} 
	\item SAUTCHUK I. Produção dialógica do texto escrito. Martins Fontes, 2003.
	\item TERRA, E.; NICOLA, J. Práticas de linguagem \& Produção de textos. Scipione, 2001.
	\item VAL, Maria da Graça Costa. Redação e textualidade. 3ª ed. São Paulo: Martins Editora, 2006
	\item LIMA, Antônio Oliveira. Manual de redação oficial. 3ª Ed. Rio de Janeiro: Campos Editora, 2009.
	\item INFANTE, U. Do texto ao texto: curso prático de leitura e redação. Scipione, 1998; 
	\item CARNEIRO, A. D. Redação em construção: a escritura do texto. Moderna, 2001;
	\item ANDRADE, M. M.; HENRIQUES, A. Língua portuguesa: noções básicas para cursos superiores. Atlas, 2004;
	\item BASTOS, L. K. A produção escrita e a gramática. Martins Fontes, 2003;
	\item BECHARA, E. O que muda com o novo acordo ortográfico. Lucerna, 2008.
	\item COSTA, José Maria da. Manual de redação jurídica. 5ª ed. São Paulo: Migalhas, 2012.
	\end{enumerate}
\end{itemize}
\noindent\rule{16.5cm}{0.4pt}\\
\\

\clearpage
%\section{Dados da Institui\c{c}\~ao}
%Sistemas Distribuídos

\begin{table}[h!]

\centering
% definindo o tamanho da fonte para small
% outros possíveis tamanhos: footnotesize, scriptsize
\begin{small} 
  % redefinindo o espaçamento das colunas
\setlength{\tabcolsep}{3pt} 
\begin{tabular}{|p{15cm}|}\hline


\begin{center}\textbf{Dados do Componente Curricular}\end{center}\\ \hline

%PREENCHER DADOS DA DISCIPLINA A SEGUIR
\textbf{Nome:} Arquitetura de Computadores \\ \hline
\textbf{Curso:} Tecnologia em Sistemas para Internet \\ \hline
\textbf{Período:} \unit{2}{\degree} \\ \hline
\textbf{Carga Horária:} \unit{67}{\hour} \\ \hline
\textbf{Docente Responsável:} Otacílio de Araújo Ramos Neto \\ \hline


\end{tabular} 
\end{small}
\label{ementa:ArquiteturadeComputadores}
\end{table} 

\begin{table}[h!]
\centering
\begin{small} 
\setlength{\tabcolsep}{1pt} 
\begin{tabular}{|p{15cm}|}\hline

%PREENCHER A EMENTA A SEGUIR
\begin{center}\textbf{Ementa}\end{center}\\ \hline

Histórico dos computadores. Fundamentos do projeto e medidas de desempenho. Circuitos lógicos combinacionais e sequenciais. Projeto do sistema de memória. Paralelismo em nível de instrução. \\ \hline

\end{tabular} 
\end{small}
%\label{dadosinstituicao}
\end{table} 

\hspace{1cm}
\begin{table}[h!]
\centering
\begin{small} 
\setlength{\tabcolsep}{3pt} 
\begin{tabular}{|p{15cm}|}\hline

%PREENCHER OS OBJETIVOS A SEGUIR
\begin{center}\textbf{Objetivos}\end{center}\\ \hline
\begin{itemize}

\item Apresentar os eventos históricos e tecnológicos que influenciaram o desenvolvimento da tecnologia de processadores até os dias atuais;
\item Capacitar os estudantes a caracterizar os sistemas de computadores com relação ao desempenho dos mesmos;
\item Capacitar os estudantes no uso das técnicas básicas de eletrônica digital utilizadas no projeto de processadores;
\item Capacitar os estudantes a compreender o funcionamento do sistema de memória \textit{cache} e do projeto de hierarquias de memória como um todo;
\item Capacitar os estudantes a compreender o funcionamento das técnicas de paralelismo a nível de instrução.

\end{itemize}
\\ \hline

\end{tabular} 
\end{small}
%\label{dadosinstituicao}
\end{table}

\hspace{1cm}
\begin{table}[h!]
\centering
\begin{small} 
\setlength{\tabcolsep}{3pt} 
\begin{tabular}{|p{15cm}|}\hline

%PREENCHER OS CONTEUDOS PROGRAMATICOS A SEGUIR (CUIDADO PARA NAO DEIXAR A TABELA MUITO GRANDE)
\begin{center}\textbf{Conteúdo Programático}\end{center}\\ \hline
\begin{itemize}
 \item \textbf{Histórico dos computadores:} Gerações dos computadores e  Lei de Moore.
 \item \textbf{Fundamentos de projeto:} Classes de computadores; Definição da arquitetura do computador; Tendências tecnológicas, tendências na alimentação dos circuitos integrados e tendências de custo.
 \item \textbf{Medidas de desempenho:} Medição, relatório e resumo do desempenho; Princípios quantitativos do projeto; Associação entre o custo e o desempenho.
 \item \textbf{Circuitos lógicos combinacionais:} Álgebra de Boole; Portas Lógicas; Funções lógicas; Minimização de funções lógicas utilizando Álgebra de Boole; Tabelas da Verdade; Minimização de Tabelas da Verdade utilizando Mapa de Karnaugh;  Circuitos Digitais e Blocos Funcionais.
 \item \textbf{Circuitos Sequenciais:} Elementos de memória (latches e flip-flops); Registradores contadores, acumuladores, deslocadores, etc; Máquinas de estado e geradores de sequências; 
 \item \textbf{Projeto do Sistema de Memória:} Técnicas de otimizações para a memória cache; Tenologias de memória; Sistema de proteção de memória;
 \item \textbf{Paralelismo em Nível de Instrução:} Conceitos de paralelismo em nível de instrução; Uso de pipelines; Previsão de desvio; \textit{Hazards}; Técnicas de implementação do Paralelismo em Nível de Instrução. %Uma descrição mais completa pode ser necessária nesse último item. Mas, para isso, eu precisaria do livro em mãos. Como ainda não tenho vou deixar desta forma mais simples mesmo.
\end{itemize}
 \\ \hline

\end{tabular} 
\end{small}
\label{dadosinstituicao}
\end{table}


\begin{table}[h!]
\centering
\begin{small} 
\setlength{\tabcolsep}{3pt} 
\begin{tabular}{|p{15cm}|}\hline

%COLOCAR A METODOLOGIA DE ENSINO A SEGUIR
\begin{center}\textbf{Metodologia de Ensino}\end{center}\\ \hline
   Aulas expositivas utilizando recursos audiovisuais e quadro. Aulas práticas utilizando placas de desenvolvimento em FPGA.
 \\ \hline
\end{tabular} 
\end{small}
\label{dadosinstituicao}
\end{table}


\begin{table}[h!]
\centering
\begin{small} 
\setlength{\tabcolsep}{3pt} 
\begin{tabular}{|p{15cm}|}\hline

%COLOCAR AVALIACAO DO PROCESSO DE ENSINO E APRENDIZAGEM A SEGUIR
\begin{center}\textbf{Avaliação do Processo de Ensino e Apendizagem}\end{center}\\ \hline
   Avaliações escritas ao final de cada unidade. Projeto baseado em estudo de caso ou problema real.
 \\ \hline

\end{tabular} 
\end{small}
\label{dadosinstituicao}
\end{table}

\begin{table}[h!]
\centering
\begin{small} 
 
\setlength{\tabcolsep}{3pt} 
\begin{tabular}{|p{15cm}|}\hline

%PREENCHER RECURSOS NECESSARIOS A SEGUIR
\begin{center}\textbf{Recursos Necessários}\end{center}\\ \hline
\begin{itemize} 
  \item Listas de Exercícios;
  \item Livros e apostilas;
  \item Utilização de recursos da web;
  \item Quadro branco;
  \item Marcadores para quadro branco;
  \item Sala de aula com acesso à Internet, microcomputador e TV ou projetor para apresentação de slides ou material multimídia;
  \item Laboratório de Arquitetura de Computadores utilizando placas de desenvolvimento em FPGA.
\end{itemize}
 \\ \hline

\end{tabular} 
\end{small}
\label{dadosinstituicao}
\end{table}


\begin{table}[h!]
\centering
\begin{small} 
\setlength{\tabcolsep}{3pt} 
\begin{tabular}{|p{15cm}|}\hline

%PREENCHER BIBLIOGRAFIA A SEGUIR
\begin{center}\textbf{Bibliografia}\end{center}\\ \hline
\begin{itemize} 
  \item Básica;
  
  HENNESSY, John L. PATTERSON, David A. Arquitetura de Computadores - Uma Abordagem Quantitativa - 5 Ed. 2014. Elsevier.
  
  HENNESSY, John L. PATTERSON, David A. Organização e Projeto de Computadores - 4 Ed. 2014. Elsevier.
  
  \item Complementar;
  
  STALLINGS, Willian. Arquitetura e Organização de Computadores 8 Ed. Pearson.

  TANENBAUM, Andrew S. Organização Estruturada de Computadores 5 Ed. Pearson.  

\end{itemize}
 \\ \hline

\end{tabular} 
\end{small}
\label{dadosinstituicao}
\end{table}

\clearpage
\paragraph{Estruturas de Dados II}

%PREENCHER DADOS DA DISCIPLINA A SEGUIR
%\vspace{-12mm}
\begin{center}\textbf{Dados do Componente Curricular}\end{center}
\vspace{-5mm}
\noindent\rule{16.5cm}{0.4pt}
\\
\textbf{Nome:} Estruturas de Dados I
\\ 
\textbf{Curso:} Tecnologia em Sistemas para Internet
\\ 
\textbf{Período:} \unit{2}{\degree}
\\ 
\textbf{Carga Horária:} \unit{67}{\hour}
\\ 
\textbf{Docente Responsável:} Otacílio de Araújo Ramos Neto
\\ 
\noindent\rule{16.5cm}{0.4pt}\\
\\
%PREENCHER A EMENTA A SEGUIR
\vspace{-12mm}
\begin{center}\textbf{Ementa}\end{center}
\vspace{-5mm}
\noindent\rule{16.5cm}{0.4pt}
\\
Conceitos básicos, crescimento de funções e recorrências; Algoritmos de ordenação e busca; Estruturas de dados elementares; Árvores de busca binária.\\ 
\noindent\rule{16.5cm}{0.4pt}\\
\\
%PREENCHER OS OBJETIVOS A SEGUIR
\vspace{-12mm}
\begin{center}\textbf{Objetivos}\end{center}
\vspace{-5mm}
\noindent\rule{16.5cm}{0.4pt}
\\
\begin{itemize}
\item Apresentar os conceitos básicos para criação e análise de algoritmos;
\item Apresentar os algoritmos básicos de ordenação e busca;
\item Capacitar os alunos a utilizarem as estruturas de dados elementares em problemas reais;
\item Apresentar aos alunos as árvores de busca binária e capacitá-los no seu uso.
\end{itemize} 
\noindent\rule{16.5cm}{0.4pt}\\
\\
%PREENCHER OS CONTEUDOS PROGRAMATICOS A SEGUIR (CUIDADO PARA NAO DEIXAR A TABELA MUITO GRANDE)
\vspace{-12mm}
\begin{center}\textbf{Conteúdo Programático}\end{center}
\vspace{-5mm}
\noindent\rule{16.5cm}{0.4pt}
\\
\begin{itemize}
 \item \textbf{Conceitos básicos:} Análise e projeto de algoritmos; Notação assintótica; O Método da Substituição, Método da Árvore de Recursão e Método Mestre.
 \item \textbf{Algoritmos de ordenação e busca:} Ordenação por inserção, Heapsort, Quicksort e ordenação em tempo linear; Busca sequencial e busca binária.
 \item \textbf{Estruturas de dados elementares:} Implementações de ponteiros e objetos; Pilhas, filas e listas ligadas.
 \item \textbf{Árvores de pesquisa binária:} Conceitos fundamentais de árvores de pesquisa binária; Algoritmos de inserção, remoção e busca; Impressão \textit{In-Order}, \textit{Post-Order} e \textit{Pre-Order} .
\end{itemize}
\noindent\rule{16.5cm}{0.4pt}\\
\\
%COLOCAR A METODOLOGIA DE ENSINO A SEGUIR
\vspace{-12mm}
\begin{center}\textbf{Metodologia de Ensino}\end{center} 
\vspace{-5mm}
\noindent\rule{16.5cm}{0.4pt}
\\
   Aulas expositivas utilizando recursos audiovisuais e quadro, além de aulas práticas utilizando computadores. Adicionalmente, serão realizadas atividades práticas individuais ou em grupo, para consolidação do conteúdo ministrado.\\
\noindent\rule{16.5cm}{0.4pt}\\
\\
%COLOCAR AVALIACAO DO PROCESSO DE ENSINO E APRENDIZAGEM A SEGUIR
\vspace{-12mm}
\begin{center}\textbf{Avaliação do Processo de Ensino e Apendizagem}\end{center}
\vspace{-5mm}
\noindent\rule{16.5cm}{0.4pt}
\\
   Avaliações escritas. Avalia\c{c}\~oes pr\'aticas envolvendo a resolu\c{c}\~ao de problemas computacionais.\\
\noindent\rule{16.5cm}{0.4pt}\\
\\
%PREENCHER RECURSOS NECESSARIOS A SEGUIR
\vspace{-12mm}
\begin{center}\textbf{Recursos Necessários}\end{center}
\vspace{-5mm}
\noindent\rule{16.5cm}{0.4pt}
\\
\begin{itemize} 
  \item Listas de Exercícios;
  \item Livros e apostilas;
  \item Utilização de recursos da web;
  \item Quadro branco;
  \item Marcadores para quadro branco;
  \item Sala de aula com acesso à internet, microcomputador e TV ou projetor para apresentação de slides ou material multimídia;
  \item Laboratório de microcomputadores contendo componentes de hardware e software específicos;
\end{itemize}
\noindent\rule{16.5cm}{0.4pt}\\ - 
\\
%PREENCHER BIBLIOGRAFIA A SEGUIR
\vspace{-12mm}
\begin{center}\textbf{Bibliografia}\end{center}
\vspace{-5mm}
\noindent\rule{16.5cm}{0.4pt}
\\
\begin{itemize} 
  \item Básica:
	\begin{enumerate}
	\item T.H. Cormen, C.E. Leiserson, R.L. Rivest, C. Stein, "Algoritmos - Teoria e Prática", 3a. ed., ISBN: 8535236996, Editora Campus, 2012.
	\end{enumerate}
  \item Complementar:
	\begin{enumerate} 
	\item Steven S Skiena, The Algorithm Design Manual, Springer; 2nd edition, ISBN: 978-1849967204, 2008.\\
	\end{enumerate}
\end{itemize}
\noindent\rule{16.5cm}{0.4pt}\\
\\

\clearpage

%\section{Dados da Institui\c{c}\~ao}
%Programação Orientada a Objetos
\begin{table}[h!]

\centering
% definindo o tamanho da fonte para small
% outros possíveis tamanhos: footnotesize, scriptsize
\begin{small} 
  % redefinindo o espaçamento das colunas
\setlength{\tabcolsep}{3pt} 
\begin{tabular}{|p{15cm}|}\hline


\begin{center}\textbf{Dados do Componente Curricular}\end{center}\\ \hline

%PREENCHER DADOS DA DISCIPLINA A SEGUIR
\textbf{Nome:} Programação Orientada a Objetos \\ \hline
\textbf{Curso:} Tecnologia em Sistemas para Internet \\ \hline
\textbf{Período:} $2^{\circ}$ \\ \hline
\textbf{Carga Horária:} 83~h \\ \hline
\textbf{Docente Responsável:} José de Sousa Barros \\ \hline


\end{tabular} 
\end{small}
\label{dadosinstituicao}
\end{table} 

\begin{table}[h!]
\centering
\begin{small} 
\setlength{\tabcolsep}{1pt} 
\begin{tabular}{|p{15cm}|}\hline

%PREENCHER A EMENTA A SEGUIR
\begin{center}\textbf{Ementa}\end{center}\\ \hline
O paradigma de programação orientada a objetos: abstração, conceito de classes e objetos, troca de mensagens entre objetos, composição de objetos, encapsulamento, empacotamento de classes, visibilidade, coleções de objetos, herança, sobrescrita, sobrecarga, interface e polimorfismo, tratamento de exceções, persistência de dados em arquivos. \\ \hline

\end{tabular} 
\end{small}
\label{dadosinstituicao}
\end{table} 

\hspace{1cm}
\begin{table}[h!]
\centering
\begin{small} 
\setlength{\tabcolsep}{3pt} 
\begin{tabular}{|p{15cm}|}\hline

%PREENCHER OS OBJETIVOS A SEGUIR
\begin{center}\textbf{Objetivos}\end{center}\\ \hline
\begin{itemize}
\item Identificar os conceitos do paradigma de programação orientado a objetos;
\item Utilizar os conceitos do paradigma de programação orientado a objetos;
\item Desenvolver aplicações em uma linguagem de programação Orientada a Objetos.
\end{itemize}
 \\ \hline

\end{tabular} 
\end{small}
\label{dadosinstituicao}
\end{table}

\hspace{1cm}
\begin{table}[h!]
\centering
\begin{small} 
\setlength{\tabcolsep}{3pt} 
\begin{tabular}{|p{15cm}|}\hline

%PREENCHER OS CONTEUDOS PROGRAMATICOS A SEGUIR (CUIDADO PARA NAO DEIXAR A TABELA MUITO GRANDE)
\begin{center}\textbf{Conteúdo Programático}\end{center}\\ \hline
\begin{itemize}
 \item \textbf{Introdução à Programação Orientada a Objetos:} Abstração; Modelagem orientada a objetos; Apresentação de uma linguagem de programação orientada a objetos;	Classes; Objetos; Construtores; Métodos; Encapsulamento e visibilidade, Pacotes.


 \item \textbf{Herança e Polimorfismo:} Membros de classe: atributos e métodos (de classe e de instância); Herança;	Classes abstratas; Métodos abstratos; Sobrescrita de métodos; Sobrecarga de métodos; Interfaces;	Polimorfismo; Coleções estáticas.

 \item \textbf{Coleções dinâmicas e Tratamento de exceções:} Generics; Coleções dinâmicas: Collection, List, Queue, Deque, Set e SortedSet; Tratamento de exceções; Interface gráfica; Manipulação de eventos; Persistência de dados em arquivos.
\end{itemize}
 \\ \hline

\end{tabular} 
\end{small}
\label{dadosinstituicao}
\end{table}


\begin{table}[h!]
\centering
\begin{small} 
\setlength{\tabcolsep}{3pt} 
\begin{tabular}{|p{15cm}|}\hline

%COLOCAR A METODOLOGIA DE ENSINO A SEGUIR
\begin{center}\textbf{Metodologia de Ensino}\end{center}\\ \hline
   Aulas expositivas utilizando recursos audiovisuais e quadro, além de aulas práticas utilizando computadores. Adicionalmente, serão realizadas atividades práticas individuais ou em grupo, para consolidação do conteúdo ministrado.
 \\ \hline
\end{tabular} 
\end{small}
\label{dadosinstituicao}
\end{table}


\begin{table}[h!]
\centering
\begin{small} 
\setlength{\tabcolsep}{3pt} 
\begin{tabular}{|p{15cm}|}\hline

%COLOCAR AVALIACAO DO PROCESSO DE ENSINO E APRENDIZAGEM A SEGUIR
\begin{center}\textbf{Avaliação do Processo de Ensino e Apendizagem}\end{center}\\ \hline
   Avaliações escritas ao final de cada unidade. Prática baseada em Estudo de Caso ou problema real.
 \\ \hline

\end{tabular} 
\end{small}
\label{dadosinstituicao}
\end{table}

\begin{table}[h!]
\centering
\begin{small} 
 
\setlength{\tabcolsep}{3pt} 
\begin{tabular}{|p{15cm}|}\hline

%PREENCHER RECURSOS NECESSARIOS A SEGUIR
\begin{center}\textbf{Recursos Necessários}\end{center}\\ \hline
\begin{itemize} 
  \item Listas de Exercícios;
  \item Livros e apostilas;
  \item Utilização de recursos da web;
  \item Quadro branco;
  \item Marcadores para quadro branco;
  \item Sala de aula com acesso à internet, microcomputador e TV ou projetor para apresentação de slides ou material multimídia;
  \item Laboratório de microcomputadores contendo componentes de hardware e software específicos;
\end{itemize}
 \\ \hline

\end{tabular} 
\end{small}
\label{dadosinstituicao}
\end{table}


\begin{table}[h!]
\centering
\begin{small} 
\setlength{\tabcolsep}{3pt} 
\begin{tabular}{|p{15cm}|}\hline

%PREENCHER BIBLIOGRAFIA A SEGUIR
\begin{center}\textbf{Bibliografia}\end{center}\\ \hline
\begin{itemize} 
  \item Básica;
  	\newline DEITEL, H. M.; DEITEL, P. J. \textbf{Java: Como Programar.} Pearson, 8ª Edição, 2010;
	\newline FURGERI, S. \textbf{Java 7 Ensino Didático.} Érica, 1ª Edição, 2010;
	\newline SIERRA K.; BATES, B. \textbf{Use a Cabeça! - Java.} Alta Books, 2ª Edição, 2007.
  \item Complementar;
  	\newline HORSTMANN, C. S. \& CORNELL, G. \textbf{Core Java, Volume 1.} Pearson, 8ª edição, 2010;
	\newline CADENHEAD, R.; LEMAY, L. \textbf{Aprenda Java em 21 Dias.} Campus, 4ª edição, 2005.
  
\end{itemize}
 \\ \hline

\end{tabular} 
\end{small}
\label{dadosinstituicao}
\end{table}


\clearpage
\paragraph{Estruturas de Dados II}

%PREENCHER DADOS DA DISCIPLINA A SEGUIR
%\vspace{-12mm}
\begin{center}\textbf{Dados do Componente Curricular}\end{center}
\vspace{-5mm}
\noindent\rule{16.5cm}{0.4pt}
\\
\textbf{Nome:} Estruturas de Dados II
\\ 
\textbf{Curso:} Tecnologia em Sistemas para Internet
\\ 
\textbf{Período:} $3^{\circ}$
\\ 
\textbf{Carga Horária:} 67~h
\\ 
\textbf{Docente Responsável:} Ruan Delgado Gomes
\\ 
\noindent\rule{16.5cm}{0.4pt}\\
\\
%PREENCHER A EMENTA A SEGUIR
\vspace{-12mm}
\begin{center}\textbf{Ementa}\end{center}
\vspace{-5mm}
\noindent\rule{16.5cm}{0.4pt}
\\
\'Arvore bin\'aria de busca; Tabelas Hash; \'Arvores vermelho e preto; Grafos; Algoritmos de busca em grafos; \'Arvore geradora m\'inima; Algoritmos de menor caminho em grafos.\\ 
\noindent\rule{16.5cm}{0.4pt}\\
\\
%PREENCHER OS OBJETIVOS A SEGUIR
\vspace{-12mm}
\begin{center}\textbf{Objetivos}\end{center}
\vspace{-5mm}
\noindent\rule{16.5cm}{0.4pt}
\\
\begin{itemize}
\item Apresentar estruturas de dados e algoritmos amplamente utilizados e discutir sua implementação e seu desempenho;
\item Aprender a utilizar estruturas de dados n\~ao lineares e algoritmos que manipulam essas estruturas de dados para resolu\c{c}\~ao de problemas computacionais.
\end{itemize} 
\noindent\rule{16.5cm}{0.4pt}\\
\\
%PREENCHER OS CONTEUDOS PROGRAMATICOS A SEGUIR (CUIDADO PARA NAO DEIXAR A TABELA MUITO GRANDE)
\vspace{-12mm}
\begin{center}\textbf{Conteúdo Programático}\end{center}
\vspace{-5mm}
\noindent\rule{16.5cm}{0.4pt}
\\
\begin{itemize}
 \item \textbf{\'Arvore bin\'aria de busca:} Defini\c{c}\~ao; Algoritmos de inser\c{c}\~ao, remo\c{c}\~ao e busca; Balanceamento; An\'alise de complexidade dos algoritmos para manipula\c{c}\~ao de \'arvores bin\'arias de busca.

% \item \textbf{Estilos Arquiteturais para SD:} Camadas; Baseada em Objetos; Baseada em Dados; Baseada em Eventos.

 \item \textbf{Tabelas Hash:} Tabelas de endere\c{c}amento direto; Tabelas Hash; Fun\c{c}\~oes Hash; Endere\c{c}amento aberto; Hashing perfeito.

 \item \textbf{\'Arvores vermelho e preto:} Propriedades; Algoritmos de rota\c{c}\~ao, inser\c{c}\~ao e remo\c{c}\~ao; An\'alise de complexaide de algoritmos para manipula\c{c}\~ao de \'arvores vermelho e preto.

 \item \textbf{Grafos:} Representa\c{c}\~ao de grafos; Modelagem de problemas utilizando grafos.

 \item \textbf{Algoritmos de busca em grafos:} Busca em largura (BFS); Busca em profundidade (DFS).

 \item \textbf{\'Arvore geradora m\'inima:} Crescimento de \'arvore geradora m\'inima; Algoritmos de Kruskal e Prim.

 \item \textbf{Algoritmos de menor caminho em grafos:} Algoritmo de Bellman-Ford; Menor caminho de \'unica fonte em grafos ac\'iclicos direcionados; Algoritmo de Dijkstra; 

\end{itemize}
\noindent\rule{16.5cm}{0.4pt}\\
\\
%COLOCAR A METODOLOGIA DE ENSINO A SEGUIR
\vspace{-12mm}
\begin{center}\textbf{Metodologia de Ensino}\end{center} 
\vspace{-5mm}
\noindent\rule{16.5cm}{0.4pt}
\\
   Aulas expositivas utilizando recursos audiovisuais e quadro, além de aulas práticas utilizando computadores. Adicionalmente, serão realizadas atividades práticas individuais ou em grupo, para consolidação do conteúdo ministrado.\\
\noindent\rule{16.5cm}{0.4pt}\\
\\
%COLOCAR AVALIACAO DO PROCESSO DE ENSINO E APRENDIZAGEM A SEGUIR
\vspace{-12mm}
\begin{center}\textbf{Avaliação do Processo de Ensino e Apendizagem}\end{center}
\vspace{-5mm}
\noindent\rule{16.5cm}{0.4pt}
\\
   Avaliações escritas. Avalia\c{c}\~oes pr\'aticas envolvendo a resolu\c{c}\~ao de problemas computacionais.\\
\noindent\rule{16.5cm}{0.4pt}\\
\\
%PREENCHER RECURSOS NECESSARIOS A SEGUIR
\vspace{-12mm}
\begin{center}\textbf{Recursos Necessários}\end{center}
\vspace{-5mm}
\noindent\rule{16.5cm}{0.4pt}
\\
\begin{itemize} 
  \item Listas de Exercícios;
  \item Livros e apostilas;
  \item Utilização de recursos da web;
  \item Quadro branco;
  \item Marcadores para quadro branco;
  \item Sala de aula com acesso à internet, microcomputador e TV ou projetor para apresentação de slides ou material multimídia;
  \item Laboratório de microcomputadores contendo componentes de hardware e software específicos;
\end{itemize}
\noindent\rule{16.5cm}{0.4pt}\\ - 
\\
%PREENCHER BIBLIOGRAFIA A SEGUIR
\vspace{-12mm}
\begin{center}\textbf{Bibliografia}\end{center}
\vspace{-5mm}
\noindent\rule{16.5cm}{0.4pt}
\\
\begin{itemize} 
  \item Básica:
	\begin{enumerate}
	\item T.H. Cormen, C.E. Leiserson, R.L. Rivest, C. Stein, "Algoritmos - Teoria e Pr\'atica", 3a. ed., ISBN: 8535236996, Editora Campus, 2012.
	\end{enumerate}
  \item Complementar:
	\begin{enumerate} 
	\item Steven S Skiena, The Algorithm Design Manual, Springer; 2nd edition, ISBN: 978-1849967204, 2008.\\
	\end{enumerate}
\end{itemize}
\noindent\rule{16.5cm}{0.4pt}\\
\\

\clearpage
\paragraph{Bancos de Dados I}

%PREENCHER DADOS DA DISCIPLINA A SEGUIR
%\vspace{-12mm}
\begin{center}\textbf{Dados do Componente Curricular}\end{center}
\vspace{-5mm}
\noindent\rule{16.5cm}{0.4pt}
\\
\textbf{Nome:} Bancos de Dados I
\\
\textbf{Curso:} Tecnologia em Sistemas para Internet
\\ 
\textbf{Período:} $3^{\circ}$ 
\\
\textbf{Carga Horária:} 67~h 
\\ 
\textbf{Docente Responsável:} José de Sousa Barros 
\\ 
\noindent\rule{16.5cm}{0.4pt}\\
\\
%PREENCHER A EMENTA A SEGUIR
\vspace{-12mm}
\begin{center}\textbf{Ementa}\end{center}
\vspace{-5mm}
\noindent\rule{16.5cm}{0.4pt}
\\ 
Introdução a bancos de dados. Conceitos básicos e terminologias de bancos de dados. Sistemas de Gerenciamento de Bancos de Dados. Modelos e esquemas de dados. Modelo entidade-relacionamento. O modelo relacional. Álgebra relacional. Linguagem de consulta estruturada (SQL). Projeto de bancos de dados relacional: normalização, restrições, índices, chaves primária e estrangeira. Visões. Subprogramas armazenados e gatilhos. Controle transacional. 
 \\

\noindent\rule{16.5cm}{0.4pt}\\
\\
%PREENCHER OS OBJETIVOS A SEGUIR
\vspace{-12mm}
\begin{center}\textbf{Objetivos}\end{center}
\vspace{-5mm}
\noindent\rule{16.5cm}{0.4pt}
\\
\begin{itemize}
\item Compreender os conceitos fundamentais de banco de dados;
\item Construir modelos conceituais de banco de dados usando o modelo de entidade-relacionamento;
\item Desenvolver modelos lógicos relacionais baseados em modelos conceituais;
\item Saber utilizar a linguagem SQL para recuperar e manipular informações em um banco de dados relacional.
\end{itemize}
\noindent\rule{16.5cm}{0.4pt}\\
\\
%PREENCHER OS CONTEUDOS PROGRAMATICOS A SEGUIR (CUIDADO PARA NAO DEIXAR A TABELA MUITO GRANDE)
\vspace{-12mm}
\begin{center}\textbf{Conteúdo Programático}\end{center}
\vspace{-5mm}
\noindent\rule{16.5cm}{0.4pt}
\\
\begin{itemize}
 \item \textbf{Conceitos Básicos de Banco de Dados:} Dados e Informação; Banco de Dados; Sistemas Gerenciadores de Bancos de Dados; Tipos de usuários.
 
 \item \textbf{Modelagem Conceitual:} Modelo de Entidade-Relacionamento: Entidades, Atributos, Relacionamentos; Modelo de Entidade-Relacionamento Estendido: Especialização e Generalização. 

 \item \textbf{Modelo Relacional:} Conceitos do Modelo Relacional; Operações com Relações; Álgebra Relacional: Operação Seleção e Projeção, União, Interseção, Diferença, Produto Cartesiano, Junção, Divisão, Projeto de Banco de Dados Relacional: Mapeamento do modelo entidade-relacionamento para o modelo relacional, Regras e Normalização.
 
 \item \textbf{Linguagem SQL:} Introdução à Linguagem SQL; Utilização das instruções das seguintes sub-linguagens: Linguagem de Definição de Dados (DDL), Linguagem de Manipulação de Dados (DML), Linguagem de Consulta de Dados (DQL), Linguagem de Controle de Dados (DCL), Linguagem de Transação de Dados (DTL); Visões, Subprogramas Armazenados e Gatilhos.

 
\end{itemize}
\noindent\rule{16.5cm}{0.4pt}\\
\\
%COLOCAR A METODOLOGIA DE ENSINO A SEGUIR
\vspace{-12mm}
\begin{center}\textbf{Metodologia de Ensino}\end{center} 
\vspace{-5mm}
\noindent\rule{16.5cm}{0.4pt}
\\
   Aulas expositivas utilizando recursos audiovisuais e quadro, além de aulas práticas utilizando computadores. Adicionalmente, serão realizadas atividades práticas individuais ou em grupo, para consolidação do conteúdo ministrado.\\
\noindent\rule{16.5cm}{0.4pt}\\
\\
%COLOCAR AVALIACAO DO PROCESSO DE ENSINO E APRENDIZAGEM A SEGUIR
\vspace{-12mm}
\begin{center}\textbf{Avaliação do Processo de Ensino e Apendizagem}\end{center}
\vspace{-5mm}
\noindent\rule{16.5cm}{0.4pt}
\\
   Avaliações escritas. Práticas baseadas em Estudos de Caso ou problemas reais.\\
\noindent\rule{16.5cm}{0.4pt}\\
\\
%PREENCHER RECURSOS NECESSARIOS A SEGUIR
\vspace{-12mm}
\begin{center}\textbf{Recursos Necessários}\end{center}
\vspace{-5mm}
\noindent\rule{16.5cm}{0.4pt}
\\
\begin{itemize} 
  \item Listas de Exercícios;
  \item Livros e apostilas;
  \item Utilização de recursos da web;
  \item Quadro branco;
  \item Marcadores para quadro branco;
  \item Sala de aula com acesso à internet, microcomputador e TV ou projetor para apresentação de slides ou material multimídia;
  \item Laboratório de microcomputadores contendo componentes de hardware e \textit{software} específicos;
\end{itemize}
\noindent\rule{16.5cm}{0.4pt}\\
\\
%PREENCHER BIBLIOGRAFIA A SEGUIR
\vspace{-12mm}
\begin{center}\textbf{Bibliografia}\end{center}
\vspace{-5mm}
\noindent\rule{16.5cm}{0.4pt}
\\
\begin{itemize} 
  \item Básica:
	\begin{enumerate}
  	\item 	ELMASRI, R.; NAVATHE, S. \textbf{Sistemas de banco de dados.} Pearson, 6ª edição, 2011;
	\item 	KORTH, H.; SILBERSCHATZ, A.; SUDARSHAN, S. \textbf{Sistemas de bancos de dados.} Campus, 5ª edição, 2006;
	\item 	DATE, C. J. \textbf{Introdução a sistemas de bancos de dados.} Campus, Tradução da 8ª edição Americana, 2004.      
	\end{enumerate}
    
  \item Complementar:
	\begin{enumerate}
  	\item 	HEUSER, C. \textbf{Projeto de Banco de Dados – Série UFRGS, Nº 4.} Sagra-Luzzatto, 5ª edição, 2004;
	\item 	GARCIA-MOLINA, H. \textbf{Implementação de Sistemas de Banco de Dados.} Campus, 1ª edição, 2010;
    \item 	RAMAKRISHNAN, R. \textbf{Sistemas de Gerenciamento de Banco de Dados.} McGraw Hill, 3ª edição, 2010.
	\end{enumerate}
\end{itemize}
\noindent\rule{16.5cm}{0.4pt}\\
\\
\clearpage
\paragraph{Sistemas Operacionais}

%PREENCHER DADOS DA DISCIPLINA A SEGUIR
%\vspace{-12mm}
\begin{center}\textbf{Dados do Componente Curricular}\end{center}
\vspace{-5mm}
\noindent\rule{16.5cm}{0.4pt}
\\
\textbf{Nome:} Sistemas Operacionais
\\ 
\textbf{Curso:} Tecnologia em Sistemas para Internet
\\ 
\textbf{Período:} $3^{\circ}$
\\ 
\textbf{Carga Horária:} 67~h
\\ 
\textbf{Docente Responsável:} Rodrigo Pinheiro Marques de Ara\'ujo
\\ 
\noindent\rule{16.5cm}{0.4pt}\\
\\
%PREENCHER A EMENTA A SEGUIR
\vspace{-12mm}
\begin{center}\textbf{Ementa}\end{center}
\vspace{-5mm}
\noindent\rule{16.5cm}{0.4pt}
\\
Conceitos básicos de sistemas operacionais; Gerência de processador; Processos e \textit{Threads}; Comunicação entre processos; Gerência de memória; Gerência de entrada/saída; Sistemas de arquivos; Segurança em sistemas operacionais; Estudo de casos.\\ 
\noindent\rule{16.5cm}{0.4pt}\\
\\
%PREENCHER OS OBJETIVOS A SEGUIR
\vspace{-12mm}
\begin{center}\textbf{Objetivos}\end{center}
\vspace{-5mm}
\noindent\rule{16.5cm}{0.4pt}
\\
\begin{itemize}
\item Entender o papel do sistema operacional dentro de um sistema computacional;
\item Entender o funcionamento dos vários módulos que compõem um sistema operacional;
\item Desenvolver uma visão crítica sobre os requisitos de confiabilidade, segurança e desempenho, associados a um sistema operacional;
\item Compreender os mecanismos básicos de: chamada ao sistema, tratamento de interrupções, bloqueio e escalonamento de processos;
\item Compreender as principais estruturas de dados de um sistema operacional;
\item Compreender os principais algoritmos utilizados para gerir a utilização dos recursos do sistema;
\item Compreender as necessidades e os mecanismos utilizados pelo sistema operacional para prover segurança para o sistema computacional.

\end{itemize} 
\noindent\rule{16.5cm}{0.4pt}\\
\\
%PREENCHER OS CONTEUDOS PROGRAMATICOS A SEGUIR (CUIDADO PARA NAO DEIXAR A TABELA MUITO GRANDE)
\vspace{-12mm}
\begin{center}\textbf{Conteúdo Programático}\end{center}
\vspace{-5mm}
\noindent\rule{16.5cm}{0.4pt}
\\
\begin{itemize}

 \item \textbf{Introdução aos Sistemas Operacionais:} Funções de um sistema operacional; Conceitos básicos.

 \item \textbf{Processos e \textit{Threads}:} Definição e estrutura de processos; Estados de um processo; Escalonamento de processos; Fluxo de execução de um processo; \textit{Multithreading}; Comunicação entre processos; Escalonamento para processadores \textit{multi-core}. Impasses; Definição de impasses; Técnicas para o tratamento de impasses.

 \item \textbf{Ger\^encia de mem\'oria:} Gerência de memória sem \textit{swap} ou paginação; \textit{Swapping}; Memória virtual; Algoritmos de reposição de páginas; Segmentação.

 \item \textbf{Entrada/Saída:} \textit{Hardware} e \textit{software} de entrada/saída; Projeto e implementação de \textit{drivers} de dispositivos.

 \item \textbf{Sistemas de Arquivos:} Arquivos e diretórios; Implementação de sistemas de arquivos; Segurança e mecanismos de proteção da informação.

\end{itemize}
\noindent\rule{16.5cm}{0.4pt}\\
\\
%COLOCAR A METODOLOGIA DE ENSINO A SEGUIR
\vspace{-12mm}
\begin{center}\textbf{Metodologia de Ensino}\end{center} 
\vspace{-5mm}
\noindent\rule{16.5cm}{0.4pt}
\\
   Aulas expositivas utilizando recursos audiovisuais e quadro, além de aulas práticas utilizando computadores. Adicionalmente, serão realizadas atividades práticas individuais ou em grupo, para consolidação do conteúdo ministrado.\\
\noindent\rule{16.5cm}{0.4pt}\\
\\
%COLOCAR AVALIACAO DO PROCESSO DE ENSINO E APRENDIZAGEM A SEGUIR
\vspace{-12mm}
\begin{center}\textbf{Avaliação do Processo de Ensino e Apendizagem}\end{center}
\vspace{-5mm}
\noindent\rule{16.5cm}{0.4pt}
\\
   Avaliações escritas e pr\'aticas.\\
\noindent\rule{16.5cm}{0.4pt}\\
\\
%PREENCHER RECURSOS NECESSARIOS A SEGUIR
\vspace{-12mm}
\begin{center}\textbf{Recursos Necessários}\end{center}
\vspace{-5mm}
\noindent\rule{16.5cm}{0.4pt}
\\
\begin{itemize} 
  \item Listas de Exercícios;
  \item Livros e apostilas;
  \item Utilização de recursos da web;
  \item Quadro branco;
  \item Marcadores para quadro branco;
  \item Sala de aula com acesso à internet, microcomputador e TV ou projetor para apresentação de slides ou material multimídia;
  \item Laboratório de microcomputadores contendo componentes de hardware e software específicos;
\end{itemize}
\noindent\rule{16.5cm}{0.4pt}\\
\\
%PREENCHER BIBLIOGRAFIA A SEGUIR
\vspace{-12mm}
\begin{center}\textbf{Bibliografia}\end{center}
\vspace{-5mm}
\noindent\rule{16.5cm}{0.4pt}
\\
\begin{itemize} 
  \item Básica:
	\begin{enumerate}
	\item Tanenbaum, A. S. Sistemas Operacionais Modernos. ISBN: 9788576052371. Editora Pearson. 3 Ed., 2010. 
	\item Silberschatz, A.; et al. ISBN: 9788521617471. Fundamentos de Sistemas Operacionais. Editora LTC, 8 Ed., 2010; 
	\end{enumerate}
  \item Complementar:
	\begin{enumerate} 
	\item Marshall Kirk McKusick, George V. Neville-Neil, Robert N.M. Watson. The Design and Implementation of the FreeBSD Operating System. ISBN: 978-0321968975. Editora Addison-Wesley. 2 Ed., 2014.
	\item Mark Russinovich, David Solomon, Alex Ionescu. Windows Internals, Part 1. Microsoft Press. ISBN: 978-0735648739. 6 Ed., 2012.
	\item Robert Love. Linux Kernel Development. ISBN: 978-0672329463. Editora Addison-Wesley. 3 Ed., 2010.
	\end{enumerate}
\end{itemize}
\noindent\rule{16.5cm}{0.4pt}\\
\\

\clearpage
\paragraph{Bancos de Dados II}

%PREENCHER DADOS DA DISCIPLINA A SEGUIR
%\vspace{-12mm}
\begin{center}\textbf{Dados do Componente Curricular}\end{center}
\vspace{-5mm}
\noindent\rule{16.5cm}{0.4pt}
\\
\textbf{Nome:} Bancos de Dados II
\\
\textbf{Curso:} Tecnologia em Sistemas para Internet
\\ 
\textbf{Período:} $4^{\circ}$ 
\\
\textbf{Carga Horária:} 67~h 
\\ 
\textbf{Docente Responsável:} José de Sousa Barros 
\\ 
\noindent\rule{16.5cm}{0.4pt}\\
\\
%PREENCHER A EMENTA A SEGUIR
\vspace{-12mm}
\begin{center}\textbf{Ementa}\end{center}
\vspace{-5mm}
\noindent\rule{16.5cm}{0.4pt}
\\
Bancos de dados orientados a objeto: ODMG, ODL e OQL. Bancos de dados objeto-relacional. Projeto de bancos de dados objeto-relacional: modelos conceitual e lógico. Consultas em bancos de dados objeto-relacional. Banco de dados geográficos. Bancos de dados distribuídos. Novas aplicações de bancos de dados.\\

\noindent\rule{16.5cm}{0.4pt}\\
\\
%PREENCHER OS OBJETIVOS A SEGUIR
\vspace{-12mm}
\begin{center}\textbf{Objetivos}\end{center}
\vspace{-5mm}
\noindent\rule{16.5cm}{0.4pt}
\\
\begin{itemize}
\item Compreender os conceitos fundamentais dos bancos de dados orientado a objetos, objeto-relacional, geográfico e distribuído;
\item Diferenciar os bancos de dados orientado a objetos, objeto-relacional, geográfico e distribuído;
\item Realizar a integração entre aplicações e os bancos de dados orientado a objetos, objeto-relacional, geográfico e distribuído.


\end{itemize}
\noindent\rule{16.5cm}{0.4pt}\\
\\
%PREENCHER OS CONTEUDOS PROGRAMATICOS A SEGUIR (CUIDADO PARA NAO DEIXAR A TABELA MUITO GRANDE)
\vspace{-12mm}
\begin{center}\textbf{Conteúdo Programático}\end{center}
\vspace{-5mm}
\noindent\rule{16.5cm}{0.4pt}
\\
\begin{itemize}
 \item \textbf{Banco de Dados Geográficos:} Conceitos básicos;	Representação de dados (Open Geospatial Consortium); PostgreSQL com PostGIS; Importação de dados espaciais; Consultas espaciais; Java Topology Suite (JTS); Representação de mapas em SVG.
 
 \item \textbf{Banco de Dados Orientados a Objetos:} Conceitos básicos; O padrão ODMG; ODL: Estrutura de classes, Construtores, Identidade de Objetos, Coleções estáticas e dinâmicas, Nomeação e alcançabilidade; OQL: Consultas, Subconsultas, Expressões de caminho.

 \item \textbf{Banco de Dados Objeto-Relacional:} Conceitos básicos; Tipos Complexos; Construtores;	Métodos; Coleções estáticas e dinâmicas; Tabelas de objetos; Tabelas aninhadas; Referências para Tipos Complexos; Herança; Consultas com tipos complexos.
 
 \item \textbf{Bancos de Dados Distribuídos:} Bancos de Dados Centralizados x Distribuídos; Tipos de Banco de Dados Distribuído; Projeto de Banco de Dados Distribuído; Processamento de Consultas.

\item \textbf{Banco de Dados NoSQL:} Conceitos básicos; Modelo de Dados: Chave-valor, Orientado a Colunas, Orientado a Documentos e Orientado a Grafos.
 
\end{itemize}
\noindent\rule{16.5cm}{0.4pt}\\
\\
%COLOCAR A METODOLOGIA DE ENSINO A SEGUIR
\vspace{-12mm}
\begin{center}\textbf{Metodologia de Ensino}\end{center} 
\vspace{-5mm}
\noindent\rule{16.5cm}{0.4pt}
\\
   Aulas expositivas utilizando recursos audiovisuais e quadro, além de aulas práticas utilizando computadores. Adicionalmente, serão realizadas atividades práticas individuais ou em grupo, para consolidação do conteúdo ministrado.\\
\noindent\rule{16.5cm}{0.4pt}\\
\\
%COLOCAR AVALIACAO DO PROCESSO DE ENSINO E APRENDIZAGEM A SEGUIR
\vspace{-12mm}
\begin{center}\textbf{Avaliação do Processo de Ensino e Apendizagem}\end{center}
\vspace{-5mm}
\noindent\rule{16.5cm}{0.4pt}
\\
   Avaliações escritas. Práticas baseadas em Estudos de Caso ou problemas reais.\\
\noindent\rule{16.5cm}{0.4pt}\\
\\
%PREENCHER RECURSOS NECESSARIOS A SEGUIR
\vspace{-12mm}
\begin{center}\textbf{Recursos Necessários}\end{center}
\vspace{-5mm}
\noindent\rule{16.5cm}{0.4pt}
\\
\begin{itemize} 
  \item Listas de Exercícios;
  \item Livros e apostilas;
  \item Utilização de recursos da web;
  \item Quadro branco;
  \item Marcadores para quadro branco;
  \item Sala de aula com acesso à internet, microcomputador e TV ou projetor para apresentação de slides ou material multimídia;
  \item Laboratório de microcomputadores contendo componentes de hardware e \textit{software} específicos;
\end{itemize}
\noindent\rule{16.5cm}{0.4pt}\\
\\
%PREENCHER BIBLIOGRAFIA A SEGUIR
\vspace{-12mm}
\begin{center}\textbf{Bibliografia}\end{center}
\vspace{-5mm}
\noindent\rule{16.5cm}{0.4pt}
\\
\begin{itemize} 
  \item Básica:
	\begin{enumerate}
  	\item 	ELMASRI, R.; NAVATHE, S. \textbf{Sistemas de banco de dados.} Pearson, 6ª edição, 2011;
	\item 	KORTH, H.; SILBERSCHATZ, A.; SUDARSHAN, S. \textbf{Sistemas de bancos de dados.} Campus, 5ª edição, 2006;
	\item 	DATE, C. J. \textbf{Introdução a sistemas de bancos de dados.} Campus, Tradução da 8ª edição Americana, 2004.      
	\end{enumerate}
    
  \item Complementar:
	\begin{enumerate}
  	\item 	CASANOVA, M, et al. \textbf{Bancos de Dados Geográficos}. INPE, 2005; 
	\item 	FOWLER, M.; SADALAGE, P. J. \textbf{NoSQL Essencial: Um Guia Conciso Para O Mundo Emergente Da Persistência Poliglota.} Novatec, 1ª Edição, 2013;
    \item 	RAMAKRISHNAN, R. \textbf{Sistemas de Gerenciamento de Banco de Dados.} McGraw Hill, 3ª edição, 2010.
	\end{enumerate}
\end{itemize}
\noindent\rule{16.5cm}{0.4pt}\\
\\
\clearpage
\paragraph{An�lise e Projeto de Sistemas}

%PREENCHER DADOS DA DISCIPLINA A SEGUIR
%\vspace{-12mm}
\begin{center}\textbf{Dados do Componente Curricular}\end{center}
\vspace{-5mm}
\noindent\rule{16.5cm}{0.4pt}
\\
\textbf{Nome:} An�lise e Projeto de Sistemas
\\
\textbf{Curso:} Tecnologia em Sistemas para Internet
\\ 
\textbf{Per�odo:} $4^{\circ}$ 
\\
\textbf{Carga Hor�ria:} 67~h 
\\ 
\textbf{Docente Respons�vel:} Jos� de Sousa Barros 
\\ 
\noindent\rule{16.5cm}{0.4pt}\\
\\
%PREENCHER A EMENTA A SEGUIR
\vspace{-12mm}
\begin{center}\textbf{Ementa}\end{center}
\vspace{-5mm}
\noindent\rule{16.5cm}{0.4pt}
\\
Conceitos de An�lise e Projeto de Sistemas. Modelos de ciclos de vida.  Metodologia para an�lise e desenvolvimento de sistemas orientados a objetos. Linguagem UML.  An�lise de requisitos, Modelagem conceitual, Ferramenta CASE para cria��o de modelos orientados a objetos. \\
\noindent\rule{16.5cm}{0.4pt}\\
\\
%PREENCHER OS OBJETIVOS A SEGUIR
\vspace{-12mm}
\begin{center}\textbf{Objetivos}\end{center}
\vspace{-5mm}
\noindent\rule{16.5cm}{0.4pt}
\\
\begin{itemize}
\item Compreender os conceitos da An�lise e Projeto Orientado a Objetos;
\item Aplicar uma Metodologia de An�lise e Projeto de Software Orientado a Objetos;
\item Analisar e Projetar solu��es orientadas a objetos utilizando UML para problemas do mundo real.
\end{itemize}
\noindent\rule{16.5cm}{0.4pt}\\
\\
%PREENCHER OS CONTEUDOS PROGRAMATICOS A SEGUIR (CUIDADO PARA NAO DEIXAR A TABELA MUITO GRANDE)
\vspace{-12mm}
\begin{center}\textbf{Conte�do Program�tico}\end{center}
\vspace{-5mm}
\noindent\rule{16.5cm}{0.4pt}
\\
\begin{itemize}
 \item \textbf{Introdu��o � An�lise e Projeto Orientados a Objetos:} Conceito de An�lise e Projeto; Conceito de An�lise e Projeto Orientados a Objetos; Modelos de ciclos de vida de software.

 \item \textbf{An�lise de Requisitos:} Requisitos funcionais e n�o Funcionais;	T�cnicas de elicita��o de requisitos; Casos de uso: Conceito de casos de uso e atores, Diagrama de casos de de uso UML, Documenta��o de casos de uso.

 \item \textbf{An�lise e Projeto:} Metodologia de An�lise e Projeto de Software Orientados a Objetos.
 
 \item \textbf{Modelagem de Software com UML:} Vis�o geral da UML; Ferramenta CASE para cria��o de modelos orientados a objetos; Diagramas de Casos de Uso; Diagramas de Classes; Diagramas de Objetos; Diagramas de Atividades e Estados;  Diagramas de Intera��o: Sequ�ncia e Comunica��o; Diagramas de Pacotes; Diagramas Implanta��o e Componentes.
 
\end{itemize}
\noindent\rule{16.5cm}{0.4pt}\\
\\
%COLOCAR A METODOLOGIA DE ENSINO A SEGUIR
\vspace{-12mm}
\begin{center}\textbf{Metodologia de Ensino}\end{center} 
\vspace{-5mm}
\noindent\rule{16.5cm}{0.4pt}
\\
   Aulas expositivas utilizando recursos audiovisuais e quadro, al�m de aulas pr�ticas utilizando computadores. Adicionalmente, ser�o realizadas atividades pr�ticas individuais ou em grupo, para consolida��o do conte�do ministrado.\\
\noindent\rule{16.5cm}{0.4pt}\\
\\
%COLOCAR AVALIACAO DO PROCESSO DE ENSINO E APRENDIZAGEM A SEGUIR
\vspace{-12mm}
\begin{center}\textbf{Avalia��o do Processo de Ensino e Apendizagem}\end{center}
\vspace{-5mm}
\noindent\rule{16.5cm}{0.4pt}
\\
   Avalia��es escritas. Pr�ticas baseadas em Estudos de Caso ou problemas reais.\\
\noindent\rule{16.5cm}{0.4pt}\\
\\
%PREENCHER RECURSOS NECESSARIOS A SEGUIR
\vspace{-12mm}
\begin{center}\textbf{Recursos Necess�rios}\end{center}
\vspace{-5mm}
\noindent\rule{16.5cm}{0.4pt}
\\
\begin{itemize} 
  \item Listas de Exerc�cios;
  \item Livros e apostilas;
  \item Utiliza��o de recursos da web;
  \item Quadro branco;
  \item Marcadores para quadro branco;
  \item Sala de aula com acesso � internet, microcomputador e TV ou projetor para apresenta��o de slides ou material multim�dia;
  \item Laborat�rio de microcomputadores contendo componentes de hardware e \textit{software} espec�ficos;
\end{itemize}
\noindent\rule{16.5cm}{0.4pt}\\
\\
%PREENCHER BIBLIOGRAFIA A SEGUIR
\vspace{-12mm}
\begin{center}\textbf{Bibliografia}\end{center}
\vspace{-5mm}
\noindent\rule{16.5cm}{0.4pt}
\\
\begin{itemize} 
  \item B�sica:
	\begin{enumerate}
  	\item LARMAN, Craig. \textbf{Utilizando UML e Padr�es: uma introdu��o � an�lise e ao projeto orientados a objetos e ao desenvolvimento iterativo.} Bookman,  3� edi��o, 2007;
	\item MCLAUGHLIN, B.; et al. \textbf{Use a Cabe�a An�lise e Projeto Orientado a Objeto.} Alta Books, 2007;
	\item PILONE, D.; PITMAN, N. \textbf{UML 2: R�pido e Pr�tico.} Alta Books, 2006.    
	\end{enumerate}
    
  \item Complementar:
	\begin{enumerate}
  	\item FOWLER, M.; SCOTT, K. \textbf{UML Essencial.} Porto Alegre: Bookman, 2005;
	\item ENGHOLM JR, H. \textbf{An�lise e Design Orientado a Objetos.} Novatec. 2013.
	\end{enumerate}
\end{itemize}
\noindent\rule{16.5cm}{0.4pt}\\
\\
\clearpage
\paragraph{Sistemas Distribu\'idos}

\begin{center}\textbf{Dados do Componente Curricular}\end{center}
\noindent\rule{16cm}{0.4pt}
%PREENCHER DADOS DA DISCIPLINA A SEGUIR
\\
\textbf{Nome:} Sistemas Distribuídos\\ 
\noindent\rule{16cm}{0.4pt}\\
\textbf{Curso:} Tecnologia em Sistemas para Internet\\ 
\noindent\rule{16cm}{0.4pt}\\
\textbf{Período:} $6^{\circ}$\\ 
\noindent\rule{16cm}{0.4pt}\\
\textbf{Carga Horária:} 67~h\\ 
\noindent\rule{16cm}{0.4pt}\\
\textbf{Docente Responsável:} Ruan Delgado Gomes\\ 
\noindent\rule{16cm}{0.4pt}\\
\\
%PREENCHER A EMENTA A SEGUIR
\begin{center}\textbf{Ementa}\end{center}
\noindent\rule{16cm}{0.4pt}
\\
Fundamentos de Sistemas Distribuídos. Estilos Arquiteturais para Sistemas Distribuídos. Arquitetura de Comunicação Cliente-Servidor. Comunicação: Sockets, RPC, RMI, MOM. Sistemas de arquivos distribuídos; Sistemas \textit{peer-to-peer}; Sincronização e estados globais; Transações; Replicação e tolerância a falhas; Serviços Web.\\ 
\noindent\rule{16cm}{0.4pt}\\
%PREENCHER OS OBJETIVOS A SEGUIR
\begin{center}\textbf{Objetivos}\end{center}
\noindent\rule{16cm}{0.4pt}
\\
\begin{itemize}
\item Proporcionar o entendimento sobre as possíveis formas de estruturação dos sistemas distribuídos;
\item Conhecer e utilizar técnicas para garantir a qualidade de sistemas distribuídos;
\item Saber como resolver problemas de faltas em sistemas distribuídos.
\end{itemize} 
\noindent\rule{16cm}{0.4pt}\\
\\
%PREENCHER OS CONTEUDOS PROGRAMATICOS A SEGUIR (CUIDADO PARA NAO DEIXAR A TABELA MUITO GRANDE)
\begin{center}\textbf{Conteúdo Programático}\end{center}
\noindent\rule{16cm}{0.4pt}
\\
\begin{itemize}
 \item \textbf{Fundamentos de Sistemas Distribuídos:} Definição de Sistemas Distribuídos; Infraestrutura básica; Tipos de Sistemas Distribuídos.

 \item \textbf{Estilos Arquiteturais para SD:} Camadas; Baseada em Objetos; Baseada em Dados; Baseada em Eventos.

 \item \textbf{Visão Cliente-Servidor:} Requisição-Resposta; Comunicação síncrona; Comunicação assíncrona.

 \item \textbf{P2P:} Arquitetura Centralizada; Arquitetura Descentralizada.
 \item \textbf{Processos e Threads}
 \item \textbf{Comunicação:} Sockets; RPC; RMI; JMS.

 \item \textbf{Serviços:} Conceitos; Arquitetura Orientada a Serviço; Tipos de Serviços; Design de Serviços; Registro e descoberta; Web Services.

 \item \textbf{Tolerância a Faltas:} Definição; Dependabilidade; Tipos; Recuperação; Mascaramento.
 \item \textbf{Sincronização:} Cálculo de Latência; Ajuste de relógios.
\end{itemize}
\noindent\rule{16cm}{0.4pt}\\
\\
%COLOCAR A METODOLOGIA DE ENSINO A SEGUIR
\begin{center}\textbf{Metodologia de Ensino}\end{center} 
\noindent\rule{16cm}{0.4pt}
\\
   Aulas expositivas utilizando recursos audiovisuais e quadro, além de aulas práticas utilizando computadores. Adicionalmente, serão realizadas atividades práticas individuais ou em grupo, para consolidação do conteúdo ministrado.\\
\noindent\rule{16cm}{0.4pt}\\
%COLOCAR AVALIACAO DO PROCESSO DE ENSINO E APRENDIZAGEM A SEGUIR
\begin{center}\textbf{Avaliação do Processo de Ensino e Apendizagem}\end{center}
\noindent\rule{16cm}{0.4pt}
\\
   Avaliações escritas ao final de cada unidade. Prática baseada em Estudo de Caso ou problema real.\\
\noindent\rule{16cm}{0.4pt}\\
\\
%PREENCHER RECURSOS NECESSARIOS A SEGUIR
\begin{center}\textbf{Recursos Necessários}\end{center}
\noindent\rule{16cm}{0.4pt}
\\
\begin{itemize} 
  \item Listas de Exercícios;
  \item Livros e apostilas;
  \item Utilização de recursos da web;
  \item Quadro branco;
  \item Marcadores para quadro branco;
  \item Sala de aula com acesso à internet, microcomputador e TV ou projetor para apresentação de slides ou material multimídia;
  \item Laboratório de microcomputadores contendo componentes de hardware e software específicos;
\end{itemize}
\noindent\rule{16cm}{0.4pt}\\
\\
%PREENCHER BIBLIOGRAFIA A SEGUIR
\begin{center}\textbf{Bibliografia}\end{center}
\noindent\rule{16cm}{0.4pt}
\\
\begin{itemize} 
  \item Básica;
  \item Complementar;
\end{itemize}
\noindent\rule{16cm}{0.4pt}\\
\\

\clearpage
\paragraph{Projeto em TSI}

%PREENCHER DADOS DA DISCIPLINA A SEGUIR
%\vspace{-12mm}
\begin{center}\textbf{Dados do Componente Curricular}\end{center}
\vspace{-5mm}
\noindent\rule{16.5cm}{0.4pt}
\\
\textbf{Nome:} Projeto em TSI
\\ 
\textbf{Curso:} Tecnologia em Sistemas para Internet
\\ 
\textbf{Período:} $6^{\circ}$
\\ 
\textbf{Carga Horária:} 67~h
\\ 
\textbf{Docente Responsável:} Ruan Delgado Gomes
\\ 
\noindent\rule{16.5cm}{0.4pt}\\
\\
%PREENCHER A EMENTA A SEGUIR
\vspace{-12mm}
\begin{center}\textbf{Ementa}\end{center}
\vspace{-5mm}
\noindent\rule{16.5cm}{0.4pt}
\\
M\'inimo produto vi\'avel (\textit{Minimum Viable Product} - MVP); Integra\c{c}\~ao dos conhecimentos obtidos durante o curso para o desenvolvimento de um produto de \textit{software}. Financiamento e investimento em projetos de \textit{software}.
\\ 
\noindent\rule{16.5cm}{0.4pt}\\
\\
%PREENCHER OS OBJETIVOS A SEGUIR
\vspace{-12mm}
\begin{center}\textbf{Objetivos}\end{center}
\vspace{-5mm}
\noindent\rule{16.5cm}{0.4pt}
\\
\begin{itemize}
\item Desenvolver uma vers\~ao reduzida (o MVP) do produto definido na disciplina Empreendedorismo em \textit{Software};
\item Testar e validar a MVP desenvolvido;
\item Aperfei\c{c}oar o plano de neg\'ocio visando uma eventual forma\c{c}\~ao de empresa;
\item Entender as formas de financiamento e investimento para projetos de \textit{software}.
\end{itemize} 
\noindent\rule{16.5cm}{0.4pt}\\
\\
%PREENCHER OS CONTEUDOS PROGRAMATICOS A SEGUIR (CUIDADO PARA NAO DEIXAR A TABELA MUITO GRANDE)
\vspace{-12mm}
\begin{center}\textbf{Conteúdo Programático}\end{center}
\vspace{-5mm}
\noindent\rule{16.5cm}{0.4pt}
\\
\begin{itemize}

 \item \textbf{M\'inimo produto vi\'avel:} Desenvolvimento de vers\~ao reduzida de produto de \textit{software}, suficiente para ser vendido ou demonstrado.

% \item \textbf{Estilos Arquiteturais para SD:} Camadas; Baseada em Objetos; Baseada em Dados; Baseada em Eventos.

 \item \textbf{Integra\c{c}\~ao dos conhecimentos obtidos durante o curso para o desenvolvimento de um produto de \textit{software}:} Utiliza\c{c}\~ao dos conhecimentos obtidos no curso para o desenvovlimento do produto; Testes e valida\c{c}\~ao do produto desenvolvido.

 \item \textbf{Financiamento e investimento em projetos de \textit{software}:} Aprimoramento do plano de neg\'ocios; Avalia\c{c}\~ao do produto; Escrita de projetos para editais de agências financiadoras e de subvenção (BNDES, FINEP, CNPq etc) e projetos para investidores (Venture Capitals, Angels etc).

\end{itemize}
\noindent\rule{16.5cm}{0.4pt}\\
\\
%COLOCAR A METODOLOGIA DE ENSINO A SEGUIR
\vspace{-12mm}
\begin{center}\textbf{Metodologia de Ensino}\end{center} 
\vspace{-5mm}
\noindent\rule{16.5cm}{0.4pt}
\\
   Reuni\~oes para discuss\~oes de ideias e problemas; Apresenta\c{c}\~oes sobre os produtos sendo desenvolvidos; Palestras sobre assuntos importantes para a disciplina.\\
\noindent\rule{16.5cm}{0.4pt}\\
\\
%COLOCAR AVALIACAO DO PROCESSO DE ENSINO E APRENDIZAGEM A SEGUIR
\vspace{-12mm}
\begin{center}\textbf{Avaliação do Processo de Ensino e Apendizagem}\end{center}
\vspace{-5mm}
\noindent\rule{16.5cm}{0.4pt}
\\
   A avalia\c{c}\~ao ser\'a baseada na an\'alise do produto de \textit{software} desenvolvido.\\
\noindent\rule{16.5cm}{0.4pt}\\
\\
%PREENCHER RECURSOS NECESSARIOS A SEGUIR
\vspace{-12mm}
\begin{center}\textbf{Recursos Necessários}\end{center}
\vspace{-5mm}
\noindent\rule{16.5cm}{0.4pt}
\\
\begin{itemize} 
  \item Livros e apostilas;
  \item Utilização de recursos da web;
  \item Quadro branco;
  \item Marcadores para quadro branco;
  \item Sala de aula com acesso à internet, microcomputador e TV ou projetor para apresentação de slides ou material multimídia;
  \item Laboratório de microcomputadores contendo componentes de hardware e \textit{software} específicos;
\end{itemize}
\noindent\rule{16.5cm}{0.4pt}\\
\\
%PREENCHER BIBLIOGRAFIA A SEGUIR
\vspace{-12mm}
\begin{center}\textbf{Bibliografia}\end{center}
\vspace{-5mm}
\noindent\rule{16.5cm}{0.4pt}
\\
\begin{itemize} 

  \item Básica:
	\begin{enumerate}
	\item A Cabeça de Steve Jobs (Inside Steve's Brain), Leander Kahney, Agir, 2008.
	\item Bilionários por Acaso: A Criação do Facebook, Ben Mezrich, Intrinseca, 2010.
	\item Google, David A. Vise, Mark Malseed, Rocco, 2007.
	\end{enumerate}
  \item Complementar:
	\begin{enumerate} 
	\item De acordo com os planos de negócio definidos pelos alunos, novos artigos e livros poderão ser sugeridos.
	\end{enumerate}
\end{itemize}
\noindent\rule{16.5cm}{0.4pt}\\
\\

\clearpage

 %arquivo que organiza as ementas

%\newpage

\subsection{Proposta Pedag\'ogica}

\subsubsection{Metodologia de Ensino}

A metodologia é entendida como um conjunto de procedimentos empregados para atingir os objetivos propostos a fim de propiciar a conexão entre os conhecimentos e as capacidades, assegurando a formação integral dos futuros tecnólogos em sistemas para internet. Este projeto pedagógico, que deve ser o norteador do currículo do Curso Superior de Tecnologia em Sistemas para Internet, deve apresentar, portanto, em sua proposta pedagógica, os princípios que embasarão o currículo, o processo de ensino-aprendizagem, as avaliações e outras atividades articuladas ao ensino, como o Estágio Curricular e o Trabalho de Conclusão de Curso (TCC).

Para este curso, que se propõe a formar profissionais comprometidos com a construção de uma sociedade justa e humana, a metodologia adotada é uma importante ferramenta para conseguir um melhor desempenho cognitivo dos acadêmicos, sabendo relacionar os conhecimentos técnico–científicos do curso com os problemas do cotidiano dos alunos, construindo assim uma consciência crítica com capacidade de intervir na relação ensino-aprendizagem de forma criativa, tendo como objetivo a participação de todos os envolvidos. Portanto, deve-se buscar um planejamento acadêmico em consonância com o conteúdo programático das disciplinas, relacionando suas aplicações no dia a dia.

Dessa forma, um dos princípios fundamentais que destacamos no presente projeto pedagógico é a relação teoria-prática, que, associada à estrutura curricular do curso, conduz a um fazer pedagógico, em que atividades como práticas interdisciplinares, seminários, oficinas, visitas técnicas e desenvolvimento de projetos, entre outros, estão presentes durante os períodos letivos.  

O trabalho coletivo entre os grupos de professores da mesma base de conhecimento e os professores de base científica e da base tecnológica específica é imprescindível à construção de práticas didático-pedagógicas integradas, resultando na construção e apreensão dos conhecimentos pelos alunos, numa perspectiva do pensamento relacional. Para tanto, os professores, articulados pela equipe técnico-pedagógica, deverão desenvolver aulas de campo, atividades laboratoriais, projetos integradores e práticas coletivas, juntamente com os alunos. Para essas atividades que preveem um planejamento coletivo, os professores terão à sua disposição horários para encontros ou reuniões de grupo.

Este plano pedagógico caracteriza-se como expressão coletiva e, portanto, deve ser avaliado periódica e sistematicamente pela comunidade escolar, apoiada por uma comissão a que compete tal função. Qualquer alteração deve ser vista sempre que se verificar, mediante avaliações sistemáticas anuais, defasagem entre o perfil de conclusão do curso, seus objetivos e sua organização curricular, frente às exigências decorrentes das transformações científicas, tecnológicas, sociais e culturais.

\subsubsection{Processo Ensino-Aprendizagem}

Considera-se a aprendizagem como construção de conhecimento em que, partindo dos conhecimentos prévios dos alunos, os professores assumem um papel fundamental, idealizando estratégias de ensino de maneira que a articulação entre o conhecimento do senso comum e o conhecimento escolar permita ao aluno desenvolver suas percepções e convicções acerca dos processos sociais e de trabalho, aprimorando-se como pessoas e profissionais responsáveis éticos e competentemente qualificados.

Para um processo ensino-aprendizagem eficiente, é recomendado, portanto, considerar algumas particularidades dos alunos, seus interesses, condições de vida e de trabalho, orientando-os na (re)construção dos conhecimentos escolares. Em razão disso, faz-se necessária à adoção de procedimentos didático-pedagógicos que possam auxiliá-los nas suas construções intelectuais, procedimentais e atitudinais, tais como:

\begin{itemize}
	\item problematizar o conhecimento, buscando confirmação em diferentes fontes;
	\item entender a totalidade como uma síntese das múltiplas relações que o homem estabelece na sociedade;
	\item reconhecer a existência de uma identidade comum do ser humano, sem esquecer-se de considerar os diferentes ritmos de aprendizagens e a subjetividade do aluno;
	\item adotar a pesquisa como um princípio educativo;
	\item articular e integrar os conhecimentos das diferentes áreas sem sobreposição de saberes;
	\item adotar atitude inter e transdisciplinar nas práticas educativas;
	\item contextualizar os conhecimentos sistematizados, valorizando as experiências dos alunos, sem perder de vista a (re)construção do saber escolar;
	\item organizar um ambiente educativo que articule múltiplas atividades voltadas às diversas dimensões de formação dos jovens e adultos, favorecendo a transformação das informações em conhecimentos diante das situações reais de vida;
	\item diagnosticar as necessidades de aprendizagem dos(as) estudantes a partir do levantamento dos seus conhecimentos prévios;
	\item elaborar materiais impressos a serem trabalhados em aulas expositivas dialogadas e atividades em grupo;
	\item elaborar e executar o planejamento, registro e análise das aulas realizadas;
	\item elaborar projetos com objetivo de articular e inter-relacionar os saberes, tendo como princípios a contextualização, a trans e a interdisciplinaridade;
	\item utilizar recursos tecnológicos para subsidiar as atividades pedagógicas;
	\item sistematizar coletivos pedagógicos que possibilitem aos estudantes e professores refletir, repensar e tomar decisões referentes ao processo ensino-aprendizagem de forma significativa;
	\item ministrar aulas interativas, por meio do desenvolvimento de projetos, seminários, debates, atividades individuais e outras atividades em grupo.
\end{itemize}

\subsubsection{Coerência do Currículo com a Proposta Pedagógica}

A formação proposta por esse curso respeita os campos de conhecimento acadêmico, estabelecendo articulações entre os saberes específicos, os cotidianos, os científicos e os dos estudantes.  Neste sentido, a avaliação da aprendizagem assume dimensões mais amplas, ultrapassando a perspectiva da mera aplicação de provas e testes, para assumir uma prática diagnóstica e processual com ênfase nos aspectos qualitativos.

Enxergando os estudantes como futuros tecnólogos, as disciplinas propostas para comporem a estrutura curricular do curso trazem em suas ementas todos os conteúdos necessários para boa formação técnica, por meio de um adequado embasamento didático-pedagógico e interdisciplinar. Assim, estão inseridos no Curso Superior em Sistemas para Internet do IFPB-Campus Guarabira, conhecimentos básicos em computação e conhecimentos específicos da área de informática que caracterizam o tecnólogo em sistemas para internet. Articulando esses conhecimentos, organiza-se o espaço curricular dos conhecimentos complementares ou interdisciplinares, compostos por disciplinas oriundas de diversos campos de conhecimento, mas que se inter-relacionam e enriquecem a formação do futuro tecnólogo.

\subsection{Atividades Articuladas ao Ensino}

 Os alunos do curso de Tecnologia em Sistemas para Internet do IFPB - Campus Guarabira têm acesso as seguintes atividades que estão articuladas ao ensino:

\begin{itemize}
\item Programa Institucional de Bolsas de Iniciação Científica do Conselho Nacional de Desenvolvimento Científico e Tecnológico (CNPq), que tem por objetivo estimular os jovens do ensino superior nas atividades, metodologias, conhecimentos e práticas próprias ao desenvolvimento tecnológico e processos de inovação. Além disso, tem o objetivo de contribuir para a formação e inserção de estudantes em atividades de pesquisa, desenvolvimento tecnológico e inovação;

\item Programa Institucional de Bolsa de Extensão do IFPB Modalidade PROBEXT, que visa a formação técnica e cidadã do estudante e pela produção e difusão de novos conhecimentos e novas metodologias oriundos das várias disciplinas e áreas do conhecimento que geram Impacto na formação do estudante nos aspectos técnico-científica, pessoal e social;

\item Bolsa de Monitoria, que é uma atividade discente, que tem como objetivo promover a interação acadêmica entre alunos, professores e visa estimular o (a) monitor (a) no desempenho de suas potencialidades docentes para subsidiar o alunado na superação de dificuldades de aprendizagem e produção de novos conhecimentos nas disciplinas objeto da monitoria.

%\item Bolsa de trabalho referente ao Programa de Iniciação ao Trabalho, que se propõe a assegurar a permanência dos estudantes em condições de vulnerabilidade social, mediante o repasse de uma bolsa mensal, para custear despesas decorrentes de seu processo sócio-educacional;

\item Visitas técnicas, que são atividades de campo supervisionadas por um professor que permitem ao aluno observar as aplicações práticas dos conceitos estudados em sala de aula e laboratórios, enriquecendo assim o seu processo de ensino-aprendizagem;


\item Participação em Atividades Complementares ao currículo, que objetivam enriquecer o processo de ensino-aprendizagem privilegiando o desenvolvimento de atividades de complementação da formação técnica, social, humana e cultural.

\item Trabalho de Conclusão de Curso (TCC), que tem como objetivo possibilitar aos alunos o desenvolvimento da capacidade de aplicação dos conceitos e teorias adquiridas durante o curso de forma integrada por meio da execução de um projeto de pesquisa e desenvolvimento;

\item Estágio Curricular, que visa a complementação do processo ensino-aprendizagem e tem como objetivos facilitar a futura inserção do estudante no mercado de trabalho, promover a articulação do IFPB com o mundo do trabalho e facilitar a adaptação social e psicológica do estudante à futura atividade profissional.

\end{itemize}

\subsubsection{Estágio Curricular}

	O estágio curricular supervisionado é um conjunto de atividades de formação, realizadas sob a supervisão de docentes da instituição formadora e acompanhado por profissionais, em que o estudante experimenta situações de efetivo exercício profissional. O estágio supervisionado tem o objetivo de consolidar e articular os conhecimentos desenvolvidos durante o curso por meio das atividades formativas de natureza teórica ou prática.

%Nos cursos superiores de tecnologia, o estágio curricular supervisionado é realizado por meio de estágio técnico e caracteriza-se como prática profissional não obrigatória.

O estágio técnico é considerado uma etapa educativa importante para consolidar os conhecimentos específicos do curso e tem por objetivos:

\begin{itemize}
	\item Possibilitar ao estudante o exercício da prática profissional, aliando a teoria à prática, como parte integrante de sua formação;
	\item Facilitar o ingresso do estudante no mundo do trabalho; e
	\item Promover a integração do IFPB com a sociedade em geral e o mundo do trabalho.
\end{itemize}
	
O estágio poderá ser realizado a partir do 4o período do curso, obedecendo às normas instituídas pelo IFPB.

O acompanhamento do estágio será realizado por um supervisor técnico da empresa/instituição na qual o estudante desenvolve o estágio, mediante acompanhamento in loco das atividades realizadas, e por um professor orientador, lastreado nos relatórios periódicos de responsabilidade do estagiário, em encontros semanais com o estagiário, contatos com o supervisor técnico e, visita ao local do estágio, sendo necessária, no mínimo, uma visita por semestre, para cada estudante orientado.

As atividades programadas para o estágio devem manter uma correspondência com os conhecimentos teórico-práticos adquiridos pelo aluno no decorrer do curso. Ao final do estágio (e somente nesse período), o estudante deverá apresentar um relatório técnico. Nos períodos de realização de estágio docente, o aluno terá momentos em sala de aula, no qual receberá as orientações.

\paragraph{Acompanhamento do estágio}\

       	O Estágio Curricular do Curso Superior de Tecnologia em Sistemas para Internet será realizado em organizações públicas, privadas ou do terceiro setor, devidamente conveniadas com o IFPB, que apresentem condições de proporcionar experiência prática na área de formação do aluno, ou desenvolvimento sociocultural ou científico, pela participação em situações de vida e de trabalho no seu meio. O acompanhamento do aluno estagiário será feito pela Coordenação de Estágios, que estabelece os seguintes procedimentos como requisitos para a efetivação do estagio curricular: 
\begin{itemize}
	\item No período de matrícula que ocorre no início do semestre letivo o aluno pode se matricular no componente curricular Estágio Obrigatório, desde que já esteja cursando ou tenha cursado as disciplinas do quarto período do curso;
	
	\item Solicitar à Coordenação de Estágios os documentos necessários para iniciar o estágio;

	\item Preencher a ficha de cadastro pessoal;
\end{itemize}

	Após esta etapa o aluno fica aguardando que a Coordenação de Estágios faça contato com as empresas parceiras do IFPB-Campus Guarabira para obter vagas de estágio na área de informática. Com a vaga de estágio garantida, passa a existir as seguintes exigências que devem ser observadas pela instituição de ensino, a empresa cedente e o aluno estagiário: 
	
\begin{itemize}
	\item O estabelecimento do Termo de Convênio, celebrado entre a organização cedente e o IFPB-Campus Guarabira, em que acordam as condições de realização do estágio;
	\item O Termo de Compromisso, celebrado entre o estagiário e a empresa cedente, com a interveniência da Instituição de Ensino, regulamentando as atividades a serem desenvolvidas pelo estagiário;
	\item A aceitação como estágio do exercício das atividades dependerá do parecer emitido pelo colegiado de curso, no caso dos cursos superiores, que levará em consideração o tipo de atividade desenvolvida e a sua contribuição para a formação profissional do estudante.
	\item Elaboração de um plano de estágio;
	\item Observar a carga horária mínima do estágio curricular obrigatório do curso, que corresponde a 300 horas.
\end {itemize}

	As atividades de extensão, de monitorias e de iniciação científica na educação superior, desenvolvidas pelo estudante, poderão ser equiparadas ao estágio, como previsto pela lei No 11.788 de 2008, desde que respeitada a carga horária mínima.  Assim como no caso do estágio em empresas, a aceitação como estágio da atividade de extensão, monitoria ou iniciação científica dependerá do parecer emitido pelo colegiado de curso, que levará em consideração o tipo de atividade desenvolvida e a sua contribuição para a formação profissional do estudante.

	A Unidade Concedente de Estágio poderá oferecer auxílio ao estagiário, mediante pagamento de bolsa ou qualquer outra forma que venha a ser acordada entre as partes, respeitando-se a legislação em vigor. O seguro contra acidentes pessoais deverá ser contratado pela Unidade Concedente de Estágio, diretamente ou através da atuação conjunta com Agentes de Integração.
       
	   Após esta etapa, a Coordenação de Estágios, juntamente com o Coordenador do Curso, designarão o professor responsável pelo acompanhamento e orientação das atividades a serem desenvolvidas pelo aluno estagiário. 
	   
       No Campo de estágio o aluno passa pelos seguintes acompanhamentos e orientações que são feitos pelo IFPB-Campus Guarabira e pela empresa cedente do estágio:
	   
\begin{itemize}

	\item A empresa cedente designa um profissional para atuar como supervisor de estagiário para verificar e acompanhando a assiduidade, o controle do horário através do registro de frequência e enviar à instituição de ensino, com periodicidade mínima de 6 (seis) meses, um relatório de atividades;
	
	\item O Coordenador do Curso de Tecnologia em Sistemas para Internet supervisiona o desenvolvimento das atividades de Estagiário e juntamente com a Coordenação de Estágios pode criar e reformular instrumentos que possam avaliar o aluno no campo de estágio; 
	
	\item O aluno terá o acompanhamento técnico de um professor orientador que irá acompanhar o estagiário, no IFPB-Campus Guarabira e na empresa concedente do estágio, através de visitas periódicas durante o período de realização do estágio. Neste trabalho o professor orientador acompanhará a elaboração e avaliação do Relatório de Estágio.

\end{itemize}

	 Durante a execução do estagio curricular na empresa cedente o aluno passa por todo um processo de avaliação que conta dos seguintes itens:

\begin{itemize}
	\item Recebimento de visita do Professor orientador de estágio na empresa cedente do estágio, incluindo reunião com o supervisor de estágio da empresa;
	\item Apresentação do Relatório de Estágio contendo as atividades desenvolvidas e as avaliações realizadas;
	\item Participação em reunião no IFPB-Campus-Guarabira com o professor orientador de Estágio, quando transcorridas aproximadamente 100 (cem) horas;
\end{itemize}

           Além destes aspectos, na avaliação das atividades desenvolvidas pelo estudante, serão consideradas:

\begin{itemize}
	\item A compatibilidade das atividades desenvolvidas com o projeto pedagógico do curso e com o plano de estágio;
	\item A qualidade e eficácia na realização das atividades;
	\item A capacidade inovadora ou criativa demonstrada por meio das atividades desenvolvidas;
	\item A capacidade de adaptar-se socialmente ao ambiente.
\end{itemize}


      Por fim, concluído o Estágio Curricular Obrigatório, o aluno terá 60 (sessenta) dias para entrega e apresentação do Relatório de Estágio.

\paragraph{Relevância do estágio e da prática profissional}\
 
      O estagio e a prática profissional são muito significativos para a formação integral do aluno, pois aparece como uma oportunidade real de crescimento pessoal e principalmente profissional, porque permite que o aluno possa colocar em prática todo o conhecimento teórico e prático que adquiriu durante o curso, podendo atuar no campo de estágio com situações reais, em áreas como o desenvolvimento de sistemas computacionais, prestação de serviços na área de tecnologia da informação, consultoria na área de tecnologia da informação, suporte à implantação e uso da tecnolgia da informação em empresas ou órgãos públicos, entre outras atividades relacionadas à área de atuação do curso.
	  
Para buscar assegurar essa relação quanti-qualitativa entre a prática profissional e o estágio, o Curso Superior de Tecnologia em Sistemas para Internet do IFPB-Campus Guarabira conta com o envolvimento de atores importantes como: gestores, professores, técnico-administrativos, empresas parceiras, pais e os próprios alunos.  Para o curso, esses atores são de grande importância pois agregam valores fundamentais à concepção do aluno em relação à importância do seu empenho, dos estudos e do curso, o que sem duvida irá contribuir para a qualidade de sua formação enquanto futuro tecnólogo em sistemas para internet. 

\subsubsection{Trabalho de Conclusão de Curso}

De acordo com o Plano de Desenvolvimento Institucional do IFPB (2015-2019), o Trabalho de Conclusão de Curso (TCC) consiste na sistematização dos resultados do Projeto correspondente, desenvolvido mediante orientação, acompanhamento e avaliação docente, conforme descrito no Projeto Pedagógico do Curso, proporcionando a articulação dos conhecimentos adquiridos com os problemas práticos do mundo do trabalho. Podendo ser de produção acadêmica, resultante de pesquisa científica sobre um determinado objeto, ato, fato ou fenômeno da realidade ou da produção técnica ou tecnológica, visando a aplicabilidade nos diversos campos do saber, com atendimento aos padrões técnicos de intervenção.

O TCC é componente curricular obrigatória para a obtenção do título de Tecnólogo. Corresponde a uma produção acadêmica que expressa as competências e as habilidades desenvolvidas (ou os conhecimentos adquiridos) pelos estudantes durante o período de formação. Desse modo, o TCC será desenvolvido no último período a partir da verticalização dos conhecimentos construídos nos projetos realizados ao longo do curso ou do aprofundamento em pesquisas acadêmico-científicas.

O estudante terá momentos de orientação e tempo destinado à elaboração da produção acadêmica correspondente. São consideradas produções acadêmicas de TCC para o curso superior de Sistemas para Internet:

\begin{itemize}
	\item artigo publicado em revista ou periódico, com ISSN;
	\item capítulo de livro publicado, com ISBN;
	\item relatório de desenvolvimento de um protótipo de um software;
	\item relatório de projeto relacionado a desenvolvimento de software realizado em estágio;
	\item trabalho ou atuação em projeto de pesquisa ou extensão; ou,
	\item outra forma definida pelo Colegiado do Curso.
\end{itemize}
	
%O TCC será acompanhado por um professor orientador e o mecanismo de planejamento, acompanhamento e avaliação é composto pelos seguintes itens:

%\begin{itemize}
%	\item elaboração de um plano de atividades, aprovado pelo professor orientador;
%	\item reuniões periódicas do aluno com o professor orientador;
%	\item elaboração da produção monográfica pelo estudante; e,
%	\item avaliação e defesa pública do trabalho perante uma banca examinadora.
%\end{itemize}

O TCC será acompanhado por um professor orientador e será apresentado a uma banca examinadora composta pelo professor orientador e mais dois componentes, podendo ser convidado, para compor essa banca, um profissional externo de reconhecida experiência profissional na área de desenvolvimento do objeto de estudo.

A avaliação do TCC terá em vista os critérios de: domínio do conteúdo; linguagem (adequação, clareza); interação; nível de participação e envolvimento; e material didático utilizado para descrever o TCC.

Será atribuída ao TCC uma pontuação entre 0 (zero) e 100 (cem) e o estudante será aprovado com, no mínimo, 70 (setenta) pontos. Caso o estudante não alcance a nota mínima de aprovação no TCC, deverá ser reorientado com o fim de realizar as necessárias adequações/correções e submeter novamente o trabalho à aprovação.

\paragraph{Acompanhamento do Trabalho de Conclusão de Curso}\

O acompanhamento dos discentes no TCC será feito por um docente orientador escolhido pelo discente ou designado pelo docente responsável pelo TCC, observando-se sempre a área de conhecimento em que será desenvolvido o projeto, a área de atuação e a disponibilidade do docente orientador. Para tal, é preciso observar e adotar os seguintes procedimentos:

\begin{itemize}
	\item Se houver necessidade, poderá existir a figura do co-orientador, para auxiliar nos trabalhos de orientação ou em outros que o orientador indicar, desde que aprovados pelo coordenador de curso;

	\item A mudança de orientador deverá ser solicitada por escrito e aprovada pelo coordenador de curso e pelo docente responsável pelo TCC;

	 \item Oacompanhamento dos Projetos de Graduação será feito por meio de de reuniões periódicas, no mínimo uma por mês, previamente agendadas entre docente orientador e orientando(s), devendo o cronograma ser apresentado ao docente responsável pelo TCC, até 20 (vinte) dias letivos após a aprovação da proposta;

	 \item Após cada reunião de orientação deverá ser atualizada a ficha de acompanhamento do TCC, segundo modelo a ser definido pelo colegiado do curso, descrevendo de forma simplificada os assuntos ali tratados. Deverá ser assinado pelos(s) discente (s)e pelo docente orientador e arquivada na pasta de acompanhamento do TCC;

	 \item é obrigatória a participação do(s) discente(s) em pelo menos 75\% das reuniões de orientação;
\end{itemize}

	 O tema para o TCC deve estar inserido em um dos campos de atuação do curso do discente, devendo ser apresentado na avaliação de propostas de TCC. Para tal é importante observar os seguintes procedimentos:
	 
	 \begin{itemize}
		 \item A avaliação da proposta de TCC será realizada em evento específico, agendado de acordo com a(s) inscrição (ões) da (s) propostas, pelas respectivas coordenações de curso;

		 \item A proposta de TCC deve ser apresentada decorridos, no máximo, 20 (vinte)dias do início do semestre.;

		 \item A não apresentação da proposta de TCC para avaliação implicará a impossibilidade de matrícula e consequente trancamento na disciplina de TCC;

		 \item A avaliação da proposta de TCC será feita por uma banca composta pelo docente orientador do trabalho, por um docente indicado pela coordenação de curso, pelo docente responsável pelo TCC ou coordenador de curso, no mínimo. As propostas de TCC serão avaliadas com base nos seguintes critérios: 
		 \begin{itemize}
		 	\item a delimitação do tema; 
			\item definição do problema; 
			\item justificativa; 
			\item objetivos; 
			\item metodologia; 
			\item relevância, inovações apresentadas ou utilidade prática do projeto; 
			\item cronograma de execução; 
			\item custos, condições e materiais disponíveis; 
		 \end{itemize}
	\end{itemize}
		
O resultado da(s) avaliação(ões) da(s) proposta(s) será divulgado, em edital da coordenação de curso, até 7 (sete) dias letivos após a realização da avaliação. 

o cronograma de execução, incluindo a defesa, deverá ser inferior ao prazo máximo de conclusão do curso, a contar da data da divulgação do resultado da avaliação da proposta, e não poderá exceder 06 (seis) meses de execução. 

 A defesa do TCC será realizada em evento público específico, cuja data, horário e local serão informados em edital da Coordenação de Curso seguindo os seguintes critérios: 
 
 \begin{itemize}
	\item A critério do Colegiado do Curso, pode ser realizada uma Pré-Defesa, que consiste numa avaliação do Trabalho Final, realizada por 02 (dois) docentes da área, excetuando-se o orientador;

	\item Não será permitida a manifestação do público, excetuando nas ocasiões em que for facultada a palavra, com a anuência dos componentes da banca;

	\item A banca de defesa do TCC será composta, no mínimo, pelo orientador do trabalho e por 02 (dois) avaliadores.
\end{itemize}

	Para participar da defesa do TCC, o discente deverá inscrever-se, junto à respectiva coordenação de curso.  A coordenação de curso terá um prazo de 15 (quinze) dias para marcar a defesa do TCC, excetuando-se os períodos de férias docentes. 
	
	No ato da inscrição para a defesa do TCC, o discente deverá entregar pelo menos 3 (três) cópias do trabalho final (sob a forma de monografia, projeto, estudo de casos, performance, produção artística, desenvolvimento de instrumentos, equipamentos, memorial descritivo de protótipos, entre outras, de acordo com a natureza e os fins do curso), conforme estrutura definida na proposta de TCC aprovada. 
	
	Na elaboração do trabalho final, devem ser seguidas as recomendações especificadas nas normas vigentes da ABNT. Discentes reprovados na defesa deverão apresentar nova proposta de projeto para avaliação. O trabalho que contemplar mais de um discente deverá ser avaliado individualmente, observando a competência de cada um no projeto, conforme apresentado para apreciação, na avaliação de propostas de TCC. Após 30 (trinta) dias da defesa do TCC, o discente deverá entregar 01 (uma) cópia corrigida e encadernada ao docente orientador de TCC, juntamente com uma versão eletrônica do trabalho. 

	Compete ao Coordenador de curso: 

\begin{itemize}
	\item proporcionar aos docentes orientadores horários para atendimento às atividades de TCC; 

	\item homologar o nome do docente responsável pelo TCC e também do docente orientador; 

	\item designar substituto do docente responsável pelo TCC, quando do impedimento deste; 

	\item participar da avaliação das propostas de Projetos de Graduação, quando do impedimento do docente responsável pelo TCC; 

	\item definir, juntamente com o Docente Responsável pelo TCC, locais, datas e horários para realização do Evento de Avaliação e Defesa dos Projetos de Graduação. 
\end{itemize}

Compete ao Docente Responsável pelo TCC: 
\begin{itemize}
	\item apoiar o Coordenador de Curso no desenvolvimento das atividades relativas ao TCC; 
	\item promover reuniões de orientação com discentes e docentes orientadores; 
	\item realizar visitas às empresas com o objetivo de acompanhar o TCC, quando essas estiverem sendo desenvolvidas em empresas;
	\item designar substitutos dos docentes orientadores, quando do impedimento destes; 
	\item definir, juntamente com a Coordenação de Curso, datas limites para entrega de projetos, relatórios; 
	\item marcar e divulgar data de defesa dos Projetos de Graduação; 
	\item coordenar a avaliação de propostas de TCC; 
	\item participar da avaliação de propostas de TCC; 
	\item efetuar o lançamento das notas finais do TCC no Sistema Acadêmico. 
\end{itemize}

Compete ao Docente Orientador: 
\begin{itemize}
	\item orientar o discente na elaboração da proposta do TCC bem como do trabalho final; 

	\item acompanhar o desenvolvimento do projeto;

	\item participar da banca examinadora de avaliação da proposta e da defesa do TCC; 

	\item realizar visitas às empresas em que o discente esteja desenvolvendo o TCC; 

	\item participar de reuniões sobre os TCC com a Coordenação de Curso e/ou com o Docente Responsável pelo TCC. 
\end{itemize}

Compete ao Discente: 

\begin{itemize}
	\item efetuar o pedido de matrícula da disciplina TCC no Sistema de Controle Acadêmico, atendendo aos prazos fixados nos Editais de Matrícula; 

	\item elaborar projeto de proposta de TCC;

	\item respeitar as normas técnicas de elaboração de trabalhos, monografias e artigos científicos; 

	\item apresentar a proposta de TCC para avaliação; 

	\item conduzir e executar o TCC; 

	\item redigir e defender o trabalho final; 

	\item entregar cópia corrigida do trabalho final; 

	\item tomar ciência e cumprir os prazos estabelecidos pela Coordenação de Curso. 
\end{itemize}

Os casos omissos desta Resolução serão resolvidos pela Câmara de Ensino do Conselho de Ensino, Pesquisa e Extensão do IFPB - CEPE. 

\paragraph{Relevância do trabalho de conclusão de curso}\
 
       O TCC possui grande relevância para a vida pessoal e principalmente profissional do egresso, porque se trata de uma experiência única em que o aluno no espaço acadêmico tem a oportunidade de colocar em prática os conhecimentos obtidos no decorrer do curso e de lidar com a resolução de problemas técnicos e científicos de sua área de atuação, por meio do conhecimento obtido.

	   %Na atualidade o TCC vem ganhando outra roupagem no espaço acadêmico, devido às mudanças da complexidade que vêm sendo implementadas pelo chamado avanço tecnológico já que neste espaço existe a exigência de uma instituição mais progressista e de professores e alunos mais flexíveis e críticos que estejam abertos a saber, saber fazer e saber ser. Assim, para poder atender o perfil de conclusão previsto no artigo 3º onde o egresso da engenharia deve ter uma formação generalista, humanista, crítica e reflexiva, capacitado á absorver e desenvolver novas tecnologias, estimulando a sua atuação crítica e criativa na identificação e resolução de problemas, considerando seus aspectos políticos, econômicos, sociais, ambientais e culturais, com visão ética e humanística, em atendimento às demandas da sociedade.

       %Neste sentido o Trabalho de Conclusão de Curso-TCC não pode ser mais aquele simples resumo e sim um trabalho cientifico e tecnológico que tenha relevância na comunidade cientifica e na sociedade como um todo, um trabalho que venha realmente contribuir para o desenvolvimento local, regional, nacional da sociedade brasileira e até á nível internacional. Para que isso aconteça o Trabalho de Conclusão de Curso-TCC que será empreendido pelo aluno Curso de Bacharelado em Engenharia Civil do IFPB-Campus-Cajazeiras será construído a partir de situações problemas que envolvem os diversos desafios que estão em volta do exercício profissional do engenheiro civil no intuito de levar o aluno a pensar e realizar atividades práticas ou simuladas que o permita apontar soluções inovadoras que venham a contribuir com a sua área de formação profissional. Essa postura vai viabilizar o debate entre os alunos haja vista que os vários trabalhos que serão feitos serão o espelho motivador. Desta forma, o Trabalho de Conclusão de Curso-TCC é um elemento que integra as disciplinas do curso correlacionando os conhecimentos teóricos e práticos ministrados ao longo do curso e aparece como elemento formado do espirito crítico e criativo permitindo o desenvolvimento intelectual, profissional e social dos alunos.  

Sendo assim, o TCC representa um componente curricular tão importante para o IFPB, seus professores e alunos, que está registrado no Plano de Desenvolvimento Institucional (2015-2019) como forma de registro de sua viabilidade e importância para a formação profissional do egresso considerando os seguintes objetivos:

\begin{itemize}
	\item Desenvolver a capacidade de aplicação dos conceitos e teorias adquiridas durante o curso de forma integrada, por meio da execução de um projeto de pesquisa e desenvolvimento;
	\item Desenvolver a capacidade de planejamento e de disciplina para resolver problemas no âmbito das diversas áreas de formação;
	\item Estimular o espírito empreendedor por meio da execução de projetos que levem ao desenvolvimento de produtos;
	\item Intensificar a extensão universitária por intermédio da resolução de problemas existentes nos diversos setores da sociedade;
	\item Estimular a interdisciplinaridade; 
	\item Estimular a inovação tecnológica e estimular a construção do conhecimento coletivo.
\end{itemize}

           Por fim, se conclui que o TCC é um momento de síntese de suma importância em que os alunos se integram num exercício de construção, reconstrução e direcionamento de sua formação profissional na área de informática.

\subsubsection{Atividades Complementares}

	Compreendem-se como atividades complementares todas e quaisquer atividades não previstas no rol das disciplinas obrigatórias do Currículo do Curso Superior em Sistemas para Internet, consideradas necessárias à formação acadêmica e ao aprimoramento pessoal e profissional do futuro tecnólogo.

	Estas atividades integram, em caráter obrigatório, e com carga horária de 40 horas, o do referido Currículo e compreende as seguintes categorias: ensino, pesquisa, extensão, práticas profissionalizantes e outras atividades oferecidas pela coordenação do curso que visem a formação complementar. As atividades complementares específicas serão descritas no regimento interno estabelecido e aprovado pelo Colegiado do Curso.

Consideram-se atividades complementares as seguintes:

\begin{itemize}
	\item Atividades de pesquisa: participação em núcleos, e/ou grupos de pesquisa, projetos científicos, apresentação ou publicação de trabalhos em eventos técnico-científicos;
	\item Participação na organização de eventos técnico-científicos de interesse da instituição em atividades afins ao curso;
	\item Atividades de extensão: participação em projetos de extensão com a comunidade ou em eventos técnico-científicos.
	\item Atividades de ensino: monitoria de disciplinas do curso de Tecnologia em Sistemas para Internet ou afins;
	\item Atividades de práticas profissionalizantes: participação em projetos realizados por empresas juniores em atividades afins ao curso de Tecnologia em Sistemas para Internet, em estágios extracurriculares na área técnica ou em projetos de desenvolvimento tecnológico junto a empresas privadas e/ou instituições públicas;
	\item Outras atividades oferecidas pela Coordenação do Curso que visem sua formação complementar;
\end{itemize}

	O aluno deverá solicitar à coordenação do curso a inclusão da carga-horária de atividades complementares em seu histórico escolar, por meio de requerimento específico e devidamente comprovado, mediante declaração ou certificado informando a carga-horária, período de realização, aproveitamento e frequência. O pedido será analisado pelo coordenador do curso ou por uma comissão designada para esse fim, que poderá deferir ou indeferir o pedido, com base nestas normas. Os casos omissos serão analisados pelo colegiado de curso.
	
	A integralização da carga horária das atividades complementares deve ser obtida em diferentes tipos de atividades estabelecidas para o curso, conforme quadro a seguir.

\begin{table}[h!]
\tiny
\begin{tabular}{|l|c|}
\hline
\rowcolor[HTML]{9B9B9B} 
\multicolumn{1}{|c|}{\cellcolor[HTML]{9B9B9B}{\bf Atividade}}                                     & {\bf \begin{tabular}[c]{@{}c@{}}Carga Horária \\ Semestral por Atividade (h)\end{tabular}} \\ \hline
Participação em conferências e palestras isoladas                                                 & 5                                                                                          \\ \hline
Cursos e mini-cursos de extensão (presencial ou à distância) na área do Curso ou diretamente afim & 15                                                                                         \\ \hline
Encontro estudantil na área do Curso ou diretamente afim.                                         & 5                                                                                          \\ \hline
Iniciação científica (como bolsista ou voluntário) na área do Curso ou diretamente afim.                                        & 25                                                                                         \\ \hline
Monitoria (como bolsista ou voluntário) na área do Curso ou diretamente afim.                                                   & 20                                                                                         \\ \hline
Atividades não previstas nos outros núcleos na área do curso ou diretamente afim.                 & 15                                                                                         \\ \hline
Atividades de voluntariado                                                                        & 15                                                                                         \\ \hline
Publicações de trabalhos em revistas técnicas/científicas, anais e revistas eletrônicas.          & 20                                                                                         \\ \hline
Viagem / visita técnica na área do Curso ou diretamente afim.                                     & 10                                                                                         \\ \hline
Atividades de extensão (como bolsista ou voluntário) na área do Curso de assistência à comunidade.                              & 10                                                                                         \\ \hline
Congressos ou seminários na área do Curso ou diretamente afim.                                    & 10                                                                                         \\ \hline
Exposição de trabalhos em eventos na área do Curso ou diretamente afim.                           & 20                                                                                         \\ \hline
Núcleos de estudos ou grupos de discussão na área do Curso ou diretamente afim.                   & 10                                                                                         \\ \hline
Membro de diretoria discente ou colegiado acadêmico no IFPB.                                      & 10                                                                                         \\ \hline
\end{tabular}
\end{table}


Para a contabilização das atividades acadêmico-científico-culturais, o aluno do Curso deverá solicitar, por meio de requerimento à Coordenação do Curso, a validação das atividades desenvolvidas com os respectivos documentos comprobatórios. Cada documento apresentado só poderá ser contabilizado uma única vez, ainda que possa ser contemplado em mais de um critério.

A cada período letivo, o Coordenador do Curso determinará os períodos de entrega das solicitações das atividades acadêmico-científico-culturais e de divulgação dos resultados. Este encaminhará os processos aos membros do Colegiado de Curso para análise e apresentação de pareceres que serão avaliados na Plenária do Colegiado. Após a aprovação e computação das horas de atividades acadêmico-científico-culturais pelo Colegiado, o Coordenador do Curso fará o devido registro relativo a cada aluno no Sistema Acadêmico. O Colegiado do Curso pode exigir os documentos que considerar importantes para computação das horas das outras atividades acadêmico-científico-culturais.

Só poderão ser contabilizadas as atividades que forem realizadas no decorrer do período em que o aluno estiver vinculado ao Curso. Os casos omissos e as situações não previstas nessas atividades serão analisados pelo Colegiado do Curso.

\paragraph{Acompanhamento das atividades complementares}\

	
	As atividades complementares serão acompanhadas pelos professores orientadores, Coordenação do Curso, Colegiado do Curso, Coordenação de Pesquisa e Extensão. Caberá:
	
	Ao professor orientador: Submeter, junto à Coordenação do Curso e ao seu Colegiado, projetos de pesquisa e extensão; acompanhar o aluno em suas atividades de iniciação científica, extensão e monitoria; estabelecer convênios interinstitucionais junto à Coordenação do Curso.
	
	À Coordenação do Curso: determinar os períodos de entrega das solicitações das atividades acadêmico-científico-culturais e de divulgação dos resultados; encaminhar os processos aos membros do Colegiado de Curso para análise e apresentação de parecer; fazer o devido registro relativo a cada aluno no Sistema Acadêmico; divulgar editais e eventos relacionados a atividades de pesquisa, extensão e monitoria, junto à Coordenação de Pesquisa e Extensão, por meio impresso e digital; estabelecer convênios interinstitucionais junto ao Colegiado do Curso.
	
	Ao Colegiado do Curso: Desenvolver a política de pesquisa e extensão do Instituto, propondo metas relacionadas ao Curso Superior de Tecnologia em Sistemas para Internet; analisar os projetos de pesquisa e extensão propostos pelos professores do curso; estabelecer convênios interinstitucionais junto à Coordenação de Pesquisa e Extensão; analisar os casos omissos.
	
	À Coordenação de Pesquisa e Extensão: divulgar editais e eventos relacionados a atividades de pesquisa, extensão e monitoria, assim como a política de pesquisa e extensão do Instituto por meio impresso e digital; distribuir bolsas de iniciação científica e de extensão; estabelecer convênios interinstitucionais; disponibilizar recursos para o deslocamento de alunos.

\paragraph{Relevância das atividades complementares}\

	Não é desejável que o estudante do curso Superior de Tecnologia em Sistemas para Internet seja simplesmente convidado a frequentar aulas ministradas segundo os termos universitários comuns, reunindo, por essa maneira, os créditos necessários para o recebimento de um diploma.
	
	Cabe ao estudante a responsabilidade pela busca do conhecimento. A curiosidade e a observação devem ser marcas permanentes do corpo discente. Para tanto, deverá perceber que o aprendizado é um processo e que o profissional do futuro deverá ter a capacidade de aprender a aprender. Deverá ser um estudante a vida toda; ou seja, seu aprendizado será permanente e esta postura deve ser incorporada no processo ensino-aprendizagem desenvolvida no curso.
	
Estas atividades privilegiarão a construção do conhecimento, complementando as atividades acadêmicas tradicionais, desenvolvidas em sala de aula. Diante dessa perspectiva, as atividades acadêmicas complementares têm a finalidade de enriquecer o processo ensino-aprendizagem, privilegiando, portanto, a complementação da formação social, humana e profissional, por meio de atividades de cunho comunitário, de interesse coletivo, de assistência acadêmica, de iniciação científica e tecnológica, como também atividades esportivas e culturais, além de intercâmbio com instituições congêneres.

\subsection{Sistema de Avalia\c{c}\~ao do Curso}

A avaliação da aprendizagem no Curso Superior de Tecnologia em Sistemas para Internet será compreendida como uma prática de investigação processual, diagnóstica, contínua e cumulativa, com a verificação da aprendizagem, análise das dificuldades e redimensionamento do processo ensino/aprendizagem. A avaliação da aprendizagem ocorrerá por meio de instrumentos próprios, buscando detectar o grau de progresso do(a) discente regularmente matriculado(a), realizada ao longo do período letivo, em cada disciplina, compreendendo: 

\begin{itemize}
	\item A apuração de frequência às atividades didáticas;
	\item A avaliação do aproveitamento escolar.
\end{itemize}

 Entende-se por frequência às atividades didáticas o comparecimento do(a) discente às aulas teóricas e práticas, aos estágios supervisionados e aos exercícios de verificação de aprendizagem previstos e realizados na programação da disciplina. 
 
O controle da frequência contabilizará a presença do(a) discente nas atividades programadas, das quais ele(a) estará obrigado(a) a participar de pelo menos 75\% da carga horária prevista na disciplina. O aproveitamento escolar deverá refletir o acompanhamento contínuo do desempenho do(a) discente em todas as atividades didáticas, avaliado por meio de exercícios de verificação. São considerados instrumentos de verificação de aprendizagem:

\begin{itemize}
	\item debates, exercícios, testes ou provas;
	\item trabalhos teórico-práticos, projetos, relatórios e seminários, aplicados individualmente ou em grupos, realizados no período letivo, abrangendo o conteúdo programático desenvolvido em sala de aula ou extra-classe, bem como o exame final. 
\end{itemize}

Os prazos definidos para conclusão e entrega dos exercícios de verificação de aprendizagem serão contabilizados em meses, dias e horas da seguinte forma:

\begin{itemize}
	\item Os prazos fixados em meses contam-se de data a data, expirando no dia de igual número do de início; 
	\item Os prazos expressos em dias contam-se de modo contínuo, expirando à zero hora; 
	\item Os prazos fixados por hora contam-se de minuto a minuto.
\end{itemize}

	As notas relativas às avaliações serão expressas numa escala de zero a 100(cem). Quando ocorrer impedimentos, por motivos de força maior, no cumprimento de prazos relativos ao recebimento por parte do(a) docente e de entrega dos instrumentos de verificação de aprendizagem por parte do(a) discente, antes de expirar o prazo estabelecido em meses ou dias, o(a) docente poderá receber estes exercícios através de protocolo, mesmo fora do horário de sua aula. 

	O(a) docente deverá registrar, sistematicamente, o conteúdo desenvolvido nas aulas, a frequência dos(as) discentes e os resultados de suas avaliações diretamente no Diário de Classe, no Sistema Acadêmico.

	No início do período letivo, o(a) docente informará a seus discentes sobre os critérios de avaliação, a periodicidade dos instrumentos de verificação de aprendizagem, a definição do conteúdo exigido em cada verificação, os quais deverão estar contidos no plano de ensino da disciplina. O(a) professor(a) deverá entregar uma cópia do plano de ensino no início do semestre à Coordenação do Curso. 

	O(a) docente responsável pela disciplina deverá discutir em sala de aula os resultados do instrumento de verificação da aprendizagem no prazo de até 07 (sete) dias úteis após a sua realização. Neste caso, o(a) discente terá direito à informação sobre o resultado obtido em cada instrumento de verificação de aprendizagem realizado, cabendo ao(à) docente da disciplina disponibilizá-los no Sistema Acadêmico ou protocolar, datar, rubricar e providenciar a aposição do documento referente aos resultados do instrumento de verificação de aprendizagem, em local apropriado. 

	O(a) discente que não comparecer às atividades de verificação de aprendizagem programadas, terá direito a apenas um exercício de reposição por disciplina, devendo o conteúdo ser o mesmo da avaliação que está sendo reposta.

	O número de verificações de aprendizagem, durante o semestre, deverá ser no mínimo de: 

\begin{itemize}
	\item 02(duas) verificações para disciplinas com até 50 h; 
	\item 03(três) verificações para disciplinas com mais de 50 h. 
\end{itemize}

	Terá direito a avaliação final o(a) discente que obtiver média igual ou superior a 40 (quarenta) e inferior a 70 (setenta) nos instrumentos de verificação de aprendizagem, além de no mínimo 75\% de frequência na disciplina. A Avaliação Final constará de uma avaliação, após o encerramento do período letivo, abrangendo todo o conteúdo programático da disciplina. As avaliações finais serão realizadas em período definido no Calendário Escolar. Não haverá segunda chamada ou reposição para avaliações finais, exceto no caso decorrente de julgamento de processo e nos casos de licença médica, amparados pelas legislações específicas apontadas no artigo 18 das Normas Didáticas do Ensino Superior do IFPB. O(a) discente que não atingir o mínimo de 40 (quarenta) na média dos instrumento de verificação da aprendizagem, terá a média obtida no semestre como nota final do período. 

	O exame de reposição e a avaliação final deverão ter seus resultados publicados no prazo estabelecido em calendário escolar. Será garantido ao(à) discente o direito de solicitar revisão de instrumento de verificação de aprendizagem escrito, até 2(dois) dias úteis, após a divulgação e revisão dos resultados pelo(a) docente da disciplina, mediante apresentação de requerimento à Coordenação do Curso, especificando o(s) critério(s) não atendidos bem como os itens e aspectos a serem revisados. 
	
	Cada requerimento atende a um pedido único de revisão de verificação de aprendizagem. O pedido será aceito mediante a confirmação de que o(a) requerente participou da aula em que o(a) docente discutiu os resultados do exercício de verificação da aprendizagem, exceto nos casos em que não tenha sido cumprido este requisito. A revisão deverá ser efetivada após os 07 (sete) dias úteis, relativos ao prazo concedido ao(à) docente para discutir em sala de aula os resultados do exercício de verificação da aprendizagem e até 05 (cinco) dias úteis a partir da data da portaria de designação da comissão revisora. 

	A revisão será efetuada por uma comissão, mediante portaria de designação emitida pelo coordenador de curso, formada por 03 (três) membros: o docente da disciplina, 01 (um) docente relacionado com a mesma disciplina ou de disciplina correlata, 01 (um) representante da coordenação pedagógica, acompanhado(a) pelo(a) interessado(a). Em caso de impedimento legal do(a) docente responsável pela disciplina, o(a) Coordenador(a) do Curso designará 02 (dois) docentes relacionados com a mesma disciplina ou de disciplinas correlatas para compor a comissão e proceder a revisão dentro de um prazo máximo de 05 (cinco) dias úteis a partir da data da portaria de designação.  Na ausência do(a) docente responsável pela disciplina, cuja justificativa não encontre respaldo no conjunto destas Normas Didáticas, findo o prazo regimental (§ 3º), a comissão será designada seguindo os mesmos parâmetros do parágrafo anterior.

	O(a) representante da Coordenação Pedagógica conduzirá a reunião de revisão de verificação da aprendizagem, sem direito a voto, mas quando no decorrer do processo ocorrer agravo pessoal para qualquer uma das partes, ele poderá encaminhar a questão para a Câmara de Ensino do Conselho de Ensino, Pesquisa e Extensão – CEPE. O processo de revisão deverá ser pautado apenas sob os aspectos específicos da solicitação do(a) discente. Uma vez concluída a revisão da verificação da aprendizagem segundo os critérios estabelecidos nos artigos anteriores, não será concedido às partes o direito de recurso. 

	As médias semestrais deverão ser registradas no Sistema Acadêmico, observando-se as datas fixadas no Calendário Escolar. Ao final do período letivo o docente deverá imprimir, assinar e encaminhar à Coordenação de Controle Acadêmico – CCA o diário de classe. 

Considerar-se-á aprovado na disciplina o (a) discente que: 

\begin{itemize}
	\item obtiver média semestral igual ou superior a 70(setenta) e frequência igual ou superior a 75\%. 
	\item após avaliação final, obtiver média maior ou igual a 50 (cinquenta). 
\end{itemize}

A média final $(MF)$ das disciplinas serão obtidas por meio da seguinte expressão:

$$ MF = \frac{MS \times 6 + AF \times 4}{10}, $$

~em que $MS$ é a média semestral da disciplina e $AF$ é a nota na avaliação final.

\subsubsection{Sistema de autoavaliação do curso}

O processo de Avaliação Institucional do IFPB é coordenado pela Comissão Própria de Avaliação, observando a Lei de Diretrizes e Bases da Educação Nacional (Lei nº 9.394, de 20/12/1996), nas Diretrizes Curriculares Nacionais de cada curso e na Lei Federal n.º 10.861, de 14 de abril de 2004, que institui o Sistema Nacional de Avaliação da Educação Superior – SINAES. Os procedimentos e processos utilizados na avaliação institucional privilegiam as abordagens qualitativas e quantitativas, contribuindo com a análise e divulgação dos resultados e buscando um sistema integrado de informações acadêmicas e administrativas. Assim, as diretrizes para implantação da Auto Avaliação Institucional no âmbito do IFPB foram elaboradas visando os seguintes objetivos:

\begin{itemize}
	\item Promover o desenvolvimento de uma cultura de avaliação no IFPB;

	\item Implantar um processo contínuo de avaliação institucional;

	\item Planejar e redirecionar as ações da Instituição a partir da avaliação institucional;

	\item Garantir a qualidade no desenvolvimento do ensino, pesquisa e extensão;

	\item Construir um planejamento institucional norteado pela gestão democrática e autônoma;

	\item Consolidar o compromisso social da Instituição;

	\item Consolidar o compromisso científico-cultural do IFPB;

	\item Manter de bancos de dados da instituição, abrangendo informações relativas à avaliação das atividades de ensino, pesquisa e extensão;
	
	\item Apoiar a integração dos sistemas de informação de cada curso e/ ou setor;
	
	\item Criar mecanismos para a divulgação dos resultados obtidos nas avaliações;
	
	\item Utilizar as tecnologias e recursos institucionais para o desenvolvimento das atividades.
\end{itemize}
  
		     O projeto de avaliação interna do IFPB considera as dimensões consideradas na Lei Federal n.º 10.861, de 14 de abril de 2004, que institui o Sistema Nacional de Avaliação da Educação Superior – SINAES, listadas a seguir:

\begin{itemize}
	\item A missão e o plano de desenvolvimento institucional;
	
	\item A política para o ensino, a pesquisa, a pós-graduação, a extensão e as respectivas formas de operacionalização, incluídos os procedimentos para estímulo à produção acadêmica, as bolsas de pesquisa, de monitoria e demais modalidades;

	\item A responsabilidade social da instituição, considerada especialmente no que se refere à sua contribuição em relação à inclusão social, ao desenvolvimento econômico e social, à defesa do meio ambiente, da memória cultural, da produção artística e do patrimônio cultural;

	\item A comunicação com a sociedade;

	\item As políticas de pessoal, as carreiras do corpo docente e do corpo técnico-administrativo, seu aperfeiçoamento, desenvolvimento profissional e suas condições de trabalho;

	\item Organização e gestão da instituição, especialmente o funcionamento e representatividade dos colegiados, sua independência e autonomia na relação com a mantenedora, e a participação dos segmentos da comunidade universitária nos processos decisórios;

	\item Infraestrutura física, especialmente a de ensino e de pesquisa, biblioteca, recursos de informação e comunicação;

	\item Planejamento e avaliação, especialmente os processos, resultados e eficácia da autoavaliação institucional;

	\item Políticas de atendimento aos estudantes;

	\item Sustentabilidade financeira, tendo em vista o significado social da continuidade dos compromissos na oferta da educação superior.

\end{itemize}


       Neste sentido, tendo como base a política de avaliação institucional do IFPB supracitada e que está inserida no corpo do Plano de Desenvolvimento Institucional (2015-2019), o processo de auto-avaliação do Curso Superior de Tecnologia em sistemas para Internet do IFPB-Campus Gurabira será estruturado nas dimensões a seguir:

\begin{itemize}
	\item Avaliação anual do Projeto Pedagógico do Curso (PPC), que terá como finalidade a verificação de possíveis problemas que estejam interferindo na qualidade do curso. Esta avaliação é de fundamental importância porque permitirá a identificação dos problemas, os encaminhamentos necessários e o estabelecimento de estratégias de reformulação curricular e alocação de recursos humanos, pedagógicos e financeiros que possam trazer soluções para os problemas apontados pela comunidade escolar. Este processo de auto-avaliação será coordenado pelo Colegiado do curso, incluindo-se aí a representação estudantil;

	\item Avaliação do curso pelo discente, envolvendo o processo ensino-aprendizagem, além de uma análise sobre a prática dos professores e a infraestrutura do curso, no que se refere aos laboratórios e equipamentos;

	\item Auto-avaliação dos alunos como forma de ensiná-los a refletirem sobre sua prática e compromisso enquanto co-responsáveis pelo processo ensino-aprendizagem;

	\item Uma avaliação do curso feita pelos professores, no que se refere às disciplinas, carga horária, plano de ensino, condições de trabalho, laboratórios e equipamentos;

	\item Avaliação dos profissionais técnicos-administrativos que atuam diretamente com o curso;

	\item Auto-avaliação dos profissionais técnicos-administrativos que atuam diretamente com o curso;

	\item Avaliação acerca da atuação do coordenador do curso;

	\item Análise acerca das condições de funcionamento do colegiado do curso;

	\item Análise acerca do desenvolvimento da pesquisa e da extensão;

	\item Avaliação acerca da biblioteca e refeitório;

	\item Análise acerca da política de assistência estudantil existente na instituição;
\end{itemize}

     Para cada dimensão a ser avaliada será criado um instrumento de coleta de dados a ser definido no colegiado do curso. Ao final do processo de auto-avaliação do curso será gerado um relatório contendo dados conclusivos a serem destinados à Direção Geral do Campus e demais setores da instituição, para a tomada das devidas providências que irão melhorar a qualidade das condições de funcionamento do curso.

\subsubsection{Avaliações oficiais do curso}

As avaliações externas realizadas pelo MEC (avaliações de curso de graduação, ENADE, IDD e CPC), são fontes de realimentação no processo de autoavaliação do Curso.
%Até o presente momento, o curso B não foi contemplado para avaliação ENADE.
