\newpage
\section{Corpo Social do Curso}

\subsection{Corpo Discente}

\subsubsection{Forma de Acesso ao Curso}

	O IFPB, enquanto instituição centenária, mantém-se na linha de discussão para melhoria do Ensino Médio, discutindo a relação entre conteúdos exigidos no ingresso na Educação Superior e habilidades fundamentais para o desempenho acadêmico e para a formação humana. Vale destacar que o IFPB já adotou o resultado do ENEM em seu Processo Seletivo, de forma parcial em 2009 e, desde 2010, o exame já é adotado como critério único de acesso aos cursos superiores.

As vantagens do ENEM revelam:

\begin{itemize}
\item a possibilidade de reestruturação e aperfeiçoamento do Ensino Médio;
\item a ampliação do acesso ao Ensino Superior;
\item a utilização de seus resultados como referência para a melhoria na Educação Básica;
\item a mobilidade do estudante para concorrer em várias instituições;
\item o atendimento às Diretrizes Curriculares Nacionais para o Ensino Médio;
\item a aplicação de provas contextualizadas que colocam o estudante diante de situações-problema que exigem além dos conceitos aprendidos, que o estudante demonstre sua aplicação.
\end{itemize}

	A admissão aos Cursos de Graduação no IFPB dar-se-á mediante processo seletivo, no período previsto em Edital Público, nas seguintes modalidades:

\begin{itemize}
	\item Via ENEM, destinado aos concluintes do Ensino Médio;

	\item Transferência Escolar Voluntária, destinado a discentes oriundos de outros cursos regulares de graduação, de mesma área ou área afim, ofertados por Instituições de Ensino Superior devidamente credenciadas;

	\item Ingresso de Graduados, destinada a discentes com diploma de cursos afins, emitidos por Instituições de Ensino Superior devidamente credenciadas;

	\item Reingresso destinado a discentes que tiveram sua matrícula cancelada em cursos de graduação regulares do IFPB nos últimos 05 (cinco) anos;

	\item Reopção, destinada a discentes regularmente matriculados em cursos de graduação no IFPB que desejam mudar de curso.
\end{itemize}

	As normas, critérios de seleção, programas e documentação dos processos seletivos para os Cursos de Graduação constarão em edital próprio aprovado pelo Reitor.
	
\paragraph{Modalidades de Ingresso Extra-ENEM}\

	O reingresso é a possibilidade dos discentes que perderam o vínculo com o IFPB, por abandono ou jubilamento, de reingressar na instituição, a fim de integralizar o seu currículo, conforme a oferta de vagas com esta finalidade no período e no curso pretendido.

	O reingresso somente poderá ser autorizado uma única vez e para o seu curso de origem. Somente serão apreciados os requerimentos de reingresso de ex-discentes que se enquadrem nas seguintes situações:

\begin{itemize}
	\item não ter sido reintegrado anteriormente;

 	\item não estar cursando qualquer curso do IFPB;

 	\item ter aprovação em todas as disciplinas exigidas para o 1º período do curso;

 	\item não ter sido reprovado 4 (quatro) vezes em uma ou mais disciplinas;

	\item não terem decorrido mais de 5 (cinco) anos, desde a interrupção do curso até o período pretendido para o reingresso.
\end{itemize}

	O reingresso condiciona, obrigatoriamente, o discente ao currículo e regime acadêmico vigente, não se admitindo, em nenhuma hipótese, complementação de carga horária em disciplinas do vínculo anterior. Será concedido ao aluno um período letivo adicional para promover a adaptação curricular.

	Para efeito de Colação de Grau dos discentes que perderam o vínculo, em período não superior a 5 (cinco) anos e que deviam apenas apresentar o Trabalho de Conclusão de Curso - TCC - ou o relatório de estágio curricular obrigatório, o reingresso poderá ser solicitado a qualquer momento, independente de prazo previsto em calendário acadêmico. Neste caso, o candidato deve protocolar uma declaração do Professor Orientador, informando o período e carga horária do estágio (no caso de estágio curricular) ou uma declaração do Professor Orientador que o aluno concluiu o TCC. Uma vez requerido o reingresso especial, o Departamento de Ensino Superior autorizará a Coordenação de Controle Acadêmico (CCA) a matricular o discente na disciplina específica, apenas para registrar a respectiva nota, emitir o Histórico Escolar de Conclusão e providenciar a Colação de Grau em separado.

	O processo de Transferência Escolar Voluntária destina-se aos discentes regularmente vinculados a curso de graduação devidamente reconhecido ou autorizado pelo MEC, mantido por instituição nacional de ensino superior credenciada, que tenham acumulado, na instituição de origem, um total de, no mínimo, 300 (trezentas) horas em disciplinas, que não tenha superado o prazo de 50\% do tempo máximo estabelecido para sua integralização. 
	
	A Transferência Escolar Voluntária poderá ser aceita pelo IFPB, para prosseguimento dos estudos no mesmo curso ao qual estava vinculado ou, quando este não existe, em curso afim, conforme a oferta de vagas com esta finalidade no período e no curso pretendido ou curso afim. A afinidade do curso será considerada quando houver afinidade curricular na formação básica, diferenciando na formação profissional. No caso de dúvida na interpretação sobre afinidade de curso, a questão deve ser encaminhada ao Colegiado do Curso, que deve emitir parecer até o prazo da matrícula.

	Somente serão apreciados os requerimentos de transferência de discentes de outra IES que se enquadrem nas seguintes situações:

	\begin{itemize}
		\item ter cursado, com aprovação, todas as disciplinas exigidas para o 1º período do curso de origem;
		\item não tiver sido desligado de um curso de graduação do IFPB;
		\item não apresentar um número igual ou superior a 3 (três) reprovações em uma mesma disciplina no curso de origem.
	\end{itemize}
	
		O processo de Ingresso de graduados possibilita ao portador de Diploma de Curso de Graduação emitido por uma IES brasileira, devidamente credenciada, e reconhecido pelo MEC, ou de instituições estrangeiras devidamente reconhecidas no seu país de origem, requerer sua admissão em curso afim ao de origem, conforme a oferta de vagas com esta finalidade no período e no curso pretendido. Só será permitido o ingresso por meio desta modalidade uma única vez no IFPB. 
		
		Só serão analisados os requerimentos de portadores de diploma que se enquadrem nas seguintes situações:
		\begin{itemize}
			\item Estar de posse do Diploma devidamente registrado, na forma da Lei;
			\item Não tiver sido desligado de um Curso de Graduação do IFPB.
		\end{itemize}

	A Reopção ou Transferência Interna oportuniza ao discente regularmente matriculado em um curso de graduação do IFPB, que tenha acumulado, no curso de origem, um total de, no mínimo, 300 (trezentas) horas em disciplinas, que não tenha superado o prazo de 50\% do tempo máximo estabelecido para sua integralização, a transferência ou mudança interna de seu curso de origem para outro curso afim, conforme a oferta de vagas com esta finalidade no período e no curso pretendido.

	A Reopção só será concedida uma única vez ao discente, sendo vedado o retorno ao curso de origem. A afinidade do curso será considerada quando houver afinidade curricular na formação básica, diferenciando na formação profissional. No caso de dúvida na interpretação sobre afinidade de curso, a questão deve ser encaminhada ao Colegiado do Curso, que deve emitir parecer até o prazo da matrícula.

	Somente serão apreciados os requerimentos de Reopção de discentes do IFPB que se enquadrem nas seguintes situações:
	\begin{itemize}
		\item ter cursado, com aprovação, todas as disciplinas exigidas para os 1º e 2º períodos do curso de origem;

		\item ter ingressado no IFPB através do ENEM;

		\item não ter se beneficiado de ingresso Extra-ENEM (em quaisquer das modalidades);
		
		\item estar regularmente matriculado no período em curso ao do pleito;
		
		\item não apresentar um número igual ou superior a 3 (três) reprovações em uma mesma disciplina.
	\end{itemize}

	Em caso de Transferência a partir de um Curso de Graduação que está apenas autorizado, mas não reconhecido pelo MEC, o discente somente poderá fazer jus ao diploma devidamente registrado após o reconhecimento do curso de origem. Caso o reconhecimento do curso de origem tenha sido negado, para a obtenção do diploma, o discente deverá prestar exames de convalidação das disciplinas do curso de origem, que tenham sido objeto de adaptação curricular, sido creditadas ou dispensadas. Se as disciplinas mencionadas no parágrafo anterior forem novamente cursadas, em curso reconhecido pelo MEC, não será necessária a referida convalidação.

	%A Comissão Permanente de Concurso – COMPEC é o órgão responsável pela execução do Processo Seletivo de que trata esta Resolução.
\vspace{4mm}
\textbf{VAGAS}
\vspace{4mm}

	Na definição do número máximo de vagas de cada curso para o processo Extra-ENEM serão considerados os seguintes números:
	\begin{itemize}

		\item Número Total de Vagas de um Curso (TV) – obtido pela multiplicação do número de vagas oferecidas no ENEM pela duração mínima de integralização curricular do curso (em períodos);

		\item Número de Ocupantes do Curso (NO) – determinado pelo somatório do número de matriculados em todos os períodos do curso, considerando todos os discentes regularmente matriculados e os que estejam com trancamento de período/matrícula, excetuando-se os que tenham ingressado por Transferência ex-officio.

		\item Número de Vagas Ociosas de um curso (VO) – é determinada pela diferença entre o Número Total de Vagas de um Curso (TV) e o Número de Ocupantes do Curso (NO), (VO = TV – NO).

	\end{itemize}
	
	Na hipótese de o número de ocupantes do curso ser maior ou igual ao número total de vagas do curso, fica estabelecida a inexistência de vagas ociosas no curso.

	Quando se tratar de um curso novo, que ainda não completou o prazo total de integralização curricular, o somatório das vagas será feito no limite dos períodos efetivamente implantados. Se ocorrer alteração de vagas ofertadas no processo seletivo de um curso, o cálculo de vagas ociosas deverá ser feito considerando o novo número de vagas.

	Considera-se discente vinculado a um curso aquele que, de acordo com as normas vigentes, não tenha sido desligado deste.

	Curso em processo de desativação ou extinção não oferecerá vagas para o processo Extra-ENEM.

	O Departamento de Ensino Superior disponibilizará para cada curso o Número de Vagas Ociosas (VO), de acordo com o critério explicado nessa subseção, que servirá de parâmetro de referência sobre a oferta de vagas para o processo Extra-ENEM. O Número de Vagas Ociosas (VO) será limitado ao número de vagas oferecidas, por período, no último ENEM realizado para o curso.

	O Colegiado do Curso poderá sugerir à Diretoria de Ensino, mediante justificativa fundamentada, o número de vagas que o Curso poderá oferecer, levando em conta as especificidades do Curso e as condições materiais, infra-estruturais e humanas disponíveis, observado o limite mínimo de 20\% em relação ao Número de Vagas Ociosas (VO). Caberá à Diretoria de Ensino, após a análise das sugestões e das justificativas apresentadas pelo Colegiado do Curso, a definição do número de vagas a serem oferecidas pelo Curso para a seleção Extra-ENEM em cada uma das modalidades, observado o disposto na presente Resolução.

	Na aplicação do percentual de que trata o caput deste artigo, não será considerada a fração inferior a 0,5 (zero vírgula cinco) e será arredondada para maior a fração igual ou superior a 0,5 (zero vírgula cinco).

	A Diretoria de Ensino fará publicar o Edital de Ingresso Extra-ENEM, no período previsto no Calendário Acadêmico. No Edital de Ingresso Extra-ENEM deverão constar: datas e local do protocolo do requerimento de ingresso, número de vagas ofertadas por curso para cada modalidade, relação de documentos a serem apresentados pelos candidatos, critérios e data da seleção, data e local de divulgação dos resultados.

\vspace{4mm}
\textbf{DISTRIBUIÇÃO DAS VAGAS OCIOSAS}
\vspace{4mm}

	Quando verificada a existência de vagas ociosas em cursos de graduação, as vagas deverão ser destinadas ao Processo Seletivo Extra-ENEM, e distribuídas de acordo com as seguintes prioridades e proporcionalidades:

\begin{itemize}
	\item Para Reingresso de ex-discente do IFPB (Reingresso) – 20\% das vagas;
	\item Para Reopção de Curso – 30\% das vagas;
	\item Para Transferência de discente de Curso de Graduação de outra Instituição de Ensino de mesmo curso ou curso afim – 40\% das vagas;
	\item Para Ingresso de Graduados – 10\% das vagas.
\end{itemize}

	A admissão para cada uma das modalidades, para o mesmo curso ou cursos afins, dar-se-á por meio de Processo Seletivo, realizado semestralmente, destinado à classificação de candidatos, até o limite de vagas oferecidas, para ingresso no período letivo seguinte ao da seleção.

	No cálculo do número de vagas por modalidade de ingresso, conforme estabelecido na subseção sobre as vagas, os resultados deverão ser apresentados em números inteiros, arredondando-se as frações decimais para o número inteiro consecutivo. Concluído o processo de arredondamento do número de vagas e ocorrendo desigualdade de resultados no cômputo do número total de vagas por curso, prevalecerá o resultado calculado após o processo de arredondamento.

	As vagas não aproveitadas em uma modalidade, por falta de candidatos inscritos ou legalmente habilitados, deverão ser remanejadas e destinadas à modalidade seguinte, observada a ordem de prioridade definida nessa seção. Caso ainda restem vagas remanescentes, após a distribuição definida anteriormente ou em decorrência de desistência ou o não comparecimento à matrícula dos candidatos classificados, estas deverão ser destinadas aos candidatos Portadores de Diploma de Curso de Graduação afim, desde que haja prazo hábil para o chamamento e matrícula dos candidatos pela Coordenação de Controle Acadêmico – CCA.

	As transferências ex-officio são regidas por legislação federal específica e ocorrem independentemente da existência de vagas nos cursos, em qualquer época do ano.

\vspace{4mm}
\textbf{INSCRIÇÃO}
\vspace{4mm}

	Em cada período letivo, o prazo destinado à inscrição para a seleção extra-ENEM será definido no Calendário Escolar. A inscrição será aberta por Edital, que especificará os documentos necessários à sua efetivação, entre outras instruções complementares, discriminação dos cursos com o respectivo número de vagas e os locais e horários de inscrição. Para requerer a inscrição, o candidato poderá ser representado por seu procurador legalmente constituído. Serão indeferidos os requerimentos de inscrição que não apresentarem a documentação exigida.

	Ao inscrever-se, o candidato firmará declaração de que aceita as condições estabelecidas neste PPC e no Edital de Inscrição.

\vspace{4mm}
\textbf{CLASSIFICAÇÃO}\
\vspace{4mm}

	A classificação final dos candidatos dar-se-á da seguinte forma:

	Procede-se à classificação dos candidatos, na ordem decrescente da média ponderada (Mp) obtida da seguinte forma:

$$ 
Mp = \frac{(CRE*7) + (RA*3)}{10}
$$
~Em que:

CRE = Coeficiente de Rendimento Escolar, definido numa escala de 0 a 100 (cem) pontos;

RA = Resultado da avaliação aplicada quando da seleção.

	No caso da não aplicação de avaliação, RA corresponderá a soma da pontuação do vestibular, definido na escala de 0 a 100 (cem) pontos. Nesse caso, o valor de RA será dado como uma proporção em relação à pontuação máxima.

	A classificação obedecerá ao limite das vagas fixadas no edital. No caso de empate na disputa pela última vaga, será classificado o candidato proveniente de instituição de ensino superior pública. Persistindo o empate, será classificado o candidato que apresentar o maior Coeficiente de Rendimento Escolar, seguido pelo critério da maior idade. O Coeficiente de Rendimento Escolar - CRE de discentes de cursos de graduação é definido como segue:

$$
CRE = \frac{\sum_{i} N_i \times H_i}{H_t}
$$
~Em que:

$N_i$ = Nota da i-\'esima disciplina

$H_i$ = Carga Horária da i-\'esima disciplina

$H_t$ = Carga Hor\'aria total $(H_t = \sum_{i} H_i)$

	Não são consideradas no cálculo do CRE as disciplinas trancadas, aproveitamento de disciplina, disciplina excluída, aceleração de estudos, disciplina dispensada e disciplinas em curso. As notas devem ser consideradas numa escala de 0 – 100 (cem). No caso de histórico escolar emitido por outra instituição de ensino que adote avaliação final numérica diferente da escala de 0 a 100 (cem), far-se-á a conversão proporcional para essa escala. Se a média final da disciplina constante do histórico escolar não for numérica, mas corresponder a intervalo numérico, ela será considerada como a média aritmética do intervalo e será expressa com uma casa decimal.

	%Em virtude da natureza do ENEM, não será permitido revisão ou recontagem de pontos.

\vspace{4mm}
\textbf{MATRÍCULA}
\vspace{4mm}

	A matrícula somente se dará no curso e turno para o qual o candidato foi classificado. A matrícula dos candidatos classificados, nos respectivos cursos, será efetuada pelo candidato ou seu procurador legalmente constituído, em duas etapas. 
	\begin{itemize}
		\item Na primeira etapa, o cadastramento, nos setores competentes, para fins de vinculação ao IFPB, gerando um correspondente número de matrícula.
		\item Na segunda etapa, a matrícula em disciplinas, na Coordenação do Curso correspondente.
	\end{itemize}

	O cadastramento é obrigatório, qualquer que tenha sido a opção de curso em que o candidato tenha obtido classificação, sob pena de perda do direito aos resultados dessa classificação, no ENEM. A matrícula em disciplinas só poderá ser realizada pelo candidato que tenha efetuado seu cadastramento.

	Perderá o direito à classificação obtida no ENEM e, consequentemente, à vaga no curso, o candidato que não apresentar a documentação exigida, nos termos do Edital do Processo Seletivo Extra-ENEM.

	As vagas que venham ocorrer após o cadastramento serão preenchidas pela classificação de candidatos, observado o disposto na subseção sobre classificação.

\subsubsection{Trancamento e Reabertura de Matrícula}

	As condições em que o discente pode requerer o trancamento e/ou a reabertura de matrícula estão enumeradas a seguir:

\begin{itemize}	
	\item O trancamento da matrícula em disciplinas será concedido mediante requerimento à Coordenação do Curso, até 45 (quarenta e cinco) dias corridos após o início do período letivo.

	\item O trancamento de uma mesma disciplina poderá ocorrer, no máximo, 02 (duas) vezes.

	\item Não será permitido o trancamento de disciplinas na blocagem oferecida no primeiro período, exceto nos seguintes casos:
	\begin{itemize}
		\item doença prolongada;
		\item convocação para o Serviço Militar;
		\item gravidez de risco;
		\item motivo de trabalho;
		\item mudança de domicílio para outro município ou unidade da federação; 
		\item acompanhamento do(a) cônjuge ou genitores.
	\end{itemize}
\end{itemize}

	O trancamento da matrícula no período letivo será concedido mediante requerimento à Coordenação do Curso, até 45 (quarenta e cinco) dias corridos após o início do período letivo. O trancamento em todo o conjunto de disciplinas matriculadas num período letivo é caracterizado como trancamento do período. O trancamento do período letivo poderá ocorrer, no máximo, 02 (duas) vezes não consecutivas. O discente não poderá requerer trancamento do período após reprovação em todas as disciplinas em que foi matriculado no período cursado anteriormente.

	O trancamento total de matrícula no período letivo não é computado no prazo máximo, fixado para integralização do respectivo curso. Não será permitido o trancamento do primeiro período letivo, exceto nos casos já listados nessa seção, no parágrafo que trata do trancamento de disciplinas na blocagem oferecida no primeiro período.

	Decorrido o prazo referente ao trancamento, o discente deverá solicitar a reabertura da matrícula, via requerimento encaminhado à coordenação do curso, protocolado em período anterior à data definida pelo IFPB para o início da matrícula. A não solicitação de reabertura de matrícula após trancamento caracteriza a situação de abandono de curso e a consequente perda da vaga.

\subsubsection{Aproveitamento de Estudos}

	Os discentes devidamente matriculados em curso de graduação do IFPB poderão solicitar reconhecimento de competências/conhecimentos, bem como aproveitamento de estudo adquiridos para fins de abreviação do tempo de integralização de seu curso.
	
	O Curso Superior de Tecnologia em Sistemas para Internet oportunizará o aproveitamento de estudos e certificará conhecimentos e experiências adquiridas na educação profissional e fora do ambiente escolar mediante avaliação, possibilitando o prosseguimento ou conclusão de estudos, conforme artigo 41 da LDB no 9394/1996.
	
Será assegurado o direito ao aproveitamento de estudos realizados ao(à) discente que:
	\begin{itemize}
		\item for classificado em novo Concurso Vestibular;
		\item tenha efetuado reopção de curso;
		\item tenha sido transferido;
		\item tenha reingressado no curso;
		\item ingressar como graduado;
		\item tenha cursado com aproveitamento a mesma disciplina ou equivalente em outro curso de graduação de outra Instituição, devidamente reconhecido.
		\end{itemize}
		
		Para o aproveitamento de estudos de componentes/disciplinas de uma matriz curricular para outra deve levar em conta os critérios.
	\begin{itemize}
		\item equivalência de conteúdos;
		\item objetivos da disciplina;
		\item atualização dos conhecimentos;
		\item condições de oferta e desenvolvimento;
		\item correspondência de no mínimo 90\% da carga horária exigida.
	\end{itemize}

	As normas mais específicas quanto aos critérios de aproveitamento e procedimentos de avaliação de competências profissionais anteriormente desenvolvidas pelo discente estão constantes nas Normas dos Cursos Superiores oferecidos pelo IFPB e nas demais resoluções que tratam do tema.
	
\subsubsection{Reconhecimento de Competências/Conhecimentos Adquiridos}

	Os discentes devidamente matriculados em curso de graduação do IFPB poderão solicitar reconhecimento de competências/conhecimentos adquiridos para fins de abreviação do tempo de integralização de seu curso. O reconhecimento de competências/conhecimentos adquiridos far-se-á mediante exames a serem prestados pelo interessado, nas épocas apropriadas, previstas no calendário acadêmico, desde que tenha seu pedido aceito.
	
	A avaliação do processo de reconhecimento de competência/conhecimento será realizada semestralmente, de acordo com as condições estabelecidas em Edital específico da Coordenação do Curso. 
	
	%Para efeito de reconhecimento de competências/conhecimentos adquiridos, as disciplinas dos cursos são divididas em:
	
	%\begin{itemize}
	%	\item Grupo I - Disciplinas Básicas, Científicas e Instrumentais: Disciplinas de formação geral pertencente à base de conhecimentos do curso.
	
	%	\item Grupo II - Disciplinas Tecnológicas: Disciplinas do núcleo específicos do curso e que aprofundam conhecimentos na área de formação.
%	\end{itemize}
	
%	As disciplinas são identificadas em cada grupo no Projeto Pedagógico do Curso e no Edital específico, emitido pela Coordenação do Curso.
	
	O reconhecimento de competências/conhecimentos adquiridos será realizado por disciplina, sendo a solicitação e avaliação realizada no período imediatamente anterior ao da sugestão de blocagem da disciplina. Não será permitido reconhecimento de competências/conhecimentos adquiridos correlatas às disciplinas da blocagem do primeiro período do curso.
	
	 O reconhecimento de competências/conhecimentos adquiridos será permitido uma única vez por disciplina, desde que o(a) discente não tenha sido reprovado(a) ou trancado a mesma. O reconhecimento de competências/conhecimentos adquiridos não se aplica ao Trabalho de Conclusão de Curso – TCC nem ao Estágio Supervisionado, ambos com regulação própria.
	
	Para cada disciplina será composta uma banca avaliadora, formada por 03 (três) professores, presidida pelo professor da disciplina no semestre em questão. A banca avaliadora será responsável pela elaboração dos instrumentos de avaliação apropriados, bem como pelo procedimento a ser adotado que pode incluir provas práticas e/ou teóricas. A avaliação deve ser realizada de forma individual e levar em consideração aspectos quantitativos e qualitativos da formação do aluno na matéria em questão.
	
	Será aprovado o aluno que tiver desempenho igual ou superior a 70 (setenta).
	
	Para a inscrição no processo de reconhecimento de competências/conhecimentos para as disciplinas básicas, científicas e instrumentais (disciplinas que não são da área de informática), o discente deve protocolar requerimento à Coordenação do Curso, no período previsto no Edital específico, devendo anexar ao requerimento os documentos que comprovem seu aproveitamento em disciplinas equivalentes ou afins daquela que está solicitando o reconhecimento de competências/conhecimentos adquiridos.
	
	Para comprovação do seu extraordinário desempenho na área de conhecimento, o discente deve comprovar exames de proficiência, histórico escolar de séries anteriores, certificados de conclusão de cursos relacionados à matéria, todos com excelente desempenho ou outros documentos que atestem sua competência na área;
	
	O coordenador do curso deve encaminhar a solicitação à banca avaliadora de cada disciplina, devendo a mesma se responsabilizar, com base na documentação apresentada, pela seleção inicial dos alunos que serão submetidos à avaliação num prazo máximo de 15 (quinze) dias. Somente terão direito a participar da avaliação os(as) discentes que comprovarem, através de documentos, que possuem competências na área da disciplina solicitada;

	Após a seleção inicial, a Coordenação do Curso publicará uma relação dos alunos selecionados para o processo de reconhecimento de competências/conhecimentos adquiridos, devendo também conter o local e horário da avaliação de cada disciplina.
	
	Para a inscrição no processo de reconhecimento de competências/conhecimentos das disciplinas do núcleo específico do curso (disciplinas da área de informática), o discente deve protocolar requerimento à Coordenação do Curso, no período previsto no Edital específico, devendo anexar ao requerimento os documentos que comprovem sua experiência profissional na área de estudo ou afins da que está solicitando o reconhecimento de competências/conhecimentos adquiridos.
	
	Para comprovação da experiência profissional na área, o discente deve comprovar através de diplomas de cursos técnicos ou superiores, certificados de cursos extracurriculares, certificados de participação em treinamentos ou cursos de qualificação, declarações de empresas, descritivos de função, contratos de trabalho, anotações de responsabilidade técnica ou outros documentos que atestem sua competência na área em avaliação.
	
	O coordenador do curso deve encaminhar a solicitação à banca avaliadora de cada disciplina, devendo a mesma se responsabilizar, com base na documentação apresentada, pela seleção inicial dos alunos que serão submetidos à avaliação num prazo máximo de 15 (quinze) dias. Somente terão direito a participar da avaliação os(as) discentes que comprovarem, através de documentos, que possuem competências na área da disciplina solicitada;
	
	Após a seleção inicial, a Coordenação do Curso publicará uma relação dos alunos selecionados para o processo de reconhecimento de competências/conhecimentos adquiridos, devendo também conter o local e horário da avaliação de cada disciplina. Após a avaliação, a banca avaliadora deve encaminhar à Coordenação do Curso, no prazo máximo de 5 (cinco) dias úteis, o resultado, em ficha individual assinada por todos os membros da banca.
	
	A Coordenação do Curso será responsável pela inserção do resultado no Sistema Acadêmico, o que deve ocorrer até o final do período letivo previsto no calendário acadêmico. Somente serão inseridos os resultados dos discentes aprovados. O resultado obtido no processo de reconhecimento de competências/conhecimentos adquiridos não será computado no Coeficiente de Rendimento Escolar – CRE do discente.
	
\subsubsection{Desligamento de Alunos}

	O discente regularmente matriculado nos cursos de graduação do IFPB, pode ter interrompido seu vínculo com o curso e, consequentemente com a instituição, quando o mesmo se encontrar nas seguintes situações:
	\begin{itemize}
		\item cancelamento de matrícula;

		\item cancelamento voluntário de matrícula; 
	
		\item jubilamento.
	\end{itemize}

	O cancelamento de matrícula ocorrerá nos seguintes casos:

	\begin{itemize}
		\item O discente com reprovação total em até 02 (dois) períodos letivos consecutivos perde o direito à vaga, ficando impedido de renovar a matrícula, entrando em processo de cancelamento da mesma.

		\item O discente com 4 (quatro) reprovações na mesma disciplina e com coeficiente de rendimento escolar inferior a 4,0 (quatro).

		\item O discente enquadrado na situação de abandono de matrícula.

		\item Considera-se abandono de matrícula quando o discente não efetuar o pedido de matrícula on-line em disciplina no prazo previsto no Calendário Acadêmico, por qualquer que seja o motivo, e não solicitá-la processualmente ou não requerer trancamento ou interrupção de estudos.
		\end{itemize}
		
		Excetuam-se os estudantes que estão com seu vínculo suspenso por interrupção de estudos.

		Cabe à Coordenação do Curso informar ao Departamento de Ensino Superior do Campus em que o mesmo está vinculado, no prazo de 20 (vinte) dias após o início do período letivo, a relação de estudantes que se enquadram na situação de abandono. O Departamento de Ensino Superior, em conjunto com a Diretoria do Campus, publicará um Edital constando a relação nominal dos discentes que terão sua matrícula cancelada por abandono, fixando um prazo para que os mesmos apresentem sua defesa. O discente também será comunicado por correspondência que está incluso no processo de cancelamento de matrícula por abandono. A correspondência será enviada ao endereço constante no seu cadastro do Sistema Acadêmico, cuja atualização é de responsabilidade de cada estudante.

		Para sua defesa, o discente deve protocolar no período previsto no Edital, toda documentação que comprove as causas alegadas para a não solicitação da matrícula, bem como a proposta para continuidade do curso, com disciplinas e horários em cada semestre, caso seja concedida a prorrogação do prazo.

		O julgamento do pedido de reconsideração, caso ocorra, será de responsabilidade do Colegiado do Curso em que o mesmo está vinculado, em reunião convocada especialmente para este fim. A deliberação do Colegiado do Curso, em ficha individual, assinada pelos membros do Colegiado, será enviada ao Departamento de Ensino Superior para processamento e comunicação ao discente.

		Cabe recurso das decisões do Colegiado a Câmara de Ensino do Conselho de Ensino, Pesquisa e Extensão do IFPB – CEPE.

		Caso o discente tenha sua justificativa aceita, o mesmo ficará o restante do semestre na condição de interrupção de estudos, devendo se matricular em disciplinas apenas no semestre seguinte, onde o mesmo não terá mais direito a recorrer em caso de não solicitação de matrícula.

		O Departamento de Ensino Superior informará a Coordenação de Controle Acadêmico – CCA sobre a situação do discente e esta ficará responsável pelo processamento final do processo.

		O cancelamento voluntário de matrícula ocorre em qualquer período, por vontade do discente, manifestada por meio de um requerimento dirigido à CCA.

		A CCA efetuará o cancelamento da matrícula, emitindo um histórico escolar atualizado, que será entregue ao mesmo, e informará a Coordenação do respectivo Curso sobre o cancelamento voluntário da matrícula.

		Jubilamento é o desligamento do IFPB de discentes que ultrapassarem o prazo máximo de tempo para a conclusão de seus cursos, contados a partir da 1a matrícula.
		
		Quanto ao jubilamento, são identificadas duas situações: 
		\begin{itemize}
			\item Discentes em risco de jubilamento;

			\item Discentes em processo de jubilamento.
		\end{itemize}
		
		Considera-se em risco de jubilamento o discente a quem resta, apenas, um período letivo para completar o prazo limite para integralização do curso. Para efeito de contagem de tempo de integralização, considera-se o período decorrido desde a matrícula inicial do discente na instituição, excetuando-se o período de trancamento. Para os alunos que fizeram reopção de curso, conta-se o período a partir da matrícula inicial, mesmo que o aluno passe a ser vinculado à outra turma em semestre distinto;

		No ato da matrícula do último período referente ao tempo máximo de integralização do curso, o discente será informado do risco de jubilamento, assinando um termo de conhecimento referente à sua situação e sendo informado que terá sua matrícula bloqueada no período seguinte.
		
		Considera-se em processo de jubilamento o discente que não concluiu o curso no prazo máximo previsto no Projeto Pedagógico do Curso. Para efeito de contagem de tempo de integralização, considera-se o tempo decorrido desde a matrícula inicial do aluno na instituição, excetuando-se o período de trancamento. Para os alunos que fizeram reopção de curso, conta-se o período a partir da matrícula inicial, mesmo que o aluno passe a ser vinculado à outra turma em semestre distinto;

		No prazo máximo de 20 (vinte) dias decorridos do encerramento do período letivo, cada coordenação elaborará uma relação nominal dos alunos que não integralizaram o curso no prazo máximo, encaminhando a mesma para o Departamento de Ensino Superior. o Departamento de Ensino Superior, em conjunto com a Diretoria do Campus publicará um Edital constando a relação nominal dos discentes que terão sua matrícula cancelada por jubilamento, fixando um prazo para que o mesmo apresente sua defesa.

		Ao discente também será comunicado por correspondência que o mesmo está incluso no processo de cancelamento de matrícula por jubilamento. A correspondência será enviada ao endereço constante no seu cadastro do Sistema Acadêmico, cuja atualização é de responsabilidade do aluno.

		Para sua defesa, o discente deve protocolar no período previsto no Edital, toda documentação que comprove as causas alegadas para a não solicitação da matrícula, bem como a proposta para continuidade do curso, com disciplinas e horários em cada semestre, caso seja concedida a prorrogação do prazo.

		O julgamento do pedido de reconsideração, caso ocorra, será de responsabilidade do Colegiado do Curso em que o mesmo está vinculado, em reunião convocada especialmente para este fim. A deliberação do Colegiado do Curso, em ficha individual, assinada pelos membros do Colegiado, será enviada ao Departamento de Ensino Superior para processamento e comunicação ao aluno.

		Cabe recurso das decisões do Colegiado a Câmara de Ensino do Conselho de Ensino, Pesquisa e Extensão do IFPB -CEPE;

		Caso o Colegiado tenha deliberado por prorrogar o prazo de integralização, o aluno deve comparecer ao Departamento de Ensino Superior e assinar um Termo de Compromisso, pelo qual se compromete a concluir o curso no prazo fixado pelo Colegiado. Neste caso, a coordenação do curso ficará responsável pela matricula em disciplinas no período.

		Cada processo deve ser avaliado individualmente pelo colegiado do curso, tendo como base os seguintes aspectos:

		\begin{itemize}
			\item Histórico Acadêmico do Aluno;

			\item Problemas de saúde;

			\item Limitações por dificuldade de aprendizagem;

			\item Convocações para Serviço Militar;

			\item Questões relativas a trabalho;

			\item Outros aspectos relevantes.
		\end{itemize}

		O aluno jubilado ou que teve sua matrícula cancelada poderá solicitar à CCA o histórico escolar parcial, em que constem as disciplinas cursadas, visando a futuro aproveitamento de estudos daquelas disciplinas em que foi aprovado. 
		
		Os casos omissos serão resolvidos pela Câmara de Ensino do Conselho de Ensino, Pesquisa e Extensão do IFPB - CEPE.

\subsubsection{Certidões e Diplomas}

	O diploma só poderá ser emitido após reconhecimento do curso de Graduação pelos órgãos competentes.

	Após a Colação de Grau, a Coordenação de Controle Acadêmico – CCA dará início ao processo de emissão de diplomas. No ato de Colação de Grau, o graduando recebe o Certificado de Conclusão de Curso. A Coordenação de Controle Acadêmico – CCA encaminhará os processos dos graduados devidamente instruídos ao Serviço de Registro de Diplomas Credenciado, para fins de registro do diploma.

\subsubsection{Aten\c{c}\~ao aos discentes}

	O Curso de Tecnologia em Sistemas para Internet oferecerá atendimento diário aos acadêmicos, professores e comunidade em geral, por intermédio de sua Coordenação do Curso no período 08h às 12h e 14h às 18h, de segunda a sexta. O acadêmico contará também com atendimento da biblioteca das 08h00min às 20h00min durante a semana.

\paragraph{Apoio ao discente}\

       No intuito de minimizar o processo de evasão e retenção o IFPB desenvolve programas de natureza assistencial, estimulando a permanência do aluno no convívio escolar. Os principais são: 

\begin{itemize}
\item Programas de apoio a permanência na Instituição;

\item Programas de natureza pedagógica para minimizar o processo de evasão e reprovação escolar;

\item Programa de bolsas, atendendo à política de ensino, pesquisa e extensão;

\item Programa de educação inclusiva;

\item Programa de atualização para o mundo do trabalho.
\end{itemize}

      Atualmente o IFPB - Campus Guarabira conta com uma equipe multidisciplinar qualificada contando com pedagogo, técnico em assuntos educacionais e assistente social, além de infraestrutura adequada com Gabinete Médico, contando com um m\'edico e um t\'ecnico em enfermagem, Biblioteca, Núcleos de Aprendizagem e Laboratórios. Há que se destacar ainda, a formação dos Conselhos Escolares e o desenvolvimento de atividades esportivas e culturais. O corpo de profissionais do IFPB Guarabira est\'a em expans\~ao e outros profissionais ser\~ao integrados ao corpo de trabalho do campus, como \'e o caso de um profissional de psicologia e um profissional de odontologia.

       %Neste sentido, o IFPB - Campus Guarabira coloca a disposição da comunidade escolar os turnos da manhã, tarde e noite, para atendimento psicopedagógico aos alunos. Os horários de atendimento são os seguintes: das 7:30 ás 11: 30, das 13: 30 ás 17:30 e das 18:00 ás 22:00. Para esse trabalho a instituição possui um quadro de profissionais formados por três pedagogos (as), um técnico em assuntos educacionais, uma psicóloga, uma assistente social, um médico clinico geral, três enfermeiros (as) e um odontólogo. 

       Os atendimentos realizados são feitos em salas específicas já que a instituição oferece salas individualizadas para cada seguimento que compõe o atendimento pedagógico e de assistência ao aluno, que é distribuída da seguinte forma: sala da coordenação pedagógica, sala do gabinete médico e a sala da assistência social. 

       As atividades relativas ao atendimento psicopedagógico aos alunos são as seguintes:

\begin{enumerate}
	\item Orientar as turmas encaminhadas pelos professores ou pela coordenação do curso; 
	\item Atendimento às dificuldades de aprendizagem;
	\item Realizar atendimento individual ou em grupo;
	\item Acompanhar e apoiar o desempenho dos alunos durante o semestre letivo;
	\item Prestar atendimento médico;
	\item Prestar atendimento odontológico;
	\item Fazer atendimento psicológico;
	\item Fazer atendimento de assistência social;
\end{enumerate}

	Os itens 6 e 7 ser\~ao disponibilizados apenas quando os profissionais habilitados forem nomeados para o IFPB - Campus Guarabira. Todos os outros \'itens j\'a s\~ao plenamente atendidos atualmente.

\paragraph{Mecanismos de nivelamento}\

	Ao longo dos últimos anos, por meio da análise de estatísticas próprias e estudos publicados por organismos nacionais, diagnosticou-se a existência de dificuldades em várias disciplinas advindas de problemas mais diversos, tais como: deficiência nos estudos de ensino básico e médio; longo tempo de afastamento da escola; suplência de ensino médio por meio de mecanismos oferecidos pelo governo, entre outros, que acabam por influenciar na educação superior.

	Portanto, ao se diagnosticar deficiência em algum campo específico, o curso de Sistemas para Internet oferecer\'a atendimento diferenciado aos acadêmicos, por meio dos professores e monitores, visando à melhoria qualitativa do trato com os assuntos, de modo a viabilizar a aprendizagem acadêmica.

\paragraph{Apoio \`as atividades acad\^emicas}\

Os acadêmicos são estimulados \`a participação e organização de congressos, palestras, seminários, encontros, simpósios, cursos, fóruns etc. O Curso de Tecnologia em Sistemas para Internet incentiva a realização de atividades extracurriculares no intuito de promover um espírito crítico e reflexivo, fatores decisivos para o desenvolvimento pessoal e profissional, envolvendo os acadêmicos em debates, projetos que primam pela iniciativa e criatividade, e possa então se transformar em um processo de construção do perfil profissional.

\subsubsection{Ouvidoria}
       
A ouvidoria do IFPB tem como base legal a Resolução Nº 017/2002 de 30 de agosto de 2002 que estabeleceu a sua criação, e constitui-se como um espaço autônomo e independente da administração.  Sua missão objetiva arbitrar demandas oriundas de diversos segmentos (alunos, técnicos administrativos, professores, comunidade externa) que compõem, direta ou indiretamente, a instituição. Nesse contexto, a ouvidoria procura otimizar encaminhamentos de questões de ordem administrativa ou pedagógicas. 

A Ouvidoria Geral é exercida por um Ouvidor-Geral, escolhido dentre servidores docentes e técnicos-administrativos de nível Superior, com pelo menos dez anos na instituição e no mínimo de 3 (três) anos no exercício de suas atividades. O mandato do Ouvidor-Geral será de 2 (dois) anos, permitida uma única recondução para mandato consecutivo. O Ouvidor escolhido deverá, necessariamente, estar submetido ao regime de Dedicação Exclusiva, se professor, e de 40 (quarenta) horas semanais, se técnico-administrativo.

                 A Ouvidoria pode ser utilizada:

\begin{itemize}
\item Por estudantes do IFPB, incluindo os de cursos extraordinários;

\item Por servidores técnicos-administrativos ativos e aposentados do IFPB;

\item Por servidores ativos e aposentados do IFPB;

\item Por pessoas da comunidade.
\end{itemize}

	A Ouvidoria não atende a solicitações anônimas, no entanto, recebe reclamações e denúncias sigilosas, quando justificáveis as razões do sigilo, até a finalização do processo.

          Compete ao Ouvidor Geral:

\begin{itemize}
\item Facilitar e simplificar ao máximo o acesso do usuário ao serviço da Ouvidoria;

\item Promover a divulgação da Ouvidoria, tornando-a conhecida por todos;

\item Receber e apurar, de forma independente e crítica, as informações, reclamações, denúncias e sugestões que lhe forem encaminhadas por membros da comunidade interna e externa, quando devidamente formalizadas;

\item Analisar as informações, reclamações, denúncias e sugestões recebidas, encaminhando o resultado da análise aos setores administrativos competentes;

\item Receber elogios, em que o requerente pode elogiar servidores, as infraestrutura, que sejam consideradas eficientes no IFPB;

\item Acompanhar as providências adotadas pelos setores competentes, mantendo o requerente informado do processo;

\item Propor ao Diretor-Geral a instauração de processo administrativo disciplinar, quando necessário, nos termos da legislação vigente;

\item Sugerir medidas de aprimoramento das atividades administrativas em proveito da comunidade e do próprio IFPB;

\item Elaborar e apresentar relatório anual de suas atividades ao Conselho Diretor;

\item Interagir com profissionais de sua área, no Brasil e no exterior, com o objetivo de aperfeiçoar o desempenho de suas atividades;

\item Propor outras atividades pertinentes à função.

\end{itemize}

      O Ouvidor-Geral no exercício de suas funções deverá:

\begin{itemize}
\item Recusar como objeto de apreciação as questões pendentes de decisão judicial;

\item Ser recebido sempre que o solicitar por todos os ocupantes de cargos do IFPB, para pedir e receber explicações orais ou por escrito, sobre questões acadêmicas ou de outras atividades;

\item Rejeitar e determinar o arquivamento de reclamações e denúncias reconhecidamente improcedentes, mediante despacho fundamentado.
\end{itemize}


%A extensão da Ouvidoria para o IFPB-Campus-Cajazeiras ocorreu em 2002 quando a instituição a época tinha a denominação de CEFET - PB - Uned -Cajazeiras através da Resolução N° 017/2002 de 30 de agosto de 2002 aprovada pelo então  Conselho Diretor do Centro  Federal de Educação Tecnológica da Paraíba, tendo como base o Artigo 28, Inciso VI do Regimento do CEFET-PB, e artigo 6º, inciso VI do Regulamento do Conselho Diretor que indicava no seu Artigo 6° a seguinte determinação:  O Ouvidor-Geral será assistido, no desempenho de suas funções, pelos seguintes auxiliares: um representante da Ouvidoria na Unidade Descentralizada (UNED - Cajazeiras).

%Sendo assim e seguindo as determinações contidas em resolução do Conselho Diretor da instituição a Ouvidoria foi instalada e funciona até os dias atuais no atual IFPB-Campus- Cajazeiras, tendo a frente o Ouvidor Severino Dantas Fernandes. A Ouvidoria tem uma sala específica e conta com horários destinados ao atendimento do público. O telefone de atendimento é o seguinte: (83)35324115.Neste sentido, compete ao Representante da Ouvidoria Geral do IFPB- Campus-Cajazeiras: Divulgar a Ouvidoria no IFPB-Campus Cajazeiras e na comunidade local. Além de Receber e processar as demandas que lhe forem encaminhadas, submetendo as à apreciação do Ouvidor-Geral.

\subsubsection{Acompanhamento dos egressos}

O acompanhamento aos egressos constitui num instrumento que possibilitará uma avaliação cont\'inua da instituição, por meio do desempenho profissional dos ex-alunos. Trata-se de um importante passo no sentido de incorporar ao processo de ensino-aprendizagem elementos da realidade externa, por meio das experiências vivenciadas pelos formados, em contrapartida ao que eles vivenciaram durante sua graduação.

São objetivos específicos:

\begin{itemize}
\item Avaliar o desempenho da instituição, por meio do acompanhamento do desenvolvimento profissional dos ex-alunos;

\item Manter registros atualizados de alunos egressos;

\item Possibilitar as condições para que os egressos possam apresentar aos graduandos os trabalhos que vêm desenvolvendo, através das Semanas Acadêmicas e outras formas de divulgação;

\item Divulgar permanentemente a inserção dos alunos formados no mercado de trabalho;

\item Identificar junto às empresas seus critérios de seleção e contratação, dando ênfase às capacitações e habilidades exigidas dos profissionais da área;

\item Incentivar a leitura de periódicos especializados, disponíveis na biblioteca do Instituto.
\end{itemize}

\paragraph{ENADE}\

  O ENADE tem por objetivo fazer o acompanhamento do processo de aprendizagem e o desempenho acadêmico dos estudantes em relação aos conteúdos programáticos previstos nas diretrizes curriculares do respectivo curso de graduação, suas habilidades para ajustamento às exigências decorrentes da evolução do conhecimento e suas competências para compreender temas exteriores ao âmbito específico de sua profissão, ligados à realidade brasileira e mundial e a outras áreas do conhecimento. Seus resultados poderão produzir dados por instituição de educação superior, categoria administrativa, organização acadêmica, município, estado e região. Assim, serão constituídos referenciais que permitam a definição de ações voltadas para a melhoria da qualidade dos cursos de graduação, por parte de professores, técnicos, dirigentes e autoridades educacionais. 

      % Até o presente momento, o Curso de Bacharelado em Engenharia Civil do IFPB-Campus-Cajazeiras ainda não foi contemplado para avaliação ENADE e aguarda esta avaliação de acordo com o calendário trienal estabelecido pelo Instituto Nacional de Estudos e Pesquisas Educacionais Anísio Teixeira-INEP.


\subsubsection{Registros acad\^emicos}

	O Departamento de Cadastro Acadêmico, Certificação e Diplomação é o organismo institucional do IFPB que faz o gerenciamento do sistema informatizado de controle acadêmico. O sistema possibilita que a instituição faça a organização, controle e o registro acadêmico do IFPB. Dentre suas atribuições algumas estão diretamente relacionadas ao sistema de controle acadêmico, como por exemplo:

\begin{itemize}
	\item Supervisionar a organização e atualização dos cadastros escolares dos alunos do ensino técnico, da graduação e da pós-graduação operados pelos campi do IFPB;

	\item Supervisionar a coleta e anotação dos resultados da verificação de rendimento escolar dos alunos realizada pelo setor de controle acadêmico de cada campus; 

	\item Supervisionar a escrituração dos créditos escolares integralizados pelos alunos e o aproveitamento de estudos feitos anteriormente realizados pelo setor de controle acadêmico de cada campus, após decisão dos órgãos competentes.
\end{itemize}

          Assim, nos campi do IFPB o gerenciamento do controle acadêmico é feito pelas coordenações de controle acadêmico. No caso do IFPB-Campus Guarabira, as atividades desenvolvidas por essa coordenação são executadas em consonância com as coordenações de cursos com o propósito de manter a Direção Geral, Departamento de Ensino, Pró-Reitoria de Ensino e Departamentos Institucionais do IFPB informados acerca das atividades setoriais, bem como a situação acadêmica dos cursos e alunos matriculados. Desta forma, muitas atividades relativas ao processo ensino-aprendizagem são desenvolvidas pela coordenação de controle acadêmico, como por exemplo:

\begin{itemize}
	\item Matrícula de alunos novatos nos Cursos Regulares da Instituição, presenciais e á distância, 
	\item Matrícula de alunos aprovados em disciplinas/turmas definindo o fluxo acadêmico para os períodos letivos;
	\item Renovação e reabertura de matrículas, emissão de documentos e  trancamento; 
	\item Alocação dos alunos nos Cursos da Instituição obedecendo aos critérios de aprovação/reprovação nos períodos cursados, além da observância de sua situação acadêmica;
	\item Acompanhamento da situação acadêmica dos alunos para emissão de relatórios destinados as Unidades Acadêmicas, Coordenações dos Cursos, Direção de Desenvolvimento do Ensino e Direção Geral;
	\item Emissão de transferências, boletins e declarações;

	\item Alimentação e operacionalização do Sistema de Controle Acadêmico, quando houver, tendo em vista o aprimoramento no uso do referido Software;
	\item Acompanhamento de resultados obtidos pelos alunos a cada final de período letivo, definindo o fluxo de matrícula para os períodos subsequentes; 

	\item Acompanhamento do quantitativo de alunos matriculados na Instituição, especificando o total de evasões, trancamentos, aprovações e reprovações, transferências e conclusão de cursos;
	\item Apoio aos docentes e discentes nos procedimentos quanto ao uso do software de controle acadêmico, quando houver, com o objetivo de solucionar dúvidas de utilização e dando suporte às rotinas administrativas pertinentes.
\end{itemize}


\subsection{Administra\c{c}\~ao do Curso}

\subsubsection{Coordena\c{c}\~ao do curso}

\begin{table}[h]
\begin{tabular}{|l|l|}
\hline
\textbf{Nome do Coordenador:} & Otac\'ilio de Ara\'ujo Ramos Neto \\ \hline
\textbf{Titulação:}           & Mestre                             \\ \hline
\textbf{Regime de Trabalho:}  & DE                                 \\ \hline
\end{tabular}
\end{table}

\paragraph{Forma\c{c}\~ao Acad\^emica e Experi\^encia Profissional}\

Otac\'ilio de Ara\'ujo Ramos Neto \'e professor do Instituto Federal de Educação, Ciência e Tecnologia da Paraíba desde 2013. é doutorando em Engenharia Mecânica pela UFPB, atuando na linha de pesquisa de automação e controle, com ênfase em robótica. Possui mestrado em Informática pela UFPB (2013). Possui especialização em Segurança da Informação pela faculdade iDez (2010), bacharelado em Engenharia Elétrica pela Universidade Federal de Campina Grande (2007) e é técnico em Processamento de Dados pelo Instituto Federal de Educação, Ci\^encia e Tecnologia da Paraíba (1997). Trabalhou como professor na Faculdade iDez/Estácio, como desenvolvedor de hardware e com Linux embarcado na DVR Tecnologia Eletrônica Ltda. Trabalhou também como desenvolvedor de firmware na Zênite Tecnologia e Teleinformática Ltda, onde desenvolveu software aplicativo e hardware. Possui experiência em Verilog, FPGAs Altera, Linux embarcado, linguagem C/C++, SystemC e síntese e simulação de circuitos integrados digitais. 

O professor Otacílio desenvolve atividades de ensino e pesquisa no IFPB-Campus Guarabira desde junho de 2013, lecionando disciplinas das áreas de algoritmos e estruturas de dados, arquitetura de computadores e banco de dados. Além disso, é líder do grupo de pesquisa em sistemas digitais, orienta quatro bolsistas de iniciação científica e coordena um grupo de preparação para olimpíadas de robótica, contando atualmente com sete alunos.

\paragraph{Atua\c{c}\~ao da coordena\c{c}\~ao}\

As funções da Coordenação do Curso serão:
\begin{itemize}
\item Formular, coordenar e avaliar objetivos e estratégias educacionais do curso;
\item Coordenar, junto aos professores, a atualização dos projetos de ensino;
\item Acompanhar, junto aos professores, a execução dos projetos de ensino;
\item Acompanhar as avaliações dos professores e controlar a entrega de provas e notas.
\item Estimular a atualização didática e científica dos professores do curso;
\item Orientar os professores nas atividades acadêmicas;
\item Orientar os alunos do curso por ocasião da matrícula;
\item Apoiar atividades científico-culturais de interesse dos alunos;
\item Coletar sugestões e elaborar o plano anual de metas do curso;
\item Avaliar os professores do curso e ser por eles e pelos concludentes avaliado;
\item Avaliar situações conflitantes entre professores e alunos.
\end{itemize}

\subsubsection{Composição e Funcionamento dos órgãos Colegiados}

\vspace{4mm}
\textbf{Do Conselho Superior}
\vspace{4mm}

O Conselho Superior, de caráter consultivo e deliberativo, é o órgão máximo do Instituto Federal da Paraíba, tendo a seguinte composição:

I.	o Reitor, como presidente;

II.	uma representação de cada Campus, destinada ao corpo docente, eleita por seus pares, na forma regimental;

III.	uma representação de cada Campus, destinada ao corpo discente, eleita por seus pares, na forma regimental;

IV.	uma representação de cada Campus, destinada ao corpo técnico-administrativos, eleita por seus pares, na forma regimental;

V.	2 (dois) representantes dos egressos, indicados por entidades representativas;

VI.	6 (seis) representantes da sociedade civil, sendo 2 (dois) indicados por entidades patronais, 2 (dois) indicados por entidades dos trabalhadores, 2 (dois) representantes do setor público e/ou empresas estatais, indicados pelas entidades e nomeados pelo Reitor;

VII.	1 (um) representante do Ministério da Educação, indicado pelo respectivo Ministério e nomeado pelo Reitor;

VIII.	1 (uma) representação dos diretores-gerais de cada Campus.

\subsubsection{Núcleo Docente Estruturante}

O Núcleo Docente Estruturante (NDE) do Curso Superior de Tecnologia em Sistemas para Internet do IFPB-Campus-Guarabira é um organismo consultivo que tem como base legal A Portaria nº 1.081, de 29 de agosto de 2008, que aprova o  instrumento de Avaliação dos Cursos de graduação do Sistema Nacional de Avaliação da Educação Superior-SINAES. Para atender a esta base legal o NDE segue também as atribuições e orientações contidas na Resolução N0 01, de 17 de junho de 2010, que trata especificamente da normatização dos NDEs dos cursos de graduação.

       Neste sentido, o NDE do Curso Superior de Tecnologia em Sistemas para Internet do IFPB-Campus Guarabira é constituído por professores, com atribuições acadêmicas de acompanhamento e atuação no processo de concepção e consolidação contínua do projeto pedagógico do curso. Sendo assim, no NDE os professores do curso devem exercer liderança acadêmica no âmbito do curso, gerando uma ação que se transforme na produção de conhecimentos da área, no desenvolvimento do ensino, e em outras dimensões entendidas como importantes pela instituição, atuando sempre na direção do desenvolvimento do curso. 

 	São atribuições do Núcleo Docente Estruturante:
 
\begin{itemize}
	\item Contribuir para a consolidação do perfil profissional do egresso do curso; 

	\item Zelar pela integração curricular interdisciplinar entre as diferentes atividades de ensino constantes no currículo; 

	\item Indicar formas de incentivo ao desenvolvimento de linhas de pesquisa e extensão, oriundas de necessidades da graduação, de exigências do mundo do trabalho e afinadas com as políticas públicas relativas à área de conhecimento do curso; 

	\item Zelar pelo cumprimento das Diretrizes Curriculares Nacionais para os Cursos de Graduação. 
\end{itemize}

	A Tabela~\ref{tab:nde} mostra os membros do NDE do Curso Superior de Tecnologia em Sistemas para Internet do IFPB-Campus Guarabira.
	
\begin{table}[h]
\caption{Rela\c{c}\~ao dos Membros do NDE}
\begin{tabular}{lllll}
\multicolumn{5}{c}{\cellcolor[HTML]{9B9B9B}Núcleo Docente Estruturante}                                                                                                                                                                                                                                                                  \\ \hline
\multicolumn{1}{|c|}{\textbf{Docente}}          & \multicolumn{1}{c|}{\textbf{Graduado em}}   & \multicolumn{1}{c|}{\textbf{Titulação}} & \multicolumn{1}{c|}{\textbf{\begin{tabular}[c]{@{}c@{}}Experiência \\ Profissional\end{tabular}}} & \multicolumn{1}{c|}{\textbf{\begin{tabular}[c]{@{}c@{}}Regime de\\ Trabalho\end{tabular}}} \\ \hline
\multicolumn{1}{|l|}{Otacílio de A. Ramos Neto} & \multicolumn{1}{l|}{Engenharia Elétrica}    & \multicolumn{1}{l|}{Mestre}             & \multicolumn{1}{l|}{9 anos}                                                                       & \multicolumn{1}{l|}{DE}                                                                    \\ \hline
\multicolumn{1}{|l|}{Ruan Delgado Gomes}        & \multicolumn{1}{l|}{Ciência da Computação}  & \multicolumn{1}{l|}{Mestre}             & \multicolumn{1}{l|}{5 anos}                                                                       & \multicolumn{1}{l|}{DE}                                                                    \\ \hline
\multicolumn{1}{|l|}{Rodrigo P. M. de Araújo}   & \multicolumn{1}{l|}{Ciência da Computação}  & \multicolumn{1}{l|}{Mestre}             & \multicolumn{1}{l|}{7 anos}                                                                       & \multicolumn{1}{l|}{DE}                                                                    \\ \hline
\multicolumn{1}{|l|}{Moisés G. de Medeiros}     & \multicolumn{1}{l|}{Sistemas para Internet} & \multicolumn{1}{l|}{Especialista}       & \multicolumn{1}{l|}{7 anos}                                                                       & \multicolumn{1}{l|}{T-40}                                                                  \\ \hline
\multicolumn{1}{|l|}{José de Sousa Barros}      & \multicolumn{1}{l|}{Sistemas de Informação} & \multicolumn{1}{l|}{Especialista}       & \multicolumn{1}{l|}{9 anos}                                                                       & \multicolumn{1}{l|}{DE}                                                                    \\ \hline
\multicolumn{5}{l}{\cellcolor[HTML]{9B9B9B}{\color[HTML]{9B9B9B} }}                                                                                                                                                                                                                                                                     
\end{tabular}
\label{tab:nde}
\end{table}
%colocar subtopicos

\subsection{Corpo Docente}

\subsubsection{Rela\c{c}\~ao nominal do corpo docente}

\begin{table}[h!]
\scriptsize
\caption{Rela\c{c}\~ao nominal do corpo docente}
\begin{tabular}{|l|l|l|l|l|l|l|l|l|l|l|}
\hline
\multicolumn{1}{|c|}{\multirow{2}{*}{Número}} & \multicolumn{1}{c|}{\multirow{2}{*}{CPF}} & \multicolumn{1}{c|}{\multirow{2}{*}{Docente}} & \multicolumn{4}{c|}{\begin{tabular}[c]{@{}c@{}}Formação \\ Acadêmica\end{tabular}}                     & \multicolumn{3}{c|}{\begin{tabular}[c]{@{}c@{}}Experiência \\ Profissional\end{tabular}} & \multicolumn{1}{c|}{\multirow{2}{*}{TC}} \\ \cline{4-10}
\multicolumn{1}{|c|}{}                        & \multicolumn{1}{c|}{}                     & \multicolumn{1}{c|}{}                         & \multicolumn{1}{c|}{GR} & \multicolumn{1}{c|}{ESP} & \multicolumn{1}{c|}{ME} & \multicolumn{1}{c|}{DO} & \multicolumn{1}{c|}{NMS}     & \multicolumn{1}{c|}{EFM}    & \multicolumn{1}{c|}{FMS}    & \multicolumn{1}{c|}{}                    \\ \hline
1                                             & 079.830.874-54                            & Ruan Delgado Gomes                            & \rotatebox[origin=c]{90}{UFPB-2010}               &                          & \rotatebox[origin=c]{90}{UFCG-2012}               &                         & 1                            & 3                           & 2                           & 3                                        \\ \hline
2                                             & 008.255.294-09                                          & Otacílio de A. Ramos Neto                     & \rotatebox[origin=c]{90}{UFCG-2007}               & \rotatebox[origin=c]{90}{iDEZ-2010}                & \rotatebox[origin=c]{90}{UFPB-2013}               &                         & 3                            & 2                           & 4                           & 2                                        \\ \hline
3                                             &  049.368.514-69                                          & Rodrigo P. M. de Araújo                       & \rotatebox[origin=c]{90}{UFRN-2007}               &                          & \rotatebox[origin=c]{90}{UFRN-2011}               &                         & 3                            & 3                           & 4                           & 3                                        \\ \hline
4                                             &  032.206.364-70                                          & Erick A. Gomes de Melo                        & \rotatebox[origin=c]{90}{IFPB-2004}               &                          & \rotatebox[origin=c]{90}{UFPB-2010}               &                         & 6                            & 3                           & 10                          & 3                                        \\ \hline
5                                             &  064.488.964-06                                          & Moisés G. de Medeiros                         & \rotatebox[origin=c]{90}{IFPB-2008}               & \rotatebox[origin=c]{90}{FATEC-2009}               &                         &                         & 2                            & 2                           & 7                           & 2                                        \\ \hline
6                                             &  044.862.414-10                                          & José de Sousa Barros                          & \rotatebox[origin=c]{90}{FIP-2007}                & \rotatebox[origin=c]{90}{FIP-2012}                 &                         &                         & 2                            & 2                           & 7                           & 2                                        \\ \hline
7                                             &  000.784.384-46                                        & Sabrina da Costa Rocha                        & \rotatebox[origin=c]{90}{UFPB-97}                 & \rotatebox[origin=c]{90}{UFPB-2002}                & \rotatebox[origin=c]{90}{UFPB-2009}               &                         & 3                            & 19                          &                             & 3                                        \\ \hline
8                                             &   039.935.324-06                                        & Erivan Lopes Tome Junior                     & \rotatebox[origin=c]{90}{UFPB-2004}               &  \rotatebox[origin=c]{90}{CBM-2012}                 & \rotatebox[origin=c]{90}{UFPB-2014}               &                         & 1                            & 7                           &                             & 0,3                                        \\ \hline
9                                             &   819.911.773-72                                         & Cícero D. V. de Barros                        & \rotatebox[origin=c]{90}{UFPB-2009}               & \rotatebox[origin=c]{90}{CESREI-2012}              & \rotatebox[origin=c]{90}{UFPB-2013}               &                         &                              & 10                          &                             & 3                                        \\ \hline
10                                            &   059.060.644-19                                        & Anna Carolina C. C. da Cunha                  & \rotatebox[origin=c]{90}{UFPB-2008}               &                          & \rotatebox[origin=c]{90}{U. of Bath-2010}         &                         & 2                            & 1                           & 2                           & 2                                        \\ \hline
\end{tabular}
\end{table}

Legenda:
\\
GR - Gradua\c{c}\~ao;\\
ESP - Especializa\c{c}\~ao;\\
ME - Mestrado;\\
DO - Doutorado;\\
NMS – tempo de experiência profissional (em ano) No Magistério Superior;\\
EFM – tempo de experiência (em ano) no Ensino Fundamental e Médio\\
FMS - tempo de experiência profissional (em ano) Fora Magistério Superior;\\
TC – Tempo (em ano) de Contrato na IES;\\

\subsubsection{Titulação e experiência do corpo docente e efetiva dedicação ao curso}

O exercício da docência no Instituto Federal da Paraíba é permitido ao profissional com formação mínima de graduação. Os requisitos para admissão são exigidos na publicação do Edital Público para concurso de admissão ao quadro, sendo importante também a comprovação de experiência profissional, que fortalece o currículo do candidato para efeito de pontuação e classificação.

O corpo docente do Curso de Sistemas para Internet a ser oferecido pelo IFPB-Campus Guarabira, é formado por especialistas e mestres, que possuem uma vasta experiência em docência. Al\'em disso, alguns professores est\~ao cursando doutorado.

\paragraph{Titula\c{c}\~ao}\

	O quadro atual \'e formado por 80\% de mestres e 20\% de especialistas.

\paragraph{Regime de trabalho do corpo docente}\

	O quadro atual \'e formado por 80\% de professores com regime de Dedica\c{c}\~ao Exclusiva e 20\% dos professores em regime de trabalho T-40.

\paragraph{Experi\^encia (acad\^emica e profissional)}\

	O corpo docente do Instituto Federal da Paraíba é constituído de profissionais que possuem experiência no Ensino Superior e que têm experiência profissional na área que lecionam, seja atuando em empresas ou como profissional liberal. Estes requisitos são considerados quando da seleção e influenciam na avaliação e na aprovação do docente.

\paragraph{Tempo de exercício no magistério superior}\

A seguir um demonstrativo da experiência do Corpo Docente do Curso de Sistemas para Internet a ser oferecido pelo IFPB - Campus Guarabira.

\begin{table}[h!]
\begin{tabular}{lll}
\rowcolor[HTML]{C0C0C0} 
\multicolumn{1}{c}{\cellcolor[HTML]{C0C0C0}\textbf{Experiência no Magistério Superior}} & \multicolumn{1}{c}{\cellcolor[HTML]{C0C0C0}\textbf{Quantidade}} & \multicolumn{1}{c}{\cellcolor[HTML]{C0C0C0}\textbf{Percentual}} \\
Sem experiência                                                                         & 1                                                               & 10\%                                                            \\
De 1 a 3 anos                                                                           & 8                                                               & 80\%                                                            \\
De 4 a 6 anos                                                                           & 1                                                               & 10\%                                                            \\
\rowcolor[HTML]{9B9B9B} 
\multicolumn{3}{l}{\cellcolor[HTML]{9B9B9B}}                                                                                                                                                                               
\end{tabular}
\end{table}

\paragraph{Tempo de exercício profissional fora do magistério}\

A seguir um demonstrativo da experiência do Corpo Docente do Curso de Sistemas para Internet fora do magistério.

\begin{table}[h!]
\begin{tabular}{lll}
\rowcolor[HTML]{C0C0C0} 
\multicolumn{1}{c}{\cellcolor[HTML]{C0C0C0}\textbf{Experiência Profissional Fora do Magistério}} & \multicolumn{1}{c}{\cellcolor[HTML]{C0C0C0}\textbf{Quantidade}} & \multicolumn{1}{c}{\cellcolor[HTML]{C0C0C0}\textbf{Percentual}} \\
Sem experiência                                                                                  & 3                                                               & 30\%                                                            \\
De 1 a 3 anos                                                                                    & 2                                                               & 20\%                                                            \\
De 4 a 9 anos                                                                                    & 4                                                               & 40\%                                                            \\
10 anos ou mais                                                                                  & 1                                                               & 10\%                                                            \\
\rowcolor[HTML]{9B9B9B} 
\multicolumn{3}{l}{\cellcolor[HTML]{9B9B9B}}                                                                                                                                                                                        
\end{tabular}
\end{table}

\subsubsection{Produção de material didático ou científico do corpo docente}

A seguir a lista de publicações ou produções científicas, técnicas, tecnológicas, pedagógicas, culturais e artísticas dos docentes do curso de Sistemas para Internet do IFPB - Campus Guarabira nos últimos 3 anos.

\begin{table}[h]
\begin{tabular}{lllll}
\rowcolor[HTML]{C0C0C0} 
\multicolumn{1}{c}{\cellcolor[HTML]{C0C0C0}}                                              & \multicolumn{3}{c}{\cellcolor[HTML]{C0C0C0}\textbf{Quantidade}} & \multicolumn{1}{c}{\cellcolor[HTML]{C0C0C0}}                                 \\
\rowcolor[HTML]{EFEFEF} 
\multicolumn{1}{c}{\multirow{-2}{*}{\cellcolor[HTML]{C0C0C0}\textbf{Tipo de Publicação}}} & 2012                & 2013                & 2014                & \multicolumn{1}{c}{\multirow{-2}{*}{\cellcolor[HTML]{C0C0C0}\textbf{Total}}} \\
Artigos publicados em periódicos científicos                                              & 3                   & 5                   & 3                   & 11                                                                           \\
\rowcolor[HTML]{EFEFEF} 
Livros ou capítulos de livros publicados                                                  & 0                   & 2                   & 0                   & 2                                                                            \\
Trabalhos publicados em anais (completos ou resumos)                                      & 6                   & 5                   & 2                   & 13                                                                           \\
\rowcolor[HTML]{EFEFEF} 
Propriedade intelectual depositada ou registrada                           & 0                   & 1                   & 3                   &          4                                                                   
\end{tabular}
\end{table}


	Vale ressaltar que a produ\c{c}\~ao acad\^emica \'e apenas com rela\c{c}\~ao aos 10 docentes que atuar\~ao no curso e que j\'a fazem parte do corpo docente do IFPB-Campus Guarabira. Dessa forma, o corpo docente apresenta uma produ\c{c}\~ao acad\^emica elevada, levando em considera\c{c}\~ao a quantidade de docentes envolvivos em atividades de pesquisa atualmente. Com o crescimento do corpo docente espera-se que a produ\c{c}\~ao acad\^emica aumente significativamente.

\subsubsection{Plano de Carreira e Incentivos ao Corpo Docente}

Plano de Carreira e Incentivos ao Corpo Docente consta como uma das preocupações do Plano de Desenvolvimento Institucional – PDI do IFPB. Com a edição da Lei nº 12.772, de 28 de dezembro de 2012, os docentes ganharam uma nova estrutura de carreira sendo denominados de Professor da Carreira do Magistério do Ensino Básico, Técnico e Tecnológico. O plano de carreira e o regime de trabalho são regidos pela Lei no 12.772, de 28 de dezembro de 2012, pela Lei no 8.112, de 11 de dezembro de 1990 e pela Constituição Federal, além da legislação vigente atrelada a essas Leis e a LDB, Nº 9.394, de 20 de dezembro de 1996. O Instituto Federal da Paraíba tem uma política de qualificação e capacitação que contempla o estímulo a participação em Seminários e Congressos, além da oferta de cursos de pós-graduação para os docentes e técnicos administrativos seja por meio da participação em programas de universidades como também dos programas interministeriais como é o caso do Minter e do Dinter.

A Política de Capacitação de Docentes e Técnicos Administrativos no âmbito Institucional foi instituída pela Portaria de n\'umero 148/2001 – GD de 22/05/2001, que criou o Comitê Gestor de Formação e Capacitação, disciplinando e regulamentando a implementação do Plano de Capacitação, bem como as condições de afastamento com esse fim. O Comitê Gestor de Formação e Capacitação tem as seguintes competências:

\begin{itemize}
\item Elaborar o plano de capacitação geral da Instituição;
\item Avaliar processos de solicitação de docentes ou técnicos administrativos para afastamento ou prorrogação de afastamento;
\item Propor à Direção Geral a liberação ou prorrogação de afastamento de docentes ou técnicos-administrativos;
\item Acompanhar os relatórios periódicos, trimestrais ou semestrais, dos servidores afastados, avaliando a continuidade da capacitação;
\item Zelar pelo cumprimento das obrigações previstas.
\end{itemize}

O Plano de capacitação do IFPB considera os seguintes níveis de qualificação profissional:
\begin{itemize}
\item Pós-Graduação stricto sensu: mestrado, doutorado e pós-doutorado;
\item Pós-Graduação lato sensu: aperfeiçoamento e especialização;
\item Graduação;
\item Capacitação profissional: cursos que favoreçam o aperfeiçoamento profissional;
\item Atividades de curta duração: cursos de atualização e participação em congressos, seminários, conclaves, simpósios, encontros e similares.
\end{itemize}

\subsubsection{Docentes x n\'umero de vagas autorizadas}

No quadro a seguir é demonstrada a relação entre as vagas anuais autorizadas e dos docentes que atuam em tempo integral.

\begin{table}[h]
\scriptsize
\begin{tabular}{|l|l|}
\hline
\rowcolor[HTML]{C0C0C0} 
\multicolumn{1}{|c|}{\cellcolor[HTML]{C0C0C0}\textbf{Número de Vagas Anuais/Docente equivalente em Tempo Integral}} & \multicolumn{1}{c|}{\cellcolor[HTML]{C0C0C0}\textbf{Quantidade}}                           \\ \hline
Vagas anuais                                                                                                        & 60                                                                                         \\ \hline
Total de docentes em tempo integral                                                                                 & 10                                                                                         \\ \hline
\rowcolor[HTML]{C0C0C0} 
\multicolumn{1}{|r|}{\cellcolor[HTML]{C0C0C0}\textbf{Média}}                                                        & \textbf{\begin{tabular}[c]{@{}l@{}}6 alunos por docente \\ em tempo integral\end{tabular}} \\ \hline
\end{tabular}
\end{table}

\subsubsection{Docente por disciplina}

\begin{table}[h!]
\scriptsize
\begin{tabular}{lllll}
\multicolumn{5}{c}{\cellcolor[HTML]{C0C0C0}\textbf{Primeiro Semestre}}                                                                                                                                                                                                                                    \\ \hline
\multicolumn{1}{|p{5.2cm}|}{\cellcolor[HTML]{FFFFFF}\textbf{Disciplina}}                                    & \multicolumn{1}{p{5.2cm}|}{\textbf{Professor}}                 & \multicolumn{1}{p{3.1cm}|}{\textbf{Graduação}}     & \multicolumn{1}{p{1.7cm}|}{\textbf{Titulação}} & \multicolumn{1}{p{0.6cm}|}{\textbf{RT}} \\ \hline
\multicolumn{1}{|l|}{Inglês Instrumental}                                                            & \multicolumn{1}{l|}{Sabrina da Costa Rocha}             & \multicolumn{1}{l|}{Letras Inglês}          & \multicolumn{1}{l|}{Mestre}             & \multicolumn{1}{l|}{DE}                          \\ \hline
\multicolumn{1}{|l|}{\begin{tabular}[c]{@{}l@{}}Fundamentos de Redes\\ de Computadores\end{tabular}} & \multicolumn{1}{l|}{Erick Augusto Gomes de Melo}        & \multicolumn{1}{l|}{\begin{tabular}[c]{@{}l@{}}Telemática\\Telecomunicações\end{tabular}}             & \multicolumn{1}{l|}{Mestre}             & \multicolumn{1}{l|}{DE}                          \\ \hline
\multicolumn{1}{|l|}{\cellcolor[HTML]{FFFFFF}Cálculo Diferencial e Integral}                         & \multicolumn{1}{l|}{Cícero Demétrio Vieira de Barros}   & \multicolumn{1}{l|}{Matemática}             & \multicolumn{1}{l|}{Mestre}             & \multicolumn{1}{l|}{T-40}                        \\ \hline
\multicolumn{1}{|l|}{Algoritmos e Lógica de Programação}                                             & \multicolumn{1}{l|}{Ruan Delgado Gomes}                 & \multicolumn{1}{l|}{Ciência da Computação}  & \multicolumn{1}{l|}{Mestre}             & \multicolumn{1}{l|}{DE}                          \\ \hline
\multicolumn{1}{|l|}{Fundamentos da Computação}                                                      & \multicolumn{1}{l|}{Rodrigo Pinheiro Marques de Araújo} & \multicolumn{1}{l|}{Ciência da Computação}  & \multicolumn{1}{l|}{Mestre}             & \multicolumn{1}{l|}{DE}                          \\ \hline
\multicolumn{1}{|l|}{Linguagens de Marcação}                                                         & \multicolumn{1}{l|}{Moisés Guimarães de Medeiros}       & \multicolumn{1}{l|}{Sistemas para Internet} & \multicolumn{1}{l|}{Especialista}       & \multicolumn{1}{l|}{T-40}                        \\ \hline
\end{tabular}
\end{table}


\begin{table}[h!]
\scriptsize
\begin{tabular}{lllll}
\multicolumn{5}{c}{\cellcolor[HTML]{C0C0C0}\textbf{Segundo Semestre}}                                                                                                                                                                                                                                    \\ \hline
\multicolumn{1}{|p{5.2cm}|}{\cellcolor[HTML]{FFFFFF}\textbf{Disciplina}}                                    & \multicolumn{1}{p{5.2cm}|}{\textbf{Professor}}                 & \multicolumn{1}{p{3.1cm}|}{\textbf{Graduação}}     & \multicolumn{1}{p{1.7cm}|}{\textbf{Titulação}} & \multicolumn{1}{p{0.6cm}|}{\textbf{RT}} \\ \hline
\multicolumn{1}{|l|}{Portugu\^es Instrumental}                                                            & \multicolumn{1}{l|}{Erivan Lopes Tome Junior}             & \multicolumn{1}{l|}{Letras}          & \multicolumn{1}{l|}{Mestre}             & \multicolumn{1}{l|}{DE}                          \\ \hline
\multicolumn{1}{|l|}{\begin{tabular}[c]{@{}l@{}}Protocolos de Interconex\~ao\\ de Redes\end{tabular}} & \multicolumn{1}{l|}{Erick Augusto Gomes de Melo}        & \multicolumn{1}{l|}{\begin{tabular}[c]{@{}l@{}}Telemática\\Telecomunicações\end{tabular}}             & \multicolumn{1}{l|}{Mestre}             & \multicolumn{1}{l|}{DE}                          \\ \hline
\multicolumn{1}{|l|}{\cellcolor[HTML]{FFFFFF}Estruturas de Dados I}                         & \multicolumn{1}{l|}{Otac\'ilio de Ara\'ujo Ramos Neto}   & \multicolumn{1}{l|}{Engenharia El\'etrica}             & \multicolumn{1}{l|}{Mestre}             & \multicolumn{1}{l|}{DE}                        \\ \hline
\multicolumn{1}{|l|}{Probabilidade e Estat\'istica}                                             & \multicolumn{1}{l|}{Cícero Demétrio Vieira de Barros}                 & \multicolumn{1}{l|}{Matem\'atica}  & \multicolumn{1}{l|}{Mestre}             & \multicolumn{1}{l|}{T-40}                          \\ \hline
\multicolumn{1}{|l|}{Arquitetura de Computadores}                                                      & \multicolumn{1}{l|}{Otac\'ilio de Ara\'ujo Ramos Neto} & \multicolumn{1}{l|}{Engenharia El\'etrica}  & \multicolumn{1}{l|}{Mestre}             & \multicolumn{1}{l|}{DE}                          \\ \hline
\multicolumn{1}{|l|}{Linguagens de Script}                                                         & \multicolumn{1}{l|}{Moisés Guimarães de Medeiros}       & \multicolumn{1}{l|}{Sistemas para Internet} & \multicolumn{1}{l|}{Especialista}       & \multicolumn{1}{l|}{T-40}                        \\ \hline
\end{tabular}
\end{table}

\begin{table}[h!]
\scriptsize
\begin{tabular}{lllll}
\multicolumn{5}{c}{\cellcolor[HTML]{C0C0C0}\textbf{Terceiro Semestre}}                                                                                                                                                                                                                                    \\ \hline
\multicolumn{1}{|p{5.2cm}|}{\cellcolor[HTML]{FFFFFF}\textbf{Disciplina}}                                    & \multicolumn{1}{p{5.2cm}|}{\textbf{Professor}}                 & \multicolumn{1}{p{3.1cm}|}{\textbf{Graduação}}     & \multicolumn{1}{p{1.7cm}|}{\textbf{Titulação}} & \multicolumn{1}{p{0.6cm}|}{\textbf{RT}} \\ \hline
\multicolumn{1}{|l|}{Intera\c{c}\~ao Humano-Computador}                                                            & \multicolumn{1}{l|}{Moisés Guimarães de Medeiros}             & \multicolumn{1}{l|}{Sistemas para Internet}          & \multicolumn{1}{l|}{Especialista}             & \multicolumn{1}{l|}{T-40}                          \\ \hline
\multicolumn{1}{|l|}{Bancos de Dados I} & \multicolumn{1}{l|}{Jos\'e de Sousa Barros}        & \multicolumn{1}{l|}{Sistemas de Informa\c{c}\~ao}             & \multicolumn{1}{l|}{Especialista}             & \multicolumn{1}{l|}{DE}                          \\ \hline
\multicolumn{1}{|l|}{\cellcolor[HTML]{FFFFFF}Estruturas de Dados II}                         & \multicolumn{1}{l|}{Ruan Delgado Gomes}   & \multicolumn{1}{l|}{Ci\^encia da Computa\c{c}\~ao}             & \multicolumn{1}{l|}{Mestre}             & \multicolumn{1}{l|}{DE}                        \\ \hline
\multicolumn{1}{|l|}{Sistemas Operacionais}                                             & \multicolumn{1}{l|}{Rodrigo Pinheiro Marques de Araújo}                 & \multicolumn{1}{l|}{Ci\^encia da Computa\c{c}\~ao}  & \multicolumn{1}{l|}{Mestre}             & \multicolumn{1}{l|}{DE}                          \\ \hline
\multicolumn{1}{|l|}{Metodologia da Pesquisa Cient\'ifica}                                                      & \multicolumn{1}{l|}{Erick Augusto Gomes de Melo} & \multicolumn{1}{l|}{\begin{tabular}[c]{@{}l@{}}Telemática\\Telecomunicações\end{tabular}}  & \multicolumn{1}{l|}{Mestre}             & \multicolumn{1}{l|}{DE}                          \\ \hline
\multicolumn{1}{|l|}{Programa\c{c}\~ao Orientada a Objetos}                                                         & \multicolumn{1}{l|}{Jos\'e de Sousa Barros}       & \multicolumn{1}{l|}{Sistemas de Informa\c{c}\~ao} & \multicolumn{1}{l|}{Especialista}       & \multicolumn{1}{l|}{DE}                        \\ \hline
\end{tabular}
\end{table}

\begin{table}[h!]
\scriptsize
\begin{tabular}{lllll}
\multicolumn{5}{c}{\cellcolor[HTML]{C0C0C0}\textbf{Quarto Semestre}}                                                                                                                                                                                                                                    \\ \hline
\multicolumn{1}{|p{5.2cm}|}{\cellcolor[HTML]{FFFFFF}\textbf{Disciplina}}                                    & \multicolumn{1}{p{5.2cm}|}{\textbf{Professor}}                 & \multicolumn{1}{p{3.1cm}|}{\textbf{Graduação}}     & \multicolumn{1}{p{1.7cm}|}{\textbf{Titulação}} & \multicolumn{1}{p{0.6cm}|}{\textbf{RT}} \\ \hline
\multicolumn{1}{|l|}{Programa\c{c}\~ao para a Web I}                                                            & \multicolumn{1}{l|}{Mois\'es Guimara\~es de Medeiros}             & \multicolumn{1}{l|}{Sistemas para Internet}          & \multicolumn{1}{l|}{Especialista}             & \multicolumn{1}{l|}{T-40}                          \\ \hline
\multicolumn{1}{|l|}{Bancos de Dados II} & \multicolumn{1}{l|}{Jos\'e de Sousa Barros}        & \multicolumn{1}{l|}{Sistemas de Informa\c{c}\~ao}             & \multicolumn{1}{l|}{Especialista}             & \multicolumn{1}{l|}{DE}                          \\ \hline
\multicolumn{1}{|l|}{\cellcolor[HTML]{FFFFFF}Programa\c{c}\~ao Paralela e Distribu\'ida}                         & \multicolumn{1}{l|}{Otac\'ilio de Ara\'ujo Ramos Neto}   & \multicolumn{1}{l|}{Engenharia El\'etrica}             & \multicolumn{1}{l|}{Mestre}             & \multicolumn{1}{l|}{DE}                        \\ \hline
\multicolumn{1}{|l|}{Legislação Social}                                             & \multicolumn{1}{l|}{Monique Ximenes Lopes de Medeiros}                 & \multicolumn{1}{l|}{Direito}  & \multicolumn{1}{l|}{Mestre}             & \multicolumn{1}{l|}{DE}                          \\ \hline
\multicolumn{1}{|l|}{Seguran\c{c}a da Informa\c{c}\~ao}                                                      & \multicolumn{1}{l|}{Mois\'es Guimar\~aes de Medeiros} & \multicolumn{1}{l|}{Sistemas para Internet}  & \multicolumn{1}{l|}{Especialista}             & \multicolumn{1}{l|}{T-40}                          \\ \hline
\multicolumn{1}{|l|}{An\'alise e Projeto de Sistemas}                                                         & \multicolumn{1}{l|}{Jos\'e de Sousa Barros}       & \multicolumn{1}{l|}{Sistemas de Informa\c{c}\~ao} & \multicolumn{1}{l|}{Especialista}       & \multicolumn{1}{l|}{DE}                        \\ \hline
\end{tabular}
\end{table}


\begin{table}[h!]
\scriptsize
\begin{tabular}{lllll}
\multicolumn{5}{c}{\cellcolor[HTML]{C0C0C0}\textbf{Quinto Semestre}}                                                                                                                                                                                                                                    \\ \hline
\multicolumn{1}{|p{5.2cm}|}{\cellcolor[HTML]{FFFFFF}\textbf{Disciplina}}                                    & \multicolumn{1}{p{5.2cm}|}{\textbf{Professor}}                 & \multicolumn{1}{p{3.1cm}|}{\textbf{Graduação}}     & \multicolumn{1}{p{1.7cm}|}{\textbf{Titulação}} & \multicolumn{1}{p{0.6cm}|}{\textbf{RT}} \\ \hline
\multicolumn{1}{|l|}{Programa\c{c}\~ao para a Web II}                                                            & \multicolumn{1}{l|}{Rodrigo Pinheiro Marques de Ara\'ujo}             & \multicolumn{1}{l|}{Ciência da Computação}          & \multicolumn{1}{l|}{Mestre}             & \multicolumn{1}{l|}{DE}                          \\ \hline
\multicolumn{1}{|l|}{Padr\~oes de Projeto de Software} & \multicolumn{1}{l|}{Rodrigo Pinheiro Marques de Ara\'ujo}        & \multicolumn{1}{l|}{Ci\^encia da Computa\c{c}\~ao}             & \multicolumn{1}{l|}{Mestre}             & \multicolumn{1}{l|}{DE}                          \\ \hline
\multicolumn{1}{|l|}{\begin{tabular}[c]{@{}l@{}}Ger\^encia e Configura\c{c}\~ao\\ de Servi\c{c}os para a Internet\end{tabular}}               & \multicolumn{1}{l|}{Erick Augusto Gomes de Melo}   & \multicolumn{1}{l|}{\begin{tabular}[c]{@{}l@{}}Telemática\\Telecomunicações\end{tabular}}             & \multicolumn{1}{l|}{Mestre}             & \multicolumn{1}{l|}{DE}                        \\ \hline
\multicolumn{1}{|l|}{Engenharia de Software}                                             & \multicolumn{1}{l|}{José de Sousa Barros}                 & \multicolumn{1}{l|}{Sistemas de Informa\c{c}\~ao}  & \multicolumn{1}{l|}{Especialista}             & \multicolumn{1}{l|}{DE}                          \\ \hline
\multicolumn{1}{|l|}{Programa\c{c}\~ao para Dispositivos M\'oveis}                                                      & \multicolumn{1}{l|}{Mois\'es Guimar\~aes de Medeiros} & \multicolumn{1}{l|}{Sistemas para Internet}  & \multicolumn{1}{l|}{Especialista}             & \multicolumn{1}{l|}{T-40}                          \\ \hline
\multicolumn{1}{|l|}{Empreendedorismo em Software}                                                         & \multicolumn{1}{l|}{Anna Carolina C. C. da Cunha}       & \multicolumn{1}{l|}{Adm. de Empresas} & \multicolumn{1}{l|}{Mestre}       & \multicolumn{1}{l|}{DE}                        \\ \hline
\end{tabular}
\end{table}

\begin{table}[h!]
\scriptsize
\begin{tabular}{lllll}
\multicolumn{5}{c}{\cellcolor[HTML]{C0C0C0}\textbf{Sexto Semestre}}                                                                                                                                                                                                                                    \\ \hline
\multicolumn{1}{|p{5.2cm}|}{\cellcolor[HTML]{FFFFFF}\textbf{Disciplina}}                                    & \multicolumn{1}{p{5.2cm}|}{\textbf{Professor}}                 & \multicolumn{1}{p{3.1cm}|}{\textbf{Graduação}}     & \multicolumn{1}{p{1.7cm}|}{\textbf{Titulação}} & \multicolumn{1}{p{0.6cm}|}{\textbf{RT}} \\ \hline
\multicolumn{1}{|l|}{Sistemas Distribu\'idos}                                                            & \multicolumn{1}{l|}{Ruan Delgado Gomes}             & \multicolumn{1}{l|}{Ciência da Computação}          & \multicolumn{1}{l|}{Mestre}             & \multicolumn{1}{l|}{DE}                          \\ \hline
\multicolumn{1}{|l|}{Com\'ercio Eletr\^onico} & \multicolumn{1}{l|}{Mois\'es Guimar\~aes de Medeiros}        & \multicolumn{1}{l|}{Sistemas para Internet}             & \multicolumn{1}{l|}{Especialista}             & \multicolumn{1}{l|}{T-40}                          \\ \hline
\multicolumn{1}{|l|}{\begin{tabular}[c]{@{}l@{}}Desenvolvimento de Aplica\c{c}\~oes\\ Corporativas\end{tabular}}               & \multicolumn{1}{l|}{José de Sousa Barros}   & \multicolumn{1}{l|}{Sistemas de Informa\c{c}\~ao}             & \multicolumn{1}{l|}{Especialista}             & \multicolumn{1}{l|}{DE}                        \\ \hline
\multicolumn{1}{|l|}{Projeto em TSI*}                                             & \multicolumn{1}{l|}{}                 & \multicolumn{1}{l|}{}  & \multicolumn{1}{l|}{}             & \multicolumn{1}{l|}{}                          \\ \hline
\multicolumn{1}{|l|}{T\'opicos Especiais*}                                                      & \multicolumn{1}{l|}{} & \multicolumn{1}{l|}{}  & \multicolumn{1}{l|}{}             & \multicolumn{1}{l|}{}                          \\ \hline
\multicolumn{1}{|l|}{TCC*}                                                         & \multicolumn{1}{l|}{}       & \multicolumn{1}{l|}{} & \multicolumn{1}{l|}{}       & \multicolumn{1}{l|}{}                        \\ \hline
\end{tabular}
\end{table}

\newpage
*As disciplinas Projeto em TSI, TCC e T\'opicos Especiais podem ser ministradas por diversos docentes, inclusive por mais de um docente em simult\^aneo.

Legenda:
\\~RT = Regime de Trabalho

\paragraph{Necessidades de contrata\c{c}\~ao de professores da \'area de inform\'atica}\

%Na tabela que indica os docentes por disciplina algumas disciplinas ainda n\~ao possuem professor com perfil adequado no campus Guarabira. No entanto, os dois primeiros per\'iodos do curso podem ser conduzidos com o corpo docente atual. 

A Tabela a seguir mostra a proje\c{c}\~ao de necessidade de contrata\c{c}\~ao de professores da \'area de inform\'atica at\'e o per\'iodo 2019.1, considerando que os professores da \'area de inform\'atica tamb\'em atuam no curso t\'ecnico integrado em inform\'atica e podem atuar tamb\'em em disciplinas relacionadas \`a inform\'atica em outros cursos, como \'e o caso do curso de gest\~ao comercial.

\begin{table}[h]
\footnotesize
\begin{tabular}{|l|l|l|c|l|l|l|l|}
\hline
\multicolumn{1}{|c|}{\textbf{Semestre}} & \multicolumn{1}{c|}{\textbf{SB}} & \multicolumn{1}{c|}{\textbf{INT}} & \textbf{GEST} & \multicolumn{1}{c|}{\textbf{TSI}} & \multicolumn{1}{c|}{\textbf{CH/SEM. TOT.}} & \multicolumn{1}{c|}{\textbf{\begin{tabular}[c]{@{}c@{}}Quantidade de\\ Professores\end{tabular}}} & \multicolumn{1}{c|}{\textbf{CH/SEM. DOC.}} \\ \hline
\rowcolor[HTML]{C0C0C0} 
2016.1                                  & 18                               & 37                                & 3             & 15                                & 73                                         & 6                                                                                                 & 12,17                                     \\ \hline
\rowcolor[HTML]{C0C0C0} 
2016.2                                  &                                  & 37                                & 3             & 30                                & 70                                         & 6                                                                                                 & 11,67                                      \\ \hline
2017.1                                  &                                  & 27                                & 3             & 51                                & 81                                        & 8                                                                                                & 10,13                                       \\ \hline
\rowcolor[HTML]{C0C0C0} 
2017.2                                  &                                  & 27                                & 3             & 72                                & 102                                       & 10                                                                                                & 10,2                                      \\ \hline
2018.1                                  &                                  & 27                                & 3             & 93                               & 123                                        & 12                                                                                                & 10,25                                      \\ \hline
\rowcolor[HTML]{C0C0C0} 
2018.2                                  &                                  & 27                                & 3             & 115                               & 145                                        & 13                                                                                                & 11,15                                      \\ \hline
\rowcolor[HTML]{C0C0C0} 
2019.1                                  &                                  & 27                                & 3             & 115                               & 145                                        & 14                                                                                                & 10,36                                      \\ \hline

\multicolumn{8}{|l|}{\begin{tabular}[c]{@{}l@{}}Legenda:\\ SB. = Informática Subsequente; \\ INT. = Inform\'atica Integrado; \\ GEST. = Gestão Comercial; \\ TSI. = Tecnologia em Sistemas para Internet;  \\ CH/SEM.TOT. = Carga Horária Semanal Total; \\ CH/SEM. DOC. = Carga Horária Semanal Média por Docente.\end{tabular}}                                      \\ \hline
\end{tabular}
\label{tab:profs}
\end{table}

É importante levar em consideração a demanda por professores da área de informática para lecionar disciplinas em outros cursos de outras áreas. Outros cursos, além dos já existentes, deverão iniciar as atividades até o período 2019.1. Após a abertura de novos cursos a tabela deverá ser atualizada para verificar se existe a necessidade de contratação de mais professores da área de informática, além do especificado até o momento. é importante notar também que o perfil “Informática Básica”, existente em todos os cursos integrados e em muitos superiores, é um perfil do núcleo comum dos cursos integrados, e talvez exista demanda para um professor desse perfil no futuro.

Atualmente contamos com 6 (seis) professores efetivos da \'area de inform\'atica. Dessa forma, para início do curso em 2016.1, o curso pode ser conduzido com o corpo docente atual. Para 2016.2 também é possível conduzir o curso com o corpo docente atual, tendo em vista o encerramento do curso subsequente. Para 2017.1, Seria necess\'aria a contrata\c{c}\~ao de mais dois professores. De maneira geral, as contrataç\~oes podem ocorrer de forma gradual, de modo que o risco se torna apenas moderado, como pode ser visto na tabela.

Essa proje\c{c}\~ao foi feita de modo que os docentes possam exercer, al\'em das atividades de ensino, atividades de apoio ao ensino (ex: n\'ucleos de aprendizagem), pesquisa, extens\~ao e cargos de gest\~ao no \^ambito do curso, sem que isso extrapole a carga hor\'aria m\'axima de trabalho dos docentes.

\subsection{Corpo T\'ecnico-Administrativo}

O Campus Guarabira do Instituto Federal da Paraíba é gerido por 1 Diretor-Geral e têm seu funcionamento estabelecido pelo Regimento Geral.

Os Diretores Gerais são escolhidos e nomeados no IFPB de acordo com o que determina o art. 13 da Lei nº. 11.892/2008, para mandato de 4 (quatro) anos, contado da data da posse, sendo permitida uma recondução.

O Diretor-Geral do Campus poderá propor à Reitoria a criação de núcleos avançados em municípios situados na micro-região do Estado da Paraíba, onde se situa ou do pólo da rede, após consulta ao respectivo Conselho Diretor.

O Campus utiliza o Regimento Interno do IFPB, o qual foi elaborado e aprovado pelo Conselho Diretor e submetido à apreciação e deliberação do Conselho Superior do Instituto Federal da Paraíba.

A organização geral do Campus Guarabira do Instituto Federal da Paraíba compreende:

a)	Diretoria Geral;

b)	Diretoria de Desenvolvimento de Ensino;

c)	Diretoria de Administração e Planejamento;

d)	Coordenação de Pesquisa e Extensão;

e)	Coordenação de Gestão de Pessoas;

f)	Secretaria – Controle Acadêmico;

g)	Coordenação do Curso Superior de Tecnologia em Gestão Comercial;

h)	Coordenação do Curso T\'ecnico Integrado em Inform\'atica;

i)	Coordenação do Curso T\'ecnico Integrado em Contabilidade;

j)	Coordenação do Curso T\'ecnico Integrado em Edifica\c{c}\~oes;

k)      Coordena\c{c}\~ao de Compras e Licita\c{c}\~ao;

l)      Coordena\c{c}\~ao de Tecnologia da Informa\c{c}\~ao;

m)      Coordena\c{c}\~ao de Execu\c{c}\~ao Or\c{c}ament\'aria e Financeira;

n)      Coordena\c{c}\~ao de Educa\c{c}\~ao \`a Dist\^ancia;

o)      Coordena\c{c}\~ao do Pronatec.


\subsubsection{Forma\c{c}\~ao e experi\^encia profissional do corpo t\'ecnico administrativo}


O quadro a seguir relaciona o corpo técnico administrativo do Campus que desenvolver\'a atividades relacionadas ao CST em Sistemas para Internet.
 
\newpage

\begin{table}[h!]
\tiny
\begin{tabular}{lllllllll}
\hline
\multicolumn{1}{|c|}{\multirow{2}{*}{MAT.}} & \multicolumn{1}{c|}{\multirow{2}{*}{Servidor}}        & \multicolumn{4}{c|}{Formação Acadêmica}                                                                                           & \multicolumn{1}{c|}{\multirow{2}{*}{Cargo}}              & \multicolumn{1}{c|}{\multirow{2}{*}{TEP}} & \multicolumn{1}{c|}{\multirow{2}{*}{TC}} \\ \cline{3-6}
\multicolumn{1}{|c|}{}                      & \multicolumn{1}{c|}{}                                 & \multicolumn{1}{l|}{GR}     & \multicolumn{1}{l|}{ESP} & \multicolumn{1}{l|}{ME} & \multicolumn{1}{l|}{DO} & \multicolumn{1}{c|}{}                                    & \multicolumn{1}{c|}{}                     & \multicolumn{1}{c|}{}                    \\ \hline
\multicolumn{1}{|l|}{1759173}               & \multicolumn{1}{l|}{Ana Carine da C. Goncalves}    & \multicolumn{1}{l|}{\rotatebox[origin=c]{90}{UFPB-2007}}    & \multicolumn{1}{l|}{\rotatebox[origin=c]{90}{UFPB-2010}}    & \multicolumn{1}{l|}{}       & \multicolumn{1}{l|}{}       & \multicolumn{1}{l|}{Bibliotecária - Documentalista}      & \multicolumn{1}{l|}{12}                   & \multicolumn{1}{l|}{32}                  \\ \hline
\multicolumn{1}{|l|}{1518883}               & \multicolumn{1}{l|}{Ana Luiza Rabelo Rolim}    & \multicolumn{1}{l|}{\rotatebox[origin=c]{90}{UFC-2004}}    & \multicolumn{1}{l|}{\rotatebox[origin=c]{90}{}}    & \multicolumn{1}{l|}{\rotatebox[origin=c]{90}{UNIFESP-2013}}       & \multicolumn{1}{l|}{}       & \multicolumn{1}{l|}{Médica}      & \multicolumn{1}{l|}{10}                   & \multicolumn{1}{l|}{15}                  \\ \hline
\multicolumn{1}{|l|}{1661421}               & \multicolumn{1}{l|}{Claudia Pereira do Nascimento}    & \multicolumn{1}{l|}{\rotatebox[origin=c]{90}{}}    & \multicolumn{1}{l|}{\rotatebox[origin=c]{90}{}}    & \multicolumn{1}{l|}{\rotatebox[origin=c]{90}{}}       & \multicolumn{1}{l|}{}       & \multicolumn{1}{l|}{Assistente em Administração}      & \multicolumn{1}{l|}{-}                   & \multicolumn{1}{l|}{40}  \\ \hline
\multicolumn{1}{|l|}{2230942}               & \multicolumn{1}{l|}{Erika Tayane Barbosa da Costa}    & \multicolumn{1}{l|}{\rotatebox[origin=c]{90}{}}    & \multicolumn{1}{l|}{\rotatebox[origin=c]{90}{}}    & \multicolumn{1}{l|}{\rotatebox[origin=c]{90}{}}       & \multicolumn{1}{l|}{}       & \multicolumn{1}{l|}{Assistente de Aluno}      & \multicolumn{1}{l|}{-}                   & \multicolumn{1}{l|}{3}  \\ \hline
\multicolumn{1}{|l|}{1958276}               & \multicolumn{1}{l|}{Genard D. de Aguiar Neto}     & \multicolumn{1}{l|}{\rotatebox[origin=c]{90}{IFPB-2010}}  & \multicolumn{1}{l|}{}             & \multicolumn{1}{l|}{}       & \multicolumn{1}{l|}{}       & \multicolumn{1}{l|}{Técnico de Tecnologia da Informação} & \multicolumn{1}{l|}{5}                    & \multicolumn{1}{l|}{38}                  \\ \hline
\multicolumn{1}{|l|}{2128967}               & \multicolumn{1}{l|}{Gilmara Henriques Ara\'ujo}   & \multicolumn{1}{l|}{\rotatebox[origin=c]{90}{UFPB - 2010}}  & \multicolumn{1}{l|}{}  & \multicolumn{1}{l|}{\rotatebox[origin=c]{90}{UFPB - 2013}}       & \multicolumn{1}{l|}{}       & \multicolumn{1}{l|}{Técnico em Assuntos Educacionais}         & \multicolumn{1}{l|}{10}                    & \multicolumn{1}{l|}{1}                  \\ \hline
\multicolumn{1}{|l|}{2079280}               & \multicolumn{1}{l|}{Helenoria de Albuquerque Mello}     & \multicolumn{1}{l|}{\rotatebox[origin=c]{90}{UFPB-2001}}  & \multicolumn{1}{l|}{}             & \multicolumn{1}{l|}{\rotatebox[origin=c]{90}{UFPB-2010}}       & \multicolumn{1}{l|}{}       & \multicolumn{1}{l|}{Assistente Social} & \multicolumn{1}{l|}{14}                    & \multicolumn{1}{l|}{7}                  \\ \hline
\multicolumn{1}{|l|}{2229572}               & \multicolumn{1}{l|}{Josenaldo Alves de Santana}     & \multicolumn{1}{l|}{\rotatebox[origin=c]{90}{}}  & \multicolumn{1}{l|}{}             & \multicolumn{1}{l|}{}       & \multicolumn{1}{l|}{}       & \multicolumn{1}{l|}{Assistente de Aluno} & \multicolumn{1}{l|}{-}                    & \multicolumn{1}{l|}{3}                  \\ \hline
\multicolumn{1}{|l|}{2185826}               & \multicolumn{1}{l|}{Lucas Leite Rangel de Pontes}   & \multicolumn{1}{l|}{\rotatebox[origin=c]{90}{UNIP\^e - 2012}}  & \multicolumn{1}{l|}{}  & \multicolumn{1}{l|}{}       & \multicolumn{1}{l|}{}       & \multicolumn{1}{l|}{Assistente de Aluno}         & \multicolumn{1}{l|}{2}                    & \multicolumn{1}{l|}{7}                  \\ \hline
\multicolumn{1}{|l|}{1598944}               & \multicolumn{1}{l|}{Jamilly de Lima A. Anizio} & \multicolumn{1}{l|}{\rotatebox[origin=c]{90}{UFPB - 2010}}  & \multicolumn{1}{l|}{}             & \multicolumn{1}{l|}{}       & \multicolumn{1}{l|}{}       & \multicolumn{1}{l|}{Bibliotecária - Documentalista}      & \multicolumn{1}{l|}{5}                    & \multicolumn{1}{l|}{39}                  \\ \hline
\multicolumn{1}{|l|}{1113575}               & \multicolumn{1}{l|}{Maria Luana Lopes de Oliveira}   & \multicolumn{1}{l|}{\rotatebox[origin=c]{90}{UFPE - 2012}}  & \multicolumn{1}{l|}{}  & \multicolumn{1}{l|}{}       & \multicolumn{1}{l|}{}       & \multicolumn{1}{l|}{Técnico em Enfermagem}         & \multicolumn{1}{l|}{1}                    & \multicolumn{1}{l|}{6}                  \\ \hline
\multicolumn{1}{|l|}{1930729}               & \multicolumn{1}{l|}{Rafael Ramos Perreira}            & \multicolumn{1}{l|}{\rotatebox[origin=c]{90}{UEPB-2012}}    & \multicolumn{1}{l|}{}             & \multicolumn{1}{l|}{}       & \multicolumn{1}{l|}{}       & \multicolumn{1}{l|}{Assistente em Administração}         & \multicolumn{1}{l|}{9}                    & \multicolumn{1}{l|}{40}                  \\ \hline
\multicolumn{1}{|l|}{1828093}               & \multicolumn{1}{l|}{Rucélio Gomes Sarmento}           & \multicolumn{1}{l|}{\rotatebox[origin=c]{90}{UFPB - 2004}}  & \multicolumn{1}{l|}{\rotatebox[origin=c]{90}{FIJ - 2012}}   & \multicolumn{1}{l|}{}       & \multicolumn{1}{l|}{}       & \multicolumn{1}{l|}{Assistente em Administração}         & \multicolumn{1}{l|}{7}                    & \multicolumn{1}{l|}{57}                  \\ \hline
\multicolumn{1}{|l|}{2125597}               & \multicolumn{1}{l|}{Simone Fernandes da Silva}   & \multicolumn{1}{l|}{\rotatebox[origin=c]{90}{UFPB - 2011}}  & \multicolumn{1}{l|}{\rotatebox[origin=c]{90}{CINTEP - 2014}}  & \multicolumn{1}{l|}{}       & \multicolumn{1}{l|}{}       & \multicolumn{1}{l|}{Pedagoga}         & \multicolumn{1}{l|}{3}                    & \multicolumn{1}{l|}{15}                  \\ \hline
\multicolumn{1}{|l|}{2019912}               & \multicolumn{1}{l|}{Sueli Pereira de Andrade}         & \multicolumn{1}{l|}{\rotatebox[origin=c]{90}{UFPB - 1994}}  & \multicolumn{1}{l|}{}             & \multicolumn{1}{l|}{}       & \multicolumn{1}{l|}{}       & \multicolumn{1}{l|}{Auxiliar em Administração}           & \multicolumn{1}{l|}{29}                   & \multicolumn{1}{l|}{25}                  \\ \hline


\end{tabular}
\end{table}
\begin{table}[h!]
\tiny
\begin{tabular}{lllllllll}
\hline
\multicolumn{1}{|c|}{\multirow{2}{*}{MAT.}} & \multicolumn{1}{c|}{\multirow{2}{*}{Servidor}}        & \multicolumn{4}{c|}{Formação Acadêmica}                                                                                           & \multicolumn{1}{c|}{\multirow{2}{*}{Cargo}}              & \multicolumn{1}{c|}{\multirow{2}{*}{TEP}} & \multicolumn{1}{c|}{\multirow{2}{*}{TC}} \\ \cline{3-6}
\multicolumn{1}{|c|}{}                      & \multicolumn{1}{c|}{}                                 & \multicolumn{1}{l|}{GR}     & \multicolumn{1}{l|}{ESP} & \multicolumn{1}{l|}{ME} & \multicolumn{1}{l|}{DO} & \multicolumn{1}{c|}{}                                    & \multicolumn{1}{c|}{}                     & \multicolumn{1}{c|}{}                    \\ \hline
\multicolumn{1}{|l|}{1934102}               & \multicolumn{1}{l|}{Ticiana Querino Guedes Cunha}     & \multicolumn{1}{l|}{\rotatebox[origin=c]{90}{UNIPê - 2004}} & \multicolumn{1}{l|}{}             & \multicolumn{1}{l|}{}       & \multicolumn{1}{l|}{}       & \multicolumn{1}{l|}{Auxiliar em Administração}           & \multicolumn{1}{l|}{11}                   & \multicolumn{1}{l|}{40}                  \\ \hline                                                                          \multicolumn{1}{|l|}{2126104}               & \multicolumn{1}{l|}{Victor Vidal Negreiros Bezerra}   & \multicolumn{1}{l|}{\rotatebox[origin=c]{90}{UFCG - 2013}}  & \multicolumn{1}{l|}{}  & \multicolumn{1}{l|}{}       & \multicolumn{1}{l|}{}       & \multicolumn{1}{l|}{Administrador}         & \multicolumn{1}{l|}{1}                    & \multicolumn{1}{l|}{15}                  \\ \hline

%\multicolumn{1}{|l|}{2185826}               & \multicolumn{1}{l|}{Diego Luis dos Santos Felix}   & \multicolumn{1}{l|}{}  & \multicolumn{1}{l|}{}  & \multicolumn{1}{l|}{}       & \multicolumn{1}{l|}{}       & \multicolumn{1}{l|}{Assistente de Aluno}         & \multicolumn{1}{l|}{9}                    & \multicolumn{1}{l|}{1}                  \\ \hline
\multicolumn{9}{l}{\begin{tabular}[c]{@{}l@{}}Legenda:\\ GR - Graduado; \\ ESP - Especialista; \\ ME - Mestre; \\ DO - Doutor; \\ MAT - Matrícula;\\ TEP – Tempo de Experiência Profissional (em ano);\\ TC – Tempo de Contrato na IES (em meses);\end{tabular}}                                                                                                                                                                                              
\end{tabular}
\end{table}

\newpage
 
\subsubsection{Plano de cargos e sal\'arios e incentivos ao pessoal t\'ecnico-administrativo}

A carreira de técnico-administrativo é regida pela Lei no 11.091, de 12 de janeiro de 2005 (PCCTAE), pela Lei no 8.112, de 11 de dezembro de 1990 e pela Constituição Federal, além da legislação vigente atreladas a essas Leis e a LDB, Lei no 9.394, de 20 de dezembro de 1996. 

O Instituto Federal da Paraíba tem uma política de qualificação e capacitação para os técnicos administrativos, que contempla a oferta de cursos de qualificação e atualização, além de propiciar oportunidades em cursos de pós-graduação por meio de parcerias com Universidades. Além disso, a implantação do CIS é uma realidade no Instituto, o que fortalece o processo de qualificação e capacitação do servidor. O Regime de Trabalho dos Técnicos Administrativos é de 40 horas semanais.


