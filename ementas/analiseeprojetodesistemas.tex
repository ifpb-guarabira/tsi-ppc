\paragraph{Análise e Projeto de Sistemas}

%PREENCHER DADOS DA DISCIPLINA A SEGUIR
%\vspace{-12mm}
\begin{center}\textbf{Dados do Componente Curricular}\end{center}
\vspace{-5mm}
\noindent\rule{16.5cm}{0.4pt}
\\
\textbf{Nome:} Análise e Projeto de Sistemas
\\
\textbf{Curso:} Tecnologia em Sistemas para Internet
\\ 
\textbf{Período:} $4^{\circ}$ 
\\
\textbf{Carga Horária:} 67~h 
\\ 
\textbf{Docente Responsável:} José de Sousa Barros 
\\ 
\noindent\rule{16.5cm}{0.4pt}\\
\\
%PREENCHER A EMENTA A SEGUIR
\vspace{-12mm}
\begin{center}\textbf{Ementa}\end{center}
\vspace{-5mm}
\noindent\rule{16.5cm}{0.4pt}
\\
Conceitos de Análise e Projeto de Sistemas. Modelos de ciclos de vida.  Metodologia para análise e desenvolvimento de sistemas orientados a objetos. Linguagem UML.  Análise de requisitos, Modelagem conceitual, Ferramenta CASE para criação de modelos orientados a objetos. \\
\noindent\rule{16.5cm}{0.4pt}\\
\\
%PREENCHER OS OBJETIVOS A SEGUIR
\vspace{-12mm}
\begin{center}\textbf{Objetivos}\end{center}
\vspace{-5mm}
\noindent\rule{16.5cm}{0.4pt}
\\
\begin{itemize}
\item Compreender os conceitos da Análise e Projeto Orientado a Objetos;
\item Aplicar uma Metodologia de Análise e Projeto de Software Orientado a Objetos;
\item Analisar e Projetar soluções orientadas a objetos utilizando UML para problemas do mundo real.
\end{itemize}
\noindent\rule{16.5cm}{0.4pt}\\
\\
%PREENCHER OS CONTEUDOS PROGRAMATICOS A SEGUIR (CUIDADO PARA NAO DEIXAR A TABELA MUITO GRANDE)
\vspace{-12mm}
\begin{center}\textbf{Conteúdo Programático}\end{center}
\vspace{-5mm}
\noindent\rule{16.5cm}{0.4pt}
\\
\begin{itemize}
 \item \textbf{Introdução à Análise e Projeto Orientados a Objetos:} Conceito de Análise e Projeto; Conceito de Análise e Projeto Orientados a Objetos; Modelos de ciclos de vida de software.

 \item \textbf{Análise de Requisitos:} Requisitos funcionais e não Funcionais;	Técnicas de elicitação de requisitos; Casos de uso: Conceito de casos de uso e atores, Diagrama de casos de de uso UML, Documentação de casos de uso.

 \item \textbf{Análise e Projeto:} Metodologia de Análise e Projeto de Software Orientados a Objetos.
 
 \item \textbf{Modelagem de Software com UML:} Visão geral da UML; Ferramenta CASE para criação de modelos orientados a objetos; Diagramas de Casos de Uso; Diagramas de Classes; Diagramas de Objetos; Diagramas de Atividades e Estados;  Diagramas de Interação: Sequência e Comunicação; Diagramas de Pacotes; Diagramas Implantação e Componentes.
 
\end{itemize}
\noindent\rule{16.5cm}{0.4pt}\\
\\
%COLOCAR A METODOLOGIA DE ENSINO A SEGUIR
\vspace{-12mm}
\begin{center}\textbf{Metodologia de Ensino}\end{center} 
\vspace{-5mm}
\noindent\rule{16.5cm}{0.4pt}
\\
   Aulas expositivas utilizando recursos audiovisuais e quadro, além de aulas práticas utilizando computadores. Adicionalmente, serão realizadas atividades práticas individuais ou em grupo, para consolidação do conteúdo ministrado.\\
\noindent\rule{16.5cm}{0.4pt}\\
\\
%COLOCAR AVALIACAO DO PROCESSO DE ENSINO E APRENDIZAGEM A SEGUIR
\vspace{-12mm}
\begin{center}\textbf{Avaliação do Processo de Ensino e Apendizagem}\end{center}
\vspace{-5mm}
\noindent\rule{16.5cm}{0.4pt}
\\
   Avaliações escritas. Práticas baseadas em Estudos de Caso ou problemas reais.\\
\noindent\rule{16.5cm}{0.4pt}\\
\\
%PREENCHER RECURSOS NECESSARIOS A SEGUIR
\vspace{-12mm}
\begin{center}\textbf{Recursos Necessários}\end{center}
\vspace{-5mm}
\noindent\rule{16.5cm}{0.4pt}
\\
\begin{itemize} 
  \item Listas de Exercícios;
  \item Livros e apostilas;
  \item Utilização de recursos da web;
  \item Quadro branco;
  \item Marcadores para quadro branco;
  \item Sala de aula com acesso à internet, microcomputador e TV ou projetor para apresentação de slides ou material multimídia;
  \item Laboratório de microcomputadores contendo componentes de hardware e \textit{software} específicos;
\end{itemize}
\noindent\rule{16.5cm}{0.4pt}\\
\\
%PREENCHER BIBLIOGRAFIA A SEGUIR
\vspace{-12mm}
\begin{center}\textbf{Bibliografia}\end{center}
\vspace{-5mm}
\noindent\rule{16.5cm}{0.4pt}
\\
\begin{itemize} 
  \item Básica:
	\begin{enumerate}
  	\item LARMAN, Craig. \textbf{Utilizando UML e Padrões: uma introdução à análise e ao projeto orientados a objetos e ao desenvolvimento iterativo.} Bookman,  3ª edição, 2007;
	\item MCLAUGHLIN, B.; et al. \textbf{Use a Cabeça Análise e Projeto Orientado a Objeto.} Alta Books, 2007;
	\item PILONE, D.; PITMAN, N. \textbf{UML 2: Rápido e Prático.} Alta Books, 2006.    
	\end{enumerate}
    
  \item Complementar:
	\begin{enumerate}
  	\item FOWLER, M.; SCOTT, K. \textbf{UML Essencial.} Porto Alegre: Bookman, 2005;
	\item ENGHOLM JR, H. \textbf{Análise e Design Orientado a Objetos.} Novatec. 2013.
	\end{enumerate}
\end{itemize}
\noindent\rule{16.5cm}{0.4pt}\\
\\
