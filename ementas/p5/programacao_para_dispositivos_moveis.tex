
% Dados do Componente Curricular

\begin{snugshade}\begin{center}\textbf{
    Dados do Componente Curricular
}\end{center}\end{snugshade}

\noindent \textbf{Nome:}                Programação para Dispositivos Móveis
\\        \textbf{Curso:}               Tecnologia em Sistemas para Internet
\\        \textbf{Período:}             \unit{5}{\degree}
\\        \textbf{Carga Horária:}       \unit{67}{\hour}
\\        \textbf{Docente Responsável:} Moisés Guimarães de Medeiros

% Ementa

\begin{snugshade}\begin{center}\textbf{
    Ementa
\vphantom{q}}\end{center}\end{snugshade}

\noindent
Visão geral das tecnologias móveis e sem fio. API de programação para dispositivos móveis e sem fio. Persistência de dados em dispositivos móveis. Integração entre dispositivos móveis e a Internet. Utilização de uma plataforma de programação para dispositivos móveis.

% Objetivos

\begin{snugshade}\begin{center}\textbf{
    Objetivos
}\end{center}\end{snugshade}

\begin{itemize}

\item Conhecer as principais plataformas de dispositivos móveis;

\item Conhecer os padrões de desenvolvimento para aplicações \textit{mobile};

\item Construir aplicações para dispositivos móveis.

\end{itemize}

% Conteúdo Programático

\begin{snugshade}\begin{center}\textbf{
    Conteúdo Programático
}\end{center}\end{snugshade}

\begin{itemize}

\item \textbf{Primeiros passos:}
    Android Studio, Gradle e ferramentas de depuração;
    \textit{User Interface} e gerenciadores de \textit{Layout};
    \textit{ListViews} e \textit{Adapters}.

\item \textbf{Interagindo com a Internet:}
    \textit{Threading} e \textit{ASyncTask};
    \textit{HTTP requests} em \textit{APIs web};
    Sistema de permissões do Android;

\item \textbf{Criando novas telas:}
    Navegação com \textit{Intents} explícitos;
    Utilizando \textit{Intents} implícitos para incorporar aplicações de terceiros;
    Compartilhando \textit{Intents} e o \textit{framework} de compartilhamento do Android;
    \textit{Broadcast Intents} e  \textit{Broadcast Receivers}.

\item \textbf{Utilizando provedores de conteúdo para persistência:}
    Ciclo de vida de uma \textit{Activity};
    Bancos de dados \textit{SQLite} e testes \textit{JUnit};
    Criando e utilizando provedores de conteúdo como uma camada de abstração;
    Utilizando \textit{Loaders} para carregar dados de forma assíncrona.
    

\item \textbf{Implementando \textit{Layouts} ricos e responsivos:}
    Princípios fundamentais de desenvolvimento do Android;
    Suportando localização e diferentes tamanhos de tela;
    Otimizando UIs para \textit{tablets} utilizando fragmentos;
    Características de Acessibilidade;
    \textit{Views} customizadas.

\item \textbf{Executando em \textit{background}:}
    Serviços em \textit{background} e alarmes para agendar tarefas em \textit{background};
    Transferência de dados efetiva com \textit{SyncAdapters};
    Notificações ricas para interagir com o usuário.

\end{itemize}

% Metodologia, Avaliação e Recursos

\begin{snugshade}\begin{center}\textbf{
    Metodologia de Ensino
}\end{center}\end{snugshade}

\noindent
Aulas expositivas utilizando recursos audiovisuais e quadro, além de aulas práticas utilizando computadores. Adicionalmente, serão realizadas atividades práticas individuais ou em grupo, para consolidação do conteúdo ministrado.

\begin{snugshade}\begin{center}\textbf{
    Avaliação do Processo de Ensino e Aprendizagem
}\end{center}\end{snugshade}

\noindent
Avaliações escritas. Avaliações práticas envolvendo a resolução de problemas computacionais.

\begin{snugshade}\begin{center}\textbf{
    Recursos Necessários
    \vphantom{q} % TODO: corrigir o depth da linha sem esta gambiarra.
}\end{center}\end{snugshade}

\begin{itemize}
  \item Listas de Exercícios;
  \item Livros e apostilas;
  \item Utilização de recursos da web;
  \item Quadro branco;
  \item Marcadores para quadro branco;
  \item Sala de aula com acesso à internet, microcomputador e TV ou projetor para apresentação de slides ou material multimídia;
  \item Laboratório de microcomputadores contendo componentes de hardware e software específicos;
\end{itemize}


% Bibliografia

\begin{snugshade}\begin{center}\textbf{
    Bibliografia
}\end{center}\end{snugshade}

\begin{itemize}

\item Básica:
    \begin{enumerate}

    \item QUERINO FILHO, Luiz Carlos.
          Desenvolvendo seu primeiro Aplicativo Android.
          Novatec, 1a edição, 2013.

    \item SAMPAIO, Cleuton.
          Manual do Indie Game Developer.
          Ciência Moderna, 1a edição, 2013.

    \item DARWIN, Ian F.
          Android Cookbook: Problemas e soluções para desenvolvedores Android.
          Bookman, 1a edição, 2013.

    \end{enumerate}

\item Complementar:
    \begin{enumerate}

    \item ALLAN, Alasdair.
          Aprendendo Programação iOS: Do Xcode à AppStore
          Novatec, 1a edição, 2013.

    \item NEIL, Theresa
          Padrões de Design para Aplicativos Móveis.
          Novatec, 1a edição, 2012.

    \end{enumerate}

\end{itemize}
