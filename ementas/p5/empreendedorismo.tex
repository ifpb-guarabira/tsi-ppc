
% Dados do Componente Curricular

\begin{snugshade}\begin{center}\textbf{
    Dados do Componente Curricular
}\end{center}\end{snugshade}

\noindent \textbf{Nome:}                Empreendedorismo em Software
\\        \textbf{Curso:}               Tecnologia em Sistemas para Internet
\\        \textbf{Período:}             \unit{5}{\degree}
\\        \textbf{Carga Horária:}       \unit{67}{\hour}
\\        \textbf{Docente Responsável:} Anna Carolina C. C. da Cunha

% Ementa

\begin{snugshade}\begin{center}\textbf{
    Ementa
\vphantom{q}}\end{center}\end{snugshade}

\noindent
Fundamentos de Administração; Conceitos de Empreendedorismo e Empreendedor; O processo empreendedor; Iniciando uma empresa: o plano de negócios; Mercado, gestão de operações e análise financeira; Modelo de negócios de software; Produtos e Serviços de software; O plano de negócios de software.

% Objetivos

\begin{snugshade}\begin{center}\textbf{
    Objetivos
}\end{center}\end{snugshade}  


\begin{itemize}

\item Incentivar o espírito empreendedor e desenvolver a capacidade de elaboração de um plano de negócios aplicado a um negócio de software;

\item Entender os conceitos básicos sobre empreendedorismo com foco em software;

\item Desenvolver e identificar idéias e oportunidades;

\item Estimular criatividade e o trabalho em equipe;

\item Aproximar o aluno da realidade dos negócios na área de software. 

\end{itemize} 

% Conteúdo Programático

%\begin{snugshade}\begin{center}\textbf{
%    Conteúdo Programático
%}\end{center}\end{snugshade}

%\begin{itemize}

%\item \textbf{Introdução à Teoria Geral da Administração:}
%Origens do pensamento empreendedor; O empreendedorismo: conceitos e características; Perfil do %empreendedor; Processo empreendedor.

%\item \textbf{Identificação e avaliação de oportunidades:}
%O que é um negócio; Entendendo o ambiente dos negócios; Modelos de negócios; Empresa: conceitos e tipos de negócios.

%\item \textbf{Plano de negócios:} 
%O que é o plano de negócios e qual a sua importância; Estrutura do plano de negócios; Entendendo o mercado; O mercado de software; Gestão de operações; Gestão de finanças; O plano de negócios de software.

%\end{itemize}

% Metodologia, Avaliação e Recursos

%\begin{snugshade}\begin{center}\textbf{
%    Metodologia de Ensino
%}\end{center}\end{snugshade}

%\noindent
%Aulas expositivas utilizando recursos audiovisuais e quadro. Adicionalmente, serão realizadas atividades práticas individuais ou em grupo, para consolidação do conteúdo ministrado.

%\begin{snugshade}\begin{center}\textbf{
 %   Avaliação do Processo de Ensino e Aprendizagem
%}\end{center}\end{snugshade}

%\noindent
%Contínua por meio de: avaliações escritas, estudos de caso, trabalhos teóricos/práticos individuais ou em grupo orientados em sala de aula, elaboração e apresentação do Projeto do Plano de Negócios de Software. 

%\begin{snugshade}\begin{center}\textbf{
 %   Recursos Necessários
  %  \vphantom{q} % TODO: corrigir o depth da linha sem esta gambiarra.
%}\end{center}\end{snugshade}

%\begin{itemize}
%  \item Listas de Exercícios;
%  \item Livros e apostilas;
%  \item Utilização de recursos da web;
%  \item Quadro branco;
%  \item Marcadores para quadro branco;
%  \item Sala de aula com acesso à internet, microcomputador e TV ou projetor para apresentação de slides ou material multimídia;
%\end{itemize}

% Bibliografia

\begin{snugshade}\begin{center}\textbf{
    Bibliografia
}\end{center}\end{snugshade}

\begin{itemize} 

\item Básica:

    \begin{enumerate}

    \item CHIAVENATO, I. Empreendedorismo:Dando Asas ao Espírito 		Empreendedor. 3 ed. São Paulo: Saraiva, 2008.

    \item GRINBERG, M.
		DOLABELA, Fernando. O segredo de Luísa. São Paulo: Sextante, 2002.
    
    \item DORNELAS, J.C.A. Empreendedorismo: transformando ideias em 		negócios. 4 ed. São Paulo: Campus, 2012.

    \end{enumerate}

\item Complementar:
	\begin{enumerate} 

    \item DORNELAS, J.C.A. Empreendedorismo Corporativo. São Paulo: Campus, 		2008.

    \item DRUCKER, Peter F. Inovação e espírito empreendedor. São Paulo: 		Thomson Learning, 2005.

    \item HISRICH, Robert; PETERS, Michael. Empreendedorismo. 5 ed. Porto 		Alegre: Bookman, 2004. (Capítulos: 7, 8, 9 e 10)
	\end{enumerate}

\end{itemize}



