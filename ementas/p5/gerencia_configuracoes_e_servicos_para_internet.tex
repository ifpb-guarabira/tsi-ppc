\paragraph{Gerência de Configuração e Serviços para Internet} \

% Dados do Componente Curricular

\begin{snugshade}\begin{center}\textbf{
    Dados do Componente Curricular
}\end{center}\end{snugshade}

\noindent \textbf{Nome:}                Gerência de Configuração e Serviços para Internet
\\        \textbf{Curso:}               Tecnologia em Sistemas para Internet
\\        \textbf{Período:}             \unit{5}{\degree}
\\        \textbf{Carga Horária:}       \unit{67}{\hour}
\\        \textbf{Docente Responsável:} Rodrigo Pinheiro Marques de Araújo

% Ementa

\begin{snugshade}\begin{center}\textbf{
    Ementa
\vphantom{q}}\end{center}\end{snugshade}

\noindent
Instalação e configuração de servidores. Configuração dos serviços de páginas para internet, banco de dados, mensagens eletrônicas, nomes e acesso remoto.

% Objetivos

\begin{snugshade}\begin{center}\textbf{
    Objetivos
}\end{center}\end{snugshade}

\begin{itemize}

\item Capacitar no uso de sistemas operacionais para servidores;

\item Habilitar na instalação de sistemas operacionais para servidores;

\item Eleger tecnologias adequadas a prestação dos serviços desejados;

\item Compreender a configuração de serviços de páginas para internet;

\item Conceber a configuração de bancos de dados;

\item Entender a configuração de serviços de mensagens eletrônicas;

\item Instruir na configuração de serviços de nomes;

\item Assimilar a configuração de serviços de acesso remoto;

\end{itemize} 

% Conteúdo Programático

\begin{snugshade}\begin{center}\textbf{
    Conteúdo Programático
}\end{center}\end{snugshade}

\begin{itemize}

\item \textbf{Sistemas operacionais para servidores:}
    Introdução; Comandos básicos;

\item \textbf{Serviço de páginas para internet:}
    Instalação, configuração e tecnologias;

\item \textbf{Serviço de banco de dados:}
    Instalação, configuração e tecnologias;
    
\item \textbf{Serviço de mensagens eletrônicas:}
    Instalação, configuração e tecnologias;
    
\item \textbf{Serviço de nomes:}
    Instalação, configuração e tecnologias;
    
\item \textbf{Serviço de acesso remoto:}
    Instalação, configuração e tecnologias;    

\end{itemize}

% Metodologia, Avaliação e Recursos

\begin{snugshade}\begin{center}\textbf{
    Metodologia de Ensino
}\end{center}\end{snugshade}

\noindent
Aulas expositivas utilizando recursos audiovisuais e quadro, além de aulas práticas utilizando computadores. Adicionalmente, serão realizadas atividades práticas individuais ou em grupo, para consolidação do conteúdo ministrado.

\begin{snugshade}\begin{center}\textbf{
    Avaliação do Processo de Ensino e Aprendizagem
}\end{center}\end{snugshade}

\noindent
Avaliações escritas. Avaliações práticas envolvendo a resolução de problemas computacionais.

\begin{snugshade}\begin{center}\textbf{
    Recursos Necessários
    \vphantom{q} % TODO: corrigir o depth da linha sem esta gambiarra.
}\end{center}\end{snugshade}

\begin{itemize}
  \item Listas de Exercícios;
  \item Livros e apostilas;
  \item Utilização de recursos da web;
  \item Quadro branco;
  \item Marcadores para quadro branco;
  \item Sala de aula com acesso à internet, microcomputador e TV ou projetor para apresentação de slides ou material multimídia;
  \item Laboratório de microcomputadores contendo componentes de hardware e software específicos;
\end{itemize}


% Bibliografia

\begin{snugshade}\begin{center}\textbf{
    Bibliografia
}\end{center}\end{snugshade}

\begin{itemize} 

\item Básica:
    \begin{enumerate}

    \item NEMETH, E.; SNYDER, G.; HEIN, T.
          Manual Completo do Linux: Guia do Administrador.
          Pearson, 2004.
    
    \item FERREIRA, R. E.
          Linux, Guia do Administrador do sistema
          Novatec, 2008.
    
    \item SIEVER, E.; WEBER A.; FIGGINS S.; LOVE R.; ROBBINS A.
          Linux, O Guia Essencial
          O'Reilly Media, 2006
	
    \end{enumerate}

\item Complementar:
	\begin{enumerate} 

    \item MORIMOTO, C.
          Linux, Guia Prático.
          GDH Press e Sul Editores, 2009.

    \item MORIMOTO, C.
          Servidores Linux, Guia Prático.
          GDH Press e Sul Editores, 2008.

    \item LAURIE, B; LAURIE, P.
          Apache: The definitive Guide.
          O'Reilly Media, 2002.

	\end{enumerate}

\end{itemize}

