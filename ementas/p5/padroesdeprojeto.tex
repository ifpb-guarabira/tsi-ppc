
% Dados do Componente Curricular

\begin{snugshade}\begin{center}\textbf{
	Dados do Componente Curricular
}\end{center}\end{snugshade}

\noindent 	\textbf{Nome:} Padrões de Projeto
\\ 			\textbf{Curso:} Tecnologia em Sistemas para Internet
\\ 			\textbf{Período:} \unit{5}{\degree}
\\ 			\textbf{Carga Horária:} \unit{67}{\hour}
\\ 			\textbf{Docente Responsável:} Rodrigo Pinheiro Marques de Araújo

% Ementa

\begin{snugshade}\begin{center}\textbf{
    Ementa
\vphantom{q}}\end{center}\end{snugshade}

\noindent
Caracterização dos padrões de projeto; Padrões e reusabilidade; Tipos de padrões de projeto; Aplicação de padrões de projeto no desenvolvimento de software orientado a objetos.

% Objetivos

\begin{snugshade}\begin{center}\textbf{
    Objetivos
}\end{center}\end{snugshade}


\begin{itemize}

\item Compreender conceitos e técnicas dos padrões de projeto de software necessárias para a modelagem e análise de sistemas;
\item Compreender os princípios da programação orientada a objetos;
\item Identificar os princípios básicos dos padrões de Projeto de software;
\item Compreender os padrões GRASP;
\item Compreender os padrões GoF;

\end{itemize}

% Conteúdo Programático

\begin{snugshade}\begin{center}\textbf{
    Conteúdo Programático
}\end{center}\end{snugshade}

\begin{itemize}

 \item \textbf{Introdução aos Padrões de Projeto:} Revisão histórica; Conceitos básicos da Orientação a Objetos; Padrões Básicos.

 \item \textbf{Padrões GRASP: } Padrão Expert; Padrão Creator; Padrão Low Coupling; Padrão High Cohesion; Padrão Model View Controller (MVC).

 \item \textbf{Padrões GoF de interface:}  Adapter; Bridge; Facade; Composite.

 \item \textbf{Padrões GoF de Responsabilidade:} Singleton; Observer; Mediator; Chain of Responsability; Proxy.

 \item \textbf{Padrões GoF de Construção:} Builder; Abstract Factory; Factory Method.

 \item \textbf{Padrões GoF de Operações:} Command; Strategy.

 \item \textbf{Padrões de Extensão:} Decorator; Iterator.

\end{itemize}

\begin{snugshade}\begin{center}\textbf{
    Metodologia de Ensino
}\end{center}\end{snugshade} 

\noindent
   Aulas expositivas utilizando recursos audiovisuais e quadro, além de aulas práticas utilizando computadores. Adicionalmente, serão realizadas atividades práticas individuais ou em grupo, para consolidação do conteúdo ministrado.

\begin{snugshade}\begin{center}\textbf{
    Avaliação do Processo de Ensino e Aprendizagem
}\end{center}\end{snugshade}   

\noindent
   Avaliações escritas e práticas.

\begin{snugshade}\begin{center}\textbf{
    Recursos Necessários
    \vphantom{q} % TODO: corrigir o depth da linha sem esta gambiarra.
}\end{center}\end{snugshade}

\begin{itemize} 
  \item Listas de Exercícios;
    \item Livros e apostilas;
    \item Utilização de recursos da web;
    \item Quadro branco;
    \item Marcadores para quadro branco;
    \item Sala de aula com acesso à internet, microcomputador e TV ou projetor para apresentação de slides ou material multimídia;
    \item Laboratório de microcomputadores contendo componentes de hardware e software específicos;
\end{itemize}


% Bibliografia

\begin{snugshade}\begin{center}\textbf{
    Bibliografia
}\end{center}\end{snugshade}

\begin{itemize} 
  \item Básica:
	\begin{enumerate}
	\item Gamma, E. et al. Padrões de Projeto: Soluções reutilizáveis de software orientado a objetos. Bookman, 2000.
	\item Freeman, E; Freeman, E. Use a cabeça! Padrões de Projeto (Design Patterns). Alta books, 2ª Edição. 2007.      
	\end{enumerate}
  \item Complementar:
	\begin{enumerate} 
	\item Metsker, S. J. Padrões de Projeto em Java. Bookman, 2004.
	\item Shalloway, A.; Trott, J. R. Explicando padrões de projeto – Uma nova perspectiva em projeto orientado a objetos. Bookman, 2004.
	\end{enumerate}
\end{itemize}
