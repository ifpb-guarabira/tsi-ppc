
% Dados do Componente Curricular

\begin{snugshade}\begin{center}\textbf{
	Dados do Componente Curricular
}\end{center}\end{snugshade}

\noindent 	\textbf{Nome:} Engenharia de Software
\\ 			\textbf{Curso:} Tecnologia em Sistemas para Internet
\\ 			\textbf{Período:} \unit{5}{\degree}
\\ 			\textbf{Carga Horária:} \unit{67}{\hour}
\\ 			\textbf{Docente Responsável:} José de Sousa Barros 


% Ementa

\begin{snugshade}\begin{center}\textbf{
    Ementa
\vphantom{q}}\end{center}\end{snugshade}

\noindent
Processos de Software: Modelos de Processo, Desenvolvimento Ágil;
Gestão da Qualidade: Técnicas de Revisão, Garantia da Qualidade de Software, Estratégias de Teste de Software, Modelagem Formal e Verificação, Métricas de Produto. Gerenciamento de Projetos de Software: Métricas de Processo e Projeto, Estimativas de Projeto de Software, Cronograma de Projeto, Gestão de Risco, Manutenção e Reengenharia.


% Objetivos

\begin{snugshade}\begin{center}\textbf{
    Objetivos
}\end{center}\end{snugshade}


\begin{itemize}

\item Compreender os modelos de processo de desenvolvimento de software e o desenvolvimento ágil;
\item Conhecer diferentes abordagens para avaliar a qualidade de um software;
\item Saber gerenciar projetos de desenvolvimento software para problemas do mundo real.

\end{itemize}

% Conteúdo Programático

%\begin{snugshade}\begin{center}\textbf{
%    Conteúdo Programático
%}\end{center}\end{snugshade}

%\begin{itemize}

% \item \textbf{Processos de Software:} Conceitos de Processos; Modelos de Processo; Desenvolvimento Ágil
 
%  \item \textbf{Gestão da Qualidade:} Conceitos de Qualidade; Técnicas de Revisão; Garantia da Qualidade de Software; Estratégias de Teste de Software; Testando Aplicativos Convencionais; Testando Aplicações Orientadas a Objeto; Testando Aplicações para Web; Modelagem Formal e Verificação; Métricas de Produto.
% \item \textbf{Gerenciamento de Projetos de Software:}  Conceitos de Gerenciamento de Projeto; Métricas de Processo e Projeto; Estimativas de Projeto de Software; Cronograma de Projeto; Gestão de Risco; Manutenção e Reengenharia.

%\end{itemize}

%\begin{snugshade}\begin{center}\textbf{
%    Metodologia de Ensino
%}\end{center}\end{snugshade} 

%\noindent
%   Aulas expositivas utilizando recursos audiovisuais e quadro, além de aulas práticas utilizando computadores. Adicionalmente, serão realizadas atividades práticas individuais ou em grupo, para consolidação do conteúdo ministrado.

%\begin{snugshade}\begin{center}\textbf{
%    Avaliação do Processo de Ensino e Aprendizagem
%}\end{center}\end{snugshade}   

%\noindent
%  Avaliações escritas. Práticas baseadas em Estudos de Caso ou problemas reais.


%\begin{snugshade}\begin{center}\textbf{
%    Recursos Necessários
%    \vphantom{q} % TODO: corrigir o depth da linha sem esta gambiarra.
%}\end{center}\end{snugshade}

%\begin{itemize} 
%  \item Listas de Exercícios;
%    \item Livros e apostilas;
%    \item Utilização de recursos da web;
%    \item Quadro branco;
%    \item Marcadores para quadro branco;
%    \item Sala de aula com acesso à internet, microcomputador e TV ou projetor para apresentação de slides ou material multimídia;
 %   \item Laboratório de microcomputadores contendo componentes de hardware e \textit{software} específicos;
%\end{itemize}

% Bibliografia

\begin{snugshade}\begin{center}\textbf{
    Bibliografia
}\end{center}\end{snugshade}

\begin{itemize} 
  \item Básica:
	\begin{enumerate}
	\item PRESSMAN, Roger S. \textbf{Engenharia de Software: uma abordagem profissional.} McGraw-Hill,  7ª edição, 2011.    
	\end{enumerate}
  \item Complementar:
	\begin{enumerate} 
	\item  SOMMERVILLE, I. \textbf{Engenharia de Software.} Pearson
	Education, 9ª Edição, 2011.
	\end{enumerate}
\end{itemize}
