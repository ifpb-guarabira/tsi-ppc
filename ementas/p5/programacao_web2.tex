\paragraph{Programação Web II} \

% Dados do Componente Curricular

\begin{snugshade}\begin{center}\textbf{
    Dados do Componente Curricular
}\end{center}\end{snugshade}

\noindent \textbf{Nome:}                Programação para a Web II
\\        \textbf{Curso:}               Tecnologia em Sistemas para Internet
\\        \textbf{Período:}             \unit{5}{\degree}
\\        \textbf{Carga Horária:}       \unit{83}{\hour}
\\        \textbf{Docente Responsável:} Rodrigo Pinheiro Marques de Araújo

% Ementa

\begin{snugshade}\begin{center}\textbf{
    Ementa
\vphantom{q}}\end{center}\end{snugshade}

\noindent
Desenvolvimento de aplicações para internet com suporte a registro de atividades e internacionalização. Construção de testes e integração contínua da aplicação Web. Gerenciamento de mudanças na base de dados. Serialização de dados. Implantação de aplicações Web em plataformas como serviço (PaaS). Sistemas de gerenciamento de conteúdo. Bibliotecas ricas para interfaces de aplicações Web.


% Objetivos

\begin{snugshade}\begin{center}\textbf{
    Objetivos
}\end{center}\end{snugshade}

\begin{itemize}

\item Entender os conceitos de testes para Web e integração contínua;

\item Compreender o gerenciamento de mudanças na base de dados;

\item Capacitar na utilização de plataformas como serviço para implantação de aplicações;

\item Assimilar os sistemas de gerenciamento de conteúdo;

\item Conhecer as bibliotecas para construção de interfaces;

\item Construir aplicações com suporte a registro de atividades e internacionalização;

\item Conhecer como serializar dados;

\end{itemize} 

% Conteúdo Programático

\begin{snugshade}\begin{center}\textbf{
    Conteúdo Programático
}\end{center}\end{snugshade}

\begin{itemize}

\item \textbf{Registro de atividades em uma aplicação Web}
\item \textbf{Internacionalização de aplicações}
\item \textbf{Testes para Web}
\item \textbf{Integração contínua}
\item \textbf{Serialização de dados}
\item \textbf{Migrações na base de dados}
\item \textbf{Sistemas de gerenciamento de conteúdo:}
    Tecnologias disponíveis; Instalação, configuração e personalização;
\item \textbf{Plataformas como Serviço:}
    Plataformas disponíveis; Desenvolvimento dirigido a implantação em plataformas como serviço;
\item \textbf{Bibliotecas ricas para interfaces de aplicações Web:}
    Construção de interfaces com bibliotecas JavaScript para gerenciamento de apresentação e responsividade;

\end{itemize}

% Metodologia, Avaliação e Recursos

\begin{snugshade}\begin{center}\textbf{
    Metodologia de Ensino
}\end{center}\end{snugshade}

\noindent
Aulas expositivas utilizando recursos audiovisuais e quadro, além de aulas práticas utilizando computadores. Adicionalmente, serão realizadas atividades práticas individuais ou em grupo, para consolidação do conteúdo ministrado.

\begin{snugshade}\begin{center}\textbf{
    Avaliação do Processo de Ensino e Aprendizagem
}\end{center}\end{snugshade}

\noindent
Avaliações escritas. Avaliações práticas envolvendo a resolução de problemas computacionais.

\begin{snugshade}\begin{center}\textbf{
    Recursos Necessários
    \vphantom{q} % TODO: corrigir o depth da linha sem esta gambiarra.
}\end{center}\end{snugshade}

\begin{itemize}
  \item Listas de Exercícios;
  \item Livros e apostilas;
  \item Utilização de recursos da web;
  \item Quadro branco;
  \item Marcadores para quadro branco;
  \item Sala de aula com acesso à internet, microcomputador e TV ou projetor para apresentação de slides ou material multimídia;
  \item Laboratório de microcomputadores contendo componentes de hardware e software específicos;
\end{itemize}


% Bibliografia

\begin{snugshade}\begin{center}\textbf{
    Bibliografia
}\end{center}\end{snugshade}

\begin{itemize} 

\item Básica:
    \begin{enumerate}

    \item GALESI, T.; SANTANA NETO, O.
          Python e Django - Desenvolvimento Ágil de Aplicações Web.
          NOVATEC, 2010

    \item GRINBERG, M.
          Flask Web Development: Developing Web Applications with Python.
          O'Reilly Media, 2014.

    \item GREENFIELD, D.; ROY, A.
          Two Scoopes of Django: Best Pratices For Django 1.6.
          Two Scoopes Press, 2014.
    

	
    \end{enumerate}

\item Complementar:
	\begin{enumerate} 

    \item RICHARDSON, L.; AMUNDSEN, M.; RUBY, S.
          Restful Web APIs.
          O'Reilly Media, 2013.

    \item SPURLOCK, J.
          Bootstrap.
          O'Reilly Media, 2013.

    \item MENEZES, N. N. C.
          Introdução a programação com Python.
          Novatec, 2014.

	\end{enumerate}

\end{itemize}



