\paragraph{Bancos de Dados I}

%PREENCHER DADOS DA DISCIPLINA A SEGUIR
%\vspace{-12mm}
\begin{center}\textbf{Dados do Componente Curricular}\end{center}
\vspace{-5mm}
\noindent\rule{16.5cm}{0.4pt}
\\
\textbf{Nome:} Bancos de Dados I
\\
\textbf{Curso:} Tecnologia em Sistemas para Internet
\\ 
\textbf{Período:} $3^{\circ}$ 
\\
\textbf{Carga Horária:} 67~h 
\\ 
\textbf{Docente Responsável:} José de Sousa Barros 
\\ 
\noindent\rule{16.5cm}{0.4pt}\\
\\
%PREENCHER A EMENTA A SEGUIR
\vspace{-12mm}
\begin{center}\textbf{Ementa}\end{center}
\vspace{-5mm}
\noindent\rule{16.5cm}{0.4pt}
\\ 
Introdução a bancos de dados. Conceitos básicos e terminologias de bancos de dados. Sistemas de Gerenciamento de Bancos de Dados. Modelos e esquemas de dados. Modelo entidade-relacionamento. O modelo relacional. Álgebra relacional. Linguagem de consulta estruturada (SQL). Projeto de bancos de dados relacional: normalização, restrições, índices, chaves primária e estrangeira. Visões. Subprogramas armazenados e gatilhos. Controle transacional. 
 \\

\noindent\rule{16.5cm}{0.4pt}\\
\\
%PREENCHER OS OBJETIVOS A SEGUIR
\vspace{-12mm}
\begin{center}\textbf{Objetivos}\end{center}
\vspace{-5mm}
\noindent\rule{16.5cm}{0.4pt}
\\
\begin{itemize}
\item Compreender os conceitos fundamentais de banco de dados;
\item Construir modelos conceituais de banco de dados usando o modelo de entidade-relacionamento;
\item Desenvolver modelos lógicos relacionais baseados em modelos conceituais;
\item Saber utilizar a linguagem SQL para recuperar e manipular informações em um banco de dados relacional.
\end{itemize}
\noindent\rule{16.5cm}{0.4pt}\\
\\
%PREENCHER OS CONTEUDOS PROGRAMATICOS A SEGUIR (CUIDADO PARA NAO DEIXAR A TABELA MUITO GRANDE)
\vspace{-12mm}
\begin{center}\textbf{Conteúdo Programático}\end{center}
\vspace{-5mm}
\noindent\rule{16.5cm}{0.4pt}
\\
\begin{itemize}
 \item \textbf{Conceitos Básicos de Banco de Dados:} Dados e Informação; Banco de Dados; Sistemas Gerenciadores de Bancos de Dados; Tipos de usuários.
 
 \item \textbf{Modelagem Conceitual:} Modelo de Entidade-Relacionamento: Entidades, Atributos, Relacionamentos; Modelo de Entidade-Relacionamento Estendido: Especialização e Generalização. 

 \item \textbf{Modelo Relacional:} Conceitos do Modelo Relacional; Operações com Relações; Álgebra Relacional: Operação Seleção e Projeção, União, Interseção, Diferença, Produto Cartesiano, Junção, Divisão, Projeto de Banco de Dados Relacional: Mapeamento do modelo entidade-relacionamento para o modelo relacional, Regras e Normalização.
 
 \item \textbf{Linguagem SQL:} Introdução à Linguagem SQL; Utilização das instruções das seguintes sub-linguagens: Linguagem de Definição de Dados (DDL), Linguagem de Manipulação de Dados (DML), Linguagem de Consulta de Dados (DQL), Linguagem de Controle de Dados (DCL), Linguagem de Transação de Dados (DTL); Visões, Subprogramas Armazenados e Gatilhos.

 
\end{itemize}
\noindent\rule{16.5cm}{0.4pt}\\
\\
%COLOCAR A METODOLOGIA DE ENSINO A SEGUIR
\vspace{-12mm}
\begin{center}\textbf{Metodologia de Ensino}\end{center} 
\vspace{-5mm}
\noindent\rule{16.5cm}{0.4pt}
\\
   Aulas expositivas utilizando recursos audiovisuais e quadro, além de aulas práticas utilizando computadores. Adicionalmente, serão realizadas atividades práticas individuais ou em grupo, para consolidação do conteúdo ministrado.\\
\noindent\rule{16.5cm}{0.4pt}\\
\\
%COLOCAR AVALIACAO DO PROCESSO DE ENSINO E APRENDIZAGEM A SEGUIR
\vspace{-12mm}
\begin{center}\textbf{Avaliação do Processo de Ensino e Apendizagem}\end{center}
\vspace{-5mm}
\noindent\rule{16.5cm}{0.4pt}
\\
   Avaliações escritas. Práticas baseadas em Estudos de Caso ou problemas reais.\\
\noindent\rule{16.5cm}{0.4pt}\\
\\
%PREENCHER RECURSOS NECESSARIOS A SEGUIR
\vspace{-12mm}
\begin{center}\textbf{Recursos Necessários}\end{center}
\vspace{-5mm}
\noindent\rule{16.5cm}{0.4pt}
\\
\begin{itemize} 
  \item Listas de Exercícios;
  \item Livros e apostilas;
  \item Utilização de recursos da web;
  \item Quadro branco;
  \item Marcadores para quadro branco;
  \item Sala de aula com acesso à internet, microcomputador e TV ou projetor para apresentação de slides ou material multimídia;
  \item Laboratório de microcomputadores contendo componentes de hardware e \textit{software} específicos;
\end{itemize}
\noindent\rule{16.5cm}{0.4pt}\\
\\
%PREENCHER BIBLIOGRAFIA A SEGUIR
\vspace{-12mm}
\begin{center}\textbf{Bibliografia}\end{center}
\vspace{-5mm}
\noindent\rule{16.5cm}{0.4pt}
\\
\begin{itemize} 
  \item Básica:
	\begin{enumerate}
  	\item 	ELMASRI, R.; NAVATHE, S. \textbf{Sistemas de banco de dados.} Pearson, 6ª edição, 2011;
	\item 	KORTH, H.; SILBERSCHATZ, A.; SUDARSHAN, S. \textbf{Sistemas de bancos de dados.} Campus, 5ª edição, 2006;
	\item 	DATE, C. J. \textbf{Introdução a sistemas de bancos de dados.} Campus, Tradução da 8ª edição Americana, 2004.      
	\end{enumerate}
    
  \item Complementar:
	\begin{enumerate}
  	\item 	HEUSER, C. \textbf{Projeto de Banco de Dados – Série UFRGS, Nº 4.} Sagra-Luzzatto, 5ª edição, 2004;
	\item 	GARCIA-MOLINA, H. \textbf{Implementação de Sistemas de Banco de Dados.} Campus, 1ª edição, 2010;
    \item 	RAMAKRISHNAN, R. \textbf{Sistemas de Gerenciamento de Banco de Dados.} McGraw Hill, 3ª edição, 2010.
	\end{enumerate}
\end{itemize}
\noindent\rule{16.5cm}{0.4pt}\\
\\