\paragraph{Estruturas de Dados II}

%PREENCHER DADOS DA DISCIPLINA A SEGUIR
%\vspace{-12mm}
\begin{center}\textbf{Dados do Componente Curricular}\end{center}
\vspace{-5mm}
\noindent\rule{16.5cm}{0.4pt}
\\
\textbf{Nome:} Estruturas de Dados I
\\ 
\textbf{Curso:} Tecnologia em Sistemas para Internet
\\ 
\textbf{Período:} \unit{2}{\degree}
\\ 
\textbf{Carga Horária:} \unit{67}{\hour}
\\ 
\textbf{Docente Responsável:} Otacílio de Araújo Ramos Neto
\\ 
\noindent\rule{16.5cm}{0.4pt}\\
\\
%PREENCHER A EMENTA A SEGUIR
\vspace{-12mm}
\begin{center}\textbf{Ementa}\end{center}
\vspace{-5mm}
\noindent\rule{16.5cm}{0.4pt}
\\
Conceitos básicos, crescimento de funções e recorrências; Algoritmos de ordenação e busca; Estruturas de dados elementares; Árvores de busca binária.\\ 
\noindent\rule{16.5cm}{0.4pt}\\
\\
%PREENCHER OS OBJETIVOS A SEGUIR
\vspace{-12mm}
\begin{center}\textbf{Objetivos}\end{center}
\vspace{-5mm}
\noindent\rule{16.5cm}{0.4pt}
\\
\begin{itemize}
\item Apresentar os conceitos básicos para criação e análise de algoritmos;
\item Apresentar os algoritmos básicos de ordenação e busca;
\item Capacitar os alunos a utilizarem as estruturas de dados elementares em problemas reais;
\item Apresentar aos alunos as árvores de busca binária e capacitá-los no seu uso.
\end{itemize} 
\noindent\rule{16.5cm}{0.4pt}\\
\\
%PREENCHER OS CONTEUDOS PROGRAMATICOS A SEGUIR (CUIDADO PARA NAO DEIXAR A TABELA MUITO GRANDE)
\vspace{-12mm}
\begin{center}\textbf{Conteúdo Programático}\end{center}
\vspace{-5mm}
\noindent\rule{16.5cm}{0.4pt}
\\
\begin{itemize}
 \item \textbf{Conceitos básicos:} Análise e projeto de algoritmos; Notação assintótica; O Método da Substituição, Método da Árvore de Recursão e Método Mestre.
 \item \textbf{Algoritmos de ordenação e busca:} Ordenação por inserção, Heapsort, Quicksort e ordenação em tempo linear; Busca sequencial e busca binária.
 \item \textbf{Estruturas de dados elementares:} Implementações de ponteiros e objetos; Pilhas, filas e listas ligadas.
 \item \textbf{Árvores de pesquisa binária:} Conceitos fundamentais de árvores de pesquisa binária; Algoritmos de inserção, remoção e busca; Impressão \textit{In-Order}, \textit{Post-Order} e \textit{Pre-Order} .
\end{itemize}
\noindent\rule{16.5cm}{0.4pt}\\
\\
%COLOCAR A METODOLOGIA DE ENSINO A SEGUIR
\vspace{-12mm}
\begin{center}\textbf{Metodologia de Ensino}\end{center} 
\vspace{-5mm}
\noindent\rule{16.5cm}{0.4pt}
\\
   Aulas expositivas utilizando recursos audiovisuais e quadro, além de aulas práticas utilizando computadores. Adicionalmente, serão realizadas atividades práticas individuais ou em grupo, para consolidação do conteúdo ministrado.\\
\noindent\rule{16.5cm}{0.4pt}\\
\\
%COLOCAR AVALIACAO DO PROCESSO DE ENSINO E APRENDIZAGEM A SEGUIR
\vspace{-12mm}
\begin{center}\textbf{Avaliação do Processo de Ensino e Apendizagem}\end{center}
\vspace{-5mm}
\noindent\rule{16.5cm}{0.4pt}
\\
   Avaliações escritas. Avalia\c{c}\~oes pr\'aticas envolvendo a resolu\c{c}\~ao de problemas computacionais.\\
\noindent\rule{16.5cm}{0.4pt}\\
\\
%PREENCHER RECURSOS NECESSARIOS A SEGUIR
\vspace{-12mm}
\begin{center}\textbf{Recursos Necessários}\end{center}
\vspace{-5mm}
\noindent\rule{16.5cm}{0.4pt}
\\
\begin{itemize} 
  \item Listas de Exercícios;
  \item Livros e apostilas;
  \item Utilização de recursos da web;
  \item Quadro branco;
  \item Marcadores para quadro branco;
  \item Sala de aula com acesso à internet, microcomputador e TV ou projetor para apresentação de slides ou material multimídia;
  \item Laboratório de microcomputadores contendo componentes de hardware e software específicos;
\end{itemize}
\noindent\rule{16.5cm}{0.4pt}\\ - 
\\
%PREENCHER BIBLIOGRAFIA A SEGUIR
\vspace{-12mm}
\begin{center}\textbf{Bibliografia}\end{center}
\vspace{-5mm}
\noindent\rule{16.5cm}{0.4pt}
\\
\begin{itemize} 
  \item Básica:
	\begin{enumerate}
	\item T.H. Cormen, C.E. Leiserson, R.L. Rivest, C. Stein, "Algoritmos - Teoria e Prática", 3a. ed., ISBN: 8535236996, Editora Campus, 2012.
	\end{enumerate}
  \item Complementar:
	\begin{enumerate} 
	\item Steven S Skiena, The Algorithm Design Manual, Springer; 2nd edition, ISBN: 978-1849967204, 2008.\\
	\end{enumerate}
\end{itemize}
\noindent\rule{16.5cm}{0.4pt}\\
\\
