\paragraph{Engenharia de Software}

%PREENCHER DADOS DA DISCIPLINA A SEGUIR
%\vspace{-12mm}
\begin{center}\textbf{Dados do Componente Curricular}\end{center}
\vspace{-5mm}
\noindent\rule{16.5cm}{0.4pt}
\\
\textbf{Nome:} Engenharia de Software
\\
\textbf{Curso:} Tecnologia em Sistemas para Internet
\\ 
\textbf{Período:} $5^{\circ}$ 
\\
\textbf{Carga Horária:} 67~h 
\\ 
\textbf{Docente Responsável:} José de Sousa Barros 
\\ 
\noindent\rule{16.5cm}{0.4pt}\\
\\
%PREENCHER A EMENTA A SEGUIR
\vspace{-12mm}
\begin{center}\textbf{Ementa}\end{center}
\vspace{-5mm}
\noindent\rule{16.5cm}{0.4pt}
\\
Processos de Software: Modelos de Processo, Desenvolvimento Ágil;
Gestão da Qualidade: Técnicas de Revisão, Garantia da Qualidade de Software, Estratégias de Teste de Software, Modelagem Formal e Verificação, Métricas de Produto. Gerenciamento de Projetos de Software: Métricas de Processo e Projeto, Estimativas de Projeto de Software, Cronograma de Projeto, Gestão de Risco, Manutenção e Reengenharia. \\
\noindent\rule{16.5cm}{0.4pt}\\
\\
%PREENCHER OS OBJETIVOS A SEGUIR
\vspace{-12mm}
\begin{center}\textbf{Objetivos}\end{center}
\vspace{-5mm}
\noindent\rule{16.5cm}{0.4pt}
\\
\begin{itemize}
\item Compreender os modelos de processo de desenvolvimento de software e o desenvolvimento ágil;
\item Conhecer diferentes abordagens para avaliar a qualidade de um software;
\item Saber gerenciar projetos de desenvolvimento software para problemas do mundo real.
\end{itemize}
\noindent\rule{16.5cm}{0.4pt}\\
\\
%PREENCHER OS CONTEUDOS PROGRAMATICOS A SEGUIR (CUIDADO PARA NAO DEIXAR A TABELA MUITO GRANDE)
\vspace{-12mm}
\begin{center}\textbf{Conteúdo Programático}\end{center}
\vspace{-5mm}
\noindent\rule{16.5cm}{0.4pt}
\\
\begin{itemize}
 \item \textbf{Processos de Software:} Conceitos de Processos; Modelos de Processo; Desenvolvimento Ágil

 \item \textbf{Gestão da Qualidade:} Conceitos de Qualidade; Técnicas de Revisão; Garantia da Qualidade de Software; Estratégias de Teste de Software; Testando Aplicativos Convencionais; Testando Aplicações Orientadas a Objeto; Testando Aplicações para Web; Modelagem Formal e Verificação; Métricas de Produto.
\item \textbf{Gerenciamento de Projetos de Software:}  Conceitos de Gerenciamento de Projeto; Métricas de Processo e Projeto; Estimativas de Projeto de Software; Cronograma de Projeto; Gestão de Risco; Manutenção e Reengenharia.
\end{itemize}
\noindent\rule{16.5cm}{0.4pt}\\
\\
%COLOCAR A METODOLOGIA DE ENSINO A SEGUIR
\vspace{-12mm}
\begin{center}\textbf{Metodologia de Ensino}\end{center} 
\vspace{-5mm}
\noindent\rule{16.5cm}{0.4pt}
\\
   Aulas expositivas utilizando recursos audiovisuais e quadro, além de aulas práticas utilizando computadores. Adicionalmente, serão realizadas atividades práticas individuais ou em grupo, para consolidação do conteúdo ministrado.\\
\noindent\rule{16.5cm}{0.4pt}\\
\\
%COLOCAR AVALIACAO DO PROCESSO DE ENSINO E APRENDIZAGEM A SEGUIR
\vspace{-12mm}
\begin{center}\textbf{Avaliação do Processo de Ensino e Apendizagem}\end{center}
\vspace{-5mm}
\noindent\rule{16.5cm}{0.4pt}
\\
   Avaliações escritas. Práticas baseadas em Estudos de Caso ou problemas reais.\\
\noindent\rule{16.5cm}{0.4pt}\\
\\
%PREENCHER RECURSOS NECESSARIOS A SEGUIR
\vspace{-12mm}
\begin{center}\textbf{Recursos Necessários}\end{center}
\vspace{-5mm}
\noindent\rule{16.5cm}{0.4pt}
\\
\begin{itemize} 
  \item Listas de Exercícios;
  \item Livros e apostilas;
  \item Utilização de recursos da web;
  \item Quadro branco;
  \item Marcadores para quadro branco;
  \item Sala de aula com acesso à internet, microcomputador e TV ou projetor para apresentação de slides ou material multimídia;
  \item Laboratório de microcomputadores contendo componentes de hardware e \textit{software} específicos;
\end{itemize}
\noindent\rule{16.5cm}{0.4pt}\\
\\
%PREENCHER BIBLIOGRAFIA A SEGUIR
\vspace{-12mm}
\begin{center}\textbf{Bibliografia}\end{center}
\vspace{-5mm}
\noindent\rule{16.5cm}{0.4pt}
\\
\begin{itemize} 
  \item Básica:
	\begin{enumerate}
  	\item PRESSMAN, Roger S. \textbf{Engenharia de Software: uma abordagem profissional.} McGraw-Hill,  7ª edição, 2011.    
	\end{enumerate}
    
  \item Complementar:
	\begin{enumerate}
  	\item  SOMMERVILLE, I. \textbf{Engenharia de Software.} Pearson
Education, 9ª Edição, 2011.
	\end{enumerate}
\end{itemize}
\noindent\rule{16.5cm}{0.4pt}\\
\\