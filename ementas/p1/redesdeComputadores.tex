\paragraph{Redes de Computadores} \

% Dados do Componente Curricular

\begin{snugshade}\begin{center}\textbf{
	Dados do Componente Curricular
}\end{center}\end{snugshade}

\noindent 	\textbf{Nome:} Redes de Computadores
\\ 			\textbf{Curso:} Tecnologia em Sistemas para Internet
\\ 			\textbf{Período:} \unit{1}{\degree}
\\ 			\textbf{Carga Horária:} \unit{67}{\hour}
\\ 			\textbf{Docente Responsável:} Erick Augusto Gomes de Melo

% Ementa

\begin{snugshade}\begin{center}\textbf{
    Ementa
\vphantom{q}}\end{center}\end{snugshade}

\noindent

Introdução à comunicação em rede; Classificação das redes quanto à área de cobertura; Processo de comunicação em redes de computadores com base nos modelos de referência OSI e TCP/IP; Funções desempenhadas pelas principais tecnologias de transmissão de dados; Mecanismo de interconexão de redes proposto pelo modelo TCP/IP; Função dos protocolos de suporte às aplicações de rede do modelo TCP/IP; Planejamento e implantação de uma rede simples.

% Objetivos

\begin{snugshade}\begin{center}\textbf{
    Objetivos
}\end{center}\end{snugshade}


\begin{itemize}

\item Compreender os fundamentos básicos sobre redes de computadores e utilizar seus recursos para suportar as atividades relacionadas ao desenvolvimento de programas computacionais;
\item Compreender a necessidade e a importância das redes de computadores;
\item Conhecer definições básicas sobre redes de computadores;
\item Compreender o processo de comunicação em redes de computadores com base nos modelos de referencia OSI e TCP/IP;
\item Compreender as funções desempenhadas pelas principais tecnologias de transmissão de dados;
\item Analisar e explicar o mecanismo de interconexão de redes proposto pelo Modelo TCP/IP;
\item Analisar e explicar a função dos protocolos de suporte às aplicações de rede do Modelo TCP/IP;
\item Planejar e Implantar uma rede simples.

\end{itemize} 

% Conteúdo Programático

\begin{snugshade}\begin{center}\textbf{
    Conteúdo Programático
}\end{center}\end{snugshade}

\begin{itemize}

 \item Histórico e evolução das redes de computadores;
 \item Conceito de redes de computadores;
 \item Visão Geral das LANS, WANS e Inter-redes;
 \item Modelo de Referência de Redes OSI e TCP/IP: Camada, funções, encapsulamento e PDUs;
 \item Serviços, protocolos e aplicações de rede;
 \item Funções da camada de rede;
 \item Vantagens da segmentação de rede provida pelo protocolo IP;
 \item Encaminhamento IP e roteamento IP (Estático e Dinâmico);
 \item Teste de conectividade entre redes e o protocolo ICMP;
 \item Funções da Camada de Enlace de Dados;
 \item Controle de acesso ao meio;
 \item Enquadramento, endereçamento físico;
 \item Funções da camada física;
 \item Sinalização e codificação;
 \item Meios físicos de transmissão (coaxial, UTP, fibra óptica e meios sem fio) e seus conectores;
 \item Tecnologia de transmissão \textit{ethernet} (visão geral);
 \item Repetidores, \textit{hubs}, \textit{bridges} e \textit{switches ethernet};
 \item Operação do \textit{ethernet}:comunicação com camadas superiores, enquadramento, detecção de erros e o CRC, controle de acesso pelo CSMA/CD, endereçamento MAC e protocolo ARP.
 
\end{itemize}

\begin{snugshade}\begin{center}\textbf{
    Metodologia de Ensino
}\end{center}\end{snugshade}

\noindent
   As aulas serão desenvolvidas por meio de metodologia participativa, com a utilização de técnicas didáticas, como: aulas expositivas, debates, seminários, trabalhos de pesquisa, práticas em laboratório (individualmente e em grupos).

\begin{snugshade}\begin{center}\textbf{
    Avaliação do Processo de Ensino e Aprendizagem
}\end{center}\end{snugshade}

\noindent
  Avaliações escritas, apresentação de seminários, atividades em grupos e elaboração de projetos.
   
\begin{snugshade}\begin{center}\textbf{
    Recursos Necessários
    \vphantom{q} % TODO: corrigir o depth da linha sem esta gambiarra.
}\end{center}\end{snugshade}

\begin{itemize} 
	\item Quadro branco;
	\item Marcadores para quadro branco;
	\item Projetor de dados multimídia;
	\item Laboratório de informática.
\end{itemize}

% Bibliografia

\begin{snugshade}\begin{center}\textbf{
    Bibliografia
}\end{center}\end{snugshade}

\begin{itemize} 
  \item Básica:
	\begin{enumerate}
	\item KUROSE, J. F., ROSSA, K. W. Redes de computadores e a internet. 5 ed. Editora Pearson. 2010.
	\item TANENBAUM, A. S., WETHERALL, D. Redes de Computadores. 5 ed. Editora Pearson. 2011. 
	\item FOROUZAN, Behrouz A.; MOSHARRAF, Firouz. Redes de Computadores - Uma Abordagem Top-Down - 2012. 1 ed. Editora Mcgraw Hill, 2012.	
	\end{enumerate}
  \item Complementar:
	\begin{enumerate} 
	\item COMER, D. E. Redes de computadores e internet. 4 ed. Editora Artmed. 2007.
	\item LOWE,Doug. Redes de Computadores Para Leigos. 9ª Edição. Editora Altabooks.
	\end{enumerate}
\end{itemize}
