\paragraph{Fundamentos da Computa\c{c}\~ao} \

% Dados do Componente Curricular

\begin{snugshade}\begin{center}\textbf{
    Dados do Componente Curricular
}\end{center}\end{snugshade}

\noindent \textbf{Nome:}                Fundamentos da Computa\c{c}\~ao
\\        \textbf{Curso:}               Tecnologia em Sistemas para Internet
\\        \textbf{Período:}             \unit{1}{\degree}
\\        \textbf{Carga Horária:}       \unit{33}{\hour}
\\        \textbf{Docente Responsável:} Rodrigo Pinheiro Marques de Araújo

% Ementa

\begin{snugshade}\begin{center}\textbf{
    Ementa
\vphantom{q}}\end{center}\end{snugshade}

\noindent
Conceitos introdut\'orios de inform\'atica; Representa\c{c}\~ao de dados e convers\~ao de base; Opera\c{c}\~oes aritm\'eticas com n\'umeros bin\'arios; L\'ogica digital; T\'opicos especiais em computa\c{c}\~ao.
% Objetivos

\begin{snugshade}\begin{center}\textbf{
    Objetivos
}\end{center}\end{snugshade}

\begin{itemize}

\item Apresentar os conceitos de \textit{hardware} e \textit{software};
\item Apresentar a representação digital de dados e informação;
\item Introduzir conceitos de l\'ogica;
\item Apresentar o funcionamento das portas lógicas;
\item Apresentar as tecnologias e aplicações de computadores.

\end{itemize} 

% Conteúdo Programático

\begin{snugshade}\begin{center}\textbf{
    Conteúdo Programático
}\end{center}\end{snugshade}

\begin{itemize}

 \item \textbf{Conceitos introdut\'orios de inform\'atica:} Histórico e evolução dos computadores; Defini\c{c}\~oes de \textit{software} e \textit{hardware}; Modelo conceitual da arquitetura de organiza\c{c}\~ao de um computador; Classifica\c{c}\~ao dos computadores; Perif\'ericos de entrada e sa\'ida.

 \item \textbf{Representa\c{c}\~ao de dados e convers\~ao de base:} Representação de dados; Representação de números inteiros na base binária; Representação de números inteiros na base octal; Representação de números inteiros nas base hexadecimal; Convers\~ao de bases.

 \item \textbf{Opera\c{c}\~oes aritm\'eticas com n\'umeros bin\'arios:} Opera\c{c}\~oes aritm\'eticas b\'asicas com n\'umeros bin\'arios.

 \item \textbf{L\'ogica digital:} Introdu\c{c}\~ao \`a l\'ogica; L\'ogica digital; Portas l\'ogicas; Constru\c{c}\~ao de circuitos combinacionais simples.

 \item \textbf{T\'opicos especiais em computa\c{c}\~ao:} Conte\'udo vari\'avel, envolvendo temas relevantes e atuais da computa\c{c}\~ao.

\end{itemize}

% Metodologia, Avaliação e Recursos
\begin{snugshade}\begin{center}\textbf{
    Metodologia de Ensino
}\end{center}\end{snugshade}

\noindent
Aulas expositivas utilizando recursos audiovisuais e quadro, além de aulas práticas utilizando computadores. Adicionalmente, serão realizadas atividades práticas individuais ou em grupo, para consolidação do conteúdo ministrado.


\begin{snugshade}\begin{center}\textbf{
    Avaliação do Processo de Ensino e Aprendizagem
}\end{center}\end{snugshade}

\noindent
          Avaliações escritas e pr\'aticas.

\begin{snugshade}\begin{center}\textbf{
    Recursos Necessários
    \vphantom{q} % TODO: corrigir o depth da linha sem esta gambiarra.
}\end{center}\end{snugshade}

\begin{itemize}
  \item Listas de Exercícios;
  \item Livros e apostilas;
  \item Utilização de recursos da web;
  \item Quadro branco;
  \item Marcadores para quadro branco;
  \item Sala de aula com acesso à internet, microcomputador e TV ou projetor para apresentação de slides ou material multimídia;
  \item Laboratório de microcomputadores contendo componentes de hardware e software específicos;
\end{itemize}


% Bibliografia

\begin{snugshade}\begin{center}\textbf{
    Bibliografia
}\end{center}\end{snugshade}

\begin{itemize} 
   \item Básica:
	\begin{enumerate}
		\item Monteiro, M. A. Introdução à Organização de Computadores. ISBN: 9788521615439. Editora LTC. 5 Ed., 2007; 
		\item Idoeta, I. V.; Capuano, F. G. Elementos de Eletrônica Digital. ISBN: 8571940193. Editora Érica, 40 Ed., 2007;
		\item Velloso, F. C. Informática: Conceitos Básicos. ISBN: 9788535243970. Editora Campus, 8 Ed., 2011. 
	\end{enumerate}
  \item Complementar:
	\begin{enumerate} 
		\item Tanenbaum, Andrew S. Organização Estruturada de Computadores. ISBN: 9788581435398. Editora Pearson. 6 Ed., 2013.
	\end{enumerate}
\end{itemize}
