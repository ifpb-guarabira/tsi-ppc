
% Dados do Componente Curricular

\begin{snugshade}\begin{center}\textbf{
    Dados do Componente Curricular
}\end{center}\end{snugshade}

\noindent \textbf{Nome:}                Ingl\^es Instrumental
\\        \textbf{Curso:}               Tecnologia em Sistemas para Internet
\\        \textbf{Período:}             \unit{1}{\degree}
\\        \textbf{Carga Horária:}       \unit{67}{\hour}
\\        \textbf{Docente Responsável:} Sabrina da Costa Rocha

% Ementa

\begin{snugshade}\begin{center}\textbf{
    Ementa
\vphantom{q}}\end{center}\end{snugshade}

\noindent
O componente curricular Inglês Instrumental desenvolve a habilidade de leitura utilizando gêneros textuais. Para a leitura e compreensão dos textos, são trabalhadas as estratégias de leitura, reconhecimento de cognatos, palavras repetidas, dicas tipográficas, skimming, scanning, prediction, selectivity, inferência, referência, marcadores discursivos, grupos nominais e  grupos verbais.

% Objetivos

\begin{snugshade}\begin{center}\textbf{
    Objetivos
}\end{center}\end{snugshade}

\begin{itemize}

\item Desenvolver a habilidade de leitura de textos autênticos em inglês sobre diferentes temas relacionados ao Curso Superior de Tecnologia em Sistemas para Internet e de outras áreas;

\item Identificar e compreender gêneros textuais diversos (notícias, propagandas, biografias, artigos de divulgação científica etc.);

\item Desenvolver os diferentes níveis de compreensão de leitura (geral, detalhada e das ideias principais);

\item Ler para obter informação geral (skimming) e específica (scanning);

\item Compreender textos usando outras estratégias de leitura (prediction, selectivity e 
flexibility);

\item Predizer informações com o uso de dicas tipográficas;

\item Identificar referenciais, marcadores discursivos, grupos nominais e grupos verbais.

\end{itemize} 

% Conteúdo Programático

%\begin{snugshade}\begin{center}\textbf{
 %   Conteúdo Programático
%}\end{center}\end{snugshade}

%\begin{itemize}

%\item \textbf{Definição e reconhecimento de Gêneros textuais;}

%\item \textbf{Objetivos da leitura;}

%\item \textbf{Níveis de compreensão da leitura;}

%\item \textbf{Estratégias de leitura:} 
%dicas tipográficas, palavras cognatas, palavras repetidas, skimming, scanning, prediction, selectivity e flexibility;

%\item \textbf{Inferência e referência;}

%\item \textbf{Marcadores Discursivos;}

%\item \textbf{Uso do dicionário;}

%\item \textbf{Grupo Nominal;}

%\item \textbf{Grupo Verbal.}

%\end{itemize}

% Metodologia, Avaliação e Recursos

%\begin{snugshade}\begin{center}\textbf{
 %   Metodologia de Ensino
%}\end{center}\end{snugshade}

%\noindent
%As aulas são teórico-expositivas, ilustradas com recursos audiovisuais. As atividades de leitura de gêneros discursivos diversos são interativas. Além disso, ocorrem aulas práticas em laboratório, utilizando roteiros e exercícios que podem ser executados individualmente ou em grupos.


%\begin{snugshade}\begin{center}\textbf{
 %   Avaliação do Processo de Ensino e Aprendizagem
%}\end{center}\end{snugshade}

%\noindent
 %          A avaliação é contínua, ou seja, deve ser entendida como uma prática processual, diagnóstica e cumulativa da aprendizagem. É feita através da observação da assiduidade, participação e interesse dos alunos. Além disso, para a verificação do domínio de conhecimentos são utilizados trabalhos de pesquisa com apresentações em sala de aula e verificações da aprendizagem.

%\begin{snugshade}\begin{center}\textbf{
%    Recursos Necessários
 %   \vphantom{q} % TODO: corrigir o depth da linha sem esta gambiarra.
%}\end{center}\end{snugshade}

%\begin{itemize}
 % \item Quadro branco;
 % \item Marcadores para quadro branco;
 % \item Sala de aula com acesso à internet, microcomputador e TV ou projetor para apresentação de slides ou material multimídia;
 % \item Micro system, CD, DVD Player.
%\end{itemize}

% Bibliografia

\begin{snugshade}\begin{center}\textbf{
    Bibliografia
}\end{center}\end{snugshade}

\begin{itemize} 

\item Básica:
    \begin{enumerate}

    \item DIONISIO, A. P.; MACHADO, A. R.; BEZERRA, M. A. (Orgs). Gêneros textuais e ensino. São Paulo: Parábola Editorial, 2010.
    
    \item GLENDINNING, E.; McEWAN, J. Basic English for Computing. Oxford, 2003;
    
    \item SOUZA, A. G.; ABSY, C. A.; COSTA, G. C.; MELLO, L. F. Leitura em língua inglesa: uma abordagem instrumental. São Paulo: Disal, 2005.
	
    \end{enumerate}

\item Complementar:
	\begin{enumerate} 

    \item FÜRSTENAU, Eugênio. Novo dicionário de termos técnicos: Inglês-Português/Português-Inglês. São Paulo: Editora Globo, 2005. Vol 1 e 2.

    \item LONGMAN. Dicionário Escolar: Inglês-Português/Português-Inglês. Pearson Longman, 2009.

    \item MURPHY, R. English Grammar in Use. Intermediate Students. New York, 2000;

	\end{enumerate}

\end{itemize}
