\paragraph{Cálculo Diferencial e Integral}  \

% Dados do Componente Curricular

\begin{snugshade}\begin{center}\textbf{
	Dados do Componente Curricular
}\end{center}\end{snugshade}

\noindent \textbf{Nome:} Cálculo Diferencial e Integral
\\        \textbf{Curso:} Tecnologia em Sistemas para Internet
\\        \textbf{Período:} \unit{1}{\degree}
\\        \textbf{Carga Horária:} \unit{100}{\hour}
\\        \textbf{Docente Responsável:} Cícero Demétrio Vieira de Barros      

% Ementa

\begin{snugshade}\begin{center}\textbf{
    Ementa
\vphantom{q}}\end{center}\end{snugshade}

\noindent
Revisão pré-cálculo; Limites e continuidade de funções reais de uma variável e suas aplicações; Derivada de funções reais de uma variável e suas aplicações; Integral indefinida, integral definida, teorema fundamental do cálculo; Técnica da Substituição e Aplicações da Integral Definida.
% Objetivos

\begin{snugshade}\begin{center}\textbf{
    Objetivos
}\end{center}\end{snugshade}

\begin{itemize}

\item Relembrar conceitos de pré-cálculo desenvolvidos no ensino médio;
\item Desenvolver o conceito de limite junto com as principais propriedades;
\item Desenvolver o conceito de continuidade de funções junto com as principais propriedades;
\item Desenvolver o conceito de Derivada, propriedades da Derivada e regras de derivação;
\item Aplicar o conceito de Derivada à problemas relacionados às áreas científicas e tecnológicas;
\item Construir modelos para resolver problemas envolvendo funções de uma variável real e suas derivadas;
\item Desenvolver o conceito de Integral de uma função de uma variável real, entender as suas diferentes representações e aplicá-lo a problemas relacionados às áreas cientificas e tecnológicas;
\item Estabelecer relações entre Derivadas e Integrais;
\item Desenvolver habilidade de calcular Derivadas, Integrais e traçar gráficos utilizando ferramentas computacionais;
\item Fazer com que o aluno aprenda a utilizar um software computacional como ferramenta auxiliar na aprendizagem do Cálculo e da Geometria Analítica;
\item  Levar o aluno a ler, interpretar e comunicar ideias matemáticas.

\end{itemize} 

% Conteúdo Programático

\begin{snugshade}\begin{center}\textbf{
    Conteúdo Programático
}\end{center}\end{snugshade}

\begin{itemize}

 \item \textbf{Revisão pré-cálculo:} Equações e inequações; Valor absoluto; Geometria Analítica; Trigonometria; Funções e gráficos; Operações com funções; Função exponencial e logarítmica.
 
 \item \textbf{Limites e continuidade de uma função de uma variável real:} Definição; Propriedades dos limites; Limites laterais; Limite de uma função composta; Teorema do Confronto; Limites no infinito; Limites infinitos; Limites fundamentais; Continuidade de funções reais; Propriedades de funções contínuas.
 
 \item \textbf{Derivadas de funções reais de uma variável e suas aplicações:} Definição e exemplos; A reta tangente; Continuidade de funções deriváveis; Derivadas laterais; regras de derivação; Regra da Cadeia; Derivadas de funções inversas; Regra de L'Hôpital; Derivadas de funções elementares (função exponencial, função logarítmica, funções trigonométricas e trigonométricas inversas); Aplicações da derivada (propriedades geométricas de gráficos e funções, máximos e mínimos relativos e absolutos de funções de uma variável real); Taxa de variação.
 
 \item \textbf{Integração:} Integral definida; Propriedades da integral definida; Tabelas de integrais imediatas; Técnicas de integração (substituição, integração por partes e frações parciais); Integral definida e propriedades; Teorema Fundamental do Cálculo; Integração e funções trigonométricas; Aplicações da integral definida (área entre curvas).
 
\end{itemize}

\begin{snugshade}\begin{center}\textbf{
    Metodologia de Ensino
}\end{center}\end{snugshade}

\noindent
   Aulas expositivas utilizando recursos audiovisuais e quadro. Aplicação e resolução de exercícios propostos, seminários individuais ou em grupo e trabalhos extra-classe.

\begin{snugshade}\begin{center}\textbf{
    Avaliação do Processo de Ensino e Aprendizagem
}\end{center}\end{snugshade}

\noindent
   Avaliações escritas e avaliações práticas envolvendo a resolução de problemas computacionais.
   
\begin{snugshade}\begin{center}\textbf{
    Recursos Necessários
    \vphantom{q} % TODO: corrigir o depth da linha sem esta gambiarra.
}\end{center}\end{snugshade}

\begin{itemize} 
  \item Listas de Exercícios;
  \item Livros e apostilas;
  \item Utilização de recursos da web;
  \item Quadro branco;
  \item Marcadores para quadro branco;
\end{itemize}

% Bibliografia

\begin{snugshade}\begin{center}\textbf{
    Bibliografia
}\end{center}\end{snugshade}

\begin{itemize} 
  \item Básica:
	\begin{enumerate}
	\item MUNEM, M. A. FOULIS, D. J. - Cálculo Vol. 1 - ISBN  85-2161-054-8, Editora LTC.
	\item GUIDORRIZZI, Hamilto Luiz. Um curso de calculo. Vol. 1. 5ª Ed. Rio de Janeiro. Editora LCT. 2001.
	\item STWART, J. Cálculo. Vol. 1. 5ª Ed. São Paulo: Pioneira Thomson Learning, 2006.
	\end{enumerate}
  \item Complementar:
	\begin{enumerate} 
	\item  HOWARD, A.; BIVENS, I.; DAVIS, S. Cálculo. Vol. 1. 8ª Ed. Porto Alegre: Bookman.  2007.
	\item  LEITHOLD, L. O. Cálculo com G'eometria Analítica. Vol. 1. 3ª. Ed. São Paulo: Harbra, 1994.
	\end{enumerate}
\end{itemize}
