
% Dados do Componente Curricular

\begin{snugshade}\begin{center}\textbf{
	Dados do Componente Curricular
}\end{center}\end{snugshade}

\noindent 	\textbf{Nome:} Programação Paralela
\\ 			\textbf{Curso:} Tecnologia em Sistemas para Internet
\\ 			\textbf{Período:} \unit{4}{\degree}
\\ 			\textbf{Carga Horária:} \unit{67}{\hour}
\\ 			\textbf{Docente Responsável:} Otacílio de Araújo Ramos Neto

% Ementa

\begin{snugshade}\begin{center}\textbf{
    Ementa
\vphantom{q}}\end{center}\end{snugshade}

\noindent
Necessidade da programação paralela; Criação de múltiplos processos threads utilizando chamadas de sistemas e bibliotecas; Exclusão mútua e sincronização entre processos e threads; Programação paralela utilizando GPGPUs; Comunicação entre threads em uma GPGPU; Aplicações de GPGPU em problemas científicos.
% Objetivos

\begin{snugshade}\begin{center}\textbf{
    Objetivos
}\end{center}\end{snugshade}

\begin{itemize}

\item Conscientizar o aluno da necessidade da programação paralela na solução de determinados problemas;
\item Apresentar ao aluno os mecanismos de criação de processos e threads;
\item Capacitar o aluno a utilizar as técnicas de exclusão mútua e sincronização;
\item Capacitar o aluno a utilizar as técnicas de comunicação entre processos e threads;
\item Apresentar ao aluno as técnicas de programação paralela utilizando GPGPU;
\item Capacitar o aluno a utilizar as técnicas de comunicação entre threads em GPGPUs;
\item Capacitar o aluno a aplicar GPGUs em problemas científicos.

\end{itemize} 

% Conteúdo Programático

\begin{snugshade}\begin{center}\textbf{
    Conteúdo Programático
}\end{center}\end{snugshade}

\begin{itemize}

 \item \textbf{Introdução à computação paralela:} Motivação; Modificações na arquitetura de von Neumann; Multiprocessamento, processos e threads; Hardware para computação paralela; Algoritmos sequências e algoritmos paralelos; Reentrância; Software paralelo.
 
 \item \textbf{Chamadas de sistema:} Chamadas de sistema para a criação de múltiplos processos e threads nos sistemas Unix, Windows e linguagens Java, Python e PHP.
 
 \item \textbf{Exclusão mútua e sincronização:} Seções críticas; Semáforos; Mutexes; Monitores; Barreiras.
 
 \item \textbf{Comunicação entre processos e threads:} Pipelines; Memória Compartilhada; Sockets.
 
 \item \textbf{Programação paralela utilizando GPGPUs:} Introdução à programação paralela com GPGPUs; Introdução ao hardware CUDA e SDK; Introdução ao CUDA C.
 
 \item \textbf{Comunicação entre threads no CUDA:} Divisão em blocos paralelos; Compartilhamento de memória e sincronização entre threads.
 
 \item \textbf{Aplicações do CUDA:} Implementação do algoritmo \textit{Ray Tracing}; Medição do desempenho com eventos; Simulando a transferência de calor; Aplicações de operações atômicas no cálculo de histogramas; Escalonamento do trabalho na GPGPU.
 
\end{itemize}

%\begin{snugshade}\begin{center}\textbf{
%    Metodologia de Ensino
%}\end{center}\end{snugshade}

%\noindent
%     Aulas expositivas utilizando recursos audiovisuais e quadro, além de aulas práticas utilizando computadores. Adicionalmente, serão realizadas atividades práticas individuais ou em grupo, para consolidação do conteúdo ministrado.

%\begin{snugshade}\begin{center}\textbf{
 %   Avaliação do Processo de Ensino e Aprendizagem
%}\end{center}\end{snugshade}

%\noindent
 %  Avaliações escritas. Avaliações práticas envolvendo a resolução de problemas computacionais.
   
%\begin{snugshade}\begin{center}\textbf{
 %   Recursos Necessários
  %  \vphantom{q} % TODO: corrigir o depth da linha sem esta gambiarra.
%}\end{center}\end{snugshade}

%\begin{itemize} 
 % \item Listas de Exercícios;
  %\item Livros e apostilas;
 % \item Utilização de recursos da web;
 % \item Quadro branco;
 % \item Marcadores para quadro branco;
 % \item Sala de aula com acesso à internet, microcomputador e TV ou projetor para apresentação de slides ou material multimídia;
%  \item Laboratório de microcomputadores contendo componentes de hardware (placas gráficas) e software específicos;
%\end{itemize}

% Bibliografia

\begin{snugshade}\begin{center}\textbf{
    Bibliografia
}\end{center}\end{snugshade}

\begin{itemize} 
  \item Básica:
	\begin{enumerate}
	\item PACHECO, Peter. A Introduction to Parallel Programming - ISBN 978-0-12-374260-5 Elsevier. 2011.
	\item HERLIHY, Maurice. SHAVIT, Nir. The Art of Multiprocessor Programming - ISBN 978-0-12-370591-4 Elsevier. 2008.
	\item SANDERS, Jason. KANDROT, Edward. Cuda by Example An Introduction to General-Purpose GPU Programming - ISBN 978-0-13-138768-3.
	\end{enumerate}
  \item Complementar:
	\begin{enumerate} 
	\item  TANENBAUM, Andrew S. Sistemas Operacionais Modernos - ISBN 978-85-7605-237-1 3ª. Ed. Pearson Prentice Hall, 2009.
	\item  DEITEL, Choffnes. Sistemas Operacionais - ISBN 8576050110 3ª. Ed. Pearson Prentice Hall. 
	\end{enumerate}
\end{itemize}
