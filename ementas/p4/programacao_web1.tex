\paragraph{Programação Web 1} \

% Dados do Componente Curricular

\begin{snugshade}\begin{center}\textbf{
    Dados do Componente Curricular
}\end{center}\end{snugshade}

\noindent \textbf{Nome:}                Programação para a Web I
\\        \textbf{Curso:}               Tecnologia em Sistemas para Internet
\\        \textbf{Período:}             \unit{4}{\degree}
\\        \textbf{Carga Horária:}       \unit{83}{\hour}
\\        \textbf{Docente Responsável:} Rodrigo Pinheiro Marques de Araújo

% Ementa

\begin{snugshade}\begin{center}\textbf{
    Ementa
\vphantom{q}}\end{center}\end{snugshade}

\noindent
Conceitos básicos sobre aplicações cliente/servidor. Construção de aplicações na Web. Interação entre aplicações na Web. Integração da aplicação com banco de dados. Mecanismos de autenticação, sessão e cache. 
% Objetivos

\begin{snugshade}\begin{center}\textbf{
    Objetivos
}\end{center}\end{snugshade}

\begin{itemize}

\item Habilitar na construção de aplicações para a internet.

\item Entender a interação de aplicações para internet;

\item Compreender o funcionamento de arcabouços para desenvolvimento web;

\end{itemize} 

% Conteúdo Programático

\begin{snugshade}\begin{center}\textbf{
    Conteúdo Programático
}\end{center}\end{snugshade}

\begin{itemize}

\item \textbf{Ciclo de vida de uma requisição Web}
\item \textbf{Construção de páginas web estáticas}
\item \textbf{Introdução a páginas web dinâmicas}
\item \textbf{Padrão MVC na construção de aplicações Web}
\item \textbf{Arcabouços para o desenvolvimento de aplicações Web}
\item \textbf{Linguagem de template para renderização de páginas}
\item \textbf{Atendimento de requisições dinâmicas}
\item \textbf{Persistência de dados da aplicação em banco de dados:}
    Persistência manual; Persistência através de mapeamento objeto-relacional.
\item \textbf{Mecanismo de Autenticação:}
    Cookies e Sessão
\item \textbf{Mecanismo de Cache}
  

\end{itemize}

% Metodologia, Avaliação e Recursos

\begin{snugshade}\begin{center}\textbf{
    Metodologia de Ensino
}\end{center}\end{snugshade}

\noindent
Aulas expositivas utilizando recursos audiovisuais e quadro, além de aulas práticas utilizando computadores. Adicionalmente, serão realizadas atividades práticas individuais ou em grupo, para consolidação do conteúdo ministrado.

\begin{snugshade}\begin{center}\textbf{
    Avaliação do Processo de Ensino e Aprendizagem
}\end{center}\end{snugshade}

\noindent
Avaliações escritas. Avaliações práticas envolvendo a resolução de problemas computacionais.

\begin{snugshade}\begin{center}\textbf{
    Recursos Necessários
    \vphantom{q} % TODO: corrigir o depth da linha sem esta gambiarra.
}\end{center}\end{snugshade}

\begin{itemize}
  \item Listas de Exercícios;
  \item Livros e apostilas;
  \item Utilização de recursos da web;
  \item Quadro branco;
  \item Marcadores para quadro branco;
  \item Sala de aula com acesso à internet, microcomputador e TV ou projetor para apresentação de slides ou material multimídia;
  \item Laboratório de microcomputadores contendo componentes de hardware e software específicos;
\end{itemize}


% Bibliografia

\begin{snugshade}\begin{center}\textbf{
    Bibliografia
}\end{center}\end{snugshade}

\begin{itemize} 

\item Básica:
    \begin{enumerate}

    \item GALESI, T.; SANTANA NETO, O.
          Python e Django - Desenvolvimento Ágil de Aplicações Web.
          NOVATEC, 2010

    \item SOARES, W.
          PHP 5 - Conceitos, Programação e Integração com Banco de Dados.
          Editora Érica, 2010.
    
    \item BASHAN, B.; et al.
          Use a Cabeça: Servlets e JSP.
          Alta Books, 2005.
    

	
    \end{enumerate}

\item Complementar:
	\begin{enumerate} 

    \item GRINBERG, M.
          Flask Web Development: Developing Web Applications with Python.
          O'Reilly Media, 2014.

    \item GREENFIELD, D.; ROY, A.
          Two Scoopes of Django: Best Pratices For Django 1.6.
          Two Scoopes Press, 2014.

    \item MENEZES, N. N. C.
          Introdução a programação com Python.
          Novatec, 2014.

	\end{enumerate}

\end{itemize}



