
% Dados do Componente Curricular

\begin{snugshade}\begin{center}\textbf{
    Dados do Componente Curricular
}\end{center}\end{snugshade}

\noindent \textbf{Nome:}                Legislação Social
\\        \textbf{Curso:}               Tecnologia em Sistemas para Internet
\\        \textbf{Período:}             \unit{4}{\degree}
\\        \textbf{Carga Horária:}       \unit{67}{\hour}
\\        \textbf{Docente Responsável:} Monique Ximenes Lopes de Medeiros

% Ementa

\begin{snugshade}\begin{center}\textbf{
    Ementa
\vphantom{q}}\end{center}\end{snugshade}

\noindent
Noções de introdução ao Direito: Normas jurídicas: conceito, características, hierarquia, conflitos, estrutura. Direito do trabalho: princípios e conceitos fundamentais, fundamentos históricos e constitucionais. Relação de Trabalho e Relação de Emprego. Empregado e empregador. Direitos Sociais dos Trabalhadores na Constituição Federal. Tutela geral do trabalho: Salário e remuneração; Duração do trabalho; Férias, Gratificação natalina. Proteção do trabalho: aviso prévio e FGTS. Contrato de trabalho. Legislação aplicada a informática: Liberdade de expressão. Responsabilidade civil: noções gerais e responsabilidade decorrente do uso dos meios informáticos. Responsabilidade penal: crimes informáticos. 

% Objetivos

\begin{snugshade}\begin{center}\textbf{
    Objetivos
}\end{center}\end{snugshade}

\begin{itemize}

\item Compreender os direitos trabalhistas e capacitar o profissional de informática para atuação dentro dos parâmetros legais;

\item Entender a estrutura e hierarquia do ordenamento jurídico brasileiro;

\item Identificar as relações de emprego e seus direitos básicos;

\item Descrever as características e requisitos de direitos trabalhistas elencados na Constituição Federal e na Consolidação das Leis Trabalhistas;

\item Reconhecer as responsabilidades civil e penal decorrentes do uso de meios informáticos.

\end{itemize}

% Conteúdo Programático

%\begin{snugshade}\begin{center}\textbf{
%    Conteúdo Programático
%}\end{center}\end{snugshade}

%\begin{itemize}

%\item \textbf{Segurança da informação:}
%    Introdução; Fundamentos; Aplicações.

%\item \textbf{Criptografia:}
%    História da criptografia;
 %   Conceitos básicos;
  %  Criptoanálise;
   % Criptografia simétrica;
    %Criptografia assimétrica;
   % Funções de \textit{hash};
   % Infraestrutura de chaves públicas.

%\item \textbf{Redes:}
 %   Comunicação segurança: IPSEC, SSL/TLS, SSH e VPNs;
  %  Redes sem fio: WEP, WPA, WPA2;
  %  \textit{Sniffers}.

%\item \textbf{\textit{Firewalls}:}
%    Histórico e evolução;
 %   Tipos de \textit{firewall} e suas aplicações.

%\item \textbf{Sistemas de arquivos:}
 %   Estrutura e permissões;
  %  Formatação física e lógica;
  %  SMART (\textit{Self-Monitoring, Analysis and Reporting Technology});
  %  Recuperação de dados.

%\item \textbf{Padrões, Normas e Certificações:}
 %   Padrões       COBIT e ITIL;
  %  Normas        ISO 27001 e ISO 27002;
   % Certificações CEH, LPT, CSSLP e outras.

%\item \textbf{Vulnerabilidades, ataques e contramedidas:}
 %   CVE (\textit{Common Vulnerabilities and Exposures});
  %  Busca de vulnerabilidades;
  %  Ataques;
   % Monitoramento, controle e auditoria.

%\end{itemize}

% Metodologia, Avaliação e Recursos

%\begin{snugshade}\begin{center}\textbf{
    Metodologia de Ensino
}\end{center}\end{snugshade}

\noindent
Aulas expositivas utilizando recursos audiovisuais e quadro, além de aulas práticas utilizando computadores. Adicionalmente, serão realizadas atividades práticas individuais ou em grupo, para consolidação do conteúdo ministrado.

%\begin{snugshade}\begin{center}\textbf{
    Avaliação do Processo de Ensino e Aprendizagem
}\end{center}\end{snugshade}

\noindent
Avaliações escritas. Avaliações práticas envolvendo a resolução de problemas computacionais.

%\begin{snugshade}\begin{center}\textbf{
    Recursos Necessários
    \vphantom{q} % TODO: corrigir o depth da linha sem esta gambiarra.
}\end{center}\end{snugshade}

\begin{itemize}
  \item Listas de Exercícios;
  \item Livros e apostilas;
  \item Utilização de recursos da web;
  \item Quadro branco;
  \item Marcadores para quadro branco;
  \item Sala de aula com acesso à internet, microcomputador e TV ou projetor para apresentação de slides ou material multimídia;
  \item Laboratório de microcomputadores contendo componentes de hardware e software específicos;
\end{itemize}


% Bibliografia

\begin{snugshade}\begin{center}\textbf{
    Bibliografia
}\end{center}\end{snugshade}

\begin{itemize}

\item Básica:
    \begin{enumerate}

    \item BARROS, Alice Monteiro de. Curso de Direito do Trabalho. São Paulo: LTR, 2013.

    \item BRASIL. Constituição da República Federativa do Brasil. Brasília: Senado Federal,2004.

    \item CARRION, Valetin. Comentários à Consolidação das Leis do Trabalho. São Paulo: Saraiva, 2013.
    \end{enumerate}

\item Complementar:
    \begin{enumerate}

    \item CRUZ, Vitor. Constituição Federal: anotada para concursos. São Paulo: Editora Ferreira,2010.

    \item DELGADO, Maurício Godinho. Curso de Direito do Trabalho. São Paulo: LTR, 2010.

    \item LENZA, Pedro. Direito Constitucional Esquematizado. 13.ed. São Paulo: Saraiva, 2009.

    \item MARTINS, Sérgio Pinto. Direito da seguridade social. São Paulo: Atlas, 2013.

	\item MORAIS FILHO, Evaristo de. Introdução ao Direito do Trabalho. São Paulo: LTR, 2010.

	\item NASCIMENTO, Amauri Mascaro. Iniciação do Direito do Trabalho. São Paulo: Saraiva, 2013.

	\item REALE, Miguel. Lições Preliminares de Direito. 27. ed. São Paulo: Saraiva, 2002.

    \end{enumerate}

\end{itemize}
