
% Dados do Componente Curricular

\begin{snugshade}\begin{center}\textbf{
	Dados do Componente Curricular
}\end{center}\end{snugshade}

\noindent 	\textbf{Nome:} Bancos de Dados II
\\ 			\textbf{Curso:} Tecnologia em Sistemas para Internet
\\ 			\textbf{Período:} \unit{4}{\degree}
\\ 			\textbf{Carga Horária:} \unit{67}{\hour}
\\ 			\textbf{Docente Responsável:} José de Sousa Barros 

% Ementa

\begin{snugshade}\begin{center}\textbf{
    Ementa
\vphantom{q}}\end{center}\end{snugshade}

\noindent
Bancos de dados orientados a objeto: ODMG, ODL e OQL. Bancos de dados objeto-relacional. Projeto de bancos de dados objeto-relacional: modelos conceitual e lógico. Consultas em bancos de dados objeto-relacional. Banco de dados geográficos. Bancos de dados distribuídos. Novas aplicações de bancos de dados.

% Objetivos

\begin{snugshade}\begin{center}\textbf{
    Objetivos
}\end{center}\end{snugshade}


\begin{itemize}

\item Compreender os conceitos fundamentais dos bancos de dados orientado a objetos, objeto-relacional, geográfico e distribuído;
\item Diferenciar os bancos de dados orientado a objetos, objeto-relacional, geográfico e distribuído;
\item Realizar a integração entre aplicações e os bancos de dados orientado a objetos, objeto-relacional, geográfico e distribuído.

\end{itemize}

%\begin{snugshade}\begin{center}\textbf{
 %   Conteúdo Programático
%}\end{center}\end{snugshade}

%\begin{itemize}

% \item \textbf{Banco de Dados Geográficos:} Conceitos básicos;	Representação de dados (Open Geospatial Consortium); PostgreSQL com PostGIS; Importação de dados espaciais; Consultas espaciais; Java Topology Suite (JTS); Representação de mapas em SVG.
 
% \item \textbf{Banco de Dados Orientados a Objetos:} Conceitos básicos; O padrão ODMG; ODL: Estrutura de classes, Construtores, Identidade de Objetos, Coleções estáticas e dinâmicas, Nomeação e alcançabilidade; OQL: Consultas, Subconsultas, Expressões de caminho.

% \item \textbf{Banco de Dados Objeto-Relacional:} Conceitos básicos; Tipos Complexos; Construtores;	Métodos; Coleções estáticas e dinâmicas; Tabelas de objetos; Tabelas aninhadas; Referências para Tipos Complexos; Herança; Consultas com tipos complexos.
 
% \item \textbf{Bancos de Dados Distribuídos:} Bancos de Dados Centralizados x Distribuídos; Tipos de Banco de Dados Distribuído; Projeto de Banco de Dados Distribuído; Processamento de Consultas.

%\item \textbf{Banco de Dados NoSQL:} Conceitos básicos; Modelo de Dados: Chave-valor, Orientado a Colunas, Orientado a Documentos e Orientado a Grafos.

%\end{itemize}

%\begin{snugshade}\begin{center}\textbf{
%    Metodologia de Ensino
%}\end{center}\end{snugshade} 

%\noindent
 %  Aulas expositivas utilizando recursos audiovisuais e quadro, além de aulas práticas utilizando computadores. Adicionalmente, serão realizadas atividades práticas individuais ou em grupo, para consolidação do conteúdo ministrado.

%\begin{snugshade}\begin{center}\textbf{
 %   Avaliação do Processo de Ensino e Aprendizagem
%}\end{center}\end{snugshade}   

%\noindent
 % Avaliações escritas. Práticas baseadas em Estudos de Caso ou problemas reais.

%\begin{snugshade}\begin{center}\textbf{
 %   Recursos Necessários
  %  \vphantom{q} % TODO: corrigir o depth da linha sem esta gambiarra.
%}\end{center}\end{snugshade}

%\begin{itemize} 
%  \item Listas de Exercícios;
 %   \item Livros e apostilas;
  %  \item Utilização de recursos da web;
   % \item Quadro branco;
   % \item Marcadores para quadro branco;
    %\item Sala de aula com acesso à internet, microcomputador e TV ou projetor para apresentação de slides ou material multimídia;
    %\item Laboratório de microcomputadores contendo componentes de hardware e \textit{software} específicos;
%\end{itemize}


\begin{snugshade}\begin{center}\textbf{
    Bibliografia
}\end{center}\end{snugshade}

\begin{itemize} 
  \item Básica:
	\begin{enumerate}
	\item 	ELMASRI, R.; NAVATHE, S. \textbf{Sistemas de banco de dados.} Pearson, 6ª edição, 2011;
	\item 	KORTH, H.; SILBERSCHATZ, A.; SUDARSHAN, S. \textbf{Sistemas de bancos de dados.} Campus, 5ª edição, 2006;
	\item 	DATE, C. J. \textbf{Introdução a sistemas de bancos de dados.} Campus, Tradução da 8ª edição Americana, 2004.
	\end{enumerate}
  \item Complementar:
	\begin{enumerate} 
	\item 	CASANOVA, M, et al. \textbf{Bancos de Dados Geográficos}. INPE, 2005; 
	\item 	FOWLER, M.; SADALAGE, P. J. \textbf{NoSQL Essencial: Um Guia Conciso Para O Mundo Emergente Da Persistência Poliglota.} Novatec, 1ª Edição, 2013;
	\item 	RAMAKRISHNAN, R. \textbf{Sistemas de Gerenciamento de Banco de Dados.} McGraw Hill, 3ª edição, 2010.
	\end{enumerate}
\end{itemize}
