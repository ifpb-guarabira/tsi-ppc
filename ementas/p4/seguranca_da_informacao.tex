
% Dados do Componente Curricular

\begin{snugshade}\begin{center}\textbf{
    Dados do Componente Curricular
}\end{center}\end{snugshade}

\noindent \textbf{Nome:}                Segurança da Informação
\\        \textbf{Curso:}               Tecnologia em Sistemas para Internet
\\        \textbf{Período:}             \unit{4}{\degree}
\\        \textbf{Carga Horária:}       \unit{67}{\hour}
\\        \textbf{Docente Responsável:} Moisés Guimarães de Medeiros

% Ementa

\begin{snugshade}\begin{center}\textbf{
    Ementa
\vphantom{q}}\end{center}\end{snugshade}

\noindent
Segurança da informação. Criptografia. Redes. \textit{Firewalls}. Sistemas de arquivos. Padrões, Normas e Certificações. Vulnerabilidades, ataques e contramedidas.

% Objetivos

\begin{snugshade}\begin{center}\textbf{
    Objetivos
}\end{center}\end{snugshade}

\begin{itemize}

\item Compreender a importância da segurança da informação.

\item Conhecer as técnicas, algoritmos e protocolos de criptografia;

\item Conhecer ferramentas de intrusão, varredura e busca de vulnerabilidades;

\item Apresentar as normas, padrões e certificações mais requisitados;

\end{itemize}

% Conteúdo Programático

\begin{snugshade}\begin{center}\textbf{
    Conteúdo Programático
}\end{center}\end{snugshade}

\begin{itemize}

\item \textbf{Segurança da informação:}
    Introdução; Fundamentos; Aplicações.

\item \textbf{Criptografia:}
    História da criptografia;
    Conceitos básicos;
    Criptoanálise;
    Criptografia simétrica;
    Criptografia assimétrica;
    Funções de \textit{hash};
    Infraestrutura de chaves públicas.

\item \textbf{Redes:}
    Comunicação segurança: IPSEC, SSL/TLS, SSH e VPNs;
    Redes sem fio: WEP, WPA, WPA2;
    \textit{Sniffers}.

\item \textbf{\textit{Firewalls}:}
    Histórico e evolução;
    Tipos de \textit{firewall} e suas aplicações.

\item \textbf{Sistemas de arquivos:}
    Estrutura e permissões;
    Formatação física e lógica;
    SMART (\textit{Self-Monitoring, Analysis and Reporting Technology});
    Recuperação de dados.

\item \textbf{Padrões, Normas e Certificações:}
    Padrões       COBIT e ITIL;
    Normas        ISO 27001 e ISO 27002;
    Certificações CEH, LPT, CSSLP e outras.

\item \textbf{Vulnerabilidades, ataques e contramedidas:}
    CVE (\textit{Common Vulnerabilities and Exposures});
    Busca de vulnerabilidades;
    Ataques;
    Monitoramento, controle e auditoria.

\end{itemize}

% Metodologia, Avaliação e Recursos

\begin{snugshade}\begin{center}\textbf{
    Metodologia de Ensino
}\end{center}\end{snugshade}

\noindent
Aulas expositivas utilizando recursos audiovisuais e quadro, além de aulas práticas utilizando computadores. Adicionalmente, serão realizadas atividades práticas individuais ou em grupo, para consolidação do conteúdo ministrado.

\begin{snugshade}\begin{center}\textbf{
    Avaliação do Processo de Ensino e Aprendizagem
}\end{center}\end{snugshade}

\noindent
Avaliações escritas. Avaliações práticas envolvendo a resolução de problemas computacionais.

\begin{snugshade}\begin{center}\textbf{
    Recursos Necessários
    \vphantom{q} % TODO: corrigir o depth da linha sem esta gambiarra.
}\end{center}\end{snugshade}

\begin{itemize}
  \item Listas de Exercícios;
  \item Livros e apostilas;
  \item Utilização de recursos da web;
  \item Quadro branco;
  \item Marcadores para quadro branco;
  \item Sala de aula com acesso à internet, microcomputador e TV ou projetor para apresentação de slides ou material multimídia;
  \item Laboratório de microcomputadores contendo componentes de hardware e software específicos;
\end{itemize}


% Bibliografia

\begin{snugshade}\begin{center}\textbf{
    Bibliografia
}\end{center}\end{snugshade}

\begin{itemize}

\item Básica:
    \begin{enumerate}

    \item STAMP, M.
          Information security: principles and practice.
          Wiley, 2nd edition, 2011.

    \item NAKAMURA, E. T.
          Segurança de redes em sistemas cooperativos.
          Editora Novatec, 2007.

    \item GOODRICH, M. T.; TAMASSIA, R.
          Introdução à Segurança de Computadores.
          Bookman, 1a edição, 2013.

    \end{enumerate}

\item Complementar:
    \begin{enumerate}

    \item STALLINGS, W.
          Criptografia e segurança de redes.
          Prentice-Hall, 4a edição, 2007.

    \item ULBRICH, H. C.; DELLA VALLE, J.
          Universidade Hacker.
          Editora Digerati Books, 2009.

    \item SHOKRANIAN, S.
          Criptografia para iniciantes.
          Ciência Moderna, 2a edição, 2012.

    \item CHAMPLAIN, J. J.
          Auditing information system.
          John Wiley \& Sons, 2a edição, 2003.

    \end{enumerate}

\end{itemize}
