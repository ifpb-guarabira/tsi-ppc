
% Dados do Componente Curricular

\begin{snugshade}\begin{center}\textbf{
    Dados do Componente Curricular
}\end{center}\end{snugshade}

\noindent \textbf{Nome:}                Intera\c{c}\~ao Humano-Computador
\\        \textbf{Curso:}               Tecnologia em Sistemas para Internet
\\        \textbf{Período:}             \unit{3}{\degree}
\\        \textbf{Carga Horária:}       \unit{67}{\hour}
\\        \textbf{Docente Responsável:} Moisés Guimarães de Medeiros

% Ementa

\begin{snugshade}\begin{center}\textbf{
    Ementa
\vphantom{q}}\end{center}\end{snugshade}

\noindent
Interação Humano-Computador: Introdução, Contextualização e Conceituação; Fatores Humanos em Sistemas Interativos; Interface com o usuário: Evolução, Princípios e Regras Básicas; Usabilidade e Acessibilidade: Definição, Aplicação e Métodos de Avaliação; Métodos e Técnicas de Análise; Projeto e Implementação de Interfaces.

% Objetivos

\begin{snugshade}\begin{center}\textbf{
    Objetivos
}\end{center}\end{snugshade}

\begin{itemize}

\item Permitir o aprendizado e a discussão sobre a concepção e construção de sistemas interativos centrados no humano.

\item Compreender os princípios da Interação Humano-Computador;

\item Compreender técnicas para projeto de interfaces centradas no humano;

\item Projetar, desenvolver e avaliar interfaces levando em consideração à usabilidade e acessibilidade.

\end{itemize} 

% Conteúdo Programático

%\begin{snugshade}\begin{center}\textbf{
 %   Conteúdo Programático
%}\end{center}\end{snugshade}

%\begin{itemize}

% \item \textbf{Introdução à IHC:} As Tecnologias da Informação e Comunicação e seu impacto no cotidiano; Sistemas Interativos: Diferentes Visões; Objetos de Estudo em IHC; IHC como Área Multidisciplinar;Benefícios de IHC.

% \item \textbf{Conceitos Básicos em IHC: }Interação, Interface, \textit{Affordance}; Qualidade em IHC(Usabilidade, Acessibilidade, Comunicabilidade).

% \item \textbf{Abordagens Teóricas em IHC}

% \item \textbf{Processos de Design de Sistemas em IHC:} Conceito de Design; Perspectivas de Design; Processo de Design e ciclos de vida; Integração das Atividades de IHC com Engenharia de Software; Métodos Ágeis e IHC.

% \item \textbf{Identificação de Necessidades dos Usuários e Requisitos de IHC:} Dados: O que, de quem e como coletar; Aspectos Éticos de Pesquisas envolvendo Pessoas.

% \item \textbf{Organização do espaço de Problema:} Perfil de Usuário; Personas; Cenários; Tarefas.

% \item \textbf{Princípios e Diretrizes para o Design de IHC:} Princípios; Padrões de Design; Guias de Estilo.

% \item \textbf{Planejamento da Avaliação de IHC} 

% \item \textbf{Métodos de Avaliação de IHC:} Inspe\c{c}\~ao; Observa\c{c}\~ao.

% \item \textbf{Prototipa\c{c}\~ao}

%\end{itemize}

% Metodologia, Avaliação e Recursos

%\begin{snugshade}\begin{center}\textbf{
 %   Metodologia de Ensino
%}\end{center}\end{snugshade}

%\noindent
%Aulas expositivas utilizando recursos áudios-visuais e quadro, além de aulas práticas utilizando computadores. Adicionalmente, serão realizadas atividades práticas individuais ou em grupo, para consolidação do conteúdo ministrado.

%\begin{snugshade}\begin{center}\textbf{
 %   Avaliação do Processo de Ensino e Aprendizagem
%}\end{center}\end{snugshade}

%\noindent
%Avaliação contínua através de exercícios e apresentação de seminários ao final de cada unidade.  Projeto prático final de prototipação e/ou avaliação da interface de um sistema interativo.

%\begin{snugshade}\begin{center}\textbf{
 %   Recursos Necessários
  %  \vphantom{q} % TODO: corrigir o depth da linha sem esta gambiarra.
%}\end{center}\end{snugshade}

%\begin{itemize}
%  \item Listas de Exercícios;
%  \item Livros e apostilas;
%  \item Utilização de recursos da web;
%  \item Quadro branco;
%  \item Marcadores para quadro branco;
%  \item Sala de aula com acesso à internet, microcomputador e TV ou projetor para apresentação de slides ou material multimídia;
%  \item Laboratório de microcomputadores contendo componentes de hardware e software específicos;
%\end{itemize}

% Bibliografia

\begin{snugshade}\begin{center}\textbf{
    Bibliografia
}\end{center}\end{snugshade}

\begin{itemize}
  \item Básica:
	\begin{enumerate}
		\item BARBOSA , S., SILVA, B. , Interação humano-computador. Elsevier. 2010.
		\item PREECE, J., ROGERS, Y., SHARP, H., Design de Interação: além da interação homem-máquina. 3ª Ed. Bookman, 2013.
		\item BENYON, D., Interação Humano-Computador . 2ª Edição. Pearson, 2011.
	\end{enumerate}
  \item Complementar:
	\begin{enumerate} 
		\item NIELSEN, J., Loranger, H. Usabilidade na Web: Projetando Websites com Qualidade. Elsevier, 2007.
		\item SHNEIDERMAN, Ben. Designing the User Interface: strategies for effective human-computer interaction. 4. ed. EUA: Addison-Wesley, 2004.
	\end{enumerate}
\end{itemize}
