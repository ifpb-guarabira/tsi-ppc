
% Dados do Componente Curricular

\begin{snugshade}\begin{center}\textbf{
	Dados do Componente Curricular
}\end{center}\end{snugshade}

\noindent 	\textbf{Nome:} Programação Orientada a Objetos
\\ 			\textbf{Curso:} Tecnologia em Sistemas para Internet
\\ 			\textbf{Período:} \unit{3}{\degree}
\\ 			\textbf{Carga Horária:} \unit{83}{\hour}
\\ 			\textbf{Docente Responsável:} José de Sousa Barros 


% Ementa

\begin{snugshade}\begin{center}\textbf{
    Ementa
\vphantom{q}}\end{center}\end{snugshade}

\noindent
O paradigma de programação orientada a objetos: abstração, conceito de classes e objetos, troca de mensagens entre objetos, composição de objetos, encapsulamento, empacotamento de classes, visibilidade, coleções de objetos, herança, sobrescrita, sobrecarga, interface e polimorfismo, tratamento de exceções, persistência de dados em arquivos.
% Objetivos

\begin{snugshade}\begin{center}\textbf{
    Objetivos
}\end{center}\end{snugshade}

\begin{itemize}

\item Identificar os conceitos do paradigma de programação orientado a objetos;
\item Utilizar os conceitos do paradigma de programação orientado a objetos;
\item Desenvolver aplicações em uma linguagem de programação Orientada a Objetos.

\end{itemize} 

% Conteúdo Programático

\begin{snugshade}\begin{center}\textbf{
    Conteúdo Programático
}\end{center}\end{snugshade}

\begin{itemize}

 \item \textbf{Introdução à Programação Orientada a Objetos:} Abstração; Modelagem orientada a objetos; Apresentação de uma linguagem de programação orientada a objetos;	Classes; Objetos; Construtores; Métodos; Encapsulamento e visibilidade, Pacotes.


 \item \textbf{Herança e Polimorfismo:} Membros de classe: atributos e métodos (de classe e de instância); Herança;	Classes abstratas; Métodos abstratos; Sobrescrita de métodos; Sobrecarga de métodos; Interfaces;	Polimorfismo; Coleções estáticas.

 \item \textbf{Coleções dinâmicas e Tratamento de exceções:} Generics; Coleções dinâmicas: Collection, List, Queue, Deque, Set e SortedSet; Tratamento de exceções; Interface gráfica; Manipulação de eventos; Persistência de dados em arquivos.

\end{itemize}

\begin{snugshade}\begin{center}\textbf{
    Metodologia de Ensino
}\end{center}\end{snugshade}

\noindent
      Aulas expositivas utilizando recursos audiovisuais e quadro, além de aulas práticas utilizando computadores. Adicionalmente, serão realizadas atividades práticas individuais ou em grupo, para consolidação do conteúdo ministrado.

\begin{snugshade}\begin{center}\textbf{
    Avaliação do Processo de Ensino e Aprendizagem
}\end{center}\end{snugshade}

\noindent
  Avaliações escritas ao final de cada unidade. Prática baseada em Estudo de Caso ou problema real.
  
\begin{snugshade}\begin{center}\textbf{
    Recursos Necessários
    \vphantom{q} % TODO: corrigir o depth da linha sem esta gambiarra.
}\end{center}\end{snugshade}

\begin{itemize} 
  	  \item Listas de Exercícios;
  	  \item Livros e apostilas;
  	  \item Utilização de recursos da web;
  	  \item Quadro branco;
  	  \item Marcadores para quadro branco;
  	  \item Sala de aula com acesso à internet, microcomputador e TV ou projetor para apresentação de slides ou material multimídia;
  	  \item Laboratório de microcomputadores contendo componentes de hardware e \textit{software} específicos;
\end{itemize}

% Bibliografia

\begin{snugshade}\begin{center}\textbf{
    Bibliografia
}\end{center}\end{snugshade}

\begin{itemize} 
  \item Básica:
	\begin{enumerate}
	\item DEITEL, H. M.; DEITEL, P. J. \textbf{Java: Como Programar.} Pearson, 8ª Edição, 2010;
	\item FURGERI, S. \textbf{Java 7 Ensino Didático.} Érica, 1ª Edição, 2010;
	\item SIERRA K.; BATES, B. \textbf{Use a Cabeça! - Java.} Alta Books, 2ª Edição, 2007.
	\end{enumerate}
  \item Complementar:
	\begin{enumerate} 
	\item HORSTMANN, C. S. \& CORNELL, G. \textbf{Core Java, Volume 1.} Pearson, 8ª edição, 2010;
	\item CADENHEAD, R.; LEMAY, L. \textbf{Aprenda Java em 21 Dias.} Campus, 4ª edição, 2005.
	\end{enumerate}
\end{itemize}

