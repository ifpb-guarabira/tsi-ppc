
% Dados do Componente Curricular

\begin{snugshade}\begin{center}\textbf{
    Dados do Componente Curricular
}\end{center}\end{snugshade}

\noindent \textbf{Nome:}                Sistemas Operacionais
\\        \textbf{Curso:}               Tecnologia em Sistemas para Internet
\\        \textbf{Período:}             \unit{3}{\degree}
\\        \textbf{Carga Horária:}       \unit{67}{\hour}
\\        \textbf{Docente Responsável:} Rodrigo Pinheiro Marques de Araújo

% Ementa

\begin{snugshade}\begin{center}\textbf{
    Ementa
\vphantom{q}}\end{center}\end{snugshade}

\noindent
Conceitos básicos de sistemas operacionais; Gerência de processador; Processos e \textit{Threads}; Comunicação entre processos; Gerência de memória; Gerência de entrada/saída; Sistemas de arquivos; Segurança em sistemas operacionais; Estudo de casos.

% Objetivos

\begin{snugshade}\begin{center}\textbf{
    Objetivos
}\end{center}\end{snugshade}

\begin{itemize}

\item Entender o papel do sistema operacional dentro de um sistema computacional;

\item Entender o funcionamento dos vários módulos que compõem um sistema operacional;

\item Desenvolver uma visão crítica sobre os requisitos de confiabilidade, segurança e desempenho, associados a um sistema operacional;

\item Compreender os mecanismos básicos de: chamada ao sistema, tratamento de interrupções, bloqueio e escalonamento de processos;

\item Compreender as principais estruturas de dados de um sistema operacional;

\item Compreender os principais algoritmos utilizados para gerir a utilização dos recursos do sistema;

\item Compreender as necessidades e os mecanismos utilizados pelo sistema operacional para prover segurança para o sistema computacional.

\end{itemize} 

% Conteúdo Programático

%\begin{snugshade}\begin{center}\textbf{
 %   Conteúdo Programático
%}\end{center}\end{snugshade}

%\begin{itemize}

 %\item \textbf{Introdução aos Sistemas Operacionais:} Funções de um sistema operacional; Conceitos básicos.

 %\item \textbf{Processos e \textit{Threads}:} Definição e estrutura de processos; Estados de um processo; Escalonamento de processos; Fluxo de execução de um processo; \textit{Multithreading}; Comunicação entre processos; Escalonamento para processadores \textit{multi-core}. Impasses; Definição de impasses; Técnicas para o tratamento de impasses.

 %\item \textbf{Ger\^encia de mem\'oria:} Gerência de memória sem \textit{swap} ou paginação; \textit{Swapping}; Memória virtual; Algoritmos de reposição de páginas; Segmentação.

 %\item \textbf{Entrada/Saída:} \textit{Hardware} e \textit{software} de entrada/saída; Projeto e implementação de \textit{drivers} de dispositivos.

 %\item \textbf{Sistemas de Arquivos:} Arquivos e diretórios; Implementação de sistemas de arquivos; Segurança e mecanismos de proteção da informação.


%\end{itemize}

% Metodologia, Avaliação e Recursos

%\begin{snugshade}\begin{center}\textbf{
    Metodologia de Ensino
}\end{center}\end{snugshade}

\noindent
Aulas expositivas utilizando recursos audiovisuais e quadro, além de aulas práticas utilizando computadores. Adicionalmente, serão realizadas atividades práticas individuais ou em grupo, para consolidação do conteúdo ministrado.

%\begin{snugshade}\begin{center}\textbf{
    Avaliação do Processo de Ensino e Aprendizagem
}\end{center}\end{snugshade}

\noindent
Avaliações escritas. Avaliações práticas envolvendo a resolução de problemas computacionais.

%\begin{snugshade}\begin{center}\textbf{
    Recursos Necessários
    \vphantom{q} % TODO: corrigir o depth da linha sem esta gambiarra.
}\end{center}\end{snugshade}

\begin{itemize}
  \item Listas de Exercícios;
  \item Livros e apostilas;
  \item Utilização de recursos da web;
  \item Quadro branco;
  \item Marcadores para quadro branco;
  \item Sala de aula com acesso à internet, microcomputador e TV ou projetor para apresentação de slides ou material multimídia;
  \item Laboratório de microcomputadores contendo componentes de hardware e software específicos;
\end{itemize}


% Bibliografia

\begin{snugshade}\begin{center}\textbf{
    Bibliografia
}\end{center}\end{snugshade}

\begin{itemize} 
  \item Básica:
	\begin{enumerate}
		\item Tanenbaum, A. S. Sistemas Operacionais Modernos. ISBN: 9788576052371. Editora Pearson. 3 Ed., 2010. 
		\item Silberschatz, A.; et al. ISBN: 9788521617471. Fundamentos de Sistemas Operacionais. Editora LTC, 8 Ed., 2010; 
	\end{enumerate}
  \item Complementar:
	\begin{enumerate} 
		\item Marshall Kirk McKusick, George V. Neville-Neil, Robert N.M. Watson. The Design and Implementation of the FreeBSD Operating System. ISBN: 978-0321968975. Editora Addison-Wesley. 2 Ed., 2014.
		\item Mark Russinovich, David Solomon, Alex Ionescu. Windows Internals, Part 1. Microsoft Press. ISBN: 978-0735648739. 6 Ed., 2012.
		\item Robert Love. Linux Kernel Development. ISBN: 978-0672329463. Editora Addison-Wesley. 3 Ed., 2010.
	\end{enumerate}
\end{itemize}
