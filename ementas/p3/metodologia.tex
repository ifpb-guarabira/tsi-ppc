
% Dados do Componente Curricular

\begin{snugshade}\begin{center}\textbf{
    Dados do Componente Curricular
}\end{center}\end{snugshade}

\noindent \textbf{Nome:}                Metodologia da Pesquisa Cient\'ifica
\\        \textbf{Curso:}               Tecnologia em Sistemas para Internet
\\        \textbf{Período:}             \unit{3}{\degree}
\\        \textbf{Carga Horária:}       \unit{33}{\hour}
\\        \textbf{Docente Responsável:} Erick Augusto Gomes de Melo 

% Ementa

\begin{snugshade}\begin{center}\textbf{
    Ementa
\vphantom{q}}\end{center}\end{snugshade}

\noindent
Estudo do texto: análise, síntese e interpretação. Sublinhando, esquematizando e resumindo. Tipos de resumo. Tipos de fichamento. Análise textual, temática e interpretativa. Estudo do processo de pesquisa científica aplicada, discutindo questões teóricas da pesquisa. Levantamento de informações para pesquisas. Conceitos utilizados na pesquisa. Tipos de pesquisa. O método científico e suas etapas. Definição de método. Tipos de método. Técnicas de pesquisa: definição e classificação. Problematização. Formulação de hipóteses. Variáveis. Coleta de dados. Amostra. Análise dos dados e conclusões. A organização do texto científico. Normas da ABNT. Tipos e caracterização de trabalhos científicos. Elaboração de projetos de pesquisa, de relatórios de pesquisa e de artigos científicos.

% Objetivos

\begin{snugshade}\begin{center}\textbf{
    Objetivos
}\end{center}\end{snugshade}

\begin{itemize}


\item Identificar e distinguir os tipos de conhecimento.

\item Caracterizar e aplicar os processos da técnica de leitura analítica para análise e interpretação de textos teóricos e/ou científicos;

\item Desenvolver habilidades de interpretação de textos técnicos e acadêmicos e de elaboração de fichamentos e resumos;

\item Identificar, distinguir e aplicar as diversas técnicas de documentação para elaboração de trabalhos acadêmicos;

\item Conhecer os principais métodos e técnicas de pesquisa científica;

\item Conceituar, diferenciar e relacionar método, técnica, método científico, pesquisa, ciência e metodologia científica;

\item Compreender e diferenciar pesquisa qualitativa e pesquisa quantitativa;

\item Conhecer os fundamentos, os métodos e as técnicas de coleta de dados e de análise presentes na produção do conhecimento científico;

\item Conceituar pesquisa, destacar sua importância na graduação e identificar suas modalidades e fases;

\item Conhecer as diversas técnicas de investigação científica e as etapas de preparação e execução da pesquisa científica;

\item Conhecer e caracterizar os diversos tipos de trabalhos científicos;

\item Definir, caracterizar e diferenciar os tipos de trabalhos acadêmicos nos cursos de graduação;

\item Desenvolver habilidades técnicas de apresentação de seminários;

\item Conhecer e aplicar normas da ABNT na produção de trabalhos científicos;

\item Compreender as diversas fases de elaboração e desenvolvimento de trabalhos acadêmicos;

\item Produzir trabalhos científicos: fichamentos; resumos; resenhas, projetos de pesquisa, artigos, papers, relatórios de pesquisa, monografias;

\item Identificar as características e normas gerais da linguagem e redação científica e aplicá-las na produção de textos acadêmicos;

\item Compreender e aplicar os princípios da metodologia científica em situações de apreensão, produção e expressão do conhecimento.

\end{itemize} 

% Conteúdo Programático

\begin{snugshade}\begin{center}\textbf{
    Conteúdo Programático
}\end{center}\end{snugshade}

\begin{itemize}

\item \textbf{Tipos de conhecimento;}

\item \textbf{O conhecimento na sociedade globalizada;}

\item \textbf{O que é ciência e conhecimento científico;}

\item \textbf{O que é método:} 
Caracterização do método científico; Fases do método científico.

\item \textbf{Tipos de pesquisa:}
Quanto à natureza: pesquisa básica ou fundamental, pesquisa aplicada ou tecnológica. Quanto aos objetivos: exploratória, descritiva e explicativa. Quanto aos procedimentos: experimental, operacional. Estudo de caso.

\item \textbf{Metodologias de pesquisa:}
Métodos, tipos e natureza.

\item \textbf{Pesquisa qualitativa e Pesquisa quantitativa:}
Tratamento dos dados em pesquisas qualitativas e quantitativas.

\item \textbf{Noções de seminário:}
Apresentação; níveis de linguagem; adequação.

\item \textbf{Aspectos da linguagem oral:}
Técnicas da oralidade; o texto argumentativo oral.

\item \textbf{Estrutura e elaboração de projetos de pesquisa:}
Levantamento do problema; hipóteses e variáveis; população e amostra; coleta de dados, cronograma.

\item \textbf{Construção dos instrumentos de pesquisa:}
Observação e estudos de caso.

\item \textbf{Tipos e estrutura de trabalhos acadêmicos e científicos;}

\item \textbf{Elaboração de trabalhos científicos:}
Fichamentos; resumos; resenhas, projetos de pesquisa, artigos, papers, relatórios de pesquisa, monografias.

\end{itemize}

% Metodologia, Avaliação e Recursos

\begin{snugshade}\begin{center}\textbf{
    Metodologia de Ensino
}\end{center}\end{snugshade}

\noindent
Nas aulas, serão adotados os seguintes procedimentos metodológicos: exposição verbal dialogada com apoio audiovisual, leituras e discussão de textos, realização de exercícios de forma individual e em pequenos grupos, apresentação oral de trabalhos e seminários. Sempre que pertinentes, serão solicitadas leituras em sala de aula. As discussões serão programadas para acontecerem, de preferência, durante o estudo dos conteúdos e serão complementadas com a efetivação de exercícios em sala ou extraclasse. Os exercícios poderão ser realizados de forma individual ou em pequenos grupos de estudo. Filmes também poderão ser utilizados, desde que pertinentes.

\begin{snugshade}\begin{center}\textbf{
    Avaliação do Processo de Ensino e Aprendizagem
}\end{center}\end{snugshade}

\noindent
         A disciplina deverá adotar como formas avaliativas os seguintes procedimentos: trabalhos individuais e coletivos em sala de aula, produção escrita de comentários de leitura, apresentações orais e apresentação de seminários. Os trabalhos individuais e coletivos em sala envolverão leituras, discussões de temas em pequenos grupos, apresentação e discussão em plenária. Os critérios básicos de avaliação serão: emprego de linguagem adequada, uso correto das normas da ABNT, compreensão, criatividade, criticidade e coerência. Também será levada em conta a participação do aluno. A periodicidade de aplicação dos procedimentos avaliativos será contínua, ao longo do semestre. Isto é, a cada aula poderá ser solicitada a execução de um dos procedimentos avaliativos acima descritos.

\begin{snugshade}\begin{center}\textbf{
    Recursos Necessários
    \vphantom{q} % TODO: corrigir o depth da linha sem esta gambiarra.
}\end{center}\end{snugshade}

\begin{itemize}
  \item Quadro branco;
  \item Marcadores para quadro branco;
  \item Sala de aula com acesso à internet, microcomputador e TV ou projetor para apresentação de slides ou material multimídia;
  \item Estudo dirigido;
  \item Filmes e textos.
\end{itemize}

% Bibliografia

\begin{snugshade}\begin{center}\textbf{
    Bibliografia
}\end{center}\end{snugshade}

\begin{itemize} 

\item Básica:
    \begin{enumerate}

    \item ANDRADE, M.M. Introdução à metodologia do trabalho científico: elaboração de trabalhos na graduação. Atlas, 2010;
    
    \item ASSOCIAÇÃO BRASILEIRA DE NORMAS TÉCNICAS. NBR 6023: Informação e documentação, referências – elaboração. Rio de Janeiro, 2002;
    
    \item BARROS, A.; LEHFELD, N. Projeto de pesquisa: propostas metodológicas. Vozes, 4ª edição, 1996;
	
    \end{enumerate}

\item Complementar:
	\begin{enumerate} 

    \item NBR 10520: Informação e documentação, apresentação de citações em documentos. Rio de Janeiro, 2002;

    \item NBR 14724: Informação e documentação, trabalhos acadêmicos – apresentação. Rio de Janeiro, 2005;

    \item CERVO, A. L.; BERVIAN, P. A. Metodologia científica.  Prentice Hall, 5ª edição, 2006;

    \item DUARTE, E. Manual técnico para a realização de trabalhos monográficos.  Universitária, 4ª Edição, 2001;

    \item DESLANDES, S F. A construção de projeto de pesquisa. In: MINAYO, M. C. de S. (Org). Pesquisa social: teoria, método e criatividade.  Petrópolis, 21ª edição, 1994, p. 31-50;
 
    \item GODOY, A. S. Introdução à pesquisa qualitativa e suas possibilidades. Revista de administração de empresas, v.35, n.2, p.57-83, mar/abr., 1995;

    \item KÖCHE, J. C. Fundamentos de metodologia científica: teoria da ciência e iniciação à pesquisa. Vozes, 26ª edição, 2009;

   % \item LAKATOS, E. M.; MARCONI, M. de A. Fundamentos de metodologia científica.  Atlas, 3ª edição,  1991;

   % \item MÁTTAR NETO, J. A. Metodologia cientifica na era da informática. Saraiva, 2007. 

   % \item MEDEIROS, J. B. Manual de redação e normalização textual: técnicas de editoração e revisão. Atlas, 2002;

   % \item OLIVEIRA NETTO, A. A. Metodologia da pesquisa científica: guia prático para apresentação de trabalhos acadêmicos. Visual Books, 2ª edição, 2008;

   % \item POLITO, R. Como falar em público corretamente e sem inibições.  Saraiva, 1999;

   % \item POSSENTI, S. Discurso, estilo e subjetividade. Martins Fontes, 2001;

   % \item RICHARDSON, R. J. Pesquisa social: métodos e técnicas. Atlas, 3ª edição, 2008;

   % \item SEVERINO, A. J. Metodologia do trabalho científico.  Cortez, 2007;

   % \item TACHIZAWA, T. ; MENDES, G. Como fazer monografia na prática.  FGV, 4ª edição, 1999.

	\end{enumerate}

\end{itemize}
