\paragraph{Estruturas de Dados II} \

% Dados do Componente Curricular

\begin{snugshade}\begin{center}\textbf{
    Dados do Componente Curricular
}\end{center}\end{snugshade}

\noindent \textbf{Nome:}                Estruturas de Dados II
\\        \textbf{Curso:}               Tecnologia em Sistemas para Internet
\\        \textbf{Período:}             \unit{3}{\degree}
\\        \textbf{Carga Horária:}       \unit{67}{\hour}
\\        \textbf{Docente Responsável:} Ruan Delgado Gomes

% Ementa

\begin{snugshade}\begin{center}\textbf{
    Ementa
\vphantom{q}}\end{center}\end{snugshade}

\noindent
\'Arvore balanceada pela altura (AVL); Tabelas Hash; \'Arvores vermelho e preto; Grafos; Algoritmos de busca em grafos; \'Arvore geradora m\'inima; Algoritmos de menor caminho em grafos.

% Objetivos

\begin{snugshade}\begin{center}\textbf{
    Objetivos
}\end{center}\end{snugshade}

\begin{itemize}

\item Apresentar estruturas de dados e algoritmos amplamente utilizados e discutir sua implementação e seu desempenho;

\item Aprender a utilizar estruturas de dados n\~ao lineares e algoritmos que manipulam essas estruturas de dados para resolu\c{c}\~ao de problemas computacionais.

\end{itemize} 

% Conteúdo Programático

\begin{snugshade}\begin{center}\textbf{
    Conteúdo Programático
}\end{center}\end{snugshade}

\begin{itemize}

 \item \textbf{\'Arvore balanceada pela altura (AVL):} Defini\c{c}\~ao; Algoritmos de inser\c{c}\~ao, remo\c{c}\~ao e busca; Balanceamento; An\'alise de complexidade dos algoritmos para manipula\c{c}\~ao de \'arvores AVL.

% \item \textbf{Estilos Arquiteturais para SD:} Camadas; Baseada em Objetos; Baseada em Dados; Baseada em Eventos.

 \item \textbf{Tabelas Hash:} Tabelas de endere\c{c}amento direto; Tabelas Hash; Fun\c{c}\~oes Hash; Endere\c{c}amento aberto; Hashing perfeito.

 \item \textbf{\'Arvores vermelho e preto:} Propriedades; Algoritmos de rota\c{c}\~ao, inser\c{c}\~ao e remo\c{c}\~ao; An\'alise de complexaide de algoritmos para manipula\c{c}\~ao de \'arvores vermelho e preto.

 \item \textbf{Grafos:} Representa\c{c}\~ao de grafos; Modelagem de problemas utilizando grafos.

 \item \textbf{Algoritmos de busca em grafos:} Busca em largura (BFS); Busca em profundidade (DFS).

 \item \textbf{\'Arvore geradora m\'inima:} Crescimento de \'arvore geradora m\'inima; Algoritmos de Kruskal e Prim.

 \item \textbf{Algoritmos de menor caminho em grafos:} Algoritmo de Bellman-Ford; Menor caminho de \'unica fonte em grafos ac\'iclicos direcionados; Algoritmo de Dijkstra; 

\end{itemize}

% Metodologia, Avaliação e Recursos

\begin{snugshade}\begin{center}\textbf{
    Metodologia de Ensino
}\end{center}\end{snugshade}

\noindent
Aulas expositivas utilizando recursos audiovisuais e quadro, além de aulas práticas utilizando computadores. Adicionalmente, serão realizadas atividades práticas individuais ou em grupo, para consolidação do conteúdo ministrado.

\begin{snugshade}\begin{center}\textbf{
    Avaliação do Processo de Ensino e Aprendizagem
}\end{center}\end{snugshade}

\noindent
Avaliações escritas. Avaliações práticas envolvendo a resolução de problemas computacionais.

\begin{snugshade}\begin{center}\textbf{
    Recursos Necessários
    \vphantom{q} % TODO: corrigir o depth da linha sem esta gambiarra.
}\end{center}\end{snugshade}

\begin{itemize}
  \item Listas de Exercícios;
  \item Livros e apostilas;
  \item Utilização de recursos da web;
  \item Quadro branco;
  \item Marcadores para quadro branco;
  \item Sala de aula com acesso à internet, microcomputador e TV ou projetor para apresentação de slides ou material multimídia;
  \item Laboratório de microcomputadores contendo componentes de hardware e software específicos;
\end{itemize}


% Bibliografia

\begin{snugshade}\begin{center}\textbf{
    Bibliografia
}\end{center}\end{snugshade}

\begin{itemize} 

  \item Básica:
	\begin{enumerate}
		\item T.H. Cormen, C.E. Leiserson, R.L. Rivest, C. Stein, "Algoritmos - Teoria e Pr\'atica", 3a. ed., ISBN: 8535236996, Editora Campus, 2012.
	\end{enumerate}
  \item Complementar:
	\begin{enumerate} 
		\item Steven S Skiena, The Algorithm Design Manual, Springer; 2nd edition, ISBN: 978-1849967204, 2008.\\
	\end{enumerate}
\end{itemize}
