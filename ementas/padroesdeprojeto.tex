\paragraph{Padr\~oes de Projeto}

%PREENCHER DADOS DA DISCIPLINA A SEGUIR
%\vspace{-12mm}
\begin{center}\textbf{Dados do Componente Curricular}\end{center}
\vspace{-5mm}
\noindent\rule{16.5cm}{0.4pt}
\\
\textbf{Nome:} Padr\~oes de Projeto
\\ 
\textbf{Curso:} Tecnologia em Sistemas para Internet
\\ 
\textbf{Período:} $5^{\circ}$
\\ 
\textbf{Carga Horária:} 67~h
\\ 
\textbf{Docente Responsável:} Rodrigo Pinheiro Marques de Ara\'ujo
\\ 
\noindent\rule{16.5cm}{0.4pt}\\
\\
%PREENCHER A EMENTA A SEGUIR
\vspace{-12mm}
\begin{center}\textbf{Ementa}\end{center}
\vspace{-5mm}
\noindent\rule{16.5cm}{0.4pt}
\\
Caracterização dos padrões de projeto; Padrões e reusabilidade; Tipos de padrões de projeto; Aplicação de padrões de projeto no desenvolvimento de software orientado a objetos.\\ 
\noindent\rule{16.5cm}{0.4pt}\\
\\
%PREENCHER OS OBJETIVOS A SEGUIR
\vspace{-12mm}
\begin{center}\textbf{Objetivos}\end{center}
\vspace{-5mm}
\noindent\rule{16.5cm}{0.4pt}
\\
\begin{itemize}
\item Compreender conceitos e técnicas dos padrões de projeto de software necessárias para a modelagem e análise de sistemas;
\item Compreender os princípios da programação orientada a objetos;
\item Identificar os princípios básicos dos padrões de Projeto de software;
\item Compreender os padrões GRASP;
\item Compreender os padrões GoF;
\end{itemize} 
\noindent\rule{16.5cm}{0.4pt}\\
\\
%PREENCHER OS CONTEUDOS PROGRAMATICOS A SEGUIR (CUIDADO PARA NAO DEIXAR A TABELA MUITO GRANDE)
\vspace{-12mm}
\begin{center}\textbf{Conteúdo Programático}\end{center}
\vspace{-5mm}
\noindent\rule{16.5cm}{0.4pt}
\\
\begin{itemize}

 \item \textbf{Introdução aos Padrões de Projeto:} Revisão histórica; Conceitos básicos da Orientação a Objetos; Padrões Básicos.

 \item \textbf{Padrões GRASP: } Padrão Expert; Padrão Creator; Padrão Low Coupling; Padrão High Cohesion; Padrão Model View Controller (MVC).

 \item \textbf{Padrões GoF de interface:}  Adapter; Bridge; Facade; Composite.

 \item \textbf{Padrões GoF de Responsabilidade:} Singleton; Observer; Mediator; Chain of Responsability; Proxy.

 \item \textbf{Padrões GoF de Construção:} Builder; Abstract Factory; Factory Method.

 \item \textbf{Padrões GoF de Operações:} Command; Strategy.

 \item \textbf{Padrões de Extensão:} Decorator; Iterator.

\end{itemize}
\noindent\rule{16.5cm}{0.4pt}\\
\\
%COLOCAR A METODOLOGIA DE ENSINO A SEGUIR
\vspace{-12mm}
\begin{center}\textbf{Metodologia de Ensino}\end{center} 
\vspace{-5mm}
\noindent\rule{16.5cm}{0.4pt}
\\
   Aulas expositivas utilizando recursos audiovisuais e quadro, além de aulas práticas utilizando computadores. Adicionalmente, serão realizadas atividades práticas individuais ou em grupo, para consolidação do conteúdo ministrado.\\
\noindent\rule{16.5cm}{0.4pt}\\
\\
%COLOCAR AVALIACAO DO PROCESSO DE ENSINO E APRENDIZAGEM A SEGUIR
\vspace{-12mm}
\begin{center}\textbf{Avaliação do Processo de Ensino e Apendizagem}\end{center}
\vspace{-5mm}
\noindent\rule{16.5cm}{0.4pt}
\\
   Avaliações escritas e pr\'aticas.\\
\noindent\rule{16.5cm}{0.4pt}\\
\\
%PREENCHER RECURSOS NECESSARIOS A SEGUIR
\vspace{-12mm}
\begin{center}\textbf{Recursos Necessários}\end{center}
\vspace{-5mm}
\noindent\rule{16.5cm}{0.4pt}
\\
\begin{itemize} 
  \item Listas de Exercícios;
  \item Livros e apostilas;
  \item Utilização de recursos da web;
  \item Quadro branco;
  \item Marcadores para quadro branco;
  \item Sala de aula com acesso à internet, microcomputador e TV ou projetor para apresentação de slides ou material multimídia;
  \item Laboratório de microcomputadores contendo componentes de hardware e software específicos;
\end{itemize}
\noindent\rule{16.5cm}{0.4pt}\\
\\
%PREENCHER BIBLIOGRAFIA A SEGUIR
\vspace{-12mm}
\begin{center}\textbf{Bibliografia}\end{center}
\vspace{-5mm}
\noindent\rule{16.5cm}{0.4pt}
\\
\begin{itemize} 
  \item Básica:
	\begin{enumerate}
	\item Gamma, E. et al. Padrões de Projeto: Soluções reutilizáveis de software orientado a objetos. Bookman, 2000.
	\item Freeman, E; Freeman, E. Use a cabeça! Padrões de Projeto (Design Patterns). Alta books, 2ª Edição. 2007.
	\end{enumerate}
  \item Complementar:
	\begin{enumerate} 
	\item Metsker, S. J. Padrões de Projeto em Java. Bookman, 2004.
	\item Shalloway, A.; Trott, J. R. Explicando padrões de projeto – Uma nova perspectiva em projeto orientado a objetos. Bookman, 2004.
	\end{enumerate}
\end{itemize}
\noindent\rule{16.5cm}{0.4pt}\\
\\
