\paragraph{Sistemas Distribu\'idos}

%PREENCHER DADOS DA DISCIPLINA A SEGUIR
%\vspace{-12mm}
\begin{center}\textbf{Dados do Componente Curricular}\end{center}
\vspace{-5mm}
\noindent\rule{16.5cm}{0.4pt}
\\
\textbf{Nome:} Sistemas Distribuídos
\\ 
\textbf{Curso:} Tecnologia em Sistemas para Internet
\\ 
\textbf{Período:} $6^{\circ}$
\\ 
\textbf{Carga Horária:} 67~h
\\ 
\textbf{Docente Responsável:} Ruan Delgado Gomes
\\ 
\noindent\rule{16.5cm}{0.4pt}\\
\\
%PREENCHER A EMENTA A SEGUIR
\vspace{-12mm}
\begin{center}\textbf{Ementa}\end{center}
\vspace{-5mm}
\noindent\rule{16.5cm}{0.4pt}
\\
Fundamentos de Sistemas Distribuídos. Arquitetura de Comunicação Cliente-Servidor. Objetos distribu\'idos e invoca\c{c}\~ao remota. Sistemas \textit{peer-to-peer} (P2P); Sistemas de arquivos distribuídos; Sincronização e coordena\c{c}\~ao; Transações e controle de concorr\^encia; Replicação e tolerância a falhas; Serviços Web; T\'opicos especiais em sistemas distribu\'idos.\\ 
\noindent\rule{16.5cm}{0.4pt}\\
\\
%PREENCHER OS OBJETIVOS A SEGUIR
\vspace{-12mm}
\begin{center}\textbf{Objetivos}\end{center}
\vspace{-5mm}
\noindent\rule{16.5cm}{0.4pt}
\\
\begin{itemize}
\item Proporcionar o entendimento sobre as possíveis formas de estruturação dos sistemas distribuídos;
\item Conhecer e utilizar técnicas para garantir a qualidade de sistemas distribuídos;
\item Saber como resolver problemas de faltas em sistemas distribuídos.
\end{itemize} 
\noindent\rule{16.5cm}{0.4pt}\\
\\
%PREENCHER OS CONTEUDOS PROGRAMATICOS A SEGUIR (CUIDADO PARA NAO DEIXAR A TABELA MUITO GRANDE)
\vspace{-12mm}
\begin{center}\textbf{Conteúdo Programático}\end{center}
\vspace{-5mm}
\noindent\rule{16.5cm}{0.4pt}
\\
\begin{itemize}
 \item \textbf{Fundamentos de Sistemas Distribuídos:} Definição de Sistemas Distribuídos; Infraestrutura básica; Tipos de Sistemas Distribuídos. Modelos de Sistemas Distribu\'idos.

% \item \textbf{Estilos Arquiteturais para SD:} Camadas; Baseada em Objetos; Baseada em Dados; Baseada em Eventos.

 \item \textbf{Arquitetura de Comunicação Cliente-Servidor:} Requisição-Resposta; Comunicação síncrona; Comunicação assíncrona; Exemplos de sistemas cliente-servidor na Web.
  
 \item \textbf{Objetos distribu\'idos e invoca\c{c}\~ao remota:} RPC, RMI, MOM:

 \item \textbf{Sistemas \textit{peer-to-peer} (P2P):} \textit{Middleware} para sistemas P2P; Sobreposi\c{c}\~ao de roteamento; Mecanismos de busca; Exemplos de sistemas P2P.

 \item \textbf{Sistemas de arquivos distribuídos:} Arquitetura e requisitos de servi\c{c}os de arquivos; Exemplos de sistemas de arquivos distribu\'idos.

 \item \textbf{Sincronização e coordena\c{c}\~ao:} Rel\'ogios, eventos e estados de processo; Sincroniza\c{c}\~ao de rel\'ogios f\'isicos; Tempo l\'ogico e rel\'ogios l\'ogicos; Estados globais; Exclus\~ao m\'utua distribu\'ida; Elei\c{c}\~ao de l\'ider; Comunica\c{c}\~ao \textit{multicast}.

 \item \textbf{Transações e controle de concorr\^encia:} Transa\c{c}\~oes; Transa\c{c}\~oes aninhadas; Travas e bloqueios; Controle de concorr\^encia; Transa\c{c}\~oes distribu\'idas planas e aninhadas; Protocolos de efetiva\c{c}\~ao at\^omica; Controle de concorr\^encia e impasses distribu\'idos; Recupera\c{c}\~ao de transa\c{c}\~oes.

\item \textbf{Replicação e tolerância a falhas:} Servi\c{c}os tolerantes a falhas; Estudos de caso de servi\c{c}oes de alta disponibilidade; Transa\c{c}\~oes com replica\c{c}\~ao de dados.

\item \textbf{Serviços Web:} Conceitos; Arquitetura orientada a servi\c{c}o; Tipos de servi\c{c}os; Design de servi\c{c}os; Registro e descoberta; Web Services; Grades computacionais.

\item \textbf{T\'opicos especiais em sistemas distribu\'idos:} Conte\'udo vari\'avel abordando t\'opicos atuais em sistemas distribu\'idos, por exemplo: sistemas m\'oveis, computa\c{c}\~ao ub\'iqua, redes de sensores sem fio, sistemas multim\'ida distribu\'idos; Computa\c{c}\~ao na nuvem.
\end{itemize}
\noindent\rule{16.5cm}{0.4pt}\\
\\
%COLOCAR A METODOLOGIA DE ENSINO A SEGUIR
\vspace{-12mm}
\begin{center}\textbf{Metodologia de Ensino}\end{center} 
\vspace{-5mm}
\noindent\rule{16.5cm}{0.4pt}
\\
   Aulas expositivas utilizando recursos audiovisuais e quadro, além de aulas práticas utilizando computadores. Adicionalmente, serão realizadas atividades práticas individuais ou em grupo, para consolidação do conteúdo ministrado.\\
\noindent\rule{16.5cm}{0.4pt}\\
\\
%COLOCAR AVALIACAO DO PROCESSO DE ENSINO E APRENDIZAGEM A SEGUIR
\vspace{-12mm}
\begin{center}\textbf{Avaliação do Processo de Ensino e Apendizagem}\end{center}
\vspace{-5mm}
\noindent\rule{16.5cm}{0.4pt}
\\
   Avaliações escritas. Práticas baseadas em Estudos de Caso ou problemas reais.\\
\noindent\rule{16.5cm}{0.4pt}\\
\\
%PREENCHER RECURSOS NECESSARIOS A SEGUIR
\vspace{-12mm}
\begin{center}\textbf{Recursos Necessários}\end{center}
\vspace{-5mm}
\noindent\rule{16.5cm}{0.4pt}
\\
\begin{itemize} 
  \item Listas de Exercícios;
  \item Livros e apostilas;
  \item Utilização de recursos da web;
  \item Quadro branco;
  \item Marcadores para quadro branco;
  \item Sala de aula com acesso à internet, microcomputador e TV ou projetor para apresentação de slides ou material multimídia;
  \item Laboratório de microcomputadores contendo componentes de hardware e \textit{software} específicos;
\end{itemize}
\noindent\rule{16.5cm}{0.4pt}\\
\\
%PREENCHER BIBLIOGRAFIA A SEGUIR
\vspace{-12mm}
\begin{center}\textbf{Bibliografia}\end{center}
\vspace{-5mm}
\noindent\rule{16.5cm}{0.4pt}
\\
\begin{itemize} 
  \item Básica:
	\begin{enumerate}
	\item Coulouris, Dollimore e Kindberg, Sistemas distribuídos, conceito e projeto (quinta edição). ISBN 9788582600535. Bookman, 2013.
	\item Tanenbaum e van Steen, Sistemas distribuídos, princípios e paradigmas (segunda edição). ISBN 9788576051428. Pearson, 2007.
	\end{enumerate}
  \item Complementar:
	\begin{enumerate} 
	\item Mullender, S. (Editor), Distributed Systems, Addison Wesley Publishing Company; 2nd edition, ISBN: 0-2016-2427-3, 1993.\\
        \\
	Artigos cient\'ificos de peri\'odicos como: 
	\item IEEE Transactions on Parallel and Distributed Systems. IEEE Computer Society. ISSN: 1045-9219;
	\item Distributed Computing Journal. Springer. ISSN: 0178-2770 (print version), ISSN: 1432-0452 (electronic version);
	\item Journal of Parallel and Distributed Systems. Elsevier. ISSN: 0743-7315.
	\end{enumerate}
\end{itemize}
\noindent\rule{16.5cm}{0.4pt}\\
\\
