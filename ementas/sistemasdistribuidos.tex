%\section{Dados da Institui\c{c}\~ao}
%Sistemas Distribuídos

\begin{table}[h!]

\centering
% definindo o tamanho da fonte para small
% outros possíveis tamanhos: footnotesize, scriptsize
\begin{small} 
  % redefinindo o espaçamento das colunas
\setlength{\tabcolsep}{3pt} 
\begin{tabular}{|p{15cm}|}\hline


\begin{center}\textbf{Dados do Componente Curricular}\end{center}\\ \hline

%PREENCHER DADOS DA DISCIPLINA A SEGUIR
\textbf{Nome:} Sistemas Distribuídos \\ \hline
\textbf{Curso:} Tecnologia em Sistemas para Internet \\ \hline
\textbf{Período:} $6^{\circ}$ \\ \hline
\textbf{Carga Horária:} 67~h \\ \hline
\textbf{Docente Responsável:} Ruan Delgado Gomes \\ \hline


\end{tabular} 
\end{small}
\label{dadosinstituicao}
\end{table} 

\begin{table}[h!]
\centering
\begin{small} 
\setlength{\tabcolsep}{1pt} 
\begin{tabular}{|p{15cm}|}\hline

%PREENCHER A EMENTA A SEGUIR
\begin{center}\textbf{Ementa}\end{center}\\ \hline
Fundamentos de Sistemas Distribuídos. Estilos Arquiteturais para Sistemas Distribuídos. Arquitetura de Comunicação Cliente-Servidor. Comunicação: Sockets, RPC, RMI, MOM. Sistemas de arquivos distribuídos; Sistemas \textit{peer-to-peer}; Sincronização e estados globais; Transações; Replicação e tolerância a falhas; Serviços Web. \\ \hline

\end{tabular} 
\end{small}
\label{dadosinstituicao}
\end{table} 

\hspace{1cm}
\begin{table}[h!]
\centering
\begin{small} 
\setlength{\tabcolsep}{3pt} 
\begin{tabular}{|p{15cm}|}\hline

%PREENCHER OS OBJETIVOS A SEGUIR
\begin{center}\textbf{Objetivos}\end{center}\\ \hline
\begin{itemize}
\item Proporcionar o entendimento sobre as possíveis formas de estruturação dos sistemas distribuídos;
\item Conhecer e utilizar técnicas para garantir a qualidade de sistemas distribuídos;
\item Saber como resolver problemas de faltas em sistemas distribuídos.
\end{itemize}
 \\ \hline

\end{tabular} 
\end{small}
\label{dadosinstituicao}
\end{table}

\hspace{1cm}
\begin{table}[h!]
\centering
\begin{small} 
\setlength{\tabcolsep}{3pt} 
\begin{tabular}{|p{15cm}|}\hline

%PREENCHER OS CONTEUDOS PROGRAMATICOS A SEGUIR (CUIDADO PARA NAO DEIXAR A TABELA MUITO GRANDE)
\begin{center}\textbf{Conteúdo Programático}\end{center}\\ \hline
\begin{itemize}
 \item \textbf{Fundamentos de Sistemas Distribuídos:} Definição de Sistemas Distribuídos; Infraestrutura básica; Tipos de Sistemas Distribuídos.

 \item \textbf{Estilos Arquiteturais para SD:} Camadas; Baseada em Objetos; Baseada em Dados; Baseada em Eventos.

 \item \textbf{Visão Cliente-Servidor:} Requisição-Resposta; Comunicação síncrona; Comunicação assíncrona.

 \item \textbf{P2P:} Arquitetura Centralizada; Arquitetura Descentralizada.
 \item \textbf{Processos e Threads}
 \item \textbf{Comunicação:} Sockets; RPC; RMI; JMS.

 \item \textbf{Serviços:} Conceitos; Arquitetura Orientada a Serviço; Tipos de Serviços; Design de Serviços; Registro e descoberta; Web Services.

 \item \textbf{Tolerância a Faltas:} Definição; Dependabilidade; Tipos; Recuperação; Mascaramento.
 \item \textbf{Sincronização:} Cálculo de Latência; Ajuste de relógios.
\end{itemize}
 \\ \hline

\end{tabular} 
\end{small}
\label{dadosinstituicao}
\end{table}


\begin{table}[h!]
\centering
\begin{small} 
\setlength{\tabcolsep}{3pt} 
\begin{tabular}{|p{15cm}|}\hline

%COLOCAR A METODOLOGIA DE ENSINO A SEGUIR
\begin{center}\textbf{Metodologia de Ensino}\end{center}\\ \hline
   Aulas expositivas utilizando recursos audiovisuais e quadro, além de aulas práticas utilizando computadores. Adicionalmente, serão realizadas atividades práticas individuais ou em grupo, para consolidação do conteúdo ministrado.
 \\ \hline
\end{tabular} 
\end{small}
\label{dadosinstituicao}
\end{table}


\begin{table}[h!]
\centering
\begin{small} 
\setlength{\tabcolsep}{3pt} 
\begin{tabular}{|p{15cm}|}\hline

%COLOCAR AVALIACAO DO PROCESSO DE ENSINO E APRENDIZAGEM A SEGUIR
\begin{center}\textbf{Avaliação do Processo de Ensino e Apendizagem}\end{center}\\ \hline
   Avaliações escritas ao final de cada unidade. Prática baseada em Estudo de Caso ou problema real.
 \\ \hline

\end{tabular} 
\end{small}
\label{dadosinstituicao}
\end{table}

\begin{table}[h!]
\centering
\begin{small} 
 
\setlength{\tabcolsep}{3pt} 
\begin{tabular}{|p{15cm}|}\hline

%PREENCHER RECURSOS NECESSARIOS A SEGUIR
\begin{center}\textbf{Recursos Necessários}\end{center}\\ \hline
\begin{itemize} 
  \item Listas de Exercícios;
  \item Livros e apostilas;
  \item Utilização de recursos da web;
  \item Quadro branco;
  \item Marcadores para quadro branco;
  \item Sala de aula com acesso à internet, microcomputador e TV ou projetor para apresentação de slides ou material multimídia;
  \item Laboratório de microcomputadores contendo componentes de hardware e software específicos;
\end{itemize}
 \\ \hline

\end{tabular} 
\end{small}
\label{dadosinstituicao}
\end{table}


\begin{table}[h!]
\centering
\begin{small} 
\setlength{\tabcolsep}{3pt} 
\begin{tabular}{|p{15cm}|}\hline

%PREENCHER BIBLIOGRAFIA A SEGUIR
\begin{center}\textbf{Bibliografia}\end{center}\\ \hline
\begin{itemize} 
  \item Básica;
  \item Complementar;
\end{itemize}
 \\ \hline

\end{tabular} 
\end{small}
\label{dadosinstituicao}
\end{table}
