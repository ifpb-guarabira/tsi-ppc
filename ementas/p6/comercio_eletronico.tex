\paragraph{Comércio Eletrônico} \

% Dados do Componente Curricular

\begin{snugshade}\begin{center}\textbf{
    Dados do Componente Curricular
}\end{center}\end{snugshade}

\noindent \textbf{Nome:}                Comércio Eletrônico
\\        \textbf{Curso:}               Tecnologia em Sistemas para Internet
\\        \textbf{Período:}             \unit{6}{\degree}
\\        \textbf{Carga Horária:}       \unit{33}{\hour}
\\        \textbf{Docente Responsável:} Moisés Guimarães de Medeiros

% Ementa

\begin{snugshade}\begin{center}\textbf{
    Ementa
\vphantom{q}}\end{center}\end{snugshade}

\noindent
Visão geral do comércio eletrônico e suas principais formas (B2B, B2C, C2B, C2C); modelos de negócios na internet e apresentação da dinâmica dos mercados de comércio eletrônico; modelos e plataformas de comércio eletrônico e seus principais componentes; identificação de requisitos específicos para a implementação de sistemas de comércio eletrônico; formas de marketing na internet; identificação e geração de oportunidades de negócios na web; ferramentas para comércio eletrônico.

% Objetivos

\begin{snugshade}\begin{center}\textbf{
    Objetivos
}\end{center}\end{snugshade}

\begin{itemize}

\item Compreender os conceitos, categorias, modelos de negócio e de receita do comércio eletrônico;

\item Compreender os aspectos distintos do comércio eletrônico \textit{mobile};

\item Projetar e construir aplicações para o comércio eletrônico;

\end{itemize}

% Conteúdo Programático

\begin{snugshade}\begin{center}\textbf{
    Conteúdo Programático
}\end{center}\end{snugshade}

\begin{itemize}

\item \textbf{Comércio eletrônico e a Internet:}
    Comércio eletrônico hoje;
    Características do comércio eletrônico;
    Mercados e mercadorias digitais.

\item \textbf{Negócios e tecnologia:}
    Categorias do comércio eletrônico;
    Modelos de negócios do comércio eletrônico;
    Modelos de receita do comércio eletrônico;
    Web 2.0;
    Marketing do comércio eletrônico;
    Comércio eletrônico de empresa para empresa.

\item \textbf{m-commerce:}
    Serviços e aplicações do comércio eletrônico no mundo \textit{mobile}.

\item \textbf{Construindo um site de comércio eletrônico:}
    Elementos da construção do site;
    Objetivos empresariais, funcionalidade do sistema e requisitos de informação;
    Construindo o site: \textit{in-house} vs \textit{outsourcing}.

\end{itemize}

% Metodologia, Avaliação e Recursos

\begin{snugshade}\begin{center}\textbf{
    Metodologia de Ensino
}\end{center}\end{snugshade}

\noindent
Aulas expositivas utilizando recursos audiovisuais e quadro, além de aulas práticas utilizando computadores. Adicionalmente, serão realizadas atividades práticas individuais ou em grupo, para consolidação do conteúdo ministrado.

\begin{snugshade}\begin{center}\textbf{
    Avaliação do Processo de Ensino e Aprendizagem
}\end{center}\end{snugshade}

\noindent
Avaliações escritas. Avaliações práticas envolvendo a resolução de problemas computacionais.

\begin{snugshade}\begin{center}\textbf{
    Recursos Necessários
    \vphantom{q} % TODO: corrigir o depth da linha sem esta gambiarra.
}\end{center}\end{snugshade}

\begin{itemize}
  \item Listas de Exercícios;
  \item Livros e apostilas;
  \item Utilização de recursos da web;
  \item Quadro branco;
  \item Marcadores para quadro branco;
  \item Sala de aula com acesso à internet, microcomputador e TV ou projetor para apresentação de slides ou material multimídia;
  \item Laboratório de microcomputadores contendo componentes de hardware e software específicos;
\end{itemize}


% Bibliografia

\begin{snugshade}\begin{center}\textbf{
    Bibliografia
}\end{center}\end{snugshade}

\begin{itemize}

\item Básica:
    \begin{enumerate}

    \item LAUDON, Kenneth C.; LAUDON, Jane P.
          Sistemas de Informações Gerenciais.
          São Paulo: Pearson Prentice Hall, 9a edição, 2010.

    \item ALBERTIN, Alberto Luiz.
          Comércio Eletrônico.
          Atlas Editora, 6a edição, 2010.

    \end{enumerate}

\item Complementar:
    \begin{enumerate}

    \item STALLINGS, W.
          Criptografia e segurança de redes.
          Prentice-Hall, 4a edição, 2007.

    \item LEDFORD, Jerri L.
          Google AdSense for Dummies.
          John Wiley Consumer, 1a edição, 2008.

    \end{enumerate}

\end{itemize}
