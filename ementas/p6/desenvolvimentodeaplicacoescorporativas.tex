
% Dados do Componente Curricular

\begin{snugshade}\begin{center}\textbf{
	Dados do Componente Curricular
}\end{center}\end{snugshade}

\noindent 	\textbf{Nome:} Desenvolvimento de Aplicações Corporativas
\\ 			\textbf{Curso:} Tecnologia em Sistemas para Internet
\\ 			\textbf{Período:} \unit{6}{\degree}
\\ 			\textbf{Carga Horária:} \unit{67}{\hour}
\\ 			\textbf{Docente Responsável:} José de Sousa Barros 

% Ementa

\begin{snugshade}\begin{center}\textbf{
    Ementa
\vphantom{q}}\end{center}\end{snugshade}

\noindent
Introdução aos sistemas corporativos. Componentes de aplicações corporativas.  Utilização de uma plataforma de programação para o desenvolvimento de aplicações corporativas. Mapeamento objeto-relacional com APIs de Persistência. Comportamento transacional dos componentes de aplicações corporativas. Segurança em sistemas corporativos.


% Objetivos

\begin{snugshade}\begin{center}\textbf{
    Objetivos
}\end{center}\end{snugshade}


\begin{itemize}

\item Compreender os conceitos fundamentais do desenvolvimento de aplicações corporativas;
\item Utilizar uma plataforma de desenvolvimento de aplicações corporativas;
\item Construir sistemas corporativos com uma arquitetura baseada em componentes.

\end{itemize}

% Conteúdo Programático

\begin{snugshade}\begin{center}\textbf{
    Conteúdo Programático
}\end{center}\end{snugshade}

\begin{itemize}

  \item \textbf{Introdução:} Introdução do desenvolvimento de aplicações corporativas;	Visão geral de uma arquitetura de aplicação corporativa baseada em componentes. 
  
  \item \textbf{Gerenciamento da camada de persistência de objetos:} Conceitos sobre persistência de objetos: O que é persistência de objetos, Persistência Transparente, Criação e manipulação de objetos persistentes, Alcançabilidade da persistência, Transação e ciclo de vida de objetos persistentes, O Gerenciador da Persistência, Padrões e Frameworks de Persistência; Persistência de Objetos com Mapeamento Objeto/Relacional (MOR): Conceitos da persistência de objetos com mapeamento objeto/relacional, Padrões e frameworks de persistência com MOR, Mapeamento de classes e atributos, Mapeamento de relacionamentos unidirecionais e bidirecionais, Mapeamento de herança, Mapeamentos avançados, Linguagem de consulta, Gerenciamento de transações.
 
 
  \item \textbf{Gerenciamento da camada de negócios:} Componentes de controle da camada de lógica de negócio: Tipos de componentes, Interfaces de acesso e Ciclo de vida; Injeção de instâncias de componentes de negócio; Integração com aplicações cliente/servidor; Acesso remoto a componentes de negócio; Interceptação de chamadas a componentes de negócio; Controle de Acesso / Segurança em componentes de negócio; Agendamento de serviços; Invocação de chamadas assíncronas; Teste de componentes na arquitetura integrada.

\end{itemize}

\begin{snugshade}\begin{center}\textbf{
    Metodologia de Ensino
}\end{center}\end{snugshade} 

\noindent
   Aulas expositivas utilizando recursos audiovisuais e quadro, além de aulas práticas utilizando computadores. Adicionalmente, serão realizadas atividades práticas individuais ou em grupo, para consolidação do conteúdo ministrado.

\begin{snugshade}\begin{center}\textbf{
    Avaliação do Processo de Ensino e Aprendizagem
}\end{center}\end{snugshade}   

\noindent
  Avaliações escritas. Práticas baseadas em Estudos de Caso ou problemas reais.

\begin{snugshade}\begin{center}\textbf{
    Recursos Necessários
    \vphantom{q} % TODO: corrigir o depth da linha sem esta gambiarra.
}\end{center}\end{snugshade}

\begin{itemize} 
  \item Listas de Exercícios;
  \item Livros e apostilas;
  \item Utilização de recursos da web;
  \item Quadro branco;
  \item Marcadores para quadro branco;
  \item Sala de aula com acesso à internet, microcomputador e TV ou projetor para apresentação de slides ou material multimídia;
  \item Laboratório de microcomputadores contendo componentes de hardware e \textit{software} específicos;
\end{itemize}

% Bibliografia

\begin{snugshade}\begin{center}\textbf{
    Bibliografia
}\end{center}\end{snugshade}

\begin{itemize} 
  \item Básica:
	\begin{enumerate}
	\item 	GONÇALVES, A. \textbf{Beginning Java EE 7.} Apress, 2013;
	\item 	GUPTA, A. \textbf{Java EE 7 Essentials.} O’Reilly, 2013;
	\item 	BURKE, B.\textbf{Enterprise Javabeans 3.0.} Pearson, 2007. 
	\end{enumerate}
  \item Complementar:
	\begin{enumerate} 
	\item 	GONÇALVES, A. \textbf{Introdução à plataforma Java EE 6 com Glassfish 3.} Ciência Moderna, 2011;
	\item 	DEREK, L. \textbf{EJB3 em Ação.} Alta Books, 2008.
	\end{enumerate}
\end{itemize}
