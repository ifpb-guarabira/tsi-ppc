\paragraph{Desenvolvimento de Aplicações Corporativas}

%PREENCHER DADOS DA DISCIPLINA A SEGUIR
%\vspace{-12mm}
\begin{center}\textbf{Dados do Componente Curricular}\end{center}
\vspace{-5mm}
\noindent\rule{16.5cm}{0.4pt}
\\
\textbf{Nome:} Desenvolvimento de Aplicações Corporativas
\\
\textbf{Curso:} Tecnologia em Sistemas para Internet
\\ 
\textbf{Período:} $6^{\circ}$ 
\\
\textbf{Carga Horária:} 67~h
\\ 
\textbf{Docente Responsável:} José de Sousa Barros 
\\ 
\noindent\rule{16.5cm}{0.4pt}\\
\\
%PREENCHER A EMENTA A SEGUIR
\vspace{-12mm}
\begin{center}\textbf{Ementa}\end{center}
\vspace{-5mm}
\noindent\rule{16.5cm}{0.4pt}
\\
Introdução aos sistemas corporativos. Componentes de aplicações corporativas.  Utilização de uma plataforma de programação para o desenvolvimento de aplicações corporativas. Mapeamento objeto-relacional com APIs de Persistência. Comportamento transacional dos componentes de aplicações corporativas. Segurança em sistemas corporativos.\\

\noindent\rule{16.5cm}{0.4pt}\\
\\
%PREENCHER OS OBJETIVOS A SEGUIR
\vspace{-12mm}
\begin{center}\textbf{Objetivos}\end{center}
\vspace{-5mm}
\noindent\rule{16.5cm}{0.4pt}
\\
\begin{itemize}
\item Compreender os conceitos fundamentais do desenvolvimento de aplicações corporativas;
\item Utilizar uma plataforma de desenvolvimento de aplicações corporativas;
\item Construir sistemas corporativos com uma arquitetura baseada em componentes.\\


\end{itemize}
\noindent\rule{16.5cm}{0.4pt}\\
\\
%PREENCHER OS CONTEUDOS PROGRAMATICOS A SEGUIR (CUIDADO PARA NAO DEIXAR A TABELA MUITO GRANDE)
\vspace{-12mm}
\begin{center}\textbf{Conteúdo Programático}\end{center}
\vspace{-5mm}
\noindent\rule{16.5cm}{0.4pt}
\\
\begin{itemize}
 \item \textbf{Introdução:} Introdução do desenvolvimento de aplicações corporativas;	Visão geral de uma arquitetura de aplicação corporativa baseada em componentes. 
 
 \item \textbf{Gerenciamento da camada de persistência de objetos:} Conceitos sobre persistência de objetos: O que é persistência de objetos, Persistência Transparente, Criação e manipulação de objetos persistentes, Alcançabilidade da persistência, Transação e ciclo de vida de objetos persistentes, O Gerenciador da Persistência, Padrões e Frameworks de Persistência; Persistência de Objetos com Mapeamento Objeto/Relacional (MOR): Conceitos da persistência de objetos com mapeamento objeto/relacional, Padrões e frameworks de persistência com MOR, Mapeamento de classes e atributos, Mapeamento de relacionamentos unidirecionais e bidirecionais, Mapeamento de herança, Mapeamentos avançados, Linguagem de consulta, Gerenciamento de transações.


 \item \textbf{Gerenciamento da camada de negócios:} Componentes de controle da camada de lógica de negócio: Tipos de componentes, Interfaces de acesso e Ciclo de vida; Injeção de instâncias de componentes de negócio; Integração com aplicações cliente/servidor; Acesso remoto a componentes de negócio; Interceptação de chamadas a componentes de negócio; Controle de Acesso / Segurança em componentes de negócio; Agendamento de serviços; Invocação de chamadas assíncronas; Teste de componentes na arquitetura integrada.

 
\end{itemize}
\noindent\rule{16.5cm}{0.4pt}\\
\\
%COLOCAR A METODOLOGIA DE ENSINO A SEGUIR
\vspace{-12mm}
\begin{center}\textbf{Metodologia de Ensino}\end{center} 
\vspace{-5mm}
\noindent\rule{16.5cm}{0.4pt}
\\
   Aulas expositivas utilizando recursos audiovisuais e quadro, além de aulas práticas utilizando computadores. Adicionalmente, serão realizadas atividades práticas individuais ou em grupo, para consolidação do conteúdo ministrado.\\
\noindent\rule{16.5cm}{0.4pt}\\
\\
%COLOCAR AVALIACAO DO PROCESSO DE ENSINO E APRENDIZAGEM A SEGUIR
\vspace{-12mm}
\begin{center}\textbf{Avaliação do Processo de Ensino e Apendizagem}\end{center}
\vspace{-5mm}
\noindent\rule{16.5cm}{0.4pt}
\\
   Avaliações escritas. Práticas baseadas em Estudos de Caso ou problemas reais.\\
\noindent\rule{16.5cm}{0.4pt}\\
\\
%PREENCHER RECURSOS NECESSARIOS A SEGUIR
\vspace{-12mm}
\begin{center}\textbf{Recursos Necessários}\end{center}
\vspace{-5mm}
\noindent\rule{16.5cm}{0.4pt}
\\
\begin{itemize} 
  \item Listas de Exercícios;
  \item Livros e apostilas;
  \item Utilização de recursos da web;
  \item Quadro branco;
  \item Marcadores para quadro branco;
  \item Sala de aula com acesso à internet, microcomputador e TV ou projetor para apresentação de slides ou material multimídia;
  \item Laboratório de microcomputadores contendo componentes de hardware e \textit{software} específicos;
\end{itemize}
\noindent\rule{16.5cm}{0.4pt}\\
\\
%PREENCHER BIBLIOGRAFIA A SEGUIR
\vspace{-12mm}
\begin{center}\textbf{Bibliografia}\end{center}
\vspace{-5mm}
\noindent\rule{16.5cm}{0.4pt}
\\
\begin{itemize} 
  \item Básica:
	\begin{enumerate}
  	\item 	GONÇALVES, A. \textbf{Beginning Java EE 7.} Apress, 2013;
	\item 	GUPTA, A. \textbf{Java EE 7 Essentials.} O’Reilly, 2013;
	\item 	BURKE, B.\textbf{Enterprise Javabeans 3.0.} Pearson, 2007. 
	\end{enumerate}
    
  \item Complementar:
	\begin{enumerate}
  	\item 	GONÇALVES, A. \textbf{Introdução à plataforma Java EE 6 com Glassfish 3.} Ciência Moderna, 2011;
	\item 	DEREK, L. \textbf{EJB3 em Ação.} Alta Books, 2008.
	\end{enumerate}
\end{itemize}
\noindent\rule{16.5cm}{0.4pt}\\
\\