
% Dados do Componente Curricular

\begin{snugshade}\begin{center}\textbf{
	Dados do Componente Curricular
}\end{center}\end{snugshade}

\noindent	\textbf{Nome:} Probabilidade e Estatística
\\ 			\textbf{Curso:} Tecnologia em Sistemas para Internet
\\ 			\textbf{Período:} \unit{2}{\degree}
\\ 			\textbf{Carga Horária:} \unit{83}{\hour}
\\ 			\textbf{Docente Responsável:} Cícero Demétrio Vieira de Barros

% Ementa

\begin{snugshade}\begin{center}\textbf{
    Ementa
\vphantom{q}}\end{center}\end{snugshade}

\noindent
Análise Estatística de Dados. Espaço Amostral. Probabilidade e seus teoremas. Probabilidade Condicional e Independência de Eventos. Teorema de Bayes. Distribuições de Variáveis Aleatórias Discretas e Contínuas Unidimensionais. Valor Esperado, Variância e Desvio Padrão. Modelos Probabilísticos Discretos: Uniforme, Bernoulli, Binomial e Poisson. Modelos Probabilísticos Contínuos: Uniforme e Normal. Estimação. Testes de Hipóteses. Tomada de decisão utilizando Redes Bayesianas.
% Objetivos

\begin{snugshade}\begin{center}\textbf{
    Objetivos
}\end{center}\end{snugshade}

\begin{itemize}

\item Utilizar métodos e técnicas estatísticas que possibilitem sumariar, calcular e analisar dados para uso na tomada de decisão auxiliada por computador;
\item Estudar resultados de experimentos aleatórios de maneira a modelar a previsão desses resultados e a probabilidade com que se pode confiar nas probabilidades obtidas;
\item Conhecer a representação gráfica, as medidas de posição e de dispersão;
\item Apresentar os principais modelos probabilísticos discretos e contínuos;
\item Conhecer a Estatística Inferencial e avaliar o tamanho do erro ao fazer generalizações;
\item Modelar a tomada de decisão por computador utilizando as redes de Bayes.

\end{itemize} 

% Conteúdo Programático

%\begin{snugshade}\begin{center}\textbf{
 %   Conteúdo Programático
%}\end{center}\end{snugshade}

%\begin{itemize}

% \item \textbf{Estatística Descritiva:} Introdução à Estatística; Importância da Estatística; Grandes áreas da Estatística; Fases do método estatístico.
 
% \item \textbf{Distribuição de Frequência:} Elementos de uma distribuição de frequência; Amplitude total; Limites de classe; Amplitude do intervalo de classe; Ponto médio da classe; Frequência absoluta, relativa e acumulada; Regras gerais para a elaboração de uma distribuição de frequência; Gráficos representativos de uma distribuição de frequência: Histograma e gráfico de coluna.
 
% \item \textbf{Medidas de Posição:} Introdução; Média aritmética simples e ponderada e suas propriedades; Moda: Dados agrupados e não  agrupados em classes; Mediana: Dados agrupados e não agrupados em classes.
 
% \item \textbf{Medidas de Dispersão:} Variância; Desvio padrão; Coeficiente de variação.
 
% \item \textbf{Probabilidade:} Experimentos aleatórios, espaço amostral e eventos; Definição clássica da Probabilidade; Frequência relativa; Tipos de eventos; Axiomas de Probabilidade; Probabilidade condicional e independência de eventos; Teorema de Bayes, do Produto e da Probabilidade Total.
 
% \item \textbf{Variáveis Aleatórias:} Conceito de variável aleatória; Distribuição de probabilidade, função densidade de probabilidade, esperança matemática, variância, desvio padrão e suas propriedades para variáveis aleatórias discretas e contínuas.
 
% \item \textbf{Distribuições Discretas:} Bernoulli, Binomial e Poisson.
 
 %\item \textbf{Distribuição Contínuas:} Uniforme; Normal Padrão (propriedades e distribuição); Aproximação Binomial da Distribuição Normal.
 
 %\item \textbf{Inferência Estatística:} População e amostra; Estatísticas e parâmetros; Distribuições amostrais.
 
 %\item \textbf{Estimação:} pontual e por intervalo.
 
% \item \textbf{Testes de Hipóteses:} Principais conceitos; Testes de hipóteses para média de populações normais com variância.
 
 %\item \textbf{Redes Bayesianas:} Cálculo de probabilidades; Aplicando a regra de Bayes; Inferência em Redes Bayesianas; Aplicações em inteligência artificial.
 
 %\end{itemize}

%\begin{snugshade}\begin{center}\textbf{
 %   Metodologia de Ensino
%}\end{center}\end{snugshade}

%\noindent
 % Aulas expositivas utilizando recursos audiovisuais e quadro.

%\begin{snugshade}\begin{center}\textbf{
 %   Avaliação do Processo de Ensino e Aprendizagem
%}\end{center}\end{snugshade}

%\noindent
 %  Avaliações escritas ao final de cada unidade.
   
%\begin{snugshade}\begin{center}\textbf{
 %   Recursos Necessários
  %  \vphantom{q} % TODO: corrigir o depth da linha sem esta gambiarra.
%}\end{center}\end{snugshade}

%\begin{itemize} 
 %  \item Listas de Exercícios;
 %  \item Livros e apostilas;
 %  \item Utilização de recursos da web;
 %  \item Quadro branco;
 %  \item Marcadores para quadro branco;
 %  \item Sala de aula com acesso à Internet, microcomputador e TV ou projetor para apresentação de slides ou material multimídia;
%\end{itemize}

% Bibliografia

\begin{snugshade}\begin{center}\textbf{
    Bibliografia
}\end{center}\end{snugshade}

\begin{itemize} 
  \item Básica:
	\begin{enumerate}
	\item BARBETTA, P.A.; REIS, M.M. e BORNIA, A.C. Estatística para cursos de engenharia e informática. Editora Atlas, São Paulo, 2004. 410 p.
	\item BUSSAB, W. O. MORETTIN, P. A.Estatística Básica.  5 ed.  São Paulo: Saraiva, 2002.
	\end{enumerate}
  \item Complementar:
	\begin{enumerate} 
	\item MEYER, P.L. Probabilidade: Aplicações à Estatística. 2 ed.  Rio de Janeiro: LTC – Livros Técnicos e Científicos, 2000.
	\item FONSECA, J.S. e Martins, G.A. Curso de Estatística. São Paulo: Atlas, 1993.
	\end{enumerate}
\end{itemize}
