
% Dados do Componente Curricular

\begin{snugshade}\begin{center}\textbf{
    Dados do Componente Curricular
}\end{center}\end{snugshade}

\noindent \textbf{Nome:}                Portugu\^es Instrumental
\\        \textbf{Curso:}               Tecnologia em Sistemas para Internet
\\        \textbf{Período:}             \unit{2}{\degree}
\\        \textbf{Carga Horária:}       \unit{67}{\hour}
\\        \textbf{Docente Responsável:} Erivan Lopes Tome Junior

% Ementa

\begin{snugshade}\begin{center}\textbf{
    Ementa
\vphantom{q}}\end{center}\end{snugshade}

\noindent
Níveis e Estratégias de leitura; Conceitos linguísticos: Norma culta, Variedades linbguísticas, Níveis de linguagem oral e escrita; Gêneros e tipos/sequências textuais. Noções metodológicas de leitura e interpretação de textos. Habilidades básicas de produção textual. Noções linguístico-gramaticais aplicada a textos de natureza diversa, inclusive, textos técnicos e científicos. Argumentação oral e escrita, a partir de diversas situações sociocomunicativas. Elementos/Fatores da Textualidade; Aspectos semânticos, pragmáticos e sintáticos aplicados ao texto. Redação oficial; Gêneros da correspondência oficial: Aviso, Ofício e Memorando. Gêneros de natureza diversa: Artigo científico, Relatório, Requerimento, Laudo técnico, Artigo de opinião, Resumo, Resenha crítica.
% Objetivos

\begin{snugshade}\begin{center}\textbf{
    Objetivos
}\end{center}\end{snugshade}

\begin{itemize}

\item Proporcionar ao aluno a aquisição de conhecimentos sobre o funcionamento da linguagem e comunicação para a estruturação e elaboração de textos diversos, considerando o perfil do egresso.;
\item Conceituar e estabelecer as diferenças que marcam a língua escrita e a falada;
\item Reconhecer os diversos registros linguísticos (formal, coloquial, informal, familiar, entre outros), com ênfase na performance formal e sua contribuição para o perfil do egresso;
\item Reconhecer os fatores que definem um texto;
\item Contribuir para o desenvolvimento de uma consciência objetiva e crítica para a compreensão e a produção de textos;
\item Desenvolver habilidades para leitura – interpretação de textos – e escrita;
\item Tornar o aluno apto a reconhecer os gêneros e tipos/sequências textuais;
\item Tornar o aluno apto a produzir textos de diversos gêneros;
\item Reconhecer a argumentatividade de gêneros diversos;
\item Produzir construções argumentativas em diversas situações sociocomunicativas;
\item Entender o contexto de produção da redação oficial;
\item Produzir gêneros da correspondência oficial;
\item Produzir com proficiência gêneros acadêmico-científicos: Artigo científico, Relatório, Resumo, Resenha crítica.
\item Produzir com proficiência o Artigo de opinião, o Laudo técnico, o requerimento.

\end{itemize} 

% Conteúdo Programático

%\begin{snugshade}\begin{center}\textbf{
 %   Conteúdo Programático
%}\end{center}\end{snugshade}

%\begin{itemize}
% \item \textbf{Elementos da teoria da comunicação:} Linguagem e comunicação; Níveis da linguagem; Funções da linguagem.

% \item \textbf{Gêneros e tipos textuais:} Tipologia textual: o texto e seus formatos físicos e eletrônicos; Gêneros textuais diversos; Estrutura e Produção de gêneros diversos: Artigo de opinião, Laudo técnico, Requerimento.

% \item \textbf{Noções metodológicas de leitura e interpretação de textos:} Mecanismo de coerência e coesão textuais; Habilidades básicas de produção textual; Noções linguístico-gramaticais aplicadas a textos de natureza diversa; Elementos/fatores da textualidade; Aspectos semânticos, sintáticos aplicados ao texto.

% \item \textbf{Gêneros acadêmico-científicos:} Estrutura e produção do Artigo científico, Relatório, Resumo, Resenha crítica.

% \item \textbf{Redação oficial:} Estrutura e produção dos gêneros oficiais: Aviso, Ofício e Memorando.

%\end{itemize}

% Metodologia, Avaliação e Recursos


\begin{snugshade}\begin{center}\textbf{
    Metodologia de Ensino
}\end{center}\end{snugshade}

\noindent
 As aulas serão desenvolvidas por meio de metodologia participativa, com a utilização de técnicas didáticas, como: aulas expositivas, debates, seminários, trabalhos de pesquisa - individualmente e em grupos.

\begin{snugshade}\begin{center}\textbf{
    Avaliação do Processo de Ensino e Aprendizagem
}\end{center}\end{snugshade}

\noindent
\begin{itemize}
	\item Observação geral do aluno como parte integrante e atuante do processo ensino-aprendizagem.
	\item Apresentação de seminários e outras atividades discursivas;
	\item Atividades escritas coletivas com o objetivo de aprofundamento do conteúdo;
	\item Avaliação oral e escrita;
	\item Avaliação contínua.
\end{itemize}

\begin{snugshade}\begin{center}\textbf{
    Recursos Necessários
    \vphantom{q} % TODO: corrigir o depth da linha sem esta gambiarra.
}\end{center}\end{snugshade}

\begin{itemize}
  \item Quadro branco;
  \item Marcadores para quadro branco;
  \item Projetor de dados multimídia;
  \item Espaços adequados para aulas extras;
  \item Mini auditório;
  \item Outros espaços circunstanciais.
\end{itemize}


% Bibliografia

\begin{snugshade}\begin{center}\textbf{
    Bibliografia
}\end{center}\end{snugshade}

\begin{itemize} 
    \item Básica:
	\begin{enumerate}
		\item SAVIOLI, F. P.; FIORIN, J. L.  Para entender o texto: leitura e redação. Ática, 1990;  
		\item SAVIOLI, F. P.; FIORIN, J. L. Lições de texto: leitura e redação. São Paulo: Ática, 1996. 
		\item MARCUSCHI, L. A.; XAVIER, A. C. Hipertexto e gêneros digitais: novas formas de construção de sentido. Lucerna, 2004;
	\end{enumerate}
  \item Complementar:
	\begin{enumerate} 
		\item SAUTCHUK I. Produção dialógica do texto escrito. Martins Fontes, 2003.
		\item TERRA, E.; NICOLA, J. Práticas de linguagem \& Produção de textos. Scipione, 2001.
		\item VAL, Maria da Graça Costa. Redação e textualidade. 3ª ed. São Paulo: Martins Editora, 2006
		\item LIMA, Antônio Oliveira. Manual de redação oficial. 3ª Ed. Rio de Janeiro: Campos Editora, 2009.
		\item INFANTE, U. Do texto ao texto: curso prático de leitura e redação. Scipione, 1998; 
		\item CARNEIRO, A. D. Redação em construção: a escritura do texto. Moderna, 2001;
		\item ANDRADE, M. M.; HENRIQUES, A. Língua portuguesa: noções básicas para cursos superiores. Atlas, 2004;
		\item BASTOS, L. K. A produção escrita e a gramática. Martins Fontes, 2003;
		\item BECHARA, E. O que muda com o novo acordo ortográfico. Lucerna, 2008.
		\item COSTA, José Maria da. Manual de redação jurídica. 5ª ed. São Paulo: Migalhas, 2012.
	\end{enumerate}
\end{itemize}
