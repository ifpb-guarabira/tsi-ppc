
% Dados do Componente Curricular

\begin{snugshade}\begin{center}\textbf{
	Dados do Componente Curricular
}\end{center}\end{snugshade}

\noindent 	\textbf{Nome:} Protocolos de Interconexão de Redes
\\ 			\textbf{Curso:} Tecnologia em Sistemas para Internet
\\ 			\textbf{Período:} \unit{2}{\degree}
\\ 			\textbf{Carga Horária:} \unit{67}{\hour}
\\ 			\textbf{Docente Responsável:} Erick Augusto Gomes de Melo

% Ementa

\begin{snugshade}\begin{center}\textbf{
    Ementa
\vphantom{q}}\end{center}\end{snugshade}

\noindent
Apresentação da natureza dos serviços prestados pela Internet; Apresentação de duas categorias de aplicações: paradigma Cliente-Servidor e \textit{Peer-to-Peer}; Discussão sobre o conceito do paradigma Cliente-Servidor e como ele fornece serviços para os usuários da Internet; Descrição de aplicações predefinidas ou padrões com base no paradigma Cliente-Servidor; Discussão sobre o conceito do paradigma \textit{Peer-to-Peer}; Apresentação de alguns protocolos \textit{Peer-to-Peer} e de aplicativos populares que utilizam tais protocolos; Protocolos multimídia.
% Objetivos

\begin{snugshade}\begin{center}\textbf{
    Objetivos
}\end{center}\end{snugshade}


\begin{itemize}

\item Compreender os serviços prestados pela camada de aplicação e como as outras quatro camadas do modelo TCP/IP dão suporte a esses serviços;
\item Compreender a natureza dos serviços prestados pela Internet;
\item Conhecer as categorias de aplicações utilizadas na Internet;
\item Discutir os conceitos dos paradigmas Cliente-Servidor e \textit{Peer-to-Peer};
\item Apresentar aplicações que fazem uso dos paradigmas Cliente-Servidor e \textit{Peer-to-Peer};
\item Conhecer os principais protocolos utilizados em aplicações multimídia.

\end{itemize} 

% Conteúdo Programático

%\begin{snugshade}\begin{center}\textbf{
 %   Conteúdo Programático
%}\end{center}\end{snugshade}

%\begin{itemize}

% \item \textbf{Introdução:} Fornecendo Serviços; Paradigmas da camada de aplicação;
% \item \textbf{Paradigma Cliente-Servidor:} Interface de programação de aplicativos; Usando serviços da camada de transporte;
% \item \textbf{Aplicações Cliente-Servidor padronizadas:} \textit{Word Wide Web} e HTTP; FTP; Correio eletrônico; Telnet; \textit{Secure shell}; Sistemas de nomes de domínio;
% \item \textbf{Paradigmas \textit{Peer-to-Peer}:} Redes P2P; Tabela de \textit{hash} distribuída; Chord; Pastry; Kademlia; A rede BitTorrent;
% \item \textbf{Programação usando a interface \textit{socket}:} Interface socket em C;
% \item \textbf{Multimídia na Internet};
% \item \textbf{Protocolos Interativos em tempo real:} Justificativa para novos protocolos; RTP; RTCP; Protocolos de iniciação de sessão; H.323; SCTP.
 
%\end{itemize}

%\begin{snugshade}\begin{center}\textbf{
 %   Metodologia de Ensino
%}\end{center}\end{snugshade}

%\noindent
 %  As aulas serão desenvolvidas por meio de metodologia participativa, com a utilização de técnicas didáticas, como: aulas expositivas, debates, seminários, trabalhos de pesquisa, práticas em laboratório (individualmente e em grupos).

%\begin{snugshade}\begin{center}\textbf{
 %   Avaliação do Processo de Ensino e Aprendizagem
%}\end{center}\end{snugshade}

%\noindent
 % Avaliações escritas, apresentação de seminários, atividades em grupos e elaboração de projetos.
   
%\begin{snugshade}\begin{center}\textbf{
 %   Recursos Necessários
  %  \vphantom{q} % TODO: corrigir o depth da linha sem esta gambiarra.
%}\end{center}\end{snugshade}

%\begin{itemize} 
%	\item Quadro branco;
%	\item Marcadores para quadro branco;
%	\item Projetor de dados multimídia;
%	\item Laboratório de informática.
%\end{itemize}

% Bibliografia

\begin{snugshade}\begin{center}\textbf{
    Bibliografia
}\end{center}\end{snugshade}

\begin{itemize} 
  \item Básica:
	\begin{enumerate}
	\item FOROUZAN, Behrouz A.; MOSHARRAF, Firouz. Redes de Computadores - Uma Abordagem Top-Down - 2012. 1 ed. Editora Mcgraw Hill, 2012.
	\item KUROSE, J. F., ROSSA, K. W. Redes de computadores e a internet. 5 ed. Editora Pearson. 2010.
	\item TANENBAUM, A. S., WETHERALL, D. Redes de Computadores. 5 ed. Editora Pearson. 2011. 
	\end{enumerate}
  \item Complementar:
	\begin{enumerate} 
	\item COMER, D. E. Redes de computadores e internet. 4 ed. Editora Artmed. 2007.
	\item LOWE,Doug. Redes de Computadores Para Leigos. 9ª Edição. Editora Altabooks.
	\end{enumerate}
\end{itemize}
