\paragraph{Estruturas de Dados I} \

% Dados do Componente Curricular

\begin{snugshade}\begin{center}\textbf{
	Dados do Componente Curricular
}\end{center}\end{snugshade}

\noindent 	\textbf{Nome:} Estruturas de Dados I
\\ 			\textbf{Curso:} Tecnologia em Sistemas para Internet
\\ 			\textbf{Período:} \unit{2}{\degree}
\\ 			\textbf{Carga Horária:} \unit{67}{\hour}
\\ 			\textbf{Docente Responsável:} Otacílio de Araújo Ramos Neto

% Ementa

\begin{snugshade}\begin{center}\textbf{
    Ementa
\vphantom{q}}\end{center}\end{snugshade}

\noindent
Conceitos básicos, crescimento de funções e recorrências; Algoritmos de ordenação e busca; Estruturas de dados elementares; Árvores de busca binária.
% Objetivos

\begin{snugshade}\begin{center}\textbf{
    Objetivos
}\end{center}\end{snugshade}

\begin{itemize}

\item Apresentar os conceitos básicos para criação e análise de algoritmos;
\item Apresentar os algoritmos básicos de ordenação e busca;
\item Capacitar os alunos a utilizarem as estruturas de dados elementares em problemas reais;
\item Apresentar aos alunos as árvores de busca binária e capacitá-los no seu uso.

\end{itemize} 

% Conteúdo Programático

\begin{snugshade}\begin{center}\textbf{
    Conteúdo Programático
}\end{center}\end{snugshade}

\begin{itemize}

 \item \textbf{Conceitos básicos:} Análise e projeto de algoritmos; Notação assintótica; O Método da Substituição, Método da Árvore de Recursão e Método Mestre.
 
 \item \textbf{Algoritmos de ordenação e busca:} Ordenação por inserção, Heapsort, Quicksort e ordenação em tempo linear; Busca sequencial e busca binária.
 
 \item \textbf{Estruturas de dados elementares:} Implementações de ponteiros e objetos; Pilhas, filas e listas ligadas.
 
 \item \textbf{Árvores de pesquisa binária:} Conceitos fundamentais de árvores de pesquisa binária; Algoritmos de inserção, remoção e busca; Impressão \textit{In-Order}, \textit{Post-Order} e \textit{Pre-Order} .
 
\end{itemize}

\begin{snugshade}\begin{center}\textbf{
    Metodologia de Ensino
}\end{center}\end{snugshade}

\noindent
   Aulas expositivas utilizando recursos audiovisuais e quadro, além de aulas práticas utilizando computadores. Adicionalmente, serão realizadas atividades práticas individuais ou em grupo, para consolidação do conteúdo ministrado.

\begin{snugshade}\begin{center}\textbf{
    Avaliação do Processo de Ensino e Aprendizagem
}\end{center}\end{snugshade}

\noindent
   Avaliações escritas. Avalia\c{c}\~oes pr\'aticas envolvendo a resolu\c{c}\~ao de problemas computacionais.
   
\begin{snugshade}\begin{center}\textbf{
    Recursos Necessários
    \vphantom{q} % TODO: corrigir o depth da linha sem esta gambiarra.
}\end{center}\end{snugshade}

\begin{itemize} 
  \item Listas de Exercícios;
  \item Livros e apostilas;
  \item Utilização de recursos da web;
  \item Quadro branco;
  \item Marcadores para quadro branco;
  \item Sala de aula com acesso à internet, microcomputador e TV ou projetor para apresentação de slides ou material multimídia;
  \item Laboratório de microcomputadores contendo componentes de hardware e software específicos;
\end{itemize}

% Bibliografia

\begin{snugshade}\begin{center}\textbf{
    Bibliografia
}\end{center}\end{snugshade}

\begin{itemize} 
  \item Básica:
	\begin{enumerate}
	\item CELES, Waldemar. Introdução a Estrutura de Dados -  ISBN 978-85-3521-228-0, Editora Campus Elsevier, 2004.
	\item T.H. Cormen, C.E. Leiserson, R.L. Rivest, C. Stein, ``Algoritmos - Teoria e Prática'', 3a. ed., ISBN: 8535236996, Editora Campus, 2012.
	\item TENENBAUM, Aaron M. LANGSAM, Yedidyah. AUGENSTEIN, Moshe J. Estrutura de Dados Usando C - ISBN 8534603480, Makron Books. 
	
	\end{enumerate}
  \item Complementar:
	\begin{enumerate} 
	\item Steven S Skiena, The Algorithm Design Manual, Springer; 2nd edition, ISBN: 978-1849967204, 2008.\\
	\end{enumerate}
\end{itemize}
