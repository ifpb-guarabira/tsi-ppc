
% Dados do Componente Curricular

\begin{snugshade}\begin{center}\textbf{
	Dados do Componente Curricular
}\end{center}\end{snugshade}

\noindent 	\textbf{Nome:} Arquitetura de Computadores
\\ 			\textbf{Curso:} Tecnologia em Sistemas para Internet
\\ 			\textbf{Período:} \unit{2}{\degree}
\\ 			\textbf{Carga Horária:} \unit{67}{\hour}
\\ 			\textbf{Docente Responsável:} Otacílio de Araújo Ramos Neto

% Ementa

\begin{snugshade}\begin{center}\textbf{
    Ementa
\vphantom{q}}\end{center}\end{snugshade}

\noindent
Histórico dos computadores. Fundamentos do projeto e medidas de desempenho. Circuitos lógicos combinacionais e sequenciais. Arquitetura do processador. Paralelismo em nível de instrução. Hierarquia de memória. 
% Objetivos

\begin{snugshade}\begin{center}\textbf{
    Objetivos
}\end{center}\end{snugshade}


\begin{itemize}

\item Apresentar os eventos históricos e tecnológicos que influenciaram o desenvolvimento da tecnologia de processadores até os dias atuais;
\item Capacitar os estudantes a caracterizar os sistemas de computadores com relação ao desempenho dos mesmos;
\item Capacitar os estudantes no uso das técnicas básicas de eletrônica digital utilizadas no projeto de processadores;
\item Capacitar os estudantes a compreender o processo de execução das instruções;
\item Capacitar os estudantes a compreender o funcionamento das técnicas de paralelismo a nível de instrução;
\item Capacitar os estudantes a compreender o funcionamento do sistema de memória \textit{cache} e do projeto de hierarquias de memória como um todo.

\end{itemize} 

% Conteúdo Programático

\begin{snugshade}\begin{center}\textbf{
    Conteúdo Programático
}\end{center}\end{snugshade}

\begin{itemize}

 \item \textbf{Histórico dos computadores:} Gerações dos computadores e  Lei de Moore.
 
 \item \textbf{Fundamentos de projeto:} Classes de computadores; Definição da arquitetura do computador; Tendências tecnológicas, tendências na alimentação dos circuitos integrados e tendências de custo.
 
 \item \textbf{Medidas de desempenho:} Medição, relatório e resumo do desempenho; Princípios quantitativos do projeto; Associação entre o custo e o desempenho.
 
 \item \textbf{Circuitos lógicos combinacionais:} Álgebra de Boole; Portas Lógicas; Funções lógicas; Minimização de funções lógicas utilizando Álgebra de Boole; Tabelas da Verdade; Minimização de Tabelas da Verdade utilizando Mapa de Karnaugh;  Circuitos Digitais e Blocos Funcionais.
 
 \item \textbf{Circuitos Sequenciais:} Elementos de memória (latches e flip-flops); Registradores contadores, acumuladores, deslocadores, etc; Máquinas de estado e geradores de sequências. 
 
 \item \textbf{Projeto do Sistema de Memória:} Técnicas de otimizações para a memória cache; Tenologias de memória; Sistema de proteção de memória.
 
 \item \textbf{Paralelismo em Nível de Instrução:} Conceitos de paralelismo em nível de instrução; Uso de pipelines; Previsão de desvio; \textit{Hazards}; Técnicas de implementação do Paralelismo em Nível de Instrução. %Uma descrição mais completa pode ser necessária nesse último item. Mas, para isso, eu precisaria do livro em mãos. Como ainda não tenho vou deixar desta forma mais simples mesmo.

\end{itemize}

%\begin{snugshade}\begin{center}\textbf{
%    Metodologia de Ensino
%}\end{center}\end{snugshade}

%\noindent
%   Aulas expositivas utilizando recursos audiovisuais e quadro. Aulas práticas utilizando placas de desenvolvimento em FPGA.

%\begin{snugshade}\begin{center}\textbf{
%    Avaliação do Processo de Ensino e Aprendizagem
%}\end{center}\end{snugshade}

%\noindent
%  Avaliações escritas ao final de cada unidade. Projeto baseado em estudo de caso ou problema real.
   
%\begin{snugshade}\begin{center}\textbf{
 %   Recursos Necessários
  %  \vphantom{q} % TODO: corrigir o depth da linha sem esta gambiarra.
%}\end{center}\end{snugshade}

%\begin{itemize} 
 % 	\item Listas de Exercícios;
  %  \item Livros e apostilas;
   % \item Utilização de recursos da web;
   % \item Quadro branco;
   % \item Marcadores para quadro branco;
   % \item Sala de aula com acesso à Internet, microcomputador e TV ou projetor para apresentação de slides ou material multimídia;
   % \item Laboratório de Arquitetura de Computadores utilizando placas de desenvolvimento em FPGA.
%\end{itemize}

% Bibliografia

\begin{snugshade}\begin{center}\textbf{
    Bibliografia
}\end{center}\end{snugshade}

\begin{itemize} 
  \item Básica:
	\begin{enumerate}
	\item HENNESSY, John L. PATTERSON, David A. Arquitetura de Computadores - Uma Abordagem Quantitativa - 5 Ed. 2014. Elsevier.
	\item HENNESSY, John L. PATTERSON, David A. Organização e Projeto de Computadores - 4 Ed. 2014. Elsevier.
	\item BRYANT, Randal E. O’Hallaron, David R. Computer Systems: A Programmer's Perspective - 3 Ed. 2015. Pearson.
	\end{enumerate}
  \item Complementar:
	\begin{enumerate} 
	\item STALLINGS, Willian. Arquitetura e Organização de Computadores 8 Ed. Pearson.
	\item TANENBAUM, Andrew S. Organização Estruturada de Computadores. ISBN: 9788581435398. Editora Pearson. 6 Ed., 2013.
	\end{enumerate}
\end{itemize}
