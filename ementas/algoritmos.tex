\paragraph{Algoritmos e L\'ogica de Programa\c{c}\~ao}

%PREENCHER DADOS DA DISCIPLINA A SEGUIR
%\vspace{-12mm}
\begin{center}\textbf{Dados do Componente Curricular}\end{center}
\vspace{-5mm}
\noindent\rule{16.5cm}{0.4pt}
\\
\textbf{Nome:} Algoritmos e L\'ogica de Programa\c{c}\~ao
\\ 
\textbf{Curso:} Tecnologia em Sistemas para Internet
\\ 
\textbf{Período:} \unit{1}{\degree}
\\ 
\textbf{Carga Horária:} \unit{100}{\hour}
\\ 
\textbf{Docente Responsável:} Ruan Delgado Gomes
\\ 
\noindent\rule{16.5cm}{0.4pt}\\
\\
%PREENCHER A EMENTA A SEGUIR
\vspace{-12mm}
\begin{center}\textbf{Ementa}\end{center}
\vspace{-5mm}
\noindent\rule{16.5cm}{0.4pt}
\\
Conceito de Algoritmos e Linguagens de Programação; Estruturas de Decisão; Estruturas de Repetição; Vetores e Matrizes; Manipulação de Strings; Modularização; Recursividade; Registros (Estruturas).\\
\noindent\rule{16.5cm}{0.4pt}\\
\\
%PREENCHER OS OBJETIVOS A SEGUIR
\vspace{-12mm}
\begin{center}\textbf{Objetivos}\end{center}
\vspace{-5mm}
\noindent\rule{16.5cm}{0.4pt}
\\
\begin{itemize}
\item Saber construir programas de computador obedecendo aos princípios da programação estruturada;
\item Conhecer conceitos básicos relacionados à construção de algoritmos;
\item Compreender e elaborar estruturas de controle;
\item Saber manipular dados por meio de Strings, Vetores e Matrizes;
\item Aprender os conceitos para criação de sub-rotinas, passagem de parâmetros e escopos de variáveis;
\item Aprender o conceito de registro (estrutura).
\end{itemize} 
\noindent\rule{16.5cm}{0.4pt}\\
\\
%PREENCHER OS CONTEUDOS PROGRAMATICOS A SEGUIR (CUIDADO PARA NAO DEIXAR A TABELA MUITO GRANDE)
\vspace{-12mm}
\begin{center}\textbf{Conteúdo Programático}\end{center}
\vspace{-5mm}
\noindent\rule{16.5cm}{0.4pt}
\\
\begin{itemize}

 \item \textbf{Conceito de Algoritmos e Linguagens de Programação:} Defini\c{c}\~ao; Caracter\'isticas; Formas de representa\c{c}\~ao de algoritmos; Diferen\c{c}a entre linguagens de baixo n\'ivel e alto n\'ivel; Conceito de vari\'avel e mem\'oria; Comandos de entrada e saída; Express\~oes aritm\'eticas; Express\~oes l\'ogicas e relacionais; Preced\^encia de operadores.

 \item \textbf{Estruturas de Decisão:} Estrutura \textit{if else}; Estruturas de decis\~ao aninhadas; Estrutura \textit{switch case}.

 \item \textbf{Estruturas de Repetição:} Estruturas de repeti\c{c}\~ao \textit{while}, \textit{for} e \textit{do while}, ou estruturas equivalentes na linguagem de programa\c{c}\~ao adotada.

 \item \textbf{Vetores e Matrizes:} Conceito de vetores; Declara\c{c}\~ao e manipula\c{c}\~ao de vetores; Conceito de matrizes; Declara\c{c}\~ao e manipula\c{c}\~ao de matrizes; Vetores multidimensionais.

 \item \textbf{Manipulação de Strings:} Declara\c{c}\~ao e manipula\c{c}\~ao de strings; Codifica\c{c}\~ao de caracteres (ex: tabela ASCII, tabela Unicode). Fun\c{c}\~oes \'uteis para manipula\c{c}\~ao de strings.

 \item \textbf{Modulariza\c{c}\~ao:} Criação de sub-rotinas, passagem de parâmetros por valor e por refer\^encia, escopo de variáveis (variáveis locais e variáveis globais), ponteiros.

 \item \textbf{Recursividade:} Defini\c{c}\~ao recursiva de algoritmos; Pilha de execu\c{c}\~ao; Resolu\c{c}\~ao de problemas utilizando recursividade.

 \item \textbf{Registros (Estruturas):} Defini\c{c}\~ao de novos tipos de dados; Agrupamento de vari\'aveis para cria\c{c}\~ao de tipos mais complexos utilizando estruturas.

\end{itemize}
\noindent\rule{16.5cm}{0.4pt}\\
\\
%COLOCAR A METODOLOGIA DE ENSINO A SEGUIR
\vspace{-12mm}
\begin{center}\textbf{Metodologia de Ensino}\end{center} 
\vspace{-5mm}
\noindent\rule{16.5cm}{0.4pt}
\\
   Aulas expositivas utilizando recursos audiovisuais e quadro, além de aulas práticas utilizando computadores. Adicionalmente, serão realizadas atividades práticas individuais ou em grupo, para consolidação do conteúdo ministrado.\\
\noindent\rule{16.5cm}{0.4pt}\\
\\
%COLOCAR AVALIACAO DO PROCESSO DE ENSINO E APRENDIZAGEM A SEGUIR
\vspace{-12mm}
\begin{center}\textbf{Avaliação do Processo de Ensino e Apendizagem}\end{center}
\vspace{-5mm}
\noindent\rule{16.5cm}{0.4pt}
\\
   Avaliações escritas. Avalia\c{c}\~oes pr\'aticas envolvendo a resolu\c{c}\~ao de problemas computacionais.\\
\noindent\rule{16.5cm}{0.4pt}\\
\\
%PREENCHER RECURSOS NECESSARIOS A SEGUIR
\vspace{-12mm}
\begin{center}\textbf{Recursos Necessários}\end{center}
\vspace{-5mm}
\noindent\rule{16.5cm}{0.4pt}
\\
\begin{itemize} 
  \item Listas de Exercícios;
  \item Livros e apostilas;
  \item Utilização de recursos da web;
  \item Quadro branco;
  \item Marcadores para quadro branco;
  \item Sala de aula com acesso à internet, microcomputador e TV ou projetor para apresentação de slides ou material multimídia;
  \item Laboratório de microcomputadores contendo componentes de hardware e software específicos;
\end{itemize}
\noindent\rule{16.5cm}{0.4pt}\\ - 
\\
%PREENCHER BIBLIOGRAFIA A SEGUIR
\vspace{-12mm}
\begin{center}\textbf{Bibliografia}\end{center}
\vspace{-5mm}
\noindent\rule{16.5cm}{0.4pt}
\\
\begin{itemize} 
  \item Básica:
	\begin{enumerate}
	\item Piva Junior, D., Engelbrecht,  A. M., Nakamiti,  G. S. e Bianchi, F.. Algoritmos e Programação de Computadores. ISBN: 9788535250312. Editora Campus. 1 ed, 2012;
	\item CELES, Waldemar. Introdução a Estrutura de Dados -  ISBN 978-85-3521-228-0, Editora Campus Elsevier, 2004.
	\end{enumerate}
  \item Complementar:
	\begin{enumerate} 
	\item Oliveira, U. Programando em C – Volume 1: Fundamentos. ISBN: 8573936592. Editora Ciência Moderna. 2007;
	\end{enumerate}
\end{itemize}
\noindent\rule{16.5cm}{0.4pt}\\
\\
