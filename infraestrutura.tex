\newpage
\section{Infraestrutura}

\subsection{Espaço Físico Geral}

O \textit{campus} do IFPB em Guarabira possui uma infraestrutura que inclui salas de aula, laboratórios, auditório, área comum para as refeições dos alunos, biblioteca e demais dependências administrativas. O \textit{campus} também está passando por um processo de expansão, onde um novo conjunto de prédios está sendo construído. As seções a seguir apresentam as instalações físicas atuais do \textit{campus} Guarabira. A tabela \ref{table:infra} contém as informações sobre a área total existente.

\begin{table}[h]
\begin{center}
\caption{Espaço físico geral.}
\begin{tabular}{|l|c|l|c|}
\hline
\multicolumn{1}{|c|}{\textbf{TIPO DE ÁREA}} & \multicolumn{1}{l|}{\textbf{QUANT.}} & \textbf{ÁREA (\squaremetre)} & \textbf{\begin{tabular}[c]{@{}c@{}}HORÁRIO DE\\ FUNCIONAMENTO\end{tabular}} \\ \hline
Salas de aula                               & 7                               & 6x7,5\meter (45 \squaremetre cada sala) & diurno/noturno                                                              \\ \hline
Sala de aula								& 1								  & 6x5\meter (30 \squaremetre) &
diurno/noturno																\\ \hline
Sala de áudio e vídeo						& 1								   & 6x7,5\meter (45 \squaremetre)  &
diurno/noturno																\\ \hline
Auditórios/Anfiteatros                      & 1                                & 187,5 \squaremetre & diurno/noturno                                                              \\ \hline
Salas de Professores                        & 3                                & 45\squaremetre, 20 \squaremetre e 7,5 \squaremetre                   & diurno/noturno                                                              \\ \hline
Áreas de Apoio Acadêmico                    & 1                                & 5x2,15\meter (10,75 \squaremetre)  & diurno/noturno                                                              \\ \hline
Áreas Administrativas                       & 4                                 & 5x6\meter (30 \squaremetre cada sala)                    & diurno/noturno                                   \\ \hline
Área Administrativa							& 1                                &  5x4\meter (20 \squaremetre) & diurno/noturno \\ \hline
Refeitório                         & 1                                &  14,90x8,75\meter (130 \squaremetre)
                  & diurno/noturno                                                              \\ \hline
Cozinha do refeitório						& 1									& 12,38x9,6\meter (119\squaremetre) & diurno/noturno																\\ \hline
Banheiros coletivos                         & 2									&  6x7,5\meter (45 \squaremetre)  & diurno/noturno                                                              \\ \hline
Banheiros individuais						& 6									& 1,92x1,25\meter (2,40\squaremetre cada) & diurno/noturno																\\ \hline
Laboratórios de informática                 & 2                                &  45\squaremetre e 72\squaremetre& diurno/noturno                                                              \\ \hline
Biblioteca (área de consulta)               & 1                                & 24,85x6\meter (149 \squaremetre)                   & diurno/noturno				            \\ \hline
Biblioteca (área da administração)          & 1									& 7,5x3,85\meter (28,9\squaremetre) & diurno/noturno																\\ \hline
\multicolumn{1}{|r|}{\textbf{Total}}        & 33                               & 1.449,05 \squaremetre    & \multicolumn{1}{l|}{}                                                       \\ \hline
\end{tabular}
\label{table:infra}
\end{center}
\end{table}
%colocar subtopicos

\subsubsection{Infraestrutura de Segurança}

O \textit{campus} do IFPB em Guarabira conta com os seguintes itens de segurança e conservação do espaço físico:

\begin{itemize}
\item Vigilância ostensiva armada, 24 horas por dia e 7 dias por semana. A equipe conta conta com 10 seguranças trabalhando em regime de 12 horas trabalhadas por 36 horas de folga. Os turnos começam das 06:00h até às 18:00h e das 18:00h às 06:00h. No primeiro turno são 2 seguranças e no segundo 3;
\item Equipe de limpeza e manutenção terceirizada contando com 7 funcionários. Dos 7, 6 são caracterizados como servente de limpeza e 1 encarregado; 
\item Equipamentos de proteção individuais diversos (botas de proteção, luvas emborrachadas, capacetes de segurança, máscaras de proteção, etc);
\item Sistema de proteção contra incêndios composto por 23 extintores posicionados em locais estratégicos.
\end{itemize} 

\subsubsection{Recursos Audiovisuais e Multimídia}

Na tabela \ref{table:audiovisual} estão especificados os equipamentos audiovisuais disponíveis na instituição para serem utilizados pelos professores e alunos do curso.

\begin{table}[h]
\caption{Relação de recursos audiovisuais e multimídia}
\begin{center}
\begin{tabular}{|p{12cm}|r|}
\hline
\multicolumn{1}{|c|}{\textbf{Item}} & \multicolumn{1}{c|}{\textbf{Quantidade}} \\ \hline
TV  LED de 50 polegadas             & 1                                        \\ \hline
Projetores Multimídia               & 10                                        \\ \hline
Projetores do tipo ``computador interativo'' (COMPUTADOR INTERATIVO PC -3500 / MARCA DURAMA) & 12 \\ \hline
Quadro branco (300 x 120mm)         & 21                                       \\ \hline
Quadro branco (120 x 90cm)          & 5											\\ \hline 
Computadores                        & 116                                       \\ \hline
\end{tabular}
\end{center}
\label{table:audiovisual}
\end{table}

\subsubsection{Manutenção e conservação das instalações físicas}

A manutenção e conservação das instalações físicas são feitas por funcionários terceirizados. Atualmente, a equipe de apoio conta com 7 funcionários. Estes são responsáveis pela limpeza e conservação das instalações, cuidar do jardim, realizar pinturas na alvenaria, implantar instalações elétricas e efetuar pequenos consertos na estrutura física do prédio.

\subsubsection{Manutenção, conservação e expansão dos equipamentos}

A manutenção dos equipamentos é realizada pelos funcionários técnicos administrativos da área de informática, no próprio campus.

\subsubsection{Condições de acesso para portadores de necessidades especiais}

	Desde o início de suas atividades, o IFPB, Campus Guarabira, tem feito esforços para promover o atendimento a pessoas com deficiência em conformidade com as diretrizes contidas no PDI da Instituição (pp. 184-185). No tocante à estrutura física atual (pr\'edio provis\'orio no antigo CAIC), ainda existem algumas defici\^encias, uma vez que nem todos os setores do IFPB são acessíveis aos portadores de defici\^encia. No entanto, os princiapais setores de apoio ao p\'ublico, bem como um conjunto de salas de aula, laborat\'orios e banheiro foram organizados de maneira a atender os portadores de defici\^encia. As aulas também são organizadas de forma que a alocação para os alunos que possuem deficiências motoras é feita em salas no térreo do edifício, facilitando o acesso tanto às salas como ao banheiro adaptado. No pr\'edio definitivo, onde provavelmente o curso superior de Tecnologia em Sistemas para Internet ir\'a funcionar, a infra-estrutura ser\'a completamente adaptada aos portadores de necessidades especiais.

	Dessa forma, o IFPB, em observância à legislação específica, tem consolidado sua política de atendimento a pessoas com deficiência, procurando assegurar-lhes o pleno direito à educação para todos e efetivar ações pedagógicas visando à redução das diferenças e à eficácia da aprendizagem.
 
	O IFPB \textit{Campus} Guarabira, especificamente, conta com um Núcleo de Apoio às pessoas com necessidades Especiais – NAPNE, que conta atualmente com 5 interpretes de libras. \`A medida que o corpo de t\'ecnicos administrativos do campus se fortalecer, novos membros poder\~ao ser designados ao NAPNE, como psic\'ologo, m\'edico e assistente social.

	%O NAPNE tem trabalhado no sentido de melhorar ainda mais a acessibilidade do Campus, solicitando, junto à direção deste, a instalação de piso tátil, faixa contrastante e a adequação dos balcões de atendimento.

%	Este Núcleo também tem trabalhado com diversas instituições que prestam assistência às pessoas com deficiência, no sentido de diagnosticar possíveis carências no acesso às pessoas com deficiência. Entre essas instituições: SCG (Associação de Surdos de Campina), Instituto dos Cegos, Escola de Auto-comunicação de Campina Grande, ICAE (Instituto Campinense de Atendimento ao Excepcional), ICACE e FDC.

	Quando se fala em ambiente universitário (especificamente a biblioteca) percebe-se ser muito intenso o processo de isolamento informacional aos portadores de necessidades especiais. Os aspectos infra e superestruturais desencadeiam um espectro de mitificação ao portador quando não têm acesso amplo ao local.
Falar dos portadores de necessidades especiais remete a um termo bastante conhecido, mas que precisa ser refletido sob vertentes sólidas: acessibilidade. Esta é a palavra chave que promoverá a discussão e as possíveis reflexões para a efetiva inserção desses indivíduos no meio acadêmico, neste contexto, a ampliação do acesso informacional pelo viés biblioteca.

	Dessa forma, fica a indagação: o que efetivamente é acessibilidade? Como ela se aplica à realidade dos portadores de necessidades especiais na biblioteca universitária? No dicionário Aurélio (2005), afirmando um conceito amplo de acessibilidade voltado, sobretudo, para a educação especial: ``Condição de acesso aos serviços de informação, documentação e comunicação, por parte de portador de necessidades especiais''.

	Vale ressaltar que o processo de acessibilidade acelerou-se sobremaneira com a difusão da rede computadores, principalmente a Internet na década de 90, ampliando o conceito de acessibilidade e tentando efetivar uma maior interação entre os portadores de necessidades especiais com os ambientes de espaços físicos (transportes, saúde, lazer) a partir do mundo digital (redes de computadores e sistemas de informação e comunicação).

	Porém, é importante fazer a ressalva de que um grande imbróglio surge em decorrência desse processo, uma vez que esse surgiu nos Estados Unidos e não foi adaptado, de maneira adequada, à realidade socioeconômica de países como o Brasil. Mesmo tendo aumentado no Brasil o acesso aos computadores e à Internet, o número de pessoas atendidas ainda não é suficiente quando se trata de ``inclusão digital''.

	Dessa forma, é possível perceber o relevante papel das bibliotecas universitárias como instrumento para essa inclusão digital. Por esta razão é que, a partir da criação da portaria nº 1.679 que dispõe acerca da exigência de requisitos de acessibilidade para pessoas com deficiência, o MEC passou a avaliar as bibliotecas dos cursos pela acessibilidade desde 1999.

	De acordo com Mazzoni (2005, p. 6), o artigo primeiro dessa Portaria determina que sejam incluídos nos instrumentos destinados à avaliação das condições de oferta de cursos superiores, para fins de autorização e reconhecimento e de credenciamento de instituições de ensino superior, bem como para a renovação, conforme as normas em vigor, requisitos de acessibilidade de pessoas portadoras de necessidades especiais. Além das determinações da referida Portaria, há a Norma Brasil 9050, da Associação Brasileira de Normas Técnicas, que trata da “Acessibilidade de Pessoas Portadoras de Deficiências e Edificações, Espaço Mobiliário e Equipamentos Urbanos”, que apresenta outras indicações para um correto atendimento às pessoas em situação de deficiência física, deficiência visual e deficiência auditiva.

	Com efeito, malgrado a criação de lei, não é esta que irá estabelecer os elos e ligações para o efetivo cumprimento das causas voltadas para o portador de necessidades especiais. A temática permeia um aspecto mais lato direcionado a questões sociais, políticas e epistemológicas. Os dois primeiros referem-se a questões de influência, que as autoridades podem encaminhar; já o terceiro concerne à concepção de que para suprir as necessidades dos portadores de necessidades especiais são necessários vários estudos, tanto de estruturas físicas, como de acesso ideológico.

	Assim, é perceptível que muitas universidades não têm avaliado a biblioteca como um importante instrumento de contribuição ao ensino, pesquisa e extensão, mas apenas como um espaço de livros armazenados. A mitificação de que a biblioteca é um espaço para uma minoria é uma realidade para grande parte da sociedade e para o próprio bibliotecário, embora na universidade essa incidência de acesso às estruturas da biblioteca seja menor.

	Percebe-se que a questão do acesso por parte do portador de necessidades especiais é, em primeiro caso, algo que permeia o aspecto sócio-político.

	Faz-se necessário tratar essa perspectiva de amplo acesso às dependências da biblioteca, disponibilizando investimentos para que possam ser feitos estudos arquitetônicos e científicos, a fim de oferecer grandes estruturas físicas, bem como de acervo e orientação para a ampliação de conhecimentos para os portadores de necessidades especiais, fomentando três condições básicas de acesso: urbanística (caminhos de acesso, estacionamento); arquitetônicos (iluminação, ventilação, banheiros, rampas adequadas) e informação e comunicação (sinalização, sistema de consulta e empréstimo, tecnologia de apoio para usuários portadores de necessidades especiais).
	
	É preciso valorizar os usuários com as mais diversas necessidades no sentido de conferir-lhe respaldo informacional, seja no aspecto sensorial (audição e visão), seja no físico (de locomoção ou coordenação), visando tornar o acesso à biblioteca pelos portadores de necessidades especiais uma realidade diferente daquela ainda presente em muitas universidades. De modo a cessar, ao menos dentro do ambiente acadêmico, especificamente, a biblioteca, constrangimentos tão comuns gerados pelas deficitárias estruturas urbanística, física e de informação que estão aquém das condições necessárias a essas pessoas.
	
	O fato do sistema de bibliotecas de uma dada universidade ser ou não centralizado pode influir nesse processo de acessibilidade. Entretanto, é preciso realçar que a estrutura da biblioteca não pode ser centralizada arbitrariamente na segmentação dos padrões geográficos e físicos da universidade.


\subsection{Biblioteca}

\subsubsection{Apresenta\c{c}\~ao}

A Biblioteca deverá operar com um sistema completamente informatizado, possibilitando fácil acesso via terminal, ao acervo da biblioteca. O sistema informatizado propicia a reserva de exemplares, cuja política de empréstimos prevê um prazo máximo de 14 (catorze) dias para os alunos e professores, além de manter pelo menos 1 (um) volume para consultas na própria Instituição.

O acervo deverá estar dividido por áreas de conhecimento, facilitando, assim, a procura por títulos específicos, com exemplares de livros e periódicos, contemplando todas as áreas de abrangência do curso. Deve oferecer serviços de empréstimo, renovação e reserva de material, consultas informatizadas às bases de dados e ao acervo, orientação na normalização de trabalhos acadêmicos, orientação bibliográfica e visitas orientadas.

\subsubsection{Espa\c{c}o f\'isico}

No pr\'edio provis\'orio do IFPB Guarabira (antigo CAIC) existe um espaço total de cerca de \unit{177,9} metros quadrados para a biblioteca, sendo este dividido em uma área de consulta com \unit{149} metros quadrados e uma área administrativa com \unit{28,9} metros quadrados. O ambiente \'e climatizado e possui atualmente um acervo de Y livros. No local também estão disponíveis aos alunos mesas e computadores com acesso \`a Internet para realiza\c{c}\~ao de estudos em grupo ou individual.

\subsubsection{Instala\c{c}\~oes para o acervo}

%colocar subtopicos

A biblioteca possui prédio próprio para guardar o acervo, que é exposto em estantes metálicas, contendo em cada uma títulos específicos. 

%Ainda sobre o ponto de vista de higienização, sugere-se a instalação de condicionadores de ar para evitar mofo.

\paragraph{Instalações para estudos individuais}\
\
Na biblioteca existem oito baias para estudos individuais, contendo computadores com acesso à Internet. 

\paragraph{Instalações para estudos em grupo}\
\
Na biblioteca existem mesas que permitem acomodar alunos para realizarem estudos em grupo.

\paragraph{Acervo atual}\
\
Na tabela a seguir, apresentamos uma descrição do acervo contido em nossa biblioteca.

%\begin{table}[h]
\begin{longtable}{|c|p{115mm}|c|}
\hline
Número & \multicolumn{1}{c|}{Título}                                                           & Quantidade \\ \hline
1      & YNEMINE, Silvana Tauhata. \textbf{Conhecendo o JavaScript}. Florianópolis: Visual Books, 2002. & 1          \\ \hline
2      & RUGGIERO, Márcia A. Gomes. \textbf{Cálculo numérico: aspectos teóricos e computacionais}. São Paulo: McGraw-Hill, 1988. & 1          \\ \hline
3      & MANCUR, Michael. \textbf{Aprenda em 24 horas JavaScript 1.3}. Rio de Janeiro: Campus, 1999. & 1          \\ \hline
4      & BASTOS, Alex C. \textbf{Programação COBOL}. Rio de Janeiro: LTC, 1993                 		 & 1          \\ \hline
5      & WEISKAMP, Keith. \textbf{Turbo Pascal 6.0}. Rio de Janeiro: LTC, 1992                 		 & 1          \\ \hline
6      & KAUFELD, John. \textbf{Acess para Windows 95 para leigos}. São Paulo: Berkeley Brasil, 1992.& 1          \\ \hline
7      & LYRA, Suzana Maria T de M. \textbf{AutoCAD R14}. João Pessoa: ETFPB, 1998                      & 1          \\ \hline
8      & \textbf{DOSSIÊ acessibilidade web: um manual para editores da web}. São Paulo: Google, 2009 & 1          \\ \hline
9      & KNIBERG, Henrik. \textbf{Scrum e XP direto das trincheiras: como nós fazemos Scrum}. São Paulo: InfoQ, 2007. & 1          \\ \hline
10     & \textbf{DESVENDANDO o Java: manual prático para programadores}. São Paulo: 2004       & 1          \\ \hline
11     & WEISSINGER, Keyton A. \textbf{ASP: guia completo}. Rio de Janeiro: Ciência Moderna Ltda, 1999. & 1          \\ \hline
12     & SHIMIZU, Tamio. \textbf{Programação Cobol: curso básico}. São Paulo: Atlas, 1985.     & 1          \\ \hline
13     & SERY, Paul G. \textbf{Ferramentas poderosas para redes em Linux: dicas e segredos}. Rio de Janeiro: Ciência Moderna Ltda, 1998. & 1          \\ \hline
14     & \textbf{GUIA do usuário: Microsoft, Windows e MS-DOS 6. 2: Sistema operacional mais utilitários avançados}. Microsoft Corporation. 2001. & 1          \\ \hline
15     & \textbf{TECNICAS de programação com PASCAL}. Rio de Janeiro: INFOBOOK, 1993            & 1          \\ \hline
16     & GREENBERG, Adele Droblas. \textbf{Photoshop 4}. São Paulo: Makron Books, 1998.        & 1          \\ \hline
17     & \textbf{ARQUITETURA de redes}. Organização de Tereza Cristina Melo de Brito Carvalho. São Paulo: Makron Books; Embratel; SGA, 1994. & 4          \\ \hline
18     & JEUNOT, Dominique. \textbf{Enterprise DBA Parte 1A: Administração e Arquitetura}. Oracle. 2000.& 2          \\ \hline
19     & SANTOS, Adauto Machado dos. \textbf{Aprendendo arte no CorelDRAW 10}. Rio de Janeiro: Brasport, 2001. & 1          \\ \hline
20     & COFFMAN, Gayle. \textbf{SQL SERVER 7: Completo e total guia de referência}. São Paulo: Makron Books, 2000. & 1          \\ \hline
21     & KIRCH, Olaf. \textbf{Guia do administrador de redes Linux}. Curitiba: Coletiva, 1999.          & 1          \\ \hline
22     & PALMER, Scott D. \textbf{Guia do programador Turbo Pascal for Windows}. Rio de Janeiro: Ciência Moderna Ltda, 1992. & 1          \\ \hline
23     & McCarty, Bill. \textbf{Aprendendo Red Hat Linux}. Rio de Janeiro: Campus, 2000.       & 1          \\ \hline
24     & LIBERTY, Jesse. \textbf{Aprenda em 24 horas C++}. Rio de Janeiro: Campus, 1998.       & 1          \\ \hline
25     & \textbf{PLENUS: O integrador de ambientes}. Belo Horizonte: Octus Informática, 1993.  & 2          \\ \hline
26     & JEUNOT, Dominique. \textbf{Enterprise DBA Parte 2A: Administração e Arquitetura}. Oracle. 2000. & 2          \\ \hline
27     & BORATTI, Isaias Camilo. \textbf{Programação orientada a objetos: usando Delphi}. Florianópolis: Visual Books, 2001. & 1          \\ \hline
28     & KOSHAFIAN, Setrag. \textbf{Banco de dados orientado a objetos}. Rio de Janeiro: Infobook, 1994.& 1          \\ \hline
29     & SOARES, Luiz Fernando Gomes; LEMOS, Guido; COLCHER, Sérgio. \textbf{Rede de computadores}. Rio de Janeiro, 1995 & 1 \\ \hline
30     & BALL, Hill. \textbf{Utilizando Linux}. Rio de Janeiro: Campus, 1999.                  & 1          \\ \hline
31     & HARRISON, Thomas H. \textbf{Intranet Data Warehouse}... São Paulo: Berkeley Brasil, 1998.      & 1          \\ \hline
32     & CANTÚ, Marco. \textbf{Dominando o Delphi 5: a bíblia}. São Paulo: Makron Books, 2000.          & 1          \\ \hline
33     & FIELDS, Duane K.; KOLB, Mark A. \textbf{Desenvolvendo na Web com JavaServer Pages}. Rio de Janeiro: Ciências Moderna, 2000. & 1          \\ \hline
34     & ARCHER, Tom. \textbf{Descobrindo C\#}. Rio de Janeiro: Campus, 2001.                   & 1          \\ \hline
35     & MUCHOW, John W. \textbf{Core J2ME Tecnologia e MIDP}. São Paulo: Pearson Makron Books, 2004.   & 1          \\ \hline
36     & HORSTMANN, Cay S.; CORNELL, Gary. \textbf{Core Java volume II - Recursos Avançados}. São Paulo: Makron Books, 2000 & 1          \\ \hline
37     & PAIXÃO, Renato Rodrigues. \textbf{Montando e Configurando PCs com Inteligência}. São Paulo: Érica,  1999. & 1          \\ \hline
38     & TACKETT JR, Jack; BURNETT, Steven. \textbf{Usando Especial Linux}. Rio de Janeiro: Campus, 2000. & 1          \\ \hline
39     & \textbf{CARTILHA de Segurança para internet}. São Paulo: Comitê Gestor da Internet no Brasil, 2006. & 1          \\ \hline
40     & SCHILDT, Herbert. \textbf{C completo e Total}. São Paulo: Makron Books, 1990.                  & 1          \\ \hline
41     & McClure, Stuart; SCAMBRAY, joel; KURTZ, George. \textbf{Hackers Expostos}. São Paulo: Makron Books, 2000. & 1          \\ \hline
42     & OLIVEIRA, Wilson José de. \textbf{Hacker Invasão e Proteção}. Santa Catarina: Visual Books, 2000. & 1          \\ \hline
43     & ALBITZ, Paul; CRICKET, Liu. \textbf{DNS e BIND}. Rio de Janeiro: Campus, 2001.        & 1          \\ \hline
44     & HOLDEN, Greg; WELLS, Nicholas; KELLER, Mathew. \textbf{Apache Server}. São Paulo: Makron Books, 2001. & 1          \\ \hline
45     & TAYLOR, Dave; ARMSTRONG JR, james C. \textbf{Aprenda em 24 horas UNIX}. Rio de Janeiro: Campus, 1998. & 1          \\ \hline
46     & OLIVEIRA, Júlio C. Peixoto de. \textbf{Controlador Programável}. São Paulo: Makron Books, 1993. & 1          \\ \hline
47     & MATEUS, César Autusto. \textbf{C ++ Builder 5: Guia Prático}. São Paulo: Érica, 2000. & 1          \\ \hline
48     & FURGERI, Sérgio. \textbf{REDES Teoria e Prática}. Campinas: KOMEDI, 2007.             & 1          \\ \hline
49     & \textbf{Segurança Digital}. São Paulo: Escala Educacional, 2009                       & 1          \\ \hline
50     & IDANKAS, Rodney José. \textbf{Manual concurso público: Informática}. São Paulo: Discovery Ltda, 2007. & 1          \\ \hline
51     & MUNIZ, Roberto. \textbf{Noções de informática para concursos}. São Paulo: CMP Editora e Livraria Ltda, 2010. & 1          \\ \hline
52     & TELLES, Reynaldo. \textbf{Descomplicando a informática para concursos}. Rio de Janeiro: Elvesier, 2008. & 1          \\ \hline
53     & MEIRELLES, Fernando de Souza. \textbf{Informática: novas aplicações com microcomputadores}. São Paulo: Pearson, 1994. & 8          \\ \hline
54     & ELMASRI, Ramez; NAVATHE, Shamkant B. \textbf{Sistemas de Banco de Dados}. São Paulo: Pearson, 2011. & 3          \\ \hline
55     & FOWLER, Martin. \textbf{UML Essencial: um breve guia para a linguagem padrão de modelagem de objetos}. Porto Alegre: BooKmam, 2005. & 1          \\ \hline
56     & LARMAN, Craig. \textbf{Utilizando UML e Padrões: uma introdução à análise e as projeto orientados a objetos e ao desenvolvimento iterativo}. Porto Alegre: BooKmam, 2007. & 1          \\ \hline
57     & TANENBAUM, Andrew S.; WOODHULL, Albert S. Sistemas Operacionais: projeto e implementação. Porto Alegre: BooKmam, 2008. & 1          \\ \hline
58     & OLIVEIRA, Ulysses de. Programando em C: fundamentos - V. 1. Rio de Janeiro: Ciência Moderna, 2008. & 3          \\ \hline
59     & HEUSER, Carlos Alberto. \textbf{Projeto de Banco de Dados}. Porto Alegre: BooKmam, 2009. & 3          \\ \hline
60     & KUROSE, James F. \textbf{Redes de computadores e a Internet: uma abordagem top-down}. São Paulo: Pearson, 2010. & 3          \\ \hline
61     & STALLINGS, William. \textbf{Arquitetura e Organização de Computadores}. São Paulo: Pearson, 2010. & 3          \\ \hline
62     & TENENBAUM, Aaron M. \textbf{Estruturas de dados usando C}. São Paulo: Pearson, 2013.  & 3          \\ \hline
63     & LAUDON, Kenneth; LAUDON, Jane. \textbf{Sistemas de Informação Gerenciais}. São Paulo: Pearson, 2010. & 8 \\ \hline
64     & GUEDES, Gilleanes T. A. \textbf{UML 2: uma abordagem prática}. São Paulo: Novatec, 2011. & 2          \\ \hline
65     & TANENBAUM, Andrew S.; AUSTIN, Todd. \textbf{Organização estrutura de Computadores}. São Paulo: Pearson, 2013. & 3          \\ \hline
66     & CAPRON, H. L.; JOHNSON, J. A. \textbf{Introdução à informática}. São Paulo: Pearson, 2004.     & 8          \\ \hline
67     & HORSTMANN, Cay S.; CORNELL, Gary. \textbf{Core Java Vol. I- Fundamentos}. São Paulo: Pearson, 2010. & 3          \\ \hline
68     & SOMMERVILLE, Ian. \textbf{Engenharia de Software}. São Paulo: Pearson, 2011.                   & 2          \\ \hline
69     & SILBERSCHATZ, Abrahm; KORTH, Henry F.; SUDARSHAN, S. \textbf{Sistema de Banco de Dados}. Rio de Janeiro: Elsevier, 2011. & 2          \\ \hline
70     & FOROUZAN, Behrouz A.; MOSHARRAF, Firouz. \textbf{Redes de computadores: uma abordagem top-down}. São Paulo: AMGH, 2010. & 3          \\ \hline
71     & ZIVIANI, Nivio. \textbf{Projetos de algoritmos: com implementações em PASCAL e C}. São Paulo: Cengage Lerning, 2013. & 1          \\ \hline
72     & TANENBAUM, Andrew S.; WETHERALL, David. \textbf{Redes de Computadores}. São Paulo: Pearson, 2011. & 3          \\ \hline
73     & GALLO, Lídia Razera. \textbf{Inglês instrumental para informática}. São Paulo: Ícone editora, 2011. & 1          \\ \hline
74     & GUIMARÃES, Célio Cardoso. \textbf{Fundamentos de Bancos de Dados: modelagem, projetos e linguagem SQL}. Campinas: Editora Unicamp, 2003. & 1          \\ \hline
75     & PRESSMAN, Roger S. \textbf{Engenharia de Software: uma abordagem profissional}. São Paulo: AMGH, 2011. & 1          \\ \hline
76     & DATE, C. J. \textbf{Introdução a sistemas de Banco de Dados}. Rio de Janeiro: Elsevier, 2003.  & 1          \\ \hline
77     & DEITEL, Paul. \textbf{Java: como programar}. São Paulo: Pearson, 2010.                & 2          \\ \hline
78     & NEMETH, Evi et al. \textbf{Manual Completo do Linux}. São Paulo: Pearson, 2007.       & 1          \\ \hline
79     & TANENBAUM, Andrew S. \textbf{Sistemas Operacionais Modernos}. São Paulo: Pearson, 2009.        & 2          \\ \hline
80     & PESSOA, Adonai Alvino. \textbf{Bibliotecas em Turbo Pascal}. Rio de Janeiro: Ciência Moderna, 1990. & 1          \\ \hline
81     & CADENHEAD, Rogers. \textbf{Aprenda em 21 dias JAVA 2}. Rio de Janeiro: Elvesier, 2005.& 1          \\ \hline
82     & TODD, Nick; SZOLKOWSKI, Mark. \textbf{JavaSever pages: o guia do desenvolvedor}. Rio de Janeiro: Elsevier, 2003. & 1          \\ \hline
83     & LARMAN, Craig. \textbf{Utilizando UML e Padrões: uma introdução à análise e as projeto orientados a objetos e ao desenvolvimento iterativo}. Porto Alegre: BooKmam, 2000. & 1          \\ \hline
84     & SILVA JR., Rubens Marques da Silva; ALVES, Maria Goretti Oliveira. \textbf{AccuRender 3: maquetes eletrônicas em AutoCAD}. São Paulo: Érica, 2002. & 1          \\ \hline
85     & SINTES, Tony. \textbf{Aprenda Programação Orientada a Objetos em 21 dias}. São Paulo: MAKRON Books, 2002. & 1          \\ \hline
86     & SYCK, Gary. \textbf{Turbo Pascal: soluções}. São Paulo: Campus, 1993.                          & 1          \\ \hline
87     & PEREIRA, Silvio do Lago. \textbf{Estrutura de Dados Fundamentais: conceitos e aplicações}. São Paulo: Érica, 2008. & 1          \\ \hline
88     & MANZANO, André L. N. G.; MANZANO, \textbf{Maria Isabel N. G. Estudo Dirigido de Informática Básica}. São Paulo: Érica, 2007. & 1          \\ \hline
89     & MORRISON, Michael. \textbf{Use a Cabeça JavaScript}. Rio de Janeiro: Alta Books, 2012.         & 2          \\ \hline
90     & ANDRADE, Denise de Fátima. \textbf{Windows 7: oficina de inclusão digital - (Coleção Smart)}. São Paulo: Viena, 2010. & 1          \\ \hline
91     & ANDRADE, Denise de Fátima. \textbf{Windows 7 - (Coleção Flex)}. São Paulo: Viena, 2010.                 & 1          \\ \hline
\end{longtable}
%\end{table}

\paragraph{Horário de Funcionamento}\
\
Atualmente a biblioteca funciona nos três turnos, de 08h00 às 12h00, 14h00 às 18h00 e 19h00 às 22h00.


\subsubsection{Acervo Específico para o Curso}

Para os componentes curriculares específicos do Curso Superior de Tecnologia em Sistemas para Internet, a quantidade mínima de exemplares deverá ser de 9 unidades por cada título da bibliografia básica e 2 unidades por cada título da bibliografia complementar.

\paragraph{Bibliografia básica}\
\
\begin{itemize}
	\item Piva Junior, D., Engelbrecht,  A. M., Nakamiti,  G. S. e Bianchi, F.. Algoritmos e Programação de Computadores. ISBN: 9788535250312. Editora Campus. 1 ed, 2012;
	\item Menezes, Nilo N. C. Introdução à Programação com Python - Algoritmos e Lógica de Programação para Iniciantes. ISBN: 9788575224083, Editora Novate, 2 ed, 2014;
	\item CELES, Waldemar. Introdução a Estrutura de Dados -  ISBN 978-85-3521-228-0, Editora Campus Elsevier, 2004.
	\item MUNEM, M. A. FOULIS, D. J. - Cálculo Vol. 1 - ISBN  85-2161-054-8, Editora LTC.
	\item GUIDORRIZZI, Hamilto Luiz. Um curso de calculo. Vol. 1. 5ª Ed. Rio de Janeiro. Editora LCT. 2001.
	\item STWART, J. Cálculo. Vol. 1. 5ª Ed. São Paulo: Pioneira Thomson Learning, 2006.
	\item Monteiro, M. A. Introdução à Organização de Computadores. ISBN: 9788521615439. Editora LTC. 5 Ed., 2007; 
	\item Idoeta, I. V.; Capuano, F. G. Elementos de Eletrônica Digital. ISBN: 8571940193. Editora Érica, 40 Ed., 2007;
	\item Velloso, F. C. Informática: Conceitos Básicos. ISBN: 9788535243970. Editora Campus, 8 Ed., 2011.
    \item DIONISIO, A. P.; MACHADO, A. R.; BEZERRA, M. A. (Orgs). Gêneros textuais e ensino. São Paulo: Parábola Editorial, 2010.
    \item GLENDINNING, E.; McEWAN, J. Basic English for Computing. Oxford, 2003;
    \item SOUZA, A. G.; ABSY, C. A.; COSTA, G. C.; MELLO, L. F. Leitura em língua inglesa: uma abordagem instrumental. São Paulo: Disal, 2005.
    \item COOPER, Nate.
          Crie seu próprio site.
          Novatec, 2015.
    
    \item LAWSON, Bruce; SHARP, Remy.
          Introdução ao HTML 5.
          Alta Books, 2011.
    
    \item HOGAN, Brian P..
          HTML 5 e CSS 3: desenvolva hoje com o padrão de amanhã.
          Ciência Moderna, 2012.
	\item KUROSE, J. F., ROSSA, K. W. Redes de computadores e a internet. 5 ed. Editora Pearson. 2010.
	\item TANENBAUM, A. S., WETHERALL, D. Redes de Computadores. 5 ed. Editora Pearson. 2011. 
	\item FOROUZAN, Behrouz A.; MOSHARRAF, Firouz. Redes de Computadores - Uma Abordagem Top-Down - 2012. 1 ed. Editora Mcgraw Hill, 2012.
	\item HENNESSY, John L. PATTERSON, David A. Arquitetura de Computadores - Uma Abordagem Quantitativa - 5 Ed. 2014. Elsevier.
	\item HENNESSY, John L. PATTERSON, David A. Organização e Projeto de Computadores - 4 Ed. 2014. Elsevier.
	\item CELES, Waldemar. Introdução a Estrutura de Dados -  ISBN 978-85-3521-228-0, Editora Campus Elsevier, 2004.
	\item T.H. Cormen, C.E. Leiserson, R.L. Rivest, C. Stein, ``Algoritmos - Teoria e Prática'', 3a. ed., ISBN: 8535236996, Editora Campus, 2012.
	\item TENENBAUM, Aaron M. LANGSAM, Yedidyah. AUGENSTEIN, Moshe J. Estrutura de Dados Usando C - ISBN 8534603480, Makron Books.
    \item FREEMAN, Eric.
          Use a Cabeça! Programação em HTML 5.
          Alta Books, 1a edição, 2014.

    \item BENEDETTI, Ryan; CRANLEY, Ronan.
          Use a Cabeça! JQuery.
          Alta Books, 2013.

    \item FLANAGAN, David.
          JavaScript: o guia definitivo.
          Bookman, 2012.
    \item SAVIOLI, F. P.; FIORIN, J. L.  Para entender o texto: leitura e redação. Ática, 1990;  
  	\item SAVIOLI, F. P.; FIORIN, J. L. Lições de texto: leitura e redação. São Paulo: Ática, 1996. 
  	\item MARCUSCHI, L. A.; XAVIER, A. C. Hipertexto e gêneros digitais: novas formas de construção de sentido. Lucerna, 2004;
	\item BARBETTA, P.A.; REIS, M.M. e BORNIA, A.C. Estatística para cursos de engenharia e informática. Editora Atlas, São Paulo, 2004. 410 p.
	\item BUSSAB, W. O. MORETTIN, P. A.Estatística Básica.  5 ed.  São Paulo: Saraiva, 2002.
	\item 	ELMASRI, R.; NAVATHE, S. \textbf{Sistemas de banco de dados.} Pearson, 6ª edição, 2011;
	\item 	KORTH, H.; SILBERSCHATZ, A.; SUDARSHAN, S. \textbf{Sistemas de bancos de dados.} Campus, 5ª edição, 2006;
	\item 	DATE, C. J. \textbf{Introdução a sistemas de bancos de dados.} Campus, Tradução da 8ª edição Americana, 2004.
	\item BARBOSA , S., SILVA, B. , Interação humano-computador. Elsevier. 2010.
	\item PREECE, J., ROGERS, Y., SHARP, H., Design de Interação: além da interação homem-máquina. 3ª Ed. Bookman, 2013.
	\item BENYON, D., Interação Humano-Computador . 2ª Edição. Pearson, 2011.
    \item ANDRADE, M.M. Introdução à metodologia do trabalho científico: elaboração de trabalhos na graduação. Atlas, 2010; 
    \item ASSOCIAÇÃO BRASILEIRA DE NORMAS TÉCNICAS. NBR 6023: Informação e documentação, referências – elaboração. Rio de Janeiro, 2002;    
    \item BARROS, A.; LEHFELD, N. Projeto de pesquisa: propostas metodológicas. Vozes, 4ª edição, 1996;
	\item DEITEL, H. M.; DEITEL, P. J. \textbf{Java: Como Programar.} Pearson, 8ª Edição, 2010;
	\item FURGERI, S. \textbf{Java 7 Ensino Didático.} Érica, 1ª Edição, 2010;
	\item SIERRA K.; BATES, B. \textbf{Use a Cabeça! - Java.} Alta Books, 2ª Edição, 2007.
	\item Tanenbaum, A. S. Sistemas Operacionais Modernos. ISBN: 9788576052371. Editora Pearson. 3 Ed., 2010. 
	\item Silberschatz, A.; et al. ISBN: 9788521617471. Fundamentos de Sistemas Operacionais. Editora LTC, 8 Ed., 2010;
  	\item LARMAN, Craig. \textbf{Utilizando UML e Padrões: uma introdução à análise e ao projeto orientados a objetos e ao desenvolvimento iterativo.} Bookman,  3ª edição, 2007;
	\item MCLAUGHLIN, B.; et al. \textbf{Use a Cabeça Análise e Projeto Orientado a Objeto.} Alta Books, 2007;
	\item PILONE, D.; PITMAN, N. \textbf{UML 2: Rápido e Prático.} Alta Books, 2006.
    \item GALESI, T.; SANTANA NETO, O.
          Python e Django - Desenvolvimento Ágil de Aplicações Web.
          NOVATEC, 2010

    \item SOARES, W.
          PHP 5 - Conceitos, Programação e Integração com Banco de Dados.
          Editora Érica, 2010.
    
    \item BASHAN, B.; et al.
          Use a Cabeça: Servlets e JSP.
          Alta Books, 2005.
	\item PACHECO, Peter. A Introduction to Parallel Programming - ISBN 978-0-12-374260-5 Elsevier. 2011.
	\item HERLIHY, Maurice. SHAVIT, Nir. The Art of Multiprocessor Programming - ISBN 978-0-12-370591-4 Elsevier. 2008.
	\item SANDERS, Jason. KANDROT, Edward. Cuda by Example An Introduction to General-Purpose GPU Programming - ISBN 978-0-13-138768-3.
    \item STAMP, M.
          Information security: principles and practice.
          Wiley, 2nd edition, 2011.

    \item NAKAMURA, E. T.
          Segurança de redes em sistemas cooperativos.
          Editora Novatec, 2007.

    \item GOODRICH, M. T.; TAMASSIA, R.
          Introdução à Segurança de Computadores.
          Bookman, 1a edição, 2013.
	\item PRESSMAN, Roger S. \textbf{Engenharia de Software: uma abordagem profissional.} McGraw-Hill,  7ª edição, 2011.    
    \item NEMETH, E.; SNYDER, G.; HEIN, T.
          Manual Completo do Linux: Guia do Administrador.
          Pearson, 2004.
    
    \item FERREIRA, R. E.
          Linux, Guia do Administrador do sistema
          Novatec, 2008.
    
    \item SIEVER, E.; WEBER A.; FIGGINS S.; LOVE R.; ROBBINS A.
          Linux, O Guia Essencial
          O'Reilly Media, 2006
	\item Gamma, E. et al. Padrões de Projeto: Soluções reutilizáveis de software orientado a objetos. Bookman, 2000.
	\item Freeman, E; Freeman, E. Use a cabeça! Padrões de Projeto (Design Patterns). Alta books, 2ª Edição. 2007.
    \item QUERINO FILHO, Luiz Carlos.
          Desenvolvendo seu primeiro Aplicativo Android.
          Novatec, 1a edição, 2013.

    \item SAMPAIO, Cleuton.
          Manual do Indie Game Developer.
          Ciência Moderna, 1a edição, 2013.

    \item DARWIN, Ian F.
          Android Cookbook: Problemas e soluções para desenvolvedores Android.
          Bookman, 1a edição, 2013.
    \item GALESI, T.; SANTANA NETO, O.
          Python e Django - Desenvolvimento Ágil de Aplicações Web.
          NOVATEC, 2010

    \item GRINBERG, M.
          Flask Web Development: Developing Web Applications with Python.
          O'Reilly Media, 2014.

    \item GREENFIELD, D.; ROY, A.
          Two Scoopes of Django: Best Pratices For Django 1.6.
          Two Scoopes Press, 2014.
    \item LAUDON, Kenneth C.; LAUDON, Jane P.
          Sistemas de Informações Gerenciais.
          São Paulo: Pearson Prentice Hall, 9a edição, 2010.

    \item ALBERTIN, Alberto Luiz.
          Comércio Eletrônico.
          Atlas Editora, 6a edição, 2010.
	\item 	GONÇALVES, A. \textbf{Beginning Java EE 7.} Apress, 2013;
	\item 	GUPTA, A. \textbf{Java EE 7 Essentials.} O’Reilly, 2013;
	\item 	BURKE, B.\textbf{Enterprise Javabeans 3.0.} Pearson, 2007.
	\item A Cabeça de Steve Jobs (Inside Steve's Brain), Leander Kahney, Agir, 2008.
	\item Bilionários por Acaso: A Criação do Facebook, Ben Mezrich, Intrinseca, 2010.
	\item Google, David A. Vise, Mark Malseed, Rocco, 2007.
	\item Coulouris, Dollimore e Kindberg, Sistemas distribuídos, conceito e projeto (quinta edição). ISBN 9788582600535. Bookman, 2013.
	\item Tanenbaum e van Steen, Sistemas distribuídos, princípios e paradigmas (segunda edição). ISBN 9788576051428. Pearson, 2007.
\end{itemize}

\paragraph{Bibliografia complementar}\
\
\begin{itemize}
	\item Oliveira, U. Programando em C – Volume 1: Fundamentos. ISBN: 8573936592. Editora Ciência Moderna. 2007;
	\item  HOWARD, A.; BIVENS, I.; DAVIS, S. Cálculo. Vol. 1. 8ª Ed. Porto Alegre: Bookman.  2007.
	\item  LEITHOLD, L. O. Cálculo com G'eometria Analítica. Vol. 1. 3ª. Ed. São Paulo: Harbra, 1994.
	\item Tanenbaum, Andrew S. Organização Estruturada de Computadores. ISBN: 9788581435398. Editora Pearson. 6 Ed., 2013.
    \item FÜRSTENAU, Eugênio. Novo dicionário de termos técnicos: Inglês-Português/Português-Inglês. São Paulo: Editora Globo, 2005. Vol 1 e 2.
    \item LONGMAN. Dicionário Escolar: Inglês-Português/Português-Inglês. Pearson Longman, 2009.
    \item MURPHY, R. English Grammar in Use. Intermediate Students. New York, 2000;
    \item FREEMAN, Eric.
          Use a Cabeça! Programação em HTML 5.
          Alta Books, 1a edição, 2014.

    \item MEYER, Eric A..
          Smashing CSS: técnicas profissionais para um layout moderno.
          Bookman, 2011.

    \item ROBSON, Elisabeth; FREEMAN, Eric.
          Head First HTML and CSS.
          O'Reilly Media, 2012.

    \item DEITEL, H. M.; DEITEL, P. J.; SADHU, P.
          XML.
          Bookman, 2003
	\item COMER, D. E. Redes de computadores e internet. 4 ed. Editora Artmed. 2007.
	\item LOWE,Doug. Redes de Computadores Para Leigos. 9ª Edição. Editora Altabooks.
	\item STALLINGS, Willian. Arquitetura e Organização de Computadores 8 Ed. Pearson.
	\item TANENBAUM, Andrew S. Organização Estruturada de Computadores. ISBN: 9788581435398. Editora Pearson. 6 Ed., 2013.
	\item Steven S Skiena, The Algorithm Design Manual, Springer; 2nd edition, ISBN: 978-1849967204, 2008.
    \item LAWSON, Bruce; SHARP, Remy.
          Introdução ao HTML 5.
          Alta Books, 2011.

    \item SILVA, Mauricio Samy.
          jQuery: a biblioteca do programador JavaScript.
          Novatec, 2008.

    \item RUTTER, Jake.
          Smashing jQuery: interatividade avançada com JavaScript simples.
          Bookman, 2012.

    \item CROCKFORD, Douglas.
          O Melhor do JavaScript.
          Alta Books, 2008.

    \item MORRISON, Michael.
          Use a Cabeça! JavaScript.
          Alta Books, 2008.
	\item SAUTCHUK I. Produção dialógica do texto escrito. Martins Fontes, 2003.
	\item TERRA, E.; NICOLA, J. Práticas de linguagem \& Produção de textos. Scipione, 2001.
	\item VAL, Maria da Graça Costa. Redação e textualidade. 3ª ed. São Paulo: Martins Editora, 2006
	\item LIMA, Antônio Oliveira. Manual de redação oficial. 3ª Ed. Rio de Janeiro: Campos Editora, 2009.
	\item INFANTE, U. Do texto ao texto: curso prático de leitura e redação. Scipione, 1998; 
	\item CARNEIRO, A. D. Redação em construção: a escritura do texto. Moderna, 2001;
	\item ANDRADE, M. M.; HENRIQUES, A. Língua portuguesa: noções básicas para cursos superiores. Atlas, 2004;
	\item BASTOS, L. K. A produção escrita e a gramática. Martins Fontes, 2003;
	\item BECHARA, E. O que muda com o novo acordo ortográfico. Lucerna, 2008.
	\item COSTA, José Maria da. Manual de redação jurídica. 5ª ed. São Paulo: Migalhas, 2012.
	\item MEYER, P.L. Probabilidade: Aplicações à Estatística. 2 ed.  Rio de Janeiro: LTC – Livros Técnicos e Científicos, 2000.
	\item FONSECA, J.S. e Martins, G.A. Curso de Estatística. São Paulo: Atlas, 1993.
	\item 	HEUSER, C. \textbf{Projeto de Banco de Dados – Série UFRGS, Nº 4.} Sagra-Luzzatto, 5ª edição, 2004;
	\item 	GARCIA-MOLINA, H. \textbf{Implementação de Sistemas de Banco de Dados.} Campus, 1ª edição, 2010;
    \item 	RAMAKRISHNAN, R. \textbf{Sistemas de Gerenciamento de Banco de Dados.} McGraw Hill, 3ª edição, 2010.
	\item NIELSEN, J., Loranger, H. Usabilidade na Web: Projetando Websites com Qualidade. Elsevier, 2007.
	\item SHNEIDERMAN, Ben. Designing the User Interface: strategies for effective human-computer interaction. 4. ed. EUA: Addison-Wesley, 2004.
    \item NBR 10520: Informação e documentação, apresentação de citações em documentos. Rio de Janeiro, 2002;
    \item NBR 14724: Informação e documentação, trabalhos acadêmicos – apresentação. Rio de Janeiro, 2005;
    \item CERVO, A. L.; BERVIAN, P. A. Metodologia científica.  Prentice Hall, 5ª edição, 2006;
    \item DUARTE, E. Manual técnico para a realização de trabalhos monográficos.  Universitária, 4ª Edição, 2001;
    \item DESLANDES, S F. A construção de projeto de pesquisa. In: MINAYO, M. C. de S. (Org). Pesquisa social: teoria, método e criatividade.  Petrópolis, 21ª edição, 1994, p. 31-50; 
    \item GODOY, A. S. Introdução à pesquisa qualitativa e suas possibilidades. Revista de administração de empresas, v.35, n.2, p.57-83, mar/abr., 1995;
    \item KÖCHE, J. C. Fundamentos de metodologia científica: teoria da ciência e iniciação à pesquisa. Vozes, 26ª edição, 2009;
	\item HORSTMANN, C. S. \& CORNELL, G. \textbf{Core Java, Volume 1.} Pearson, 8ª edição, 2010;
	\item CADENHEAD, R.; LEMAY, L. \textbf{Aprenda Java em 21 Dias.} Campus, 4ª edição, 2005.
	\item Marshall Kirk McKusick, George V. Neville-Neil, Robert N.M. Watson. The Design and Implementation of the FreeBSD Operating System. ISBN: 978-0321968975. Editora Addison-Wesley. 2 Ed., 2014.
	\item Mark Russinovich, David Solomon, Alex Ionescu. Windows Internals, Part 1. Microsoft Press. ISBN: 978-0735648739. 6 Ed., 2012.
	\item Robert Love. Linux Kernel Development. ISBN: 978-0672329463. Editora Addison-Wesley. 3 Ed., 2010.
  	\item FOWLER, M.; SCOTT, K. \textbf{UML Essencial.} Porto Alegre: Bookman, 2005;
	\item ENGHOLM JR, H. \textbf{Análise e Design Orientado a Objetos.} Novatec. 2013.
	\item 	CASANOVA, M, et al. \textbf{Bancos de Dados Geográficos}. INPE, 2005; 
	\item 	FOWLER, M.; SADALAGE, P. J. \textbf{NoSQL Essencial: Um Guia Conciso Para O Mundo Emergente Da Persistência Poliglota.} Novatec, 1ª Edição, 2013;
	\item 	RAMAKRISHNAN, R. \textbf{Sistemas de Gerenciamento de Banco de Dados.} McGraw Hill, 3ª edição, 2010.
    \item GRINBERG, M.
          Flask Web Development: Developing Web Applications with Python.
          O'Reilly Media, 2014.

    \item GREENFIELD, D.; ROY, A.
          Two Scoopes of Django: Best Pratices For Django 1.6.
          Two Scoopes Press, 2014.

    \item MENEZES, N. N. C.
          Introdução a programação com Python.
          Novatec, 2014.
	\item STALLINGS, W.
	    Criptografia e segurança de redes.
	    Prentice-Hall, 4a edição, 2007.

	\item ULBRICH, H. C.; DELLA VALLE, J.
	    Universidade Hacker.
	    Editora Digerati Books, 2009.

	\item SHOKRANIAN, S.
	    Criptografia para iniciantes.
	    Ciência Moderna, 2a edição, 2012.

	\item CHAMPLAIN, J. J.
	    Auditing information system.
	    John Wiley \& Sons, 2a edição, 2003.
	\item  SOMMERVILLE, I. \textbf{Engenharia de Software.} Pearson
    \item MORIMOTO, C.
          Linux, Guia Prático.
          GDH Press e Sul Editores, 2009.

    \item MORIMOTO, C.
          Servidores Linux, Guia Prático.
          GDH Press e Sul Editores, 2008.

    \item LAURIE, B; LAURIE, P.
          Apache: The definitive Guide.
          O'Reilly Media, 2002.
	\item Metsker, S. J. Padrões de Projeto em Java. Bookman, 2004.
	\item Shalloway, A.; Trott, J. R. Explicando padrões de projeto – Uma nova perspectiva em projeto orientado a objetos. Bookman, 2004.
    \item ALLAN, Alasdair.
          Aprendendo Programação iOS: Do Xcode à AppStore
          Novatec, 1a edição, 2013.

    \item NEIL, Theresa
          Padrões de Design para Aplicativos Móveis.
          Novatec, 1a edição, 2012.
    \item RICHARDSON, L.; AMUNDSEN, M.; RUBY, S.
          Restful Web APIs.
          O'Reilly Media, 2013.

    \item SPURLOCK, J.
          Bootstrap.
          O'Reilly Media, 2013.

    \item MENEZES, N. N. C.
          Introdução a programação com Python.
          Novatec, 2014.
    \item STALLINGS, W.
          Criptografia e segurança de redes.
          Prentice-Hall, 4a edição, 2007.

    \item LEDFORD, Jerri L.
          Google AdSense for Dummies.
          John Wiley Consumer, 1a edição, 2008.
	\item 	GONÇALVES, A. \textbf{Introdução à plataforma Java EE 6 com Glassfish 3.} Ciência Moderna, 2011;
	\item 	DEREK, L. \textbf{EJB3 em Ação.} Alta Books, 2008.
	\item Mullender, S. (Editor), Distributed Systems, Addison Wesley Publishing Company; 2nd edition, ISBN: 0-2016-2427-3, 1993.
\end{itemize}

\paragraph{Periódicos, bases de dados específicas, revistas e acervo em multimídia}\
\
O IFPB-Campus Guarabira possui acesso a periódicos nos principais portais de periódicos da área de computação, como o IEEE Xplore e a ACM Digital Library.

\subsubsection{Serviço de acesso ao acervo}

Cada aluno fará seu cadastro na biblioteca apresentando documentação comprobatória de matrícula. Após a ratificação do cadastro, os funcionários da biblioteca, designados para este fim, deverão emprestar, bem como reservar por tempo determinado, as referências bibliográficas solicitadas pelos alunos. Contudo, alguns exemplares, por possuírem poucas unidades, não poderão ser retirados da instituição. Quanto a empréstimo entre bibliotecas e/ou comutação bibliográfica, esta instituição ainda não possui condições favoráveis para  tal atividade.

A Biblioteca deverá operar com um sistema completamente informatizado, possibilitando fácil acesso via terminal ao acervo da biblioteca. O sistema informatizado propicia a reserva de exemplares cuja política de empréstimos prevê um prazo máximo de 14 (catorze) dias para o aluno e professores, além de manter pelo menos 1 (um) volume para consultas na própria Instituição. 

O acervo deverá estar dividido por áreas de conhecimento, facilitando, assim, a procura por títulos específicos, com exemplares de livros e periódicos, contemplando todas as áreas de abrangência do curso. Deve oferecer serviços de empréstimo, renovação e reserva de material, consultas informatizadas a bases de dados e ao acervo, orientação na normalização de trabalhos acadêmicos, orientação bibliográfica e visitas orientadas.

\subsubsection{Pessoal técnico-administrativo}


\subsection{Espa\c{c}os F\'isicos Utilizados no Desenvolvimento do Curso}

\subsubsection{Sala de professores e sala de reuni\~oes}

Nas instalações atuais do IFPB \textit{campus} Guarabira estão disponíveis três salas para professores, sendo as áreas das salas de \unit{45}{\squaremetre}, \unit{20}{\squaremetre} e \unit{7,5}{\squaremetre}. As salas possuem bancada de trabalho e a maior possui uma grande mesa. As reuniões são realizadas nesta sala maior ou na sala de áudio e vídeo.

%\subsubsection{Gabinetes de trabalho para docentes}

\subsubsection{Salas de aula}

Nas salas de aula serão realizadas a maioria das atividades pedagógicas do curso. Devido a isso, é necessário que elas estejam muito bem preparadas para a recepção aos alunos e professores e para a ministração do conteúdo abordado em cada disciplina. Também é importante notar que, tratando-se do curso Superior de Tecnologia em Sistemas para a Internet, para que as aulas práticas sejam realizadas, é necessário que existam laboratórios equipados com computadores,  sendo estes laboratórios uma sala de aula um pouco diferenciada da tradicional. Tratando-se de salas de aula tradicionais, as instalações atuais possuem um total de 7 salas de \unit{45}{\squaremetre} e uma de \unit{30}{\squaremetre}. As salas são climatizadas, possuem marcador para quadro branco e um projetor de vídeo fixo no teto.

\subsubsection{Laborat\'orios e Ambientes Espec\'ificos para o Curso}

	
	De acordo com o Catálogo Nacional dos Cursos Superiores de Tecnologia, a seguinte infraestrutura é recomendada para o Curso Superior de Tecnologia em Sistemas para Internet:
	
\begin{itemize}
	\item laboratório de arquitetura de computadores
	\item laboratório de informática com programas específicos e conectados à Internet 
	\item laboratório de redes de computadores
\end{itemize}

	Atualmente as instalações possuem dois laboratórios de informática, com programas específicos da área de informática instalados e com todos os computadores conectados à Internet. Um dos laboratórios é localizado no térreo e o outro no primeiro andar do prédio atual. O laboratório do térreo é climatizado, possui \unit{72}{\squaremetre}, quadro branco, projetor de vídeo e 31 computadores. O laboratório do primeiro andar também é climatizado, possui \unit{45}{\squaremetre}, quadro branco, projetor de vídeo e 24 computadores. Todos os computadores já são configuradas para atender às disciplinas da área de informática e atualmente já são utilizados nas disciplinas do curso técnico integrado em informática. Nesses laboratórios serão realizadas as aulas práticas das disciplinas relacionadas a informática do curso de Sistemas para Internet.
	
	Para melhor atender ao curso, foi criada uma comissão de projetos da área de informática, por meio da portaria de número 005/2015-Campus Guarabira, visando elaborar projetos para aquisição de novos equipamentos para melhor estruturar os laboratórios, incluindo a aquisição de equipamentos de redes e kits de prototipagem para estudos em eletrônica digital e arquitetura de computadores. Esses equipamentos poderão ser colocados em um novo espaço ou ser integrados aos laboratórios de informática existentes.


%\subsubsection{Espaço Físico}


