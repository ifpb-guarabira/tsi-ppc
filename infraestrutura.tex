\newpage
\section{Infraestrutura}

\subsection{Espaço Físico Geral}

O \textit{campus} do IFPB em Guarabira conta com uma infraestrutura que inclui salas de aula, laboratórios, auditório, área comum para as refeições dos alunos, biblioteca e demais dependências administrativas. O \textit{campus} também está passando por um processo de expansão, onde um novo conjunto de prédios está sendo construído. As seções a seguir apresentam as instalações físicas atuais do \textit{campus} Guarabira e as que estão sendo construídas. A tabela \ref{table:infra} contém as informações sobre a área total existente e a que está sendo construída. O cronograma de realização das obras e os detalhes sobre o plano de expansão do \textit{campus} estão contidos na seção [por a referência].

\begin{table}[h]
\begin{center}
\caption{Espaço físico geral.}
\begin{tabular}{|l|c|l|c|}
\hline
\multicolumn{1}{|c|}{\textbf{TIPO DE ÁREA}} & \multicolumn{1}{l|}{\textbf{QUANT.}} & \textbf{ÁREA (\squaremetre)} & \textbf{\begin{tabular}[c]{@{}c@{}}HORÁRIO DE\\ FUNCIONAMENTO\end{tabular}} \\ \hline
Salas de aula                               & 12                               & 46,88 \squaremetre cada sala & diurno/noturno                                                              \\ \hline
Sala de áudio e vídeo						& 1								   & 46,88 \squaremetre  &
diurno/noturno																\\ \hline
Auditórios/Anfiteatros                      & 1                                & 156,25 \squaremetre & diurno/noturno                                                              \\ \hline
Salas de Professores                        & 2                                & 15,62 \squaremetre e 46,88 \squaremetre                   & diurno/noturno                                                              \\ \hline
Áreas de Apoio Acadêmico                    & 1                                & 22,20 \squaremetre  & diurno/noturno                                                              \\ \hline
Áreas Administrativas                       & 16                                 & 419,20 \squaremetre (total)                   & diurno/noturno                                                              \\ \hline
Conveniência/Praças                         & 1                                &  171,88 \squaremetre
                  & diurno/noturno                                                              \\ \hline
Banheiros                                   & 4 (uso dos alunos)               &  103 \squaremetre (total)  & diurno/noturno                                                              \\ \hline
Laboratórios                                & 3                                &  156,25 \squaremetre (total)& diurno/noturno                                                              \\ \hline
Biblioteca                                  & 1                                & 156 \squaremetre                   & diurno/noturno                                                              \\ \hline
\multicolumn{1}{|r|}{\textbf{Total}}        & 42                               & 1856.72 \squaremetre    & \multicolumn{1}{l|}{}                                                       \\ \hline
\end{tabular}
\label{table:infra}
\end{center}
\end{table}
%colocar subtopicos

\subsubsection{Infraestrutura de Segurança}

O \textit{campus} do IFPB em Guarabira conta com os seguintes itens de segurança e conservação do espaço físico:

\begin{itemize}
\item Serviço de vigilância privada contando com 3 vigilantes 24 horas por dia 7 dias por semana;
\item Pessoal de limpeza terceirizado (7 pessoas) responsável pela conservação das instalações do \textit{campus} (espaços administrativos, salas de aula, laboratórios, etc);
\item Equipamentos de proteção individuais diversos;
\item Equipamentos de prevenção de incêndio (extintores);
\end{itemize} 

\subsubsection{Recursos Audiovisuais e Multimídia}

Na tabela \ref{table:audiovisual} estão especificados os equipamentos audiovisuais disponíveis na instituição para serem utilizados pelos professores e alunos do curso.

\begin{table}[h]
\caption{Relação de recursos audiovisuais e multimídia}
\begin{center}
\begin{tabular}{|p{12cm}|r|}
\hline
\multicolumn{1}{|c|}{\textbf{Item}} & \multicolumn{1}{c|}{\textbf{Quantidade}} \\ \hline
TV  LED de 50 polegadas             & 1                                        \\ \hline
Projetor Multimídia                 & 4                                        \\ \hline
Quadro branco                       & 36                                       \\ \hline
Computadores                        & 90                                       \\ \hline
\end{tabular}
\end{center}
\label{table:audiovisual}
\end{table}

\subsubsection{Manutenção e conservação das instalações físicas}

A manutenção e conservação das instalações são feitas por funcionários terceirizados. Atualmente, a equipe de apoio conta com 7 funcionários. Estes são responsáveis pela limpeza e conservação das instalações, cuidar do jardim, realizar pinturas na alvenaria, implantar instalações elétricas e efetuar pequenos consertos na estrutura física do prédio.

\subsubsection{Manutenção e conservação e expansão dos equipamentos}

A manutenção dos equipamentos 

\subsubsection{Condições de acesso para portadores de necessidades especiais}

	Desde o início de suas atividades, o IFPB, Campus Guarabira, tem feito esforços para promover o atendimento a pessoas com deficiência em conformidade com as diretrizes contidas no PDI da Instituição (pp. 184-185). No tocante à estrutura física atual (pr\'edio provis\'orio no antigo CAIC), ainda existem algumas defici\^encias, uma vez que nem todos os setores do IFPB \'e acess\'ivel a portadores de defici\^encia. No entanto, os princiapais setores de apoio ao p\'ublico, bem como um conjunto de salas de aula, laborat\'orio e banheiro foram organizados de forma a atender portadores de defici\^encia. No pr\'edio definitivo, onde provavelmente o curso superior de Tecnologia em Sistemas para Internet ir\'a funcionar, a infra-estrutura ser\'a completamente adaptada aos portadores de necessidades especiais.

	Dessa forma, o IFPB, em observância à legislação específica, tem consolidado sua política de atendimento a pessoas com deficiência, procurando assegurar-lhes o pleno direito à educação para todos e efetivar ações pedagógicas visando à redução das diferenças e à eficácia da aprendizagem.
 
	O IFPB \textit{Campus} Guarabira, especificamente, conta com um Núcleo de Apoio às pessoas com necessidades Especiais – NAPNE, que conta atualmente com 2 interpretes. \`A medida que o corpo de t\'ecnicos administrativos do campus se fortalecer, novos membros poder\~ao ser designados ao NAPNE, como psic\'ologo, m\'edico e assistente social.

	%O NAPNE tem trabalhado no sentido de melhorar ainda mais a acessibilidade do Campus, solicitando, junto à direção deste, a instalação de piso tátil, faixa contrastante e a adequação dos balcões de atendimento.

%	Este Núcleo também tem trabalhado com diversas instituições que prestam assistência às pessoas com deficiência, no sentido de diagnosticar possíveis carências no acesso às pessoas com deficiência. Entre essas instituições: SCG (Associação de Surdos de Campina), Instituto dos Cegos, Escola de Auto-comunicação de Campina Grande, ICAE (Instituto Campinense de Atendimento ao Excepcional), ICACE e FDC.

	Quando se fala em ambiente universitário (especificamente a biblioteca) percebe-se ser muito intenso o processo de isolamento informacional aos portadores de necessidades especiais. Os aspectos infra e superestruturais desencadeiam um espectro de mitificação ao portador quando não têm acesso amplo ao local.
Falar dos portadores de necessidades especiais remete a um termo bastante conhecido, mas que precisa ser refletido sob vertentes sólidas: acessibilidade. Esta é a palavra chave que promoverá a discussão e as possíveis reflexões para a efetiva inserção desses indivíduos no meio acadêmico, neste contexto, a ampliação do acesso informacional pelo viés biblioteca.

	Dessa forma, fica a indagação: o que efetivamente é acessibilidade? Como ela se aplica à realidade dos portadores de necessidades especiais na biblioteca universitária? No dicionário Aurélio (2005), afirmando um conceito amplo de acessibilidade voltado, sobretudo, para a educação especial: “Condição de acesso aos serviços de informação, documentação e comunicação, por parte de portador de necessidades especiais”.

	Vale ressaltar que o processo de acessibilidade acelerou-se sobremaneira com a difusão da rede computadores, principalmente a internet na década de 90, ampliando o conceito de acessibilidade e tentando efetivar uma maior interação entre os portadores de necessidades especiais com os ambientes de espaços físicos (transportes, saúde, lazer) a partir do mundo digital (redes de computadores e sistemas de informação e comunicação).

	Porém, é importante fazer a ressalva de que um grande imbróglio surge em decorrência desse processo, uma vez que esse surgiu nos Estados Unidos e não foi adaptado, de maneira adequada, à realidade socioeconômica de países como o Brasil. Mesmo os acessos ao computador e internet terem aumentado no Brasil, o número de pessoas atendidas ainda é insignificante quando se trata de “inclusão digital”.

	Dessa forma, é possível perceber o relevante papel das bibliotecas universitárias como instrumento para essa inclusão digital. Por esta razão é que, a partir da criação da portaria nº 1.679 que dispõe acerca da exigência de requisitos de acessibilidade para pessoas com deficiência, o MEC passou a avaliar as bibliotecas dos cursos pela acessibilidade desde 1999.

	De acordo com Mazzoni (2005, p. 6), o artigo primeiro dessa Portaria determina que sejam incluídos nos instrumentos destinados à avaliação das condições de oferta de cursos superiores, para fins de autorização e reconhecimento e de credenciamento de instituições de ensino superior, bem como para a renovação, conforme as normas em vigor, requisitos de acessibilidade de pessoas portadoras de necessidades especiais. Além das determinações da referida Portaria, há a Norma Brasil 9050, da Associação Brasileira de Normas Técnicas, que trata da “Acessibilidade de Pessoas Portadoras de Deficiências e Edificações, Espaço Mobiliário e Equipamentos Urbanos”, que apresenta outras indicações para um correto atendimento às pessoas em situação de deficiência física, deficiência visual e deficiência auditiva.

	Com efeito, malgrado a criação de lei, não é esta que irá estabelecer os elos e ligações para o efetivo cumprimento das causas voltadas para o portador de necessidades especiais. A temática permeia um aspecto mais lato direcionado a questões sociais, políticas e epistemológicas. Os dois primeiros referem-se a questões de influência, que as autoridades podem encaminhar; já o terceiro concerne à concepção de que para suprir as necessidades dos portadores de necessidades especiais são necessários vários estudos, tanto de estruturas físicas, como de acesso ideológico.

	Assim, é perceptível que muitas universidades não têm avaliado a biblioteca como um importante instrumento de contribuição ao ensino, pesquisa e extensão, mas apenas como um espaço de livros armazenados. A mitificação de que a biblioteca é um espaço para uma minoria é uma realidade para grande parte da sociedade e para o próprio bibliotecário, embora na universidade essa incidência de acesso às estruturas da biblioteca seja menor.

	Percebe-se que a questão do acesso por parte do portador de necessidades especiais é, em primeiro caso, algo que permeia o aspecto sócio-político.

	Faz-se necessário tratar essa perspectiva de amplo acesso às dependências da biblioteca, disponibilizando investimentos para que possam ser feitos estudos arquitetônicos e científicos, a fim de oferecer grandes estruturas físicas, bem como de acervo e orientação para a ampliação de conhecimentos para os portadores de necessidades especiais, fomentando três condições básicas de acesso: urbanística (caminhos de acesso, estacionamento); arquitetônicos (iluminação, ventilação, banheiros, rampas adequadas) e informação e comunicação (sinalização, sistema de consulta e empréstimo, tecnologia de apoio para usuários portadores de necessidades especiais).
	
	É preciso valorizar os usuários com as mais diversas necessidades no sentido de conferir-lhe respaldo informacional, seja no aspecto sensorial (audição e visão), seja no físico (de locomoção ou coordenação), visando tornar o acesso à biblioteca pelos portadores de necessidades especiais uma realidade diferente daquela ainda presente em muitas universidades. De modo a cessar, ao menos dentro do ambiente acadêmico, especificamente, a biblioteca, constrangimentos tão comuns gerados pelas deficitárias estruturas urbanística, física e de informação que estão aquém das condições necessárias a essas pessoas.
	
	O fato do sistema de bibliotecas de uma dada universidade ser ou não centralizado pode influir nesse processo de acessibilidade. Entretanto, é preciso realçar que a estrutura da biblioteca não pode ser centralizada arbitrariamente na segmentação dos padrões geográficos e físicos da universidade.


\subsection{Espa\c{c}os F\'isicos Utilizados no Desenvolvimento do Curso}

\subsubsection{Sala de professores e sala de reuni\~oes}

\subsubsection{Gabinetes de trabalho para docentes}

\subsubsection{Salas de aula}

%colocar subtopicos

\subsection{Biblioteca}

\subsubsection{Apresenta\c{c}\~ao}

A Biblioteca deverá operar com um sistema completamente informatizado, possibilitando fácil acesso via terminal, ao acervo da biblioteca. O sistema informatizado propicia a reserva de exemplares, cuja política de empréstimos prevê um prazo máximo de 14 (catorze) dias para os alunos e professores, além de manter pelo menos 1 (um) volume para consultas na própria Instituição.

O acervo deverá estar dividido por áreas de conhecimento, facilitando, assim, a procura por títulos específicos, com exemplares de livros e periódicos, contemplando todas as áreas de abrangência do curso. Deve oferecer serviços de empréstimo, renovação e reserva de material, consultas informatizadas às bases de dados e ao acervo, orientação na normalização de trabalhos acadêmicos, orientação bibliográfica e visitas orientadas.

\subsubsection{Espa\c{c}o f\'isico}

No pr\'edio provis\'orio do IFPB Guarabira (antigo CAIC) existe um espa\c{c} de cerca de X metros quadrados para a biblioteca. O ambiente \'e climatizado e possui atualmente um acervo de Y livros. O ambiente possui mesas e computadores com acesso \`a internet para realiza\c{c}\~ao de estudos em grupo ou individual.

\subsubsection{Instala\c{c}\~oes para o acervo}

%colocar subtopicos

A biblioteca possui prédio próprio para guarda o acervo que deverá ser exposto em estantes metálicas, contendo em cada uma títulos específicos. 

%Ainda sobre o ponto de vista de higienização, sugere-se a instalação de condicionadores de ar para evitar mofo.

\paragraph{Acervo atual}

No quadro a seguir, apresentamos uma descrição do acervo contido em nossa biblioteca.

\subsection{Laborat\'orios e Ambientes Espec\'ificos para o Curso}

%colocar subtopicos 


