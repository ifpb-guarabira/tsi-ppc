\newpage
\section{Infraestrutura}

\subsection{Espa\c{c}o F\'isico Geral}

O quadro a seguir apresenta a estrutura física m\'inima necessária ao funcionamento do Curso Tecn\'ologo em Sistemas para Internet e dispon\'ivel no campus Guarabira. Os demais quadros apresentam a relação detalhada dos equipamentos para os laboratórios.

\begin{table}[h]
\caption{Estrutura F\'isica do Campus}
\begin{center}
\begin{tabular}{|p{4.5cm}|p{2.2cm}|p{2.2cm}|p{2.2cm}|p{3.0cm}|}
\hline
Tipo de \'Area & Quantidade necess\'ario & Quantidade dispon\'ivel & \'Area  & Hor\'ario de Funcionamento\\
\hline 
\hline
Salas de aula &  6 & 28 (verificar) & (verificar) & diurno/noturno \\
\hline
Audit\'orios/Anfiteatros &  1 & 1 & (verificar) & diurno/noturno \\
\hline
Salas de professores & 1 & 2 (verificar) & (verificar) & diurno/noturno \\
\hline
\'Areas de apoio acad\^emico & 1 & 1 (verificar) & (verificar) & diurno/noturno \\
\hline
\'Areas Administrativas & X & X (verificar) & (verificar) & diurno/noturno \\
\hline
Coveni\^encia/pra\c{c}as & 1 & 1 (verificar) & (verificar) & diurno/noturno \\
\hline
Banheiros & 1 & 4 (verificar) & 4 (verificar) & diurno/noturno \\
\hline
Laborat\'orios & 2 & 2 (verificar) & 4 (verificar) & diurno/noturno \\
\hline
Biblioteca & 1 & 1 (verificar) & 4 (verificar) & diurno/noturno \\
\hline
\end{tabular} 
\end{center}
\label{tab:pl}
\end{table}
%colocar subtopicos

\subsubsection{Infraestrutura de Seguran\c{c}a}

A prevenção de lesões aos trabalhadores requer a introdução de alterações, dos padrões de trabalho, tais como a passagem de horários noturnos para diurnos, o melhoramento das condições de contratação, valorizando a qualidade do serviço em detrimento do preço, e melhorando a relação entre o docente e discente, podem reduzir diretamente o risco de les\~oes.

Os perigos e riscos que os professores enfrentam incluem:

\begin{itemize}
\item Exposição a substâncias perigosas, incluindo agentes biológicos que podem causar asma, alergias, e infecções no sangue;
Ruído e vibração;
\item Escorregamento, tropeções e quedas durante ``o trabalho em piso molhado'';
\item Acidentes de origem elétrica provocados pelo equipamento de trabalho;
\item Risco de lesões musculoesqueléticas;
\item Trabalho solitário, estresse profissional, violência, e assédio moral (\textit{bullying});
\item Ritmos e horários de trabalho irregulares.
\end{itemize}

\subsubsection{Recursos Audiovisuais e Multim\'idia}

No quadro a seguir estão especificados os equipamentos audiovisuais a serem utilizados pelos professores e alunos do curso.

\begin{table}[h]
\caption{Rela\c{c}\~ao de recursos audiovisuais e multim\'idia}
\begin{center}
\begin{tabular}{|p{4.5cm}|p{2.5cm}|p{4.5cm}|}
\hline
Tipo de Equipamento & Quantidade necess\'ario & Observa\c{c}\~oes\\
\hline 
\hline
TV LED 50'' ou Projetor multim\'idia & 32 (verificar) &  Localizadas em cada sala de aula\\
\hline
Projetor multim\'idia &  4 (verificar) & Dispon\'ivel para os laborat\'orios \\
\hline
Quadro branco & 36 (verificar) & Localizados em cada sala de aula e laborat\'orios \\
\hline
Computadores & 90 (verificar) & Distribu\'idos nos laborat\'orios \\
\hline
\end{tabular} 
\end{center}
\label{tab:pl}
\end{table}

\subsubsection{Manuten\c{c}\~ao e conserva\c{c}\~ao das instala\c{c}\~oes f\'isicas}


\textbf{Corretiva} – corrige falhas detectadas que prejudicam o funcionamento normal dos equipamentos. A quebra de uma máquina pode deixar outros equipamentos ociosos.

\vspace{2mm}
\textbf{Preventiva} – Tem vantagens óbvias, mas por ser um programa de implantação difícil, tem um custo elevado.


\subsubsection{Manutenção, conservação e expansão dos equipamentos}

\textbf{Atendimento: } o setor que necessitar de algum dos serviços prestados pelo Setor de Manutenção e Conservação deverá solicitar o atendimento via memorando, telefone ou pessoalmente, no horário das 7h às 18h. 

	A possibilidade de solicitações via Internet está em estudo. No atendimento, o setor solicitante deverá fazer uma descrição preliminar do serviço a ser realizado, informando, também, o nome do setor e o número do ramal.

\vspace{2mm}
\textbf{Manuten\c{c}\~ao:} após o diagnóstico da solicitação, o Setor de Manutenção e Conservação informará ao setor requerente uma previsão de atendimento, esclarecendo que este ficará condicionado à disponibilidade dos materiais à execução do serviço, se necessário.

	Caso o equipamento exija assistência técnica especializada, que não conste no quadro do referido setor será encaminhado para empresas que estejam aptas a prestarem serviços para o estado, cabendo àquele acompanhar e fiscalizar a qualidade dos serviços prestados, bem como os prazos de entrega e de garantia do serviço.

\subsubsection{Condições de acesso para portadores de necessidades especiais}

	Desde o início de suas atividades, o IFPB, Campus Guarabira, tem feito esforços para promover o atendimento a pessoas com deficiência em conformidade com as diretrizes contidas no PDI da Instituição (pp. 184-185). No tocante à estrutura física atual (pr\'edio provis\'orio no antigo CAIC), ainda existem algumas defici\^encias, uma vez que nem todos os setores do IFPB \'e acess\'ivel a portadores de defici\^encia. No entanto, os princiapais setores de apoio ao p\'ublico, bem como um conjunto de salas de aula, laborat\'orio e banheiro foram organizados de forma a atender portadores de defici\^encia. No pr\'edio definitivo, onde provavelmente o curso superior de Tecnologia em Sistemas para Internet ir\'a funcionar, a infra-estrutura ser\'a completamente adaptada aos portadores de necessidades especiais.

	Dessa forma, o IFPB, em observância à legislação específica, tem consolidado sua política de atendimento a pessoas com deficiência, procurando assegurar-lhes o pleno direito à educação para todos e efetivar ações pedagógicas visando à redução das diferenças e à eficácia da aprendizagem.
 
	O IFPB Campus Guarabira, especificamente, conta com um Núcleo de Apoio às pessoas com necessidades Especiais – NAPNE, que conta atualmente com 2 interpretes. \`A medida que o corpo de t\'ecnicos administrativos do campus se fortalecer, novos membros poder\~ao ser designados ao NAPNE, como psic\'ologo, m\'edico e assistente social.

	%O NAPNE tem trabalhado no sentido de melhorar ainda mais a acessibilidade do Campus, solicitando, junto à direção deste, a instalação de piso tátil, faixa contrastante e a adequação dos balcões de atendimento.

%	Este Núcleo também tem trabalhado com diversas instituições que prestam assistência às pessoas com deficiência, no sentido de diagnosticar possíveis carências no acesso às pessoas com deficiência. Entre essas instituições: SCG (Associação de Surdos de Campina), Instituto dos Cegos, Escola de Auto-comunicação de Campina Grande, ICAE (Instituto Campinense de Atendimento ao Excepcional), ICACE e FDC.

	Quando se fala em ambiente universitário (especificamente a biblioteca) percebe-se ser muito intenso o processo de isolamento informacional aos portadores de necessidades especiais. Os aspectos infra e superestruturais desencadeiam um espectro de mitificação ao portador quando não têm acesso amplo ao local.
Falar dos portadores de necessidades especiais remete a um termo bastante conhecido, mas que precisa ser refletido sob vertentes sólidas: acessibilidade. Esta é a palavra chave que promoverá a discussão e as possíveis reflexões para a efetiva inserção desses indivíduos no meio acadêmico, neste contexto, a ampliação do acesso informacional pelo viés biblioteca.

	Dessa forma, fica a indagação: o que efetivamente é acessibilidade? Como ela se aplica à realidade dos portadores de necessidades especiais na biblioteca universitária? No dicionário Aurélio (2005), afirmando um conceito amplo de acessibilidade voltado, sobretudo, para a educação especial: “Condição de acesso aos serviços de informação, documentação e comunicação, por parte de portador de necessidades especiais”.

	Vale ressaltar que o processo de acessibilidade acelerou-se sobremaneira com a difusão da rede computadores, principalmente a internet na década de 90, ampliando o conceito de acessibilidade e tentando efetivar uma maior interação entre os portadores de necessidades especiais com os ambientes de espaços físicos (transportes, saúde, lazer) a partir do mundo digital (redes de computadores e sistemas de informação e comunicação).

	Porém, é importante fazer a ressalva de que um grande imbróglio surge em decorrência desse processo, uma vez que esse surgiu nos Estados Unidos e não foi adaptado, de maneira adequada, à realidade socioeconômica de países como o Brasil. Mesmo os acessos ao computador e internet terem aumentado no Brasil, o número de pessoas atendidas ainda é insignificante quando se trata de “inclusão digital”.

	Dessa forma, é possível perceber o relevante papel das bibliotecas universitárias como instrumento para essa inclusão digital. Por esta razão é que, a partir da criação da portaria nº 1.679 que dispõe acerca da exigência de requisitos de acessibilidade para pessoas com deficiência, o MEC passou a avaliar as bibliotecas dos cursos pela acessibilidade desde 1999.

	De acordo com Mazzoni (2005, p. 6), o artigo primeiro dessa Portaria determina que sejam incluídos nos instrumentos destinados à avaliação das condições de oferta de cursos superiores, para fins de autorização e reconhecimento e de credenciamento de instituições de ensino superior, bem como para a renovação, conforme as normas em vigor, requisitos de acessibilidade de pessoas portadoras de necessidades especiais. Além das determinações da referida Portaria, há a Norma Brasil 9050, da Associação Brasileira de Normas Técnicas, que trata da “Acessibilidade de Pessoas Portadoras de Deficiências e Edificações, Espaço Mobiliário e Equipamentos Urbanos”, que apresenta outras indicações para um correto atendimento às pessoas em situação de deficiência física, deficiência visual e deficiência auditiva.

	Com efeito, malgrado a criação de lei, não é esta que irá estabelecer os elos e ligações para o efetivo cumprimento das causas voltadas para o portador de necessidades especiais. A temática permeia um aspecto mais lato direcionado a questões sociais, políticas e epistemológicas. Os dois primeiros referem-se a questões de influência, que as autoridades podem encaminhar; já o terceiro concerne à concepção de que para suprir as necessidades dos portadores de necessidades especiais são necessários vários estudos, tanto de estruturas físicas, como de acesso ideológico.

	Assim, é perceptível que muitas universidades não têm avaliado a biblioteca como um importante instrumento de contribuição ao ensino, pesquisa e extensão, mas apenas como um espaço de livros armazenados. A mitificação de que a biblioteca é um espaço para uma minoria é uma realidade para grande parte da sociedade e para o próprio bibliotecário, embora na universidade essa incidência de acesso às estruturas da biblioteca seja menor.

	Percebe-se que a questão do acesso por parte do portador de necessidades especiais é, em primeiro caso, algo que permeia o aspecto sócio-político.

	Faz-se necessário tratar essa perspectiva de amplo acesso às dependências da biblioteca, disponibilizando investimentos para que possam ser feitos estudos arquitetônicos e científicos, a fim de oferecer grandes estruturas físicas, bem como de acervo e orientação para a ampliação de conhecimentos para os portadores de necessidades especiais, fomentando três condições básicas de acesso: urbanística (caminhos de acesso, estacionamento); arquitetônicos (iluminação, ventilação, banheiros, rampas adequadas) e informação e comunicação (sinalização, sistema de consulta e empréstimo, tecnologia de apoio para usuários portadores de necessidades especiais).
	
	É preciso valorizar os usuários com as mais diversas necessidades no sentido de conferir-lhe respaldo informacional, seja no aspecto sensorial (audição e visão), seja no físico (de locomoção ou coordenação), visando tornar o acesso à biblioteca pelos portadores de necessidades especiais uma realidade diferente daquela ainda presente em muitas universidades. De modo a cessar, ao menos dentro do ambiente acadêmico, especificamente, a biblioteca, constrangimentos tão comuns gerados pelas deficitárias estruturas urbanística, física e de informação que estão aquém das condições necessárias a essas pessoas.
	
	O fato do sistema de bibliotecas de uma dada universidade ser ou não centralizado pode influir nesse processo de acessibilidade. Entretanto, é preciso realçar que a estrutura da biblioteca não pode ser centralizada arbitrariamente na segmentação dos padrões geográficos e físicos da universidade.


\subsection{Espa\c{c}os F\'isicos Utilizados no Desenvolvimento do Curso}

\subsubsection{Sala de professores e sala de reuni\~oes}

\subsubsection{Gabinetes de trabalho para docentes}

\subsubsection{Salas de aula}

%colocar subtopicos

\subsection{Biblioteca}

\subsubsection{Apresenta\c{c}\~ao}

A Biblioteca deverá operar com um sistema completamente informatizado, possibilitando fácil acesso via terminal, ao acervo da biblioteca. O sistema informatizado propicia a reserva de exemplares, cuja política de empréstimos prevê um prazo máximo de 14 (catorze) dias para os alunos e professores, além de manter pelo menos 1 (um) volume para consultas na própria Instituição.

O acervo deverá estar dividido por áreas de conhecimento, facilitando, assim, a procura por títulos específicos, com exemplares de livros e periódicos, contemplando todas as áreas de abrangência do curso. Deve oferecer serviços de empréstimo, renovação e reserva de material, consultas informatizadas às bases de dados e ao acervo, orientação na normalização de trabalhos acadêmicos, orientação bibliográfica e visitas orientadas.

\subsubsection{Espa\c{c}o f\'isico}

No pr\'edio provis\'orio do IFPB Guarabira (antigo CAIC) existe um espa\c{c} de cerca de X metros quadrados para a biblioteca. O ambiente \'e climatizado e possui atualmente um acervo de Y livros. O ambiente possui mesas e computadores com acesso \`a internet para realiza\c{c}\~ao de estudos em grupo ou individual.

\subsubsection{Instala\c{c}\~oes para o acervo}

%colocar subtopicos

A biblioteca possui prédio próprio para guarda o acervo que deverá ser exposto em estantes metálicas, contendo em cada uma títulos específicos. 

%Ainda sobre o ponto de vista de higienização, sugere-se a instalação de condicionadores de ar para evitar mofo.

\paragraph{Acervo atual}

No quadro a seguir, apresentamos uma descrição do acervo contido em nossa biblioteca.

\subsection{Laborat\'orios e Ambientes Espec\'ificos para o Curso}

%colocar subtopicos 


